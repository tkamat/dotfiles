% \iffalse meta-comment
%
% File: mleftright.dtx
% Version: 2016/05/16 v1.1
% Info: Math left/right delim. as open/close
%
% Copyright (C) 2010 by
%    Heiko Oberdiek <heiko.oberdiek at googlemail.com>
%    2016
%    https://github.com/ho-tex/oberdiek/issues
%
% This work may be distributed and/or modified under the
% conditions of the LaTeX Project Public License, either
% version 1.3c of this license or (at your option) any later
% version. This version of this license is in
%    http://www.latex-project.org/lppl/lppl-1-3c.txt
% and the latest version of this license is in
%    http://www.latex-project.org/lppl.txt
% and version 1.3 or later is part of all distributions of
% LaTeX version 2005/12/01 or later.
%
% This work has the LPPL maintenance status "maintained".
%
% This Current Maintainer of this work is Heiko Oberdiek.
%
% The Base Interpreter refers to any `TeX-Format',
% because some files are installed in TDS:tex/generic//.
%
% This work consists of the main source file mleftright.dtx
% and the derived files
%    mleftright.sty, mleftright.pdf, mleftright.ins, mleftright.drv,
%    mleftright-test1.tex.
%
% Distribution:
%    CTAN:macros/latex/contrib/oberdiek/mleftright.dtx
%    CTAN:macros/latex/contrib/oberdiek/mleftright.pdf
%
% Unpacking:
%    (a) If mleftright.ins is present:
%           tex mleftright.ins
%    (b) Without mleftright.ins:
%           tex mleftright.dtx
%    (c) If you insist on using LaTeX
%           latex \let\install=y% \iffalse meta-comment
%
% File: mleftright.dtx
% Version: 2016/05/16 v1.1
% Info: Math left/right delim. as open/close
%
% Copyright (C) 2010 by
%    Heiko Oberdiek <heiko.oberdiek at googlemail.com>
%    2016
%    https://github.com/ho-tex/oberdiek/issues
%
% This work may be distributed and/or modified under the
% conditions of the LaTeX Project Public License, either
% version 1.3c of this license or (at your option) any later
% version. This version of this license is in
%    http://www.latex-project.org/lppl/lppl-1-3c.txt
% and the latest version of this license is in
%    http://www.latex-project.org/lppl.txt
% and version 1.3 or later is part of all distributions of
% LaTeX version 2005/12/01 or later.
%
% This work has the LPPL maintenance status "maintained".
%
% This Current Maintainer of this work is Heiko Oberdiek.
%
% The Base Interpreter refers to any `TeX-Format',
% because some files are installed in TDS:tex/generic//.
%
% This work consists of the main source file mleftright.dtx
% and the derived files
%    mleftright.sty, mleftright.pdf, mleftright.ins, mleftright.drv,
%    mleftright-test1.tex.
%
% Distribution:
%    CTAN:macros/latex/contrib/oberdiek/mleftright.dtx
%    CTAN:macros/latex/contrib/oberdiek/mleftright.pdf
%
% Unpacking:
%    (a) If mleftright.ins is present:
%           tex mleftright.ins
%    (b) Without mleftright.ins:
%           tex mleftright.dtx
%    (c) If you insist on using LaTeX
%           latex \let\install=y% \iffalse meta-comment
%
% File: mleftright.dtx
% Version: 2016/05/16 v1.1
% Info: Math left/right delim. as open/close
%
% Copyright (C) 2010 by
%    Heiko Oberdiek <heiko.oberdiek at googlemail.com>
%    2016
%    https://github.com/ho-tex/oberdiek/issues
%
% This work may be distributed and/or modified under the
% conditions of the LaTeX Project Public License, either
% version 1.3c of this license or (at your option) any later
% version. This version of this license is in
%    http://www.latex-project.org/lppl/lppl-1-3c.txt
% and the latest version of this license is in
%    http://www.latex-project.org/lppl.txt
% and version 1.3 or later is part of all distributions of
% LaTeX version 2005/12/01 or later.
%
% This work has the LPPL maintenance status "maintained".
%
% This Current Maintainer of this work is Heiko Oberdiek.
%
% The Base Interpreter refers to any `TeX-Format',
% because some files are installed in TDS:tex/generic//.
%
% This work consists of the main source file mleftright.dtx
% and the derived files
%    mleftright.sty, mleftright.pdf, mleftright.ins, mleftright.drv,
%    mleftright-test1.tex.
%
% Distribution:
%    CTAN:macros/latex/contrib/oberdiek/mleftright.dtx
%    CTAN:macros/latex/contrib/oberdiek/mleftright.pdf
%
% Unpacking:
%    (a) If mleftright.ins is present:
%           tex mleftright.ins
%    (b) Without mleftright.ins:
%           tex mleftright.dtx
%    (c) If you insist on using LaTeX
%           latex \let\install=y% \iffalse meta-comment
%
% File: mleftright.dtx
% Version: 2016/05/16 v1.1
% Info: Math left/right delim. as open/close
%
% Copyright (C) 2010 by
%    Heiko Oberdiek <heiko.oberdiek at googlemail.com>
%    2016
%    https://github.com/ho-tex/oberdiek/issues
%
% This work may be distributed and/or modified under the
% conditions of the LaTeX Project Public License, either
% version 1.3c of this license or (at your option) any later
% version. This version of this license is in
%    http://www.latex-project.org/lppl/lppl-1-3c.txt
% and the latest version of this license is in
%    http://www.latex-project.org/lppl.txt
% and version 1.3 or later is part of all distributions of
% LaTeX version 2005/12/01 or later.
%
% This work has the LPPL maintenance status "maintained".
%
% This Current Maintainer of this work is Heiko Oberdiek.
%
% The Base Interpreter refers to any `TeX-Format',
% because some files are installed in TDS:tex/generic//.
%
% This work consists of the main source file mleftright.dtx
% and the derived files
%    mleftright.sty, mleftright.pdf, mleftright.ins, mleftright.drv,
%    mleftright-test1.tex.
%
% Distribution:
%    CTAN:macros/latex/contrib/oberdiek/mleftright.dtx
%    CTAN:macros/latex/contrib/oberdiek/mleftright.pdf
%
% Unpacking:
%    (a) If mleftright.ins is present:
%           tex mleftright.ins
%    (b) Without mleftright.ins:
%           tex mleftright.dtx
%    (c) If you insist on using LaTeX
%           latex \let\install=y\input{mleftright.dtx}
%        (quote the arguments according to the demands of your shell)
%
% Documentation:
%    (a) If mleftright.drv is present:
%           latex mleftright.drv
%    (b) Without mleftright.drv:
%           latex mleftright.dtx; ...
%    The class ltxdoc loads the configuration file ltxdoc.cfg
%    if available. Here you can specify further options, e.g.
%    use A4 as paper format:
%       \PassOptionsToClass{a4paper}{article}
%
%    Programm calls to get the documentation (example):
%       pdflatex mleftright.dtx
%       makeindex -s gind.ist mleftright.idx
%       pdflatex mleftright.dtx
%       makeindex -s gind.ist mleftright.idx
%       pdflatex mleftright.dtx
%
% Installation:
%    TDS:tex/generic/oberdiek/mleftright.sty
%    TDS:doc/latex/oberdiek/mleftright.pdf
%    TDS:doc/latex/oberdiek/test/mleftright-test1.tex
%    TDS:source/latex/oberdiek/mleftright.dtx
%
%<*ignore>
\begingroup
  \catcode123=1 %
  \catcode125=2 %
  \def\x{LaTeX2e}%
\expandafter\endgroup
\ifcase 0\ifx\install y1\fi\expandafter
         \ifx\csname processbatchFile\endcsname\relax\else1\fi
         \ifx\fmtname\x\else 1\fi\relax
\else\csname fi\endcsname
%</ignore>
%<*install>
\input docstrip.tex
\Msg{************************************************************************}
\Msg{* Installation}
\Msg{* Package: mleftright 2016/05/16 v1.1 Math left/right delim. as open/close (HO)}
\Msg{************************************************************************}

\keepsilent
\askforoverwritefalse

\let\MetaPrefix\relax
\preamble

This is a generated file.

Project: mleftright
Version: 2016/05/16 v1.1

Copyright (C) 2010 by
   Heiko Oberdiek <heiko.oberdiek at googlemail.com>

This work may be distributed and/or modified under the
conditions of the LaTeX Project Public License, either
version 1.3c of this license or (at your option) any later
version. This version of this license is in
   http://www.latex-project.org/lppl/lppl-1-3c.txt
and the latest version of this license is in
   http://www.latex-project.org/lppl.txt
and version 1.3 or later is part of all distributions of
LaTeX version 2005/12/01 or later.

This work has the LPPL maintenance status "maintained".

This Current Maintainer of this work is Heiko Oberdiek.

The Base Interpreter refers to any `TeX-Format',
because some files are installed in TDS:tex/generic//.

This work consists of the main source file mleftright.dtx
and the derived files
   mleftright.sty, mleftright.pdf, mleftright.ins, mleftright.drv,
   mleftright-test1.tex.

\endpreamble
\let\MetaPrefix\DoubleperCent

\generate{%
  \file{mleftright.ins}{\from{mleftright.dtx}{install}}%
  \file{mleftright.drv}{\from{mleftright.dtx}{driver}}%
  \usedir{tex/generic/oberdiek}%
  \file{mleftright.sty}{\from{mleftright.dtx}{package}}%
  \usedir{doc/latex/oberdiek/test}%
  \file{mleftright-test1.tex}{\from{mleftright.dtx}{test1}}%
  \nopreamble
  \nopostamble
  \usedir{source/latex/oberdiek/catalogue}%
  \file{mleftright.xml}{\from{mleftright.dtx}{catalogue}}%
}

\catcode32=13\relax% active space
\let =\space%
\Msg{************************************************************************}
\Msg{*}
\Msg{* To finish the installation you have to move the following}
\Msg{* file into a directory searched by TeX:}
\Msg{*}
\Msg{*     mleftright.sty}
\Msg{*}
\Msg{* To produce the documentation run the file `mleftright.drv'}
\Msg{* through LaTeX.}
\Msg{*}
\Msg{* Happy TeXing!}
\Msg{*}
\Msg{************************************************************************}

\endbatchfile
%</install>
%<*ignore>
\fi
%</ignore>
%<*driver>
\NeedsTeXFormat{LaTeX2e}
\ProvidesFile{mleftright.drv}%
  [2016/05/16 v1.1 Math left/right delim. as open/close (HO)]%
\documentclass{ltxdoc}
\usepackage{holtxdoc}[2011/11/22]
\usepackage{mleftright}[2016/05/16]
\begin{document}
  \DocInput{mleftright.dtx}%
\end{document}
%</driver>
% \fi
%
%
% \CharacterTable
%  {Upper-case    \A\B\C\D\E\F\G\H\I\J\K\L\M\N\O\P\Q\R\S\T\U\V\W\X\Y\Z
%   Lower-case    \a\b\c\d\e\f\g\h\i\j\k\l\m\n\o\p\q\r\s\t\u\v\w\x\y\z
%   Digits        \0\1\2\3\4\5\6\7\8\9
%   Exclamation   \!     Double quote  \"     Hash (number) \#
%   Dollar        \$     Percent       \%     Ampersand     \&
%   Acute accent  \'     Left paren    \(     Right paren   \)
%   Asterisk      \*     Plus          \+     Comma         \,
%   Minus         \-     Point         \.     Solidus       \/
%   Colon         \:     Semicolon     \;     Less than     \<
%   Equals        \=     Greater than  \>     Question mark \?
%   Commercial at \@     Left bracket  \[     Backslash     \\
%   Right bracket \]     Circumflex    \^     Underscore    \_
%   Grave accent  \`     Left brace    \{     Vertical bar  \|
%   Right brace   \}     Tilde         \~}
%
% \GetFileInfo{mleftright.drv}
%
% \title{The \xpackage{mleftright} package}
% \date{2016/05/16 v1.1}
% \author{Heiko Oberdiek\thanks
% {Please report any issues at https://github.com/ho-tex/oberdiek/issues}\\
% \xemail{heiko.oberdiek at googlemail.com}}
%
% \maketitle
%
% \begin{abstract}
% \TeX\ sets subformulas by \cs{left} and \cs{right} as inner formulas
% with additional surrounding spaces in some situations. This package
% provides \cs{mleft} and \cs{mright} that call \cs{left} and \cs{right},
% but the delimiters will act as normal \cs{mathopen} and \cs{mathclose}
% delimiters without the additional space of an inner formula.
% \end{abstract}
%
% \tableofcontents
%
% \section{Documentation}
%
% The package is a result of a thread in the newsgroup \textsf{comp.text.tex}
% with the subject \textit{spacing after \cs{right}\texttt{)}
% and before \cs{left}\texttt{)}} \cite{dave}.
% The problem: \cs{left} and \cs{right} adjust the size of the
% delimiters automatically. However, \TeX\ treats the whole expression
% as inner formula. In some circumstances \TeX\ adds extra space
% before or after an inner formula.
% Example:
% \begin{quote}
%   \thinmuskip=1.5\thinmuskip
%   \begin{tabular}{@{}l@{\quad$\Rightarrow$\quad}l@{}}
%     |$\sin(x^2), x$|
%     & $\sin(x^2), x$\\
%     |$\sin\left(x^2\right), x$|
%     & $\sin\left(x^2\right), x$\\
%   ^^A  \multicolumn{1}{@{}r@{\quad$\Rightarrow$\quad}}{^^A
%   ^^A    \itshape with exaggerated spacing^^A
%   ^^A  }
%   ^^A  & $\thinmuskip=4\thinmuskip
%   ^^A    \sin\left(x^2\right){,}\mskip.25\thinmuskip x$\\
%     |$\sin\mleft(x^2\mright), x$|
%     & $\sin\mleft(x^2\mright), x$\\
%   \end{tabular}\\*[.5ex]
%   (\cs{mleft} and \cs{mright} are provided by this package.)
% \end{quote}
%
% In the newsgroup Donald Arseneau answered with clever macros \cite{arseneau}:
% \begin{quote}
%\begin{verbatim}
%\newcommand\lft{\mathopen{}\left}
%\newcommand\rgt{\aftergroup\mathclose\aftergroup{\aftergroup}\right}
%\end{verbatim}
% \end{quote}
% However one problem remains, a following subscript or superscript
% is not applied to the right delimiter but the empty
% \cs{mathclose}.
% Thus Philipp Stephani provided an improvement \cite{stephani}:
%\begin{quote}
%\begin{verbatim}
%\mathopen{} \mathclose{\left\| A^2 \right\|}_2
%\end{verbatim}
%\end{quote}
% Heiko Oberdiek converted this into macro form \cite{oberdiek}:
%\begin{quote}
%\begin{verbatim}
%\newcommand\lft{\mathopen{}\mathclose\bgroup\left}
%\newcommand\rgt{\aftergroup\egroup\right}
%\end{verbatim}
%\end{quote}
%
% The package uses longer macro names \cs{mleft} and \cs{mright}
% to avoid name clashes. Also it adds some checks for error conditions.
%
% \subsection{Use}
%
% \begin{declcs}{mleft}\meta{delimL} \dots\unkern\ \cs{mright}\meta{delimR}
% \end{declcs}
% Macros \cs{mleft} and \cs{mright} are used in the same way as
% \cs{left} and \cs{right}. Also \cs{middle} can be used inbetween if
% \eTeX\ is present.
%
% \begin{declcs}{mleftright}
% \end{declcs}
% Macro \cs{mleftright} redefines \cs{left} as \cs{mleft} and
% \cs{right} as \cs{mright}. The redefinition is local to the group.
%
% \begin{declcs}{mleftrightrestore}
% \end{declcs}
% Macro \cs{mleftright} restores \cs{left} and \cs{right} with
% the original meaning if they were previously redefined by
% \cs{mleftright} (also locally).
%
%
% \StopEventually{
% }
%
% \section{Implementation}
%    \begin{macrocode}
%<*package>
%    \end{macrocode}
%    Reload check, especially if the package is not used with \LaTeX.
%    \begin{macrocode}
\begingroup\catcode61\catcode48\catcode32=10\relax%
  \catcode13=5 % ^^M
  \endlinechar=13 %
  \catcode35=6 % #
  \catcode39=12 % '
  \catcode44=12 % ,
  \catcode45=12 % -
  \catcode46=12 % .
  \catcode58=12 % :
  \catcode64=11 % @
  \catcode123=1 % {
  \catcode125=2 % }
  \expandafter\let\expandafter\x\csname ver@mleftright.sty\endcsname
  \ifx\x\relax % plain-TeX, first loading
  \else
    \def\empty{}%
    \ifx\x\empty % LaTeX, first loading,
      % variable is initialized, but \ProvidesPackage not yet seen
    \else
      \expandafter\ifx\csname PackageInfo\endcsname\relax
        \def\x#1#2{%
          \immediate\write-1{Package #1 Info: #2.}%
        }%
      \else
        \def\x#1#2{\PackageInfo{#1}{#2, stopped}}%
      \fi
      \x{mleftright}{The package is already loaded}%
      \aftergroup\endinput
    \fi
  \fi
\endgroup%
%    \end{macrocode}
%    Package identification:
%    \begin{macrocode}
\begingroup\catcode61\catcode48\catcode32=10\relax%
  \catcode13=5 % ^^M
  \endlinechar=13 %
  \catcode35=6 % #
  \catcode39=12 % '
  \catcode40=12 % (
  \catcode41=12 % )
  \catcode44=12 % ,
  \catcode45=12 % -
  \catcode46=12 % .
  \catcode47=12 % /
  \catcode58=12 % :
  \catcode64=11 % @
  \catcode91=12 % [
  \catcode93=12 % ]
  \catcode123=1 % {
  \catcode125=2 % }
  \expandafter\ifx\csname ProvidesPackage\endcsname\relax
    \def\x#1#2#3[#4]{\endgroup
      \immediate\write-1{Package: #3 #4}%
      \xdef#1{#4}%
    }%
  \else
    \def\x#1#2[#3]{\endgroup
      #2[{#3}]%
      \ifx#1\@undefined
        \xdef#1{#3}%
      \fi
      \ifx#1\relax
        \xdef#1{#3}%
      \fi
    }%
  \fi
\expandafter\x\csname ver@mleftright.sty\endcsname
\ProvidesPackage{mleftright}%
  [2016/05/16 v1.1 Math left/right delim. as open/close (HO)]%
%    \end{macrocode}
%
%    \begin{macrocode}
\begingroup\catcode61\catcode48\catcode32=10\relax%
  \catcode13=5 % ^^M
  \endlinechar=13 %
  \catcode123=1 % {
  \catcode125=2 % }
  \catcode64=11 % @
  \def\x{\endgroup
    \expandafter\edef\csname mleftright@AtEnd\endcsname{%
      \endlinechar=\the\endlinechar\relax
      \catcode13=\the\catcode13\relax
      \catcode32=\the\catcode32\relax
      \catcode35=\the\catcode35\relax
      \catcode61=\the\catcode61\relax
      \catcode64=\the\catcode64\relax
      \catcode123=\the\catcode123\relax
      \catcode125=\the\catcode125\relax
    }%
  }%
\x\catcode61\catcode48\catcode32=10\relax%
\catcode13=5 % ^^M
\endlinechar=13 %
\catcode35=6 % #
\catcode64=11 % @
\catcode123=1 % {
\catcode125=2 % }
\def\TMP@EnsureCode#1#2{%
  \edef\mleftright@AtEnd{%
    \mleftright@AtEnd
    \catcode#1=\the\catcode#1\relax
  }%
  \catcode#1=#2\relax
}
\TMP@EnsureCode{38}{4}% &
\TMP@EnsureCode{39}{12}% '
\TMP@EnsureCode{40}{12}% (
\TMP@EnsureCode{41}{12}% )
\TMP@EnsureCode{42}{12}% *
\TMP@EnsureCode{43}{12}% +
\TMP@EnsureCode{44}{12}% ,
\TMP@EnsureCode{45}{12}% -
\TMP@EnsureCode{46}{12}% .
\TMP@EnsureCode{47}{12}% /
\TMP@EnsureCode{60}{12}% <
\TMP@EnsureCode{91}{12}% [
\TMP@EnsureCode{93}{12}% ]
\edef\mleftright@AtEnd{%
  \mleftright@AtEnd
  \escapechar\the\escapechar\relax
  \noexpand\endinput
}
\escapechar=92 %
%    \end{macrocode}
%
%    \begin{macrocode}
\begingroup\expandafter\expandafter\expandafter\endgroup
\expandafter\ifx\csname RequirePackage\endcsname\relax
  \input infwarerr.sty\relax
  \input ltxcmds.sty\relax
\else
  \RequirePackage{infwarerr}[2010/04/08]%
  \RequirePackage{ltxcmds}[2010/04/26]%
\fi
%    \end{macrocode}
%
%    The original commands \cs{left} and \cs{right}
%    are saved and later used in \cs{mleft} and
%    \cs{mright} in order to deal with:
%    \begin{quote}
%\begin{verbatim}
%\let\left\mleft
%\let\right\mright
%\end{verbatim}
%    \end{quote}
%    \begin{macro}{\mleftright@OrgLeft}
%    \begin{macrocode}
\let\mleftright@OrgLeft\left
%    \end{macrocode}
%    \end{macro}
%    \begin{macro}{\mleftright@OrgRight}
%    \begin{macrocode}
\let\mleftright@OrgRight\right
%    \end{macrocode}
%    \end{macro}
%
%    \begin{macro}{\mleftright@Def}
%    Macro \cs{mleftright@Def} defines a macro as robust macro
%    if \eTeX\ or \LaTeX\ is available.
%    \begin{macrocode}
\ltx@IfUndefined{protected}{%
  \ltx@IfUndefined{DeclareRobustCommand}{%
    \def\mleftright@Def{\def}%
  }{%
    \def\mleftright@Def{\DeclareRobustCommand*}%
  }%
}{%
  \def\mleftright@Def{\protected\def}%
}
\edef\mleftright@Def#1{%
  \noexpand\ltx@IfUndefined{%
    \noexpand\expandafter\noexpand\ltx@gobble\noexpand\string#1%
  }{%
    \expandafter\noexpand\mleftright@Def#1%
  }{%
    \noexpand\@PackageError{mleftright}{%
      Command \noexpand\string#1 already defined%
    }\noexpand\@ehd
    \noexpand\ltx@gobble
  }%
}
%    \end{macrocode}
%    \end{macro}
%
%    In case of \eTeX\ the group status after the left symbol
%    is saved and later checked at the beginning of \cs{mright}.
%    \begin{macrocode}
\ltx@IfUndefined{currentgrouplevel}{%
  \catcode38=14 % & = comment
}{%
  \catcode38=9 % & = ignore
}
%    \end{macrocode}
%
%    \begin{macro}{\mleftright@GroupLevel}
%    \begin{macrocode}
& \def\mleftright@GroupLevel{-1}%
%    \end{macrocode}
%    \end{macro}
%
%    \begin{macro}{\mleftright@WrongGroup}
%    \begin{macrocode}
& \def\mleftright@WrongGroup#1(#2){%
&   \ifnum\mleftright@GroupLevel<\ltx@zero
&     \@PackageError{mleftright}{%
&       Missing previous \string\mleft
&     }\@ehc
&   \else
&     \@PackageError{mleftright}{%
&       Unexpected group status for \string\mright%
&       \ifnum\mleftright@GroupLevel=#1 %
&       \else
&         .\MessageBreak
&         Group level is #1, %
&           expected is \mleftright@GroupLevel
&       \fi
&       \ifnum16=#2 %
&       \else
&         .\MessageBreak
&         Group type is #2 (%
&         \ifcase#2 %
&           bottom level%
&           \expandafter\expandafter\expandafter\ltx@gobblefour
&           \expandafter\ltx@gobbletwo
&         \or simple%
&         \or hbox%
&         \or adjusted hbox%
&         \or vbox%
&         \or vtop%
&         \or align%
&         \or no align%
&         \or output%
&         \or math%
&         \or disc%
&         \or insert%
&         \or vcenter%
&         \or math choice%
&         \or semi simple%
&         \or math shift%
&         \or math left%
&         \else
&           unknown%
&         \fi
&         \space group),\MessageBreak
&         expected is 16 (math left group)%
&       \fi
&     }\@ehd
&   \fi
& }%
%    \end{macrocode}
%    \end{macro}
%
%    \begin{macro}{\mleft}
%    \begin{macrocode}
\mleftright@Def\mleft{%
  \mathopen{}\mathclose\bgroup
& \edef\mleftright@GroupLevel{\the\numexpr\the\currentgrouplevel+1}%
  \mleftright@OrgLeft
}
%    \end{macrocode}
%    \end{macro}
%    \begin{macro}{\mright}
%    \begin{macrocode}
\mleftright@Def\mright{%
& \ifnum\mleftright@GroupLevel=\currentgrouplevel
&   \ifnum16=\currentgrouptype
      \aftergroup\egroup
&   \else
&     \expandafter\mleftright@WrongGroup
&     \the\expandafter\currentgrouplevel
&     \expandafter(\the\currentgrouptype)%
&   \fi
& \else
&   \expandafter\mleftright@WrongGroup
&   \the\expandafter\currentgrouplevel
&   \expandafter(\the\currentgrouptype)%
& \fi
  \mleftright@OrgRight
}
%    \end{macrocode}
%    \end{macro}
%
%    \begin{macro}{\mleftright}
%    \begin{macrocode}
\mleftright@Def\mleftright{%
  \let\left\mleft
  \let\right\mright
}
%    \end{macrocode}
%    \end{macro}
%
%    \begin{macro}{\mleftrightrestore}
%    \begin{macrocode}
\mleftright@Def\mleftrightrestore{%
  \ifx\left\mleft
    \let\left\mleftright@OrgLeft
  \fi
  \ifx\right\mright
    \let\right\mleftright@OrgRight
  \fi
}
%    \end{macrocode}
%    \end{macro}
%
%    \begin{macrocode}
\mleftright@AtEnd%
%</package>
%    \end{macrocode}
%
% \section{Test}
%
% \subsection{Catcode checks for loading}
%
%    \begin{macrocode}
%<*test1>
%    \end{macrocode}
%    \begin{macrocode}
\catcode`\{=1 %
\catcode`\}=2 %
\catcode`\#=6 %
\catcode`\@=11 %
\expandafter\ifx\csname count@\endcsname\relax
  \countdef\count@=255 %
\fi
\expandafter\ifx\csname @gobble\endcsname\relax
  \long\def\@gobble#1{}%
\fi
\expandafter\ifx\csname @firstofone\endcsname\relax
  \long\def\@firstofone#1{#1}%
\fi
\expandafter\ifx\csname loop\endcsname\relax
  \expandafter\@firstofone
\else
  \expandafter\@gobble
\fi
{%
  \def\loop#1\repeat{%
    \def\body{#1}%
    \iterate
  }%
  \def\iterate{%
    \body
      \let\next\iterate
    \else
      \let\next\relax
    \fi
    \next
  }%
  \let\repeat=\fi
}%
\def\RestoreCatcodes{}
\count@=0 %
\loop
  \edef\RestoreCatcodes{%
    \RestoreCatcodes
    \catcode\the\count@=\the\catcode\count@\relax
  }%
\ifnum\count@<255 %
  \advance\count@ 1 %
\repeat

\def\RangeCatcodeInvalid#1#2{%
  \count@=#1\relax
  \loop
    \catcode\count@=15 %
  \ifnum\count@<#2\relax
    \advance\count@ 1 %
  \repeat
}
\def\RangeCatcodeCheck#1#2#3{%
  \count@=#1\relax
  \loop
    \ifnum#3=\catcode\count@
    \else
      \errmessage{%
        Character \the\count@\space
        with wrong catcode \the\catcode\count@\space
        instead of \number#3%
      }%
    \fi
  \ifnum\count@<#2\relax
    \advance\count@ 1 %
  \repeat
}
\def\space{ }
\expandafter\ifx\csname LoadCommand\endcsname\relax
  \def\LoadCommand{\input mleftright.sty\relax}%
\fi
\def\Test{%
  \RangeCatcodeInvalid{0}{47}%
  \RangeCatcodeInvalid{58}{64}%
  \RangeCatcodeInvalid{91}{96}%
  \RangeCatcodeInvalid{123}{255}%
  \catcode`\@=12 %
  \catcode`\\=0 %
  \catcode`\%=14 %
  \LoadCommand
  \RangeCatcodeCheck{0}{36}{15}%
  \RangeCatcodeCheck{37}{37}{14}%
  \RangeCatcodeCheck{38}{47}{15}%
  \RangeCatcodeCheck{48}{57}{12}%
  \RangeCatcodeCheck{58}{63}{15}%
  \RangeCatcodeCheck{64}{64}{12}%
  \RangeCatcodeCheck{65}{90}{11}%
  \RangeCatcodeCheck{91}{91}{15}%
  \RangeCatcodeCheck{92}{92}{0}%
  \RangeCatcodeCheck{93}{96}{15}%
  \RangeCatcodeCheck{97}{122}{11}%
  \RangeCatcodeCheck{123}{255}{15}%
  \RestoreCatcodes
}
\Test
\csname @@end\endcsname
\end
%    \end{macrocode}
%    \begin{macrocode}
%</test1>
%    \end{macrocode}
%
% \section{Installation}
%
% \subsection{Download}
%
% \paragraph{Package.} This package is available on
% CTAN\footnote{\url{http://ctan.org/pkg/mleftright}}:
% \begin{description}
% \item[\CTAN{macros/latex/contrib/oberdiek/mleftright.dtx}] The source file.
% \item[\CTAN{macros/latex/contrib/oberdiek/mleftright.pdf}] Documentation.
% \end{description}
%
%
% \paragraph{Bundle.} All the packages of the bundle `oberdiek'
% are also available in a TDS compliant ZIP archive. There
% the packages are already unpacked and the documentation files
% are generated. The files and directories obey the TDS standard.
% \begin{description}
% \item[\CTAN{install/macros/latex/contrib/oberdiek.tds.zip}]
% \end{description}
% \emph{TDS} refers to the standard ``A Directory Structure
% for \TeX\ Files'' (\CTAN{tds/tds.pdf}). Directories
% with \xfile{texmf} in their name are usually organized this way.
%
% \subsection{Bundle installation}
%
% \paragraph{Unpacking.} Unpack the \xfile{oberdiek.tds.zip} in the
% TDS tree (also known as \xfile{texmf} tree) of your choice.
% Example (linux):
% \begin{quote}
%   |unzip oberdiek.tds.zip -d ~/texmf|
% \end{quote}
%
% \paragraph{Script installation.}
% Check the directory \xfile{TDS:scripts/oberdiek/} for
% scripts that need further installation steps.
% Package \xpackage{attachfile2} comes with the Perl script
% \xfile{pdfatfi.pl} that should be installed in such a way
% that it can be called as \texttt{pdfatfi}.
% Example (linux):
% \begin{quote}
%   |chmod +x scripts/oberdiek/pdfatfi.pl|\\
%   |cp scripts/oberdiek/pdfatfi.pl /usr/local/bin/|
% \end{quote}
%
% \subsection{Package installation}
%
% \paragraph{Unpacking.} The \xfile{.dtx} file is a self-extracting
% \docstrip\ archive. The files are extracted by running the
% \xfile{.dtx} through \plainTeX:
% \begin{quote}
%   \verb|tex mleftright.dtx|
% \end{quote}
%
% \paragraph{TDS.} Now the different files must be moved into
% the different directories in your installation TDS tree
% (also known as \xfile{texmf} tree):
% \begin{quote}
% \def\t{^^A
% \begin{tabular}{@{}>{\ttfamily}l@{ $\rightarrow$ }>{\ttfamily}l@{}}
%   mleftright.sty & tex/generic/oberdiek/mleftright.sty\\
%   mleftright.pdf & doc/latex/oberdiek/mleftright.pdf\\
%   test/mleftright-test1.tex & doc/latex/oberdiek/test/mleftright-test1.tex\\
%   mleftright.dtx & source/latex/oberdiek/mleftright.dtx\\
% \end{tabular}^^A
% }^^A
% \sbox0{\t}^^A
% \ifdim\wd0>\linewidth
%   \begingroup
%     \advance\linewidth by\leftmargin
%     \advance\linewidth by\rightmargin
%   \edef\x{\endgroup
%     \def\noexpand\lw{\the\linewidth}^^A
%   }\x
%   \def\lwbox{^^A
%     \leavevmode
%     \hbox to \linewidth{^^A
%       \kern-\leftmargin\relax
%       \hss
%       \usebox0
%       \hss
%       \kern-\rightmargin\relax
%     }^^A
%   }^^A
%   \ifdim\wd0>\lw
%     \sbox0{\small\t}^^A
%     \ifdim\wd0>\linewidth
%       \ifdim\wd0>\lw
%         \sbox0{\footnotesize\t}^^A
%         \ifdim\wd0>\linewidth
%           \ifdim\wd0>\lw
%             \sbox0{\scriptsize\t}^^A
%             \ifdim\wd0>\linewidth
%               \ifdim\wd0>\lw
%                 \sbox0{\tiny\t}^^A
%                 \ifdim\wd0>\linewidth
%                   \lwbox
%                 \else
%                   \usebox0
%                 \fi
%               \else
%                 \lwbox
%               \fi
%             \else
%               \usebox0
%             \fi
%           \else
%             \lwbox
%           \fi
%         \else
%           \usebox0
%         \fi
%       \else
%         \lwbox
%       \fi
%     \else
%       \usebox0
%     \fi
%   \else
%     \lwbox
%   \fi
% \else
%   \usebox0
% \fi
% \end{quote}
% If you have a \xfile{docstrip.cfg} that configures and enables \docstrip's
% TDS installing feature, then some files can already be in the right
% place, see the documentation of \docstrip.
%
% \subsection{Refresh file name databases}
%
% If your \TeX~distribution
% (\teTeX, \mikTeX, \dots) relies on file name databases, you must refresh
% these. For example, \teTeX\ users run \verb|texhash| or
% \verb|mktexlsr|.
%
% \subsection{Some details for the interested}
%
% \paragraph{Attached source.}
%
% The PDF documentation on CTAN also includes the
% \xfile{.dtx} source file. It can be extracted by
% AcrobatReader 6 or higher. Another option is \textsf{pdftk},
% e.g. unpack the file into the current directory:
% \begin{quote}
%   \verb|pdftk mleftright.pdf unpack_files output .|
% \end{quote}
%
% \paragraph{Unpacking with \LaTeX.}
% The \xfile{.dtx} chooses its action depending on the format:
% \begin{description}
% \item[\plainTeX:] Run \docstrip\ and extract the files.
% \item[\LaTeX:] Generate the documentation.
% \end{description}
% If you insist on using \LaTeX\ for \docstrip\ (really,
% \docstrip\ does not need \LaTeX), then inform the autodetect routine
% about your intention:
% \begin{quote}
%   \verb|latex \let\install=y\input{mleftright.dtx}|
% \end{quote}
% Do not forget to quote the argument according to the demands
% of your shell.
%
% \paragraph{Generating the documentation.}
% You can use both the \xfile{.dtx} or the \xfile{.drv} to generate
% the documentation. The process can be configured by the
% configuration file \xfile{ltxdoc.cfg}. For instance, put this
% line into this file, if you want to have A4 as paper format:
% \begin{quote}
%   \verb|\PassOptionsToClass{a4paper}{article}|
% \end{quote}
% An example follows how to generate the
% documentation with pdf\LaTeX:
% \begin{quote}
%\begin{verbatim}
%pdflatex mleftright.dtx
%makeindex -s gind.ist mleftright.idx
%pdflatex mleftright.dtx
%makeindex -s gind.ist mleftright.idx
%pdflatex mleftright.dtx
%\end{verbatim}
% \end{quote}
%
% \section{Catalogue}
%
% The following XML file can be used as source for the
% \href{http://mirror.ctan.org/help/Catalogue/catalogue.html}{\TeX\ Catalogue}.
% The elements \texttt{caption} and \texttt{description} are imported
% from the original XML file from the Catalogue.
% The name of the XML file in the Catalogue is \xfile{mleftright.xml}.
%    \begin{macrocode}
%<*catalogue>
<?xml version='1.0' encoding='us-ascii'?>
<!DOCTYPE entry SYSTEM 'catalogue.dtd'>
<entry datestamp='$Date$' modifier='$Author$' id='mleftright'>
  <name>mleftright</name>
  <caption>Variants of delimiters that act as maths open/close.</caption>
  <authorref id='auth:oberdiek'/>
  <copyright owner='Heiko Oberdiek' year='2010'/>
  <license type='lppl1.3'/>
  <version number='1.1'/>
  <description>
    The package defines variants <tt>\mleft</tt> and <tt>\mright</tt>
    of <tt>\left</tt> and <tt>\right</tt>, that make the delimiters
    act as <tt>\mathopen</tt> and <tt>\mathclose</tt>.  These commands
    address spacing difficulties in subformulas.
    <p/>
    The package is part of the <xref refid='oberdiek'>oberdiek</xref> bundle.
  </description>
  <documentation details='Package documentation'
      href='ctan:/macros/latex/contrib/oberdiek/mleftright.pdf'/>
  <ctan file='true' path='/macros/latex/contrib/oberdiek/mleftright.dtx'/>
  <miktex location='oberdiek'/>
  <texlive location='oberdiek'/>
  <install path='/macros/latex/contrib/oberdiek/oberdiek.tds.zip'/>
</entry>
%</catalogue>
%    \end{macrocode}
%
% \section{Acknowledgement}
%
% \begin{description}
% \item[Donald Arsenau:]
% He provided the main trick and the first macros.
% \item[Philipp Stephani:]
% He solved the subscript problem.
% \end{description}
%
% \begin{thebibliography}{9}
% \raggedright
% \bibitem{dave}
%   Dave94705,
%   \textit{spacing after \cs{right}\texttt{)} and before \cs{left}\texttt{)}},
%   newsgroup comp.text.tex,
%   Message-ID: \texttt{\small 5d264909-7c3d-4c9d-9b22-434178b2bf90@g21g2000prn.googlegroups.com},
%   2010-08-12.
%   \newblock
%   {\small\url{http://groups.google.com/group/comp.text.tex/msg/e5b6833da7dc29bf}}
%
% \bibitem{arseneau}
%   Donald Arseneau,
%   \textit{Re: spacing after \cs{right}\texttt) and before \cs{left}\texttt)},
%   newsgroup comp.text.tex,
%   Message-ID: \texttt{\small yfivd6svl8y.fsf@mutant.triumf.ca},
%   2010-08-30.
%   \newblock
%   {\small\url{http://groups.google.com/group/comp.text.tex/msg/e0b2e4386e5d04e4}}
%
% \bibitem{stephani}
%   Philipp Stephani,
%   \textit{Re: spacing after \cs{right}\texttt) and before \cs{left}\texttt)},
%   newsgroup comp.text.tex,
%   Message-ID: \texttt{\small 4c8c8c1e\$0\$6981\$9b4e6d93@newsspool4.arcor-online.net},
%   2010-09-12.
%   \newblock
%   {\small\url{http://groups.google.com/group/comp.text.tex/msg/87ac1f61321de3ef}}
%
% \bibitem{oberdiek}
%   Heiko Oberdiek,
%   \textit{Re: spacing after \cs{right}\texttt) and before \cs{left}\texttt)},
%   newsgroup comp.text.tex,
%   Message-ID: \texttt{\small i6jcc2\$8of\$1@news.eternal-september.org},
%   2010-09-12.
%   \newblock
%   {\small\url{http://groups.google.com/group/comp.text.tex/msg/257aa6119bef878b}}
%
% \end{thebibliography}
%
% \begin{History}
%   \begin{Version}{2010/09/25 v1.0}
%   \item
%     The first version.
%   \end{Version}
%   \begin{Version}{2016/05/16 v1.1}
%   \item
%     Documentation updates.
%   \end{Version}
% \end{History}
%
% \PrintIndex
%
% \Finale
\endinput

%        (quote the arguments according to the demands of your shell)
%
% Documentation:
%    (a) If mleftright.drv is present:
%           latex mleftright.drv
%    (b) Without mleftright.drv:
%           latex mleftright.dtx; ...
%    The class ltxdoc loads the configuration file ltxdoc.cfg
%    if available. Here you can specify further options, e.g.
%    use A4 as paper format:
%       \PassOptionsToClass{a4paper}{article}
%
%    Programm calls to get the documentation (example):
%       pdflatex mleftright.dtx
%       makeindex -s gind.ist mleftright.idx
%       pdflatex mleftright.dtx
%       makeindex -s gind.ist mleftright.idx
%       pdflatex mleftright.dtx
%
% Installation:
%    TDS:tex/generic/oberdiek/mleftright.sty
%    TDS:doc/latex/oberdiek/mleftright.pdf
%    TDS:doc/latex/oberdiek/test/mleftright-test1.tex
%    TDS:source/latex/oberdiek/mleftright.dtx
%
%<*ignore>
\begingroup
  \catcode123=1 %
  \catcode125=2 %
  \def\x{LaTeX2e}%
\expandafter\endgroup
\ifcase 0\ifx\install y1\fi\expandafter
         \ifx\csname processbatchFile\endcsname\relax\else1\fi
         \ifx\fmtname\x\else 1\fi\relax
\else\csname fi\endcsname
%</ignore>
%<*install>
\input docstrip.tex
\Msg{************************************************************************}
\Msg{* Installation}
\Msg{* Package: mleftright 2016/05/16 v1.1 Math left/right delim. as open/close (HO)}
\Msg{************************************************************************}

\keepsilent
\askforoverwritefalse

\let\MetaPrefix\relax
\preamble

This is a generated file.

Project: mleftright
Version: 2016/05/16 v1.1

Copyright (C) 2010 by
   Heiko Oberdiek <heiko.oberdiek at googlemail.com>

This work may be distributed and/or modified under the
conditions of the LaTeX Project Public License, either
version 1.3c of this license or (at your option) any later
version. This version of this license is in
   http://www.latex-project.org/lppl/lppl-1-3c.txt
and the latest version of this license is in
   http://www.latex-project.org/lppl.txt
and version 1.3 or later is part of all distributions of
LaTeX version 2005/12/01 or later.

This work has the LPPL maintenance status "maintained".

This Current Maintainer of this work is Heiko Oberdiek.

The Base Interpreter refers to any `TeX-Format',
because some files are installed in TDS:tex/generic//.

This work consists of the main source file mleftright.dtx
and the derived files
   mleftright.sty, mleftright.pdf, mleftright.ins, mleftright.drv,
   mleftright-test1.tex.

\endpreamble
\let\MetaPrefix\DoubleperCent

\generate{%
  \file{mleftright.ins}{\from{mleftright.dtx}{install}}%
  \file{mleftright.drv}{\from{mleftright.dtx}{driver}}%
  \usedir{tex/generic/oberdiek}%
  \file{mleftright.sty}{\from{mleftright.dtx}{package}}%
  \usedir{doc/latex/oberdiek/test}%
  \file{mleftright-test1.tex}{\from{mleftright.dtx}{test1}}%
  \nopreamble
  \nopostamble
  \usedir{source/latex/oberdiek/catalogue}%
  \file{mleftright.xml}{\from{mleftright.dtx}{catalogue}}%
}

\catcode32=13\relax% active space
\let =\space%
\Msg{************************************************************************}
\Msg{*}
\Msg{* To finish the installation you have to move the following}
\Msg{* file into a directory searched by TeX:}
\Msg{*}
\Msg{*     mleftright.sty}
\Msg{*}
\Msg{* To produce the documentation run the file `mleftright.drv'}
\Msg{* through LaTeX.}
\Msg{*}
\Msg{* Happy TeXing!}
\Msg{*}
\Msg{************************************************************************}

\endbatchfile
%</install>
%<*ignore>
\fi
%</ignore>
%<*driver>
\NeedsTeXFormat{LaTeX2e}
\ProvidesFile{mleftright.drv}%
  [2016/05/16 v1.1 Math left/right delim. as open/close (HO)]%
\documentclass{ltxdoc}
\usepackage{holtxdoc}[2011/11/22]
\usepackage{mleftright}[2016/05/16]
\begin{document}
  \DocInput{mleftright.dtx}%
\end{document}
%</driver>
% \fi
%
%
% \CharacterTable
%  {Upper-case    \A\B\C\D\E\F\G\H\I\J\K\L\M\N\O\P\Q\R\S\T\U\V\W\X\Y\Z
%   Lower-case    \a\b\c\d\e\f\g\h\i\j\k\l\m\n\o\p\q\r\s\t\u\v\w\x\y\z
%   Digits        \0\1\2\3\4\5\6\7\8\9
%   Exclamation   \!     Double quote  \"     Hash (number) \#
%   Dollar        \$     Percent       \%     Ampersand     \&
%   Acute accent  \'     Left paren    \(     Right paren   \)
%   Asterisk      \*     Plus          \+     Comma         \,
%   Minus         \-     Point         \.     Solidus       \/
%   Colon         \:     Semicolon     \;     Less than     \<
%   Equals        \=     Greater than  \>     Question mark \?
%   Commercial at \@     Left bracket  \[     Backslash     \\
%   Right bracket \]     Circumflex    \^     Underscore    \_
%   Grave accent  \`     Left brace    \{     Vertical bar  \|
%   Right brace   \}     Tilde         \~}
%
% \GetFileInfo{mleftright.drv}
%
% \title{The \xpackage{mleftright} package}
% \date{2016/05/16 v1.1}
% \author{Heiko Oberdiek\thanks
% {Please report any issues at https://github.com/ho-tex/oberdiek/issues}\\
% \xemail{heiko.oberdiek at googlemail.com}}
%
% \maketitle
%
% \begin{abstract}
% \TeX\ sets subformulas by \cs{left} and \cs{right} as inner formulas
% with additional surrounding spaces in some situations. This package
% provides \cs{mleft} and \cs{mright} that call \cs{left} and \cs{right},
% but the delimiters will act as normal \cs{mathopen} and \cs{mathclose}
% delimiters without the additional space of an inner formula.
% \end{abstract}
%
% \tableofcontents
%
% \section{Documentation}
%
% The package is a result of a thread in the newsgroup \textsf{comp.text.tex}
% with the subject \textit{spacing after \cs{right}\texttt{)}
% and before \cs{left}\texttt{)}} \cite{dave}.
% The problem: \cs{left} and \cs{right} adjust the size of the
% delimiters automatically. However, \TeX\ treats the whole expression
% as inner formula. In some circumstances \TeX\ adds extra space
% before or after an inner formula.
% Example:
% \begin{quote}
%   \thinmuskip=1.5\thinmuskip
%   \begin{tabular}{@{}l@{\quad$\Rightarrow$\quad}l@{}}
%     |$\sin(x^2), x$|
%     & $\sin(x^2), x$\\
%     |$\sin\left(x^2\right), x$|
%     & $\sin\left(x^2\right), x$\\
%   ^^A  \multicolumn{1}{@{}r@{\quad$\Rightarrow$\quad}}{^^A
%   ^^A    \itshape with exaggerated spacing^^A
%   ^^A  }
%   ^^A  & $\thinmuskip=4\thinmuskip
%   ^^A    \sin\left(x^2\right){,}\mskip.25\thinmuskip x$\\
%     |$\sin\mleft(x^2\mright), x$|
%     & $\sin\mleft(x^2\mright), x$\\
%   \end{tabular}\\*[.5ex]
%   (\cs{mleft} and \cs{mright} are provided by this package.)
% \end{quote}
%
% In the newsgroup Donald Arseneau answered with clever macros \cite{arseneau}:
% \begin{quote}
%\begin{verbatim}
%\newcommand\lft{\mathopen{}\left}
%\newcommand\rgt{\aftergroup\mathclose\aftergroup{\aftergroup}\right}
%\end{verbatim}
% \end{quote}
% However one problem remains, a following subscript or superscript
% is not applied to the right delimiter but the empty
% \cs{mathclose}.
% Thus Philipp Stephani provided an improvement \cite{stephani}:
%\begin{quote}
%\begin{verbatim}
%\mathopen{} \mathclose{\left\| A^2 \right\|}_2
%\end{verbatim}
%\end{quote}
% Heiko Oberdiek converted this into macro form \cite{oberdiek}:
%\begin{quote}
%\begin{verbatim}
%\newcommand\lft{\mathopen{}\mathclose\bgroup\left}
%\newcommand\rgt{\aftergroup\egroup\right}
%\end{verbatim}
%\end{quote}
%
% The package uses longer macro names \cs{mleft} and \cs{mright}
% to avoid name clashes. Also it adds some checks for error conditions.
%
% \subsection{Use}
%
% \begin{declcs}{mleft}\meta{delimL} \dots\unkern\ \cs{mright}\meta{delimR}
% \end{declcs}
% Macros \cs{mleft} and \cs{mright} are used in the same way as
% \cs{left} and \cs{right}. Also \cs{middle} can be used inbetween if
% \eTeX\ is present.
%
% \begin{declcs}{mleftright}
% \end{declcs}
% Macro \cs{mleftright} redefines \cs{left} as \cs{mleft} and
% \cs{right} as \cs{mright}. The redefinition is local to the group.
%
% \begin{declcs}{mleftrightrestore}
% \end{declcs}
% Macro \cs{mleftright} restores \cs{left} and \cs{right} with
% the original meaning if they were previously redefined by
% \cs{mleftright} (also locally).
%
%
% \StopEventually{
% }
%
% \section{Implementation}
%    \begin{macrocode}
%<*package>
%    \end{macrocode}
%    Reload check, especially if the package is not used with \LaTeX.
%    \begin{macrocode}
\begingroup\catcode61\catcode48\catcode32=10\relax%
  \catcode13=5 % ^^M
  \endlinechar=13 %
  \catcode35=6 % #
  \catcode39=12 % '
  \catcode44=12 % ,
  \catcode45=12 % -
  \catcode46=12 % .
  \catcode58=12 % :
  \catcode64=11 % @
  \catcode123=1 % {
  \catcode125=2 % }
  \expandafter\let\expandafter\x\csname ver@mleftright.sty\endcsname
  \ifx\x\relax % plain-TeX, first loading
  \else
    \def\empty{}%
    \ifx\x\empty % LaTeX, first loading,
      % variable is initialized, but \ProvidesPackage not yet seen
    \else
      \expandafter\ifx\csname PackageInfo\endcsname\relax
        \def\x#1#2{%
          \immediate\write-1{Package #1 Info: #2.}%
        }%
      \else
        \def\x#1#2{\PackageInfo{#1}{#2, stopped}}%
      \fi
      \x{mleftright}{The package is already loaded}%
      \aftergroup\endinput
    \fi
  \fi
\endgroup%
%    \end{macrocode}
%    Package identification:
%    \begin{macrocode}
\begingroup\catcode61\catcode48\catcode32=10\relax%
  \catcode13=5 % ^^M
  \endlinechar=13 %
  \catcode35=6 % #
  \catcode39=12 % '
  \catcode40=12 % (
  \catcode41=12 % )
  \catcode44=12 % ,
  \catcode45=12 % -
  \catcode46=12 % .
  \catcode47=12 % /
  \catcode58=12 % :
  \catcode64=11 % @
  \catcode91=12 % [
  \catcode93=12 % ]
  \catcode123=1 % {
  \catcode125=2 % }
  \expandafter\ifx\csname ProvidesPackage\endcsname\relax
    \def\x#1#2#3[#4]{\endgroup
      \immediate\write-1{Package: #3 #4}%
      \xdef#1{#4}%
    }%
  \else
    \def\x#1#2[#3]{\endgroup
      #2[{#3}]%
      \ifx#1\@undefined
        \xdef#1{#3}%
      \fi
      \ifx#1\relax
        \xdef#1{#3}%
      \fi
    }%
  \fi
\expandafter\x\csname ver@mleftright.sty\endcsname
\ProvidesPackage{mleftright}%
  [2016/05/16 v1.1 Math left/right delim. as open/close (HO)]%
%    \end{macrocode}
%
%    \begin{macrocode}
\begingroup\catcode61\catcode48\catcode32=10\relax%
  \catcode13=5 % ^^M
  \endlinechar=13 %
  \catcode123=1 % {
  \catcode125=2 % }
  \catcode64=11 % @
  \def\x{\endgroup
    \expandafter\edef\csname mleftright@AtEnd\endcsname{%
      \endlinechar=\the\endlinechar\relax
      \catcode13=\the\catcode13\relax
      \catcode32=\the\catcode32\relax
      \catcode35=\the\catcode35\relax
      \catcode61=\the\catcode61\relax
      \catcode64=\the\catcode64\relax
      \catcode123=\the\catcode123\relax
      \catcode125=\the\catcode125\relax
    }%
  }%
\x\catcode61\catcode48\catcode32=10\relax%
\catcode13=5 % ^^M
\endlinechar=13 %
\catcode35=6 % #
\catcode64=11 % @
\catcode123=1 % {
\catcode125=2 % }
\def\TMP@EnsureCode#1#2{%
  \edef\mleftright@AtEnd{%
    \mleftright@AtEnd
    \catcode#1=\the\catcode#1\relax
  }%
  \catcode#1=#2\relax
}
\TMP@EnsureCode{38}{4}% &
\TMP@EnsureCode{39}{12}% '
\TMP@EnsureCode{40}{12}% (
\TMP@EnsureCode{41}{12}% )
\TMP@EnsureCode{42}{12}% *
\TMP@EnsureCode{43}{12}% +
\TMP@EnsureCode{44}{12}% ,
\TMP@EnsureCode{45}{12}% -
\TMP@EnsureCode{46}{12}% .
\TMP@EnsureCode{47}{12}% /
\TMP@EnsureCode{60}{12}% <
\TMP@EnsureCode{91}{12}% [
\TMP@EnsureCode{93}{12}% ]
\edef\mleftright@AtEnd{%
  \mleftright@AtEnd
  \escapechar\the\escapechar\relax
  \noexpand\endinput
}
\escapechar=92 %
%    \end{macrocode}
%
%    \begin{macrocode}
\begingroup\expandafter\expandafter\expandafter\endgroup
\expandafter\ifx\csname RequirePackage\endcsname\relax
  \input infwarerr.sty\relax
  \input ltxcmds.sty\relax
\else
  \RequirePackage{infwarerr}[2010/04/08]%
  \RequirePackage{ltxcmds}[2010/04/26]%
\fi
%    \end{macrocode}
%
%    The original commands \cs{left} and \cs{right}
%    are saved and later used in \cs{mleft} and
%    \cs{mright} in order to deal with:
%    \begin{quote}
%\begin{verbatim}
%\let\left\mleft
%\let\right\mright
%\end{verbatim}
%    \end{quote}
%    \begin{macro}{\mleftright@OrgLeft}
%    \begin{macrocode}
\let\mleftright@OrgLeft\left
%    \end{macrocode}
%    \end{macro}
%    \begin{macro}{\mleftright@OrgRight}
%    \begin{macrocode}
\let\mleftright@OrgRight\right
%    \end{macrocode}
%    \end{macro}
%
%    \begin{macro}{\mleftright@Def}
%    Macro \cs{mleftright@Def} defines a macro as robust macro
%    if \eTeX\ or \LaTeX\ is available.
%    \begin{macrocode}
\ltx@IfUndefined{protected}{%
  \ltx@IfUndefined{DeclareRobustCommand}{%
    \def\mleftright@Def{\def}%
  }{%
    \def\mleftright@Def{\DeclareRobustCommand*}%
  }%
}{%
  \def\mleftright@Def{\protected\def}%
}
\edef\mleftright@Def#1{%
  \noexpand\ltx@IfUndefined{%
    \noexpand\expandafter\noexpand\ltx@gobble\noexpand\string#1%
  }{%
    \expandafter\noexpand\mleftright@Def#1%
  }{%
    \noexpand\@PackageError{mleftright}{%
      Command \noexpand\string#1 already defined%
    }\noexpand\@ehd
    \noexpand\ltx@gobble
  }%
}
%    \end{macrocode}
%    \end{macro}
%
%    In case of \eTeX\ the group status after the left symbol
%    is saved and later checked at the beginning of \cs{mright}.
%    \begin{macrocode}
\ltx@IfUndefined{currentgrouplevel}{%
  \catcode38=14 % & = comment
}{%
  \catcode38=9 % & = ignore
}
%    \end{macrocode}
%
%    \begin{macro}{\mleftright@GroupLevel}
%    \begin{macrocode}
& \def\mleftright@GroupLevel{-1}%
%    \end{macrocode}
%    \end{macro}
%
%    \begin{macro}{\mleftright@WrongGroup}
%    \begin{macrocode}
& \def\mleftright@WrongGroup#1(#2){%
&   \ifnum\mleftright@GroupLevel<\ltx@zero
&     \@PackageError{mleftright}{%
&       Missing previous \string\mleft
&     }\@ehc
&   \else
&     \@PackageError{mleftright}{%
&       Unexpected group status for \string\mright%
&       \ifnum\mleftright@GroupLevel=#1 %
&       \else
&         .\MessageBreak
&         Group level is #1, %
&           expected is \mleftright@GroupLevel
&       \fi
&       \ifnum16=#2 %
&       \else
&         .\MessageBreak
&         Group type is #2 (%
&         \ifcase#2 %
&           bottom level%
&           \expandafter\expandafter\expandafter\ltx@gobblefour
&           \expandafter\ltx@gobbletwo
&         \or simple%
&         \or hbox%
&         \or adjusted hbox%
&         \or vbox%
&         \or vtop%
&         \or align%
&         \or no align%
&         \or output%
&         \or math%
&         \or disc%
&         \or insert%
&         \or vcenter%
&         \or math choice%
&         \or semi simple%
&         \or math shift%
&         \or math left%
&         \else
&           unknown%
&         \fi
&         \space group),\MessageBreak
&         expected is 16 (math left group)%
&       \fi
&     }\@ehd
&   \fi
& }%
%    \end{macrocode}
%    \end{macro}
%
%    \begin{macro}{\mleft}
%    \begin{macrocode}
\mleftright@Def\mleft{%
  \mathopen{}\mathclose\bgroup
& \edef\mleftright@GroupLevel{\the\numexpr\the\currentgrouplevel+1}%
  \mleftright@OrgLeft
}
%    \end{macrocode}
%    \end{macro}
%    \begin{macro}{\mright}
%    \begin{macrocode}
\mleftright@Def\mright{%
& \ifnum\mleftright@GroupLevel=\currentgrouplevel
&   \ifnum16=\currentgrouptype
      \aftergroup\egroup
&   \else
&     \expandafter\mleftright@WrongGroup
&     \the\expandafter\currentgrouplevel
&     \expandafter(\the\currentgrouptype)%
&   \fi
& \else
&   \expandafter\mleftright@WrongGroup
&   \the\expandafter\currentgrouplevel
&   \expandafter(\the\currentgrouptype)%
& \fi
  \mleftright@OrgRight
}
%    \end{macrocode}
%    \end{macro}
%
%    \begin{macro}{\mleftright}
%    \begin{macrocode}
\mleftright@Def\mleftright{%
  \let\left\mleft
  \let\right\mright
}
%    \end{macrocode}
%    \end{macro}
%
%    \begin{macro}{\mleftrightrestore}
%    \begin{macrocode}
\mleftright@Def\mleftrightrestore{%
  \ifx\left\mleft
    \let\left\mleftright@OrgLeft
  \fi
  \ifx\right\mright
    \let\right\mleftright@OrgRight
  \fi
}
%    \end{macrocode}
%    \end{macro}
%
%    \begin{macrocode}
\mleftright@AtEnd%
%</package>
%    \end{macrocode}
%
% \section{Test}
%
% \subsection{Catcode checks for loading}
%
%    \begin{macrocode}
%<*test1>
%    \end{macrocode}
%    \begin{macrocode}
\catcode`\{=1 %
\catcode`\}=2 %
\catcode`\#=6 %
\catcode`\@=11 %
\expandafter\ifx\csname count@\endcsname\relax
  \countdef\count@=255 %
\fi
\expandafter\ifx\csname @gobble\endcsname\relax
  \long\def\@gobble#1{}%
\fi
\expandafter\ifx\csname @firstofone\endcsname\relax
  \long\def\@firstofone#1{#1}%
\fi
\expandafter\ifx\csname loop\endcsname\relax
  \expandafter\@firstofone
\else
  \expandafter\@gobble
\fi
{%
  \def\loop#1\repeat{%
    \def\body{#1}%
    \iterate
  }%
  \def\iterate{%
    \body
      \let\next\iterate
    \else
      \let\next\relax
    \fi
    \next
  }%
  \let\repeat=\fi
}%
\def\RestoreCatcodes{}
\count@=0 %
\loop
  \edef\RestoreCatcodes{%
    \RestoreCatcodes
    \catcode\the\count@=\the\catcode\count@\relax
  }%
\ifnum\count@<255 %
  \advance\count@ 1 %
\repeat

\def\RangeCatcodeInvalid#1#2{%
  \count@=#1\relax
  \loop
    \catcode\count@=15 %
  \ifnum\count@<#2\relax
    \advance\count@ 1 %
  \repeat
}
\def\RangeCatcodeCheck#1#2#3{%
  \count@=#1\relax
  \loop
    \ifnum#3=\catcode\count@
    \else
      \errmessage{%
        Character \the\count@\space
        with wrong catcode \the\catcode\count@\space
        instead of \number#3%
      }%
    \fi
  \ifnum\count@<#2\relax
    \advance\count@ 1 %
  \repeat
}
\def\space{ }
\expandafter\ifx\csname LoadCommand\endcsname\relax
  \def\LoadCommand{\input mleftright.sty\relax}%
\fi
\def\Test{%
  \RangeCatcodeInvalid{0}{47}%
  \RangeCatcodeInvalid{58}{64}%
  \RangeCatcodeInvalid{91}{96}%
  \RangeCatcodeInvalid{123}{255}%
  \catcode`\@=12 %
  \catcode`\\=0 %
  \catcode`\%=14 %
  \LoadCommand
  \RangeCatcodeCheck{0}{36}{15}%
  \RangeCatcodeCheck{37}{37}{14}%
  \RangeCatcodeCheck{38}{47}{15}%
  \RangeCatcodeCheck{48}{57}{12}%
  \RangeCatcodeCheck{58}{63}{15}%
  \RangeCatcodeCheck{64}{64}{12}%
  \RangeCatcodeCheck{65}{90}{11}%
  \RangeCatcodeCheck{91}{91}{15}%
  \RangeCatcodeCheck{92}{92}{0}%
  \RangeCatcodeCheck{93}{96}{15}%
  \RangeCatcodeCheck{97}{122}{11}%
  \RangeCatcodeCheck{123}{255}{15}%
  \RestoreCatcodes
}
\Test
\csname @@end\endcsname
\end
%    \end{macrocode}
%    \begin{macrocode}
%</test1>
%    \end{macrocode}
%
% \section{Installation}
%
% \subsection{Download}
%
% \paragraph{Package.} This package is available on
% CTAN\footnote{\url{http://ctan.org/pkg/mleftright}}:
% \begin{description}
% \item[\CTAN{macros/latex/contrib/oberdiek/mleftright.dtx}] The source file.
% \item[\CTAN{macros/latex/contrib/oberdiek/mleftright.pdf}] Documentation.
% \end{description}
%
%
% \paragraph{Bundle.} All the packages of the bundle `oberdiek'
% are also available in a TDS compliant ZIP archive. There
% the packages are already unpacked and the documentation files
% are generated. The files and directories obey the TDS standard.
% \begin{description}
% \item[\CTAN{install/macros/latex/contrib/oberdiek.tds.zip}]
% \end{description}
% \emph{TDS} refers to the standard ``A Directory Structure
% for \TeX\ Files'' (\CTAN{tds/tds.pdf}). Directories
% with \xfile{texmf} in their name are usually organized this way.
%
% \subsection{Bundle installation}
%
% \paragraph{Unpacking.} Unpack the \xfile{oberdiek.tds.zip} in the
% TDS tree (also known as \xfile{texmf} tree) of your choice.
% Example (linux):
% \begin{quote}
%   |unzip oberdiek.tds.zip -d ~/texmf|
% \end{quote}
%
% \paragraph{Script installation.}
% Check the directory \xfile{TDS:scripts/oberdiek/} for
% scripts that need further installation steps.
% Package \xpackage{attachfile2} comes with the Perl script
% \xfile{pdfatfi.pl} that should be installed in such a way
% that it can be called as \texttt{pdfatfi}.
% Example (linux):
% \begin{quote}
%   |chmod +x scripts/oberdiek/pdfatfi.pl|\\
%   |cp scripts/oberdiek/pdfatfi.pl /usr/local/bin/|
% \end{quote}
%
% \subsection{Package installation}
%
% \paragraph{Unpacking.} The \xfile{.dtx} file is a self-extracting
% \docstrip\ archive. The files are extracted by running the
% \xfile{.dtx} through \plainTeX:
% \begin{quote}
%   \verb|tex mleftright.dtx|
% \end{quote}
%
% \paragraph{TDS.} Now the different files must be moved into
% the different directories in your installation TDS tree
% (also known as \xfile{texmf} tree):
% \begin{quote}
% \def\t{^^A
% \begin{tabular}{@{}>{\ttfamily}l@{ $\rightarrow$ }>{\ttfamily}l@{}}
%   mleftright.sty & tex/generic/oberdiek/mleftright.sty\\
%   mleftright.pdf & doc/latex/oberdiek/mleftright.pdf\\
%   test/mleftright-test1.tex & doc/latex/oberdiek/test/mleftright-test1.tex\\
%   mleftright.dtx & source/latex/oberdiek/mleftright.dtx\\
% \end{tabular}^^A
% }^^A
% \sbox0{\t}^^A
% \ifdim\wd0>\linewidth
%   \begingroup
%     \advance\linewidth by\leftmargin
%     \advance\linewidth by\rightmargin
%   \edef\x{\endgroup
%     \def\noexpand\lw{\the\linewidth}^^A
%   }\x
%   \def\lwbox{^^A
%     \leavevmode
%     \hbox to \linewidth{^^A
%       \kern-\leftmargin\relax
%       \hss
%       \usebox0
%       \hss
%       \kern-\rightmargin\relax
%     }^^A
%   }^^A
%   \ifdim\wd0>\lw
%     \sbox0{\small\t}^^A
%     \ifdim\wd0>\linewidth
%       \ifdim\wd0>\lw
%         \sbox0{\footnotesize\t}^^A
%         \ifdim\wd0>\linewidth
%           \ifdim\wd0>\lw
%             \sbox0{\scriptsize\t}^^A
%             \ifdim\wd0>\linewidth
%               \ifdim\wd0>\lw
%                 \sbox0{\tiny\t}^^A
%                 \ifdim\wd0>\linewidth
%                   \lwbox
%                 \else
%                   \usebox0
%                 \fi
%               \else
%                 \lwbox
%               \fi
%             \else
%               \usebox0
%             \fi
%           \else
%             \lwbox
%           \fi
%         \else
%           \usebox0
%         \fi
%       \else
%         \lwbox
%       \fi
%     \else
%       \usebox0
%     \fi
%   \else
%     \lwbox
%   \fi
% \else
%   \usebox0
% \fi
% \end{quote}
% If you have a \xfile{docstrip.cfg} that configures and enables \docstrip's
% TDS installing feature, then some files can already be in the right
% place, see the documentation of \docstrip.
%
% \subsection{Refresh file name databases}
%
% If your \TeX~distribution
% (\teTeX, \mikTeX, \dots) relies on file name databases, you must refresh
% these. For example, \teTeX\ users run \verb|texhash| or
% \verb|mktexlsr|.
%
% \subsection{Some details for the interested}
%
% \paragraph{Attached source.}
%
% The PDF documentation on CTAN also includes the
% \xfile{.dtx} source file. It can be extracted by
% AcrobatReader 6 or higher. Another option is \textsf{pdftk},
% e.g. unpack the file into the current directory:
% \begin{quote}
%   \verb|pdftk mleftright.pdf unpack_files output .|
% \end{quote}
%
% \paragraph{Unpacking with \LaTeX.}
% The \xfile{.dtx} chooses its action depending on the format:
% \begin{description}
% \item[\plainTeX:] Run \docstrip\ and extract the files.
% \item[\LaTeX:] Generate the documentation.
% \end{description}
% If you insist on using \LaTeX\ for \docstrip\ (really,
% \docstrip\ does not need \LaTeX), then inform the autodetect routine
% about your intention:
% \begin{quote}
%   \verb|latex \let\install=y% \iffalse meta-comment
%
% File: mleftright.dtx
% Version: 2016/05/16 v1.1
% Info: Math left/right delim. as open/close
%
% Copyright (C) 2010 by
%    Heiko Oberdiek <heiko.oberdiek at googlemail.com>
%    2016
%    https://github.com/ho-tex/oberdiek/issues
%
% This work may be distributed and/or modified under the
% conditions of the LaTeX Project Public License, either
% version 1.3c of this license or (at your option) any later
% version. This version of this license is in
%    http://www.latex-project.org/lppl/lppl-1-3c.txt
% and the latest version of this license is in
%    http://www.latex-project.org/lppl.txt
% and version 1.3 or later is part of all distributions of
% LaTeX version 2005/12/01 or later.
%
% This work has the LPPL maintenance status "maintained".
%
% This Current Maintainer of this work is Heiko Oberdiek.
%
% The Base Interpreter refers to any `TeX-Format',
% because some files are installed in TDS:tex/generic//.
%
% This work consists of the main source file mleftright.dtx
% and the derived files
%    mleftright.sty, mleftright.pdf, mleftright.ins, mleftright.drv,
%    mleftright-test1.tex.
%
% Distribution:
%    CTAN:macros/latex/contrib/oberdiek/mleftright.dtx
%    CTAN:macros/latex/contrib/oberdiek/mleftright.pdf
%
% Unpacking:
%    (a) If mleftright.ins is present:
%           tex mleftright.ins
%    (b) Without mleftright.ins:
%           tex mleftright.dtx
%    (c) If you insist on using LaTeX
%           latex \let\install=y\input{mleftright.dtx}
%        (quote the arguments according to the demands of your shell)
%
% Documentation:
%    (a) If mleftright.drv is present:
%           latex mleftright.drv
%    (b) Without mleftright.drv:
%           latex mleftright.dtx; ...
%    The class ltxdoc loads the configuration file ltxdoc.cfg
%    if available. Here you can specify further options, e.g.
%    use A4 as paper format:
%       \PassOptionsToClass{a4paper}{article}
%
%    Programm calls to get the documentation (example):
%       pdflatex mleftright.dtx
%       makeindex -s gind.ist mleftright.idx
%       pdflatex mleftright.dtx
%       makeindex -s gind.ist mleftright.idx
%       pdflatex mleftright.dtx
%
% Installation:
%    TDS:tex/generic/oberdiek/mleftright.sty
%    TDS:doc/latex/oberdiek/mleftright.pdf
%    TDS:doc/latex/oberdiek/test/mleftright-test1.tex
%    TDS:source/latex/oberdiek/mleftright.dtx
%
%<*ignore>
\begingroup
  \catcode123=1 %
  \catcode125=2 %
  \def\x{LaTeX2e}%
\expandafter\endgroup
\ifcase 0\ifx\install y1\fi\expandafter
         \ifx\csname processbatchFile\endcsname\relax\else1\fi
         \ifx\fmtname\x\else 1\fi\relax
\else\csname fi\endcsname
%</ignore>
%<*install>
\input docstrip.tex
\Msg{************************************************************************}
\Msg{* Installation}
\Msg{* Package: mleftright 2016/05/16 v1.1 Math left/right delim. as open/close (HO)}
\Msg{************************************************************************}

\keepsilent
\askforoverwritefalse

\let\MetaPrefix\relax
\preamble

This is a generated file.

Project: mleftright
Version: 2016/05/16 v1.1

Copyright (C) 2010 by
   Heiko Oberdiek <heiko.oberdiek at googlemail.com>

This work may be distributed and/or modified under the
conditions of the LaTeX Project Public License, either
version 1.3c of this license or (at your option) any later
version. This version of this license is in
   http://www.latex-project.org/lppl/lppl-1-3c.txt
and the latest version of this license is in
   http://www.latex-project.org/lppl.txt
and version 1.3 or later is part of all distributions of
LaTeX version 2005/12/01 or later.

This work has the LPPL maintenance status "maintained".

This Current Maintainer of this work is Heiko Oberdiek.

The Base Interpreter refers to any `TeX-Format',
because some files are installed in TDS:tex/generic//.

This work consists of the main source file mleftright.dtx
and the derived files
   mleftright.sty, mleftright.pdf, mleftright.ins, mleftright.drv,
   mleftright-test1.tex.

\endpreamble
\let\MetaPrefix\DoubleperCent

\generate{%
  \file{mleftright.ins}{\from{mleftright.dtx}{install}}%
  \file{mleftright.drv}{\from{mleftright.dtx}{driver}}%
  \usedir{tex/generic/oberdiek}%
  \file{mleftright.sty}{\from{mleftright.dtx}{package}}%
  \usedir{doc/latex/oberdiek/test}%
  \file{mleftright-test1.tex}{\from{mleftright.dtx}{test1}}%
  \nopreamble
  \nopostamble
  \usedir{source/latex/oberdiek/catalogue}%
  \file{mleftright.xml}{\from{mleftright.dtx}{catalogue}}%
}

\catcode32=13\relax% active space
\let =\space%
\Msg{************************************************************************}
\Msg{*}
\Msg{* To finish the installation you have to move the following}
\Msg{* file into a directory searched by TeX:}
\Msg{*}
\Msg{*     mleftright.sty}
\Msg{*}
\Msg{* To produce the documentation run the file `mleftright.drv'}
\Msg{* through LaTeX.}
\Msg{*}
\Msg{* Happy TeXing!}
\Msg{*}
\Msg{************************************************************************}

\endbatchfile
%</install>
%<*ignore>
\fi
%</ignore>
%<*driver>
\NeedsTeXFormat{LaTeX2e}
\ProvidesFile{mleftright.drv}%
  [2016/05/16 v1.1 Math left/right delim. as open/close (HO)]%
\documentclass{ltxdoc}
\usepackage{holtxdoc}[2011/11/22]
\usepackage{mleftright}[2016/05/16]
\begin{document}
  \DocInput{mleftright.dtx}%
\end{document}
%</driver>
% \fi
%
%
% \CharacterTable
%  {Upper-case    \A\B\C\D\E\F\G\H\I\J\K\L\M\N\O\P\Q\R\S\T\U\V\W\X\Y\Z
%   Lower-case    \a\b\c\d\e\f\g\h\i\j\k\l\m\n\o\p\q\r\s\t\u\v\w\x\y\z
%   Digits        \0\1\2\3\4\5\6\7\8\9
%   Exclamation   \!     Double quote  \"     Hash (number) \#
%   Dollar        \$     Percent       \%     Ampersand     \&
%   Acute accent  \'     Left paren    \(     Right paren   \)
%   Asterisk      \*     Plus          \+     Comma         \,
%   Minus         \-     Point         \.     Solidus       \/
%   Colon         \:     Semicolon     \;     Less than     \<
%   Equals        \=     Greater than  \>     Question mark \?
%   Commercial at \@     Left bracket  \[     Backslash     \\
%   Right bracket \]     Circumflex    \^     Underscore    \_
%   Grave accent  \`     Left brace    \{     Vertical bar  \|
%   Right brace   \}     Tilde         \~}
%
% \GetFileInfo{mleftright.drv}
%
% \title{The \xpackage{mleftright} package}
% \date{2016/05/16 v1.1}
% \author{Heiko Oberdiek\thanks
% {Please report any issues at https://github.com/ho-tex/oberdiek/issues}\\
% \xemail{heiko.oberdiek at googlemail.com}}
%
% \maketitle
%
% \begin{abstract}
% \TeX\ sets subformulas by \cs{left} and \cs{right} as inner formulas
% with additional surrounding spaces in some situations. This package
% provides \cs{mleft} and \cs{mright} that call \cs{left} and \cs{right},
% but the delimiters will act as normal \cs{mathopen} and \cs{mathclose}
% delimiters without the additional space of an inner formula.
% \end{abstract}
%
% \tableofcontents
%
% \section{Documentation}
%
% The package is a result of a thread in the newsgroup \textsf{comp.text.tex}
% with the subject \textit{spacing after \cs{right}\texttt{)}
% and before \cs{left}\texttt{)}} \cite{dave}.
% The problem: \cs{left} and \cs{right} adjust the size of the
% delimiters automatically. However, \TeX\ treats the whole expression
% as inner formula. In some circumstances \TeX\ adds extra space
% before or after an inner formula.
% Example:
% \begin{quote}
%   \thinmuskip=1.5\thinmuskip
%   \begin{tabular}{@{}l@{\quad$\Rightarrow$\quad}l@{}}
%     |$\sin(x^2), x$|
%     & $\sin(x^2), x$\\
%     |$\sin\left(x^2\right), x$|
%     & $\sin\left(x^2\right), x$\\
%   ^^A  \multicolumn{1}{@{}r@{\quad$\Rightarrow$\quad}}{^^A
%   ^^A    \itshape with exaggerated spacing^^A
%   ^^A  }
%   ^^A  & $\thinmuskip=4\thinmuskip
%   ^^A    \sin\left(x^2\right){,}\mskip.25\thinmuskip x$\\
%     |$\sin\mleft(x^2\mright), x$|
%     & $\sin\mleft(x^2\mright), x$\\
%   \end{tabular}\\*[.5ex]
%   (\cs{mleft} and \cs{mright} are provided by this package.)
% \end{quote}
%
% In the newsgroup Donald Arseneau answered with clever macros \cite{arseneau}:
% \begin{quote}
%\begin{verbatim}
%\newcommand\lft{\mathopen{}\left}
%\newcommand\rgt{\aftergroup\mathclose\aftergroup{\aftergroup}\right}
%\end{verbatim}
% \end{quote}
% However one problem remains, a following subscript or superscript
% is not applied to the right delimiter but the empty
% \cs{mathclose}.
% Thus Philipp Stephani provided an improvement \cite{stephani}:
%\begin{quote}
%\begin{verbatim}
%\mathopen{} \mathclose{\left\| A^2 \right\|}_2
%\end{verbatim}
%\end{quote}
% Heiko Oberdiek converted this into macro form \cite{oberdiek}:
%\begin{quote}
%\begin{verbatim}
%\newcommand\lft{\mathopen{}\mathclose\bgroup\left}
%\newcommand\rgt{\aftergroup\egroup\right}
%\end{verbatim}
%\end{quote}
%
% The package uses longer macro names \cs{mleft} and \cs{mright}
% to avoid name clashes. Also it adds some checks for error conditions.
%
% \subsection{Use}
%
% \begin{declcs}{mleft}\meta{delimL} \dots\unkern\ \cs{mright}\meta{delimR}
% \end{declcs}
% Macros \cs{mleft} and \cs{mright} are used in the same way as
% \cs{left} and \cs{right}. Also \cs{middle} can be used inbetween if
% \eTeX\ is present.
%
% \begin{declcs}{mleftright}
% \end{declcs}
% Macro \cs{mleftright} redefines \cs{left} as \cs{mleft} and
% \cs{right} as \cs{mright}. The redefinition is local to the group.
%
% \begin{declcs}{mleftrightrestore}
% \end{declcs}
% Macro \cs{mleftright} restores \cs{left} and \cs{right} with
% the original meaning if they were previously redefined by
% \cs{mleftright} (also locally).
%
%
% \StopEventually{
% }
%
% \section{Implementation}
%    \begin{macrocode}
%<*package>
%    \end{macrocode}
%    Reload check, especially if the package is not used with \LaTeX.
%    \begin{macrocode}
\begingroup\catcode61\catcode48\catcode32=10\relax%
  \catcode13=5 % ^^M
  \endlinechar=13 %
  \catcode35=6 % #
  \catcode39=12 % '
  \catcode44=12 % ,
  \catcode45=12 % -
  \catcode46=12 % .
  \catcode58=12 % :
  \catcode64=11 % @
  \catcode123=1 % {
  \catcode125=2 % }
  \expandafter\let\expandafter\x\csname ver@mleftright.sty\endcsname
  \ifx\x\relax % plain-TeX, first loading
  \else
    \def\empty{}%
    \ifx\x\empty % LaTeX, first loading,
      % variable is initialized, but \ProvidesPackage not yet seen
    \else
      \expandafter\ifx\csname PackageInfo\endcsname\relax
        \def\x#1#2{%
          \immediate\write-1{Package #1 Info: #2.}%
        }%
      \else
        \def\x#1#2{\PackageInfo{#1}{#2, stopped}}%
      \fi
      \x{mleftright}{The package is already loaded}%
      \aftergroup\endinput
    \fi
  \fi
\endgroup%
%    \end{macrocode}
%    Package identification:
%    \begin{macrocode}
\begingroup\catcode61\catcode48\catcode32=10\relax%
  \catcode13=5 % ^^M
  \endlinechar=13 %
  \catcode35=6 % #
  \catcode39=12 % '
  \catcode40=12 % (
  \catcode41=12 % )
  \catcode44=12 % ,
  \catcode45=12 % -
  \catcode46=12 % .
  \catcode47=12 % /
  \catcode58=12 % :
  \catcode64=11 % @
  \catcode91=12 % [
  \catcode93=12 % ]
  \catcode123=1 % {
  \catcode125=2 % }
  \expandafter\ifx\csname ProvidesPackage\endcsname\relax
    \def\x#1#2#3[#4]{\endgroup
      \immediate\write-1{Package: #3 #4}%
      \xdef#1{#4}%
    }%
  \else
    \def\x#1#2[#3]{\endgroup
      #2[{#3}]%
      \ifx#1\@undefined
        \xdef#1{#3}%
      \fi
      \ifx#1\relax
        \xdef#1{#3}%
      \fi
    }%
  \fi
\expandafter\x\csname ver@mleftright.sty\endcsname
\ProvidesPackage{mleftright}%
  [2016/05/16 v1.1 Math left/right delim. as open/close (HO)]%
%    \end{macrocode}
%
%    \begin{macrocode}
\begingroup\catcode61\catcode48\catcode32=10\relax%
  \catcode13=5 % ^^M
  \endlinechar=13 %
  \catcode123=1 % {
  \catcode125=2 % }
  \catcode64=11 % @
  \def\x{\endgroup
    \expandafter\edef\csname mleftright@AtEnd\endcsname{%
      \endlinechar=\the\endlinechar\relax
      \catcode13=\the\catcode13\relax
      \catcode32=\the\catcode32\relax
      \catcode35=\the\catcode35\relax
      \catcode61=\the\catcode61\relax
      \catcode64=\the\catcode64\relax
      \catcode123=\the\catcode123\relax
      \catcode125=\the\catcode125\relax
    }%
  }%
\x\catcode61\catcode48\catcode32=10\relax%
\catcode13=5 % ^^M
\endlinechar=13 %
\catcode35=6 % #
\catcode64=11 % @
\catcode123=1 % {
\catcode125=2 % }
\def\TMP@EnsureCode#1#2{%
  \edef\mleftright@AtEnd{%
    \mleftright@AtEnd
    \catcode#1=\the\catcode#1\relax
  }%
  \catcode#1=#2\relax
}
\TMP@EnsureCode{38}{4}% &
\TMP@EnsureCode{39}{12}% '
\TMP@EnsureCode{40}{12}% (
\TMP@EnsureCode{41}{12}% )
\TMP@EnsureCode{42}{12}% *
\TMP@EnsureCode{43}{12}% +
\TMP@EnsureCode{44}{12}% ,
\TMP@EnsureCode{45}{12}% -
\TMP@EnsureCode{46}{12}% .
\TMP@EnsureCode{47}{12}% /
\TMP@EnsureCode{60}{12}% <
\TMP@EnsureCode{91}{12}% [
\TMP@EnsureCode{93}{12}% ]
\edef\mleftright@AtEnd{%
  \mleftright@AtEnd
  \escapechar\the\escapechar\relax
  \noexpand\endinput
}
\escapechar=92 %
%    \end{macrocode}
%
%    \begin{macrocode}
\begingroup\expandafter\expandafter\expandafter\endgroup
\expandafter\ifx\csname RequirePackage\endcsname\relax
  \input infwarerr.sty\relax
  \input ltxcmds.sty\relax
\else
  \RequirePackage{infwarerr}[2010/04/08]%
  \RequirePackage{ltxcmds}[2010/04/26]%
\fi
%    \end{macrocode}
%
%    The original commands \cs{left} and \cs{right}
%    are saved and later used in \cs{mleft} and
%    \cs{mright} in order to deal with:
%    \begin{quote}
%\begin{verbatim}
%\let\left\mleft
%\let\right\mright
%\end{verbatim}
%    \end{quote}
%    \begin{macro}{\mleftright@OrgLeft}
%    \begin{macrocode}
\let\mleftright@OrgLeft\left
%    \end{macrocode}
%    \end{macro}
%    \begin{macro}{\mleftright@OrgRight}
%    \begin{macrocode}
\let\mleftright@OrgRight\right
%    \end{macrocode}
%    \end{macro}
%
%    \begin{macro}{\mleftright@Def}
%    Macro \cs{mleftright@Def} defines a macro as robust macro
%    if \eTeX\ or \LaTeX\ is available.
%    \begin{macrocode}
\ltx@IfUndefined{protected}{%
  \ltx@IfUndefined{DeclareRobustCommand}{%
    \def\mleftright@Def{\def}%
  }{%
    \def\mleftright@Def{\DeclareRobustCommand*}%
  }%
}{%
  \def\mleftright@Def{\protected\def}%
}
\edef\mleftright@Def#1{%
  \noexpand\ltx@IfUndefined{%
    \noexpand\expandafter\noexpand\ltx@gobble\noexpand\string#1%
  }{%
    \expandafter\noexpand\mleftright@Def#1%
  }{%
    \noexpand\@PackageError{mleftright}{%
      Command \noexpand\string#1 already defined%
    }\noexpand\@ehd
    \noexpand\ltx@gobble
  }%
}
%    \end{macrocode}
%    \end{macro}
%
%    In case of \eTeX\ the group status after the left symbol
%    is saved and later checked at the beginning of \cs{mright}.
%    \begin{macrocode}
\ltx@IfUndefined{currentgrouplevel}{%
  \catcode38=14 % & = comment
}{%
  \catcode38=9 % & = ignore
}
%    \end{macrocode}
%
%    \begin{macro}{\mleftright@GroupLevel}
%    \begin{macrocode}
& \def\mleftright@GroupLevel{-1}%
%    \end{macrocode}
%    \end{macro}
%
%    \begin{macro}{\mleftright@WrongGroup}
%    \begin{macrocode}
& \def\mleftright@WrongGroup#1(#2){%
&   \ifnum\mleftright@GroupLevel<\ltx@zero
&     \@PackageError{mleftright}{%
&       Missing previous \string\mleft
&     }\@ehc
&   \else
&     \@PackageError{mleftright}{%
&       Unexpected group status for \string\mright%
&       \ifnum\mleftright@GroupLevel=#1 %
&       \else
&         .\MessageBreak
&         Group level is #1, %
&           expected is \mleftright@GroupLevel
&       \fi
&       \ifnum16=#2 %
&       \else
&         .\MessageBreak
&         Group type is #2 (%
&         \ifcase#2 %
&           bottom level%
&           \expandafter\expandafter\expandafter\ltx@gobblefour
&           \expandafter\ltx@gobbletwo
&         \or simple%
&         \or hbox%
&         \or adjusted hbox%
&         \or vbox%
&         \or vtop%
&         \or align%
&         \or no align%
&         \or output%
&         \or math%
&         \or disc%
&         \or insert%
&         \or vcenter%
&         \or math choice%
&         \or semi simple%
&         \or math shift%
&         \or math left%
&         \else
&           unknown%
&         \fi
&         \space group),\MessageBreak
&         expected is 16 (math left group)%
&       \fi
&     }\@ehd
&   \fi
& }%
%    \end{macrocode}
%    \end{macro}
%
%    \begin{macro}{\mleft}
%    \begin{macrocode}
\mleftright@Def\mleft{%
  \mathopen{}\mathclose\bgroup
& \edef\mleftright@GroupLevel{\the\numexpr\the\currentgrouplevel+1}%
  \mleftright@OrgLeft
}
%    \end{macrocode}
%    \end{macro}
%    \begin{macro}{\mright}
%    \begin{macrocode}
\mleftright@Def\mright{%
& \ifnum\mleftright@GroupLevel=\currentgrouplevel
&   \ifnum16=\currentgrouptype
      \aftergroup\egroup
&   \else
&     \expandafter\mleftright@WrongGroup
&     \the\expandafter\currentgrouplevel
&     \expandafter(\the\currentgrouptype)%
&   \fi
& \else
&   \expandafter\mleftright@WrongGroup
&   \the\expandafter\currentgrouplevel
&   \expandafter(\the\currentgrouptype)%
& \fi
  \mleftright@OrgRight
}
%    \end{macrocode}
%    \end{macro}
%
%    \begin{macro}{\mleftright}
%    \begin{macrocode}
\mleftright@Def\mleftright{%
  \let\left\mleft
  \let\right\mright
}
%    \end{macrocode}
%    \end{macro}
%
%    \begin{macro}{\mleftrightrestore}
%    \begin{macrocode}
\mleftright@Def\mleftrightrestore{%
  \ifx\left\mleft
    \let\left\mleftright@OrgLeft
  \fi
  \ifx\right\mright
    \let\right\mleftright@OrgRight
  \fi
}
%    \end{macrocode}
%    \end{macro}
%
%    \begin{macrocode}
\mleftright@AtEnd%
%</package>
%    \end{macrocode}
%
% \section{Test}
%
% \subsection{Catcode checks for loading}
%
%    \begin{macrocode}
%<*test1>
%    \end{macrocode}
%    \begin{macrocode}
\catcode`\{=1 %
\catcode`\}=2 %
\catcode`\#=6 %
\catcode`\@=11 %
\expandafter\ifx\csname count@\endcsname\relax
  \countdef\count@=255 %
\fi
\expandafter\ifx\csname @gobble\endcsname\relax
  \long\def\@gobble#1{}%
\fi
\expandafter\ifx\csname @firstofone\endcsname\relax
  \long\def\@firstofone#1{#1}%
\fi
\expandafter\ifx\csname loop\endcsname\relax
  \expandafter\@firstofone
\else
  \expandafter\@gobble
\fi
{%
  \def\loop#1\repeat{%
    \def\body{#1}%
    \iterate
  }%
  \def\iterate{%
    \body
      \let\next\iterate
    \else
      \let\next\relax
    \fi
    \next
  }%
  \let\repeat=\fi
}%
\def\RestoreCatcodes{}
\count@=0 %
\loop
  \edef\RestoreCatcodes{%
    \RestoreCatcodes
    \catcode\the\count@=\the\catcode\count@\relax
  }%
\ifnum\count@<255 %
  \advance\count@ 1 %
\repeat

\def\RangeCatcodeInvalid#1#2{%
  \count@=#1\relax
  \loop
    \catcode\count@=15 %
  \ifnum\count@<#2\relax
    \advance\count@ 1 %
  \repeat
}
\def\RangeCatcodeCheck#1#2#3{%
  \count@=#1\relax
  \loop
    \ifnum#3=\catcode\count@
    \else
      \errmessage{%
        Character \the\count@\space
        with wrong catcode \the\catcode\count@\space
        instead of \number#3%
      }%
    \fi
  \ifnum\count@<#2\relax
    \advance\count@ 1 %
  \repeat
}
\def\space{ }
\expandafter\ifx\csname LoadCommand\endcsname\relax
  \def\LoadCommand{\input mleftright.sty\relax}%
\fi
\def\Test{%
  \RangeCatcodeInvalid{0}{47}%
  \RangeCatcodeInvalid{58}{64}%
  \RangeCatcodeInvalid{91}{96}%
  \RangeCatcodeInvalid{123}{255}%
  \catcode`\@=12 %
  \catcode`\\=0 %
  \catcode`\%=14 %
  \LoadCommand
  \RangeCatcodeCheck{0}{36}{15}%
  \RangeCatcodeCheck{37}{37}{14}%
  \RangeCatcodeCheck{38}{47}{15}%
  \RangeCatcodeCheck{48}{57}{12}%
  \RangeCatcodeCheck{58}{63}{15}%
  \RangeCatcodeCheck{64}{64}{12}%
  \RangeCatcodeCheck{65}{90}{11}%
  \RangeCatcodeCheck{91}{91}{15}%
  \RangeCatcodeCheck{92}{92}{0}%
  \RangeCatcodeCheck{93}{96}{15}%
  \RangeCatcodeCheck{97}{122}{11}%
  \RangeCatcodeCheck{123}{255}{15}%
  \RestoreCatcodes
}
\Test
\csname @@end\endcsname
\end
%    \end{macrocode}
%    \begin{macrocode}
%</test1>
%    \end{macrocode}
%
% \section{Installation}
%
% \subsection{Download}
%
% \paragraph{Package.} This package is available on
% CTAN\footnote{\url{http://ctan.org/pkg/mleftright}}:
% \begin{description}
% \item[\CTAN{macros/latex/contrib/oberdiek/mleftright.dtx}] The source file.
% \item[\CTAN{macros/latex/contrib/oberdiek/mleftright.pdf}] Documentation.
% \end{description}
%
%
% \paragraph{Bundle.} All the packages of the bundle `oberdiek'
% are also available in a TDS compliant ZIP archive. There
% the packages are already unpacked and the documentation files
% are generated. The files and directories obey the TDS standard.
% \begin{description}
% \item[\CTAN{install/macros/latex/contrib/oberdiek.tds.zip}]
% \end{description}
% \emph{TDS} refers to the standard ``A Directory Structure
% for \TeX\ Files'' (\CTAN{tds/tds.pdf}). Directories
% with \xfile{texmf} in their name are usually organized this way.
%
% \subsection{Bundle installation}
%
% \paragraph{Unpacking.} Unpack the \xfile{oberdiek.tds.zip} in the
% TDS tree (also known as \xfile{texmf} tree) of your choice.
% Example (linux):
% \begin{quote}
%   |unzip oberdiek.tds.zip -d ~/texmf|
% \end{quote}
%
% \paragraph{Script installation.}
% Check the directory \xfile{TDS:scripts/oberdiek/} for
% scripts that need further installation steps.
% Package \xpackage{attachfile2} comes with the Perl script
% \xfile{pdfatfi.pl} that should be installed in such a way
% that it can be called as \texttt{pdfatfi}.
% Example (linux):
% \begin{quote}
%   |chmod +x scripts/oberdiek/pdfatfi.pl|\\
%   |cp scripts/oberdiek/pdfatfi.pl /usr/local/bin/|
% \end{quote}
%
% \subsection{Package installation}
%
% \paragraph{Unpacking.} The \xfile{.dtx} file is a self-extracting
% \docstrip\ archive. The files are extracted by running the
% \xfile{.dtx} through \plainTeX:
% \begin{quote}
%   \verb|tex mleftright.dtx|
% \end{quote}
%
% \paragraph{TDS.} Now the different files must be moved into
% the different directories in your installation TDS tree
% (also known as \xfile{texmf} tree):
% \begin{quote}
% \def\t{^^A
% \begin{tabular}{@{}>{\ttfamily}l@{ $\rightarrow$ }>{\ttfamily}l@{}}
%   mleftright.sty & tex/generic/oberdiek/mleftright.sty\\
%   mleftright.pdf & doc/latex/oberdiek/mleftright.pdf\\
%   test/mleftright-test1.tex & doc/latex/oberdiek/test/mleftright-test1.tex\\
%   mleftright.dtx & source/latex/oberdiek/mleftright.dtx\\
% \end{tabular}^^A
% }^^A
% \sbox0{\t}^^A
% \ifdim\wd0>\linewidth
%   \begingroup
%     \advance\linewidth by\leftmargin
%     \advance\linewidth by\rightmargin
%   \edef\x{\endgroup
%     \def\noexpand\lw{\the\linewidth}^^A
%   }\x
%   \def\lwbox{^^A
%     \leavevmode
%     \hbox to \linewidth{^^A
%       \kern-\leftmargin\relax
%       \hss
%       \usebox0
%       \hss
%       \kern-\rightmargin\relax
%     }^^A
%   }^^A
%   \ifdim\wd0>\lw
%     \sbox0{\small\t}^^A
%     \ifdim\wd0>\linewidth
%       \ifdim\wd0>\lw
%         \sbox0{\footnotesize\t}^^A
%         \ifdim\wd0>\linewidth
%           \ifdim\wd0>\lw
%             \sbox0{\scriptsize\t}^^A
%             \ifdim\wd0>\linewidth
%               \ifdim\wd0>\lw
%                 \sbox0{\tiny\t}^^A
%                 \ifdim\wd0>\linewidth
%                   \lwbox
%                 \else
%                   \usebox0
%                 \fi
%               \else
%                 \lwbox
%               \fi
%             \else
%               \usebox0
%             \fi
%           \else
%             \lwbox
%           \fi
%         \else
%           \usebox0
%         \fi
%       \else
%         \lwbox
%       \fi
%     \else
%       \usebox0
%     \fi
%   \else
%     \lwbox
%   \fi
% \else
%   \usebox0
% \fi
% \end{quote}
% If you have a \xfile{docstrip.cfg} that configures and enables \docstrip's
% TDS installing feature, then some files can already be in the right
% place, see the documentation of \docstrip.
%
% \subsection{Refresh file name databases}
%
% If your \TeX~distribution
% (\teTeX, \mikTeX, \dots) relies on file name databases, you must refresh
% these. For example, \teTeX\ users run \verb|texhash| or
% \verb|mktexlsr|.
%
% \subsection{Some details for the interested}
%
% \paragraph{Attached source.}
%
% The PDF documentation on CTAN also includes the
% \xfile{.dtx} source file. It can be extracted by
% AcrobatReader 6 or higher. Another option is \textsf{pdftk},
% e.g. unpack the file into the current directory:
% \begin{quote}
%   \verb|pdftk mleftright.pdf unpack_files output .|
% \end{quote}
%
% \paragraph{Unpacking with \LaTeX.}
% The \xfile{.dtx} chooses its action depending on the format:
% \begin{description}
% \item[\plainTeX:] Run \docstrip\ and extract the files.
% \item[\LaTeX:] Generate the documentation.
% \end{description}
% If you insist on using \LaTeX\ for \docstrip\ (really,
% \docstrip\ does not need \LaTeX), then inform the autodetect routine
% about your intention:
% \begin{quote}
%   \verb|latex \let\install=y\input{mleftright.dtx}|
% \end{quote}
% Do not forget to quote the argument according to the demands
% of your shell.
%
% \paragraph{Generating the documentation.}
% You can use both the \xfile{.dtx} or the \xfile{.drv} to generate
% the documentation. The process can be configured by the
% configuration file \xfile{ltxdoc.cfg}. For instance, put this
% line into this file, if you want to have A4 as paper format:
% \begin{quote}
%   \verb|\PassOptionsToClass{a4paper}{article}|
% \end{quote}
% An example follows how to generate the
% documentation with pdf\LaTeX:
% \begin{quote}
%\begin{verbatim}
%pdflatex mleftright.dtx
%makeindex -s gind.ist mleftright.idx
%pdflatex mleftright.dtx
%makeindex -s gind.ist mleftright.idx
%pdflatex mleftright.dtx
%\end{verbatim}
% \end{quote}
%
% \section{Catalogue}
%
% The following XML file can be used as source for the
% \href{http://mirror.ctan.org/help/Catalogue/catalogue.html}{\TeX\ Catalogue}.
% The elements \texttt{caption} and \texttt{description} are imported
% from the original XML file from the Catalogue.
% The name of the XML file in the Catalogue is \xfile{mleftright.xml}.
%    \begin{macrocode}
%<*catalogue>
<?xml version='1.0' encoding='us-ascii'?>
<!DOCTYPE entry SYSTEM 'catalogue.dtd'>
<entry datestamp='$Date$' modifier='$Author$' id='mleftright'>
  <name>mleftright</name>
  <caption>Variants of delimiters that act as maths open/close.</caption>
  <authorref id='auth:oberdiek'/>
  <copyright owner='Heiko Oberdiek' year='2010'/>
  <license type='lppl1.3'/>
  <version number='1.1'/>
  <description>
    The package defines variants <tt>\mleft</tt> and <tt>\mright</tt>
    of <tt>\left</tt> and <tt>\right</tt>, that make the delimiters
    act as <tt>\mathopen</tt> and <tt>\mathclose</tt>.  These commands
    address spacing difficulties in subformulas.
    <p/>
    The package is part of the <xref refid='oberdiek'>oberdiek</xref> bundle.
  </description>
  <documentation details='Package documentation'
      href='ctan:/macros/latex/contrib/oberdiek/mleftright.pdf'/>
  <ctan file='true' path='/macros/latex/contrib/oberdiek/mleftright.dtx'/>
  <miktex location='oberdiek'/>
  <texlive location='oberdiek'/>
  <install path='/macros/latex/contrib/oberdiek/oberdiek.tds.zip'/>
</entry>
%</catalogue>
%    \end{macrocode}
%
% \section{Acknowledgement}
%
% \begin{description}
% \item[Donald Arsenau:]
% He provided the main trick and the first macros.
% \item[Philipp Stephani:]
% He solved the subscript problem.
% \end{description}
%
% \begin{thebibliography}{9}
% \raggedright
% \bibitem{dave}
%   Dave94705,
%   \textit{spacing after \cs{right}\texttt{)} and before \cs{left}\texttt{)}},
%   newsgroup comp.text.tex,
%   Message-ID: \texttt{\small 5d264909-7c3d-4c9d-9b22-434178b2bf90@g21g2000prn.googlegroups.com},
%   2010-08-12.
%   \newblock
%   {\small\url{http://groups.google.com/group/comp.text.tex/msg/e5b6833da7dc29bf}}
%
% \bibitem{arseneau}
%   Donald Arseneau,
%   \textit{Re: spacing after \cs{right}\texttt) and before \cs{left}\texttt)},
%   newsgroup comp.text.tex,
%   Message-ID: \texttt{\small yfivd6svl8y.fsf@mutant.triumf.ca},
%   2010-08-30.
%   \newblock
%   {\small\url{http://groups.google.com/group/comp.text.tex/msg/e0b2e4386e5d04e4}}
%
% \bibitem{stephani}
%   Philipp Stephani,
%   \textit{Re: spacing after \cs{right}\texttt) and before \cs{left}\texttt)},
%   newsgroup comp.text.tex,
%   Message-ID: \texttt{\small 4c8c8c1e\$0\$6981\$9b4e6d93@newsspool4.arcor-online.net},
%   2010-09-12.
%   \newblock
%   {\small\url{http://groups.google.com/group/comp.text.tex/msg/87ac1f61321de3ef}}
%
% \bibitem{oberdiek}
%   Heiko Oberdiek,
%   \textit{Re: spacing after \cs{right}\texttt) and before \cs{left}\texttt)},
%   newsgroup comp.text.tex,
%   Message-ID: \texttt{\small i6jcc2\$8of\$1@news.eternal-september.org},
%   2010-09-12.
%   \newblock
%   {\small\url{http://groups.google.com/group/comp.text.tex/msg/257aa6119bef878b}}
%
% \end{thebibliography}
%
% \begin{History}
%   \begin{Version}{2010/09/25 v1.0}
%   \item
%     The first version.
%   \end{Version}
%   \begin{Version}{2016/05/16 v1.1}
%   \item
%     Documentation updates.
%   \end{Version}
% \end{History}
%
% \PrintIndex
%
% \Finale
\endinput
|
% \end{quote}
% Do not forget to quote the argument according to the demands
% of your shell.
%
% \paragraph{Generating the documentation.}
% You can use both the \xfile{.dtx} or the \xfile{.drv} to generate
% the documentation. The process can be configured by the
% configuration file \xfile{ltxdoc.cfg}. For instance, put this
% line into this file, if you want to have A4 as paper format:
% \begin{quote}
%   \verb|\PassOptionsToClass{a4paper}{article}|
% \end{quote}
% An example follows how to generate the
% documentation with pdf\LaTeX:
% \begin{quote}
%\begin{verbatim}
%pdflatex mleftright.dtx
%makeindex -s gind.ist mleftright.idx
%pdflatex mleftright.dtx
%makeindex -s gind.ist mleftright.idx
%pdflatex mleftright.dtx
%\end{verbatim}
% \end{quote}
%
% \section{Catalogue}
%
% The following XML file can be used as source for the
% \href{http://mirror.ctan.org/help/Catalogue/catalogue.html}{\TeX\ Catalogue}.
% The elements \texttt{caption} and \texttt{description} are imported
% from the original XML file from the Catalogue.
% The name of the XML file in the Catalogue is \xfile{mleftright.xml}.
%    \begin{macrocode}
%<*catalogue>
<?xml version='1.0' encoding='us-ascii'?>
<!DOCTYPE entry SYSTEM 'catalogue.dtd'>
<entry datestamp='$Date$' modifier='$Author$' id='mleftright'>
  <name>mleftright</name>
  <caption>Variants of delimiters that act as maths open/close.</caption>
  <authorref id='auth:oberdiek'/>
  <copyright owner='Heiko Oberdiek' year='2010'/>
  <license type='lppl1.3'/>
  <version number='1.1'/>
  <description>
    The package defines variants <tt>\mleft</tt> and <tt>\mright</tt>
    of <tt>\left</tt> and <tt>\right</tt>, that make the delimiters
    act as <tt>\mathopen</tt> and <tt>\mathclose</tt>.  These commands
    address spacing difficulties in subformulas.
    <p/>
    The package is part of the <xref refid='oberdiek'>oberdiek</xref> bundle.
  </description>
  <documentation details='Package documentation'
      href='ctan:/macros/latex/contrib/oberdiek/mleftright.pdf'/>
  <ctan file='true' path='/macros/latex/contrib/oberdiek/mleftright.dtx'/>
  <miktex location='oberdiek'/>
  <texlive location='oberdiek'/>
  <install path='/macros/latex/contrib/oberdiek/oberdiek.tds.zip'/>
</entry>
%</catalogue>
%    \end{macrocode}
%
% \section{Acknowledgement}
%
% \begin{description}
% \item[Donald Arsenau:]
% He provided the main trick and the first macros.
% \item[Philipp Stephani:]
% He solved the subscript problem.
% \end{description}
%
% \begin{thebibliography}{9}
% \raggedright
% \bibitem{dave}
%   Dave94705,
%   \textit{spacing after \cs{right}\texttt{)} and before \cs{left}\texttt{)}},
%   newsgroup comp.text.tex,
%   Message-ID: \texttt{\small 5d264909-7c3d-4c9d-9b22-434178b2bf90@g21g2000prn.googlegroups.com},
%   2010-08-12.
%   \newblock
%   {\small\url{http://groups.google.com/group/comp.text.tex/msg/e5b6833da7dc29bf}}
%
% \bibitem{arseneau}
%   Donald Arseneau,
%   \textit{Re: spacing after \cs{right}\texttt) and before \cs{left}\texttt)},
%   newsgroup comp.text.tex,
%   Message-ID: \texttt{\small yfivd6svl8y.fsf@mutant.triumf.ca},
%   2010-08-30.
%   \newblock
%   {\small\url{http://groups.google.com/group/comp.text.tex/msg/e0b2e4386e5d04e4}}
%
% \bibitem{stephani}
%   Philipp Stephani,
%   \textit{Re: spacing after \cs{right}\texttt) and before \cs{left}\texttt)},
%   newsgroup comp.text.tex,
%   Message-ID: \texttt{\small 4c8c8c1e\$0\$6981\$9b4e6d93@newsspool4.arcor-online.net},
%   2010-09-12.
%   \newblock
%   {\small\url{http://groups.google.com/group/comp.text.tex/msg/87ac1f61321de3ef}}
%
% \bibitem{oberdiek}
%   Heiko Oberdiek,
%   \textit{Re: spacing after \cs{right}\texttt) and before \cs{left}\texttt)},
%   newsgroup comp.text.tex,
%   Message-ID: \texttt{\small i6jcc2\$8of\$1@news.eternal-september.org},
%   2010-09-12.
%   \newblock
%   {\small\url{http://groups.google.com/group/comp.text.tex/msg/257aa6119bef878b}}
%
% \end{thebibliography}
%
% \begin{History}
%   \begin{Version}{2010/09/25 v1.0}
%   \item
%     The first version.
%   \end{Version}
%   \begin{Version}{2016/05/16 v1.1}
%   \item
%     Documentation updates.
%   \end{Version}
% \end{History}
%
% \PrintIndex
%
% \Finale
\endinput

%        (quote the arguments according to the demands of your shell)
%
% Documentation:
%    (a) If mleftright.drv is present:
%           latex mleftright.drv
%    (b) Without mleftright.drv:
%           latex mleftright.dtx; ...
%    The class ltxdoc loads the configuration file ltxdoc.cfg
%    if available. Here you can specify further options, e.g.
%    use A4 as paper format:
%       \PassOptionsToClass{a4paper}{article}
%
%    Programm calls to get the documentation (example):
%       pdflatex mleftright.dtx
%       makeindex -s gind.ist mleftright.idx
%       pdflatex mleftright.dtx
%       makeindex -s gind.ist mleftright.idx
%       pdflatex mleftright.dtx
%
% Installation:
%    TDS:tex/generic/oberdiek/mleftright.sty
%    TDS:doc/latex/oberdiek/mleftright.pdf
%    TDS:doc/latex/oberdiek/test/mleftright-test1.tex
%    TDS:source/latex/oberdiek/mleftright.dtx
%
%<*ignore>
\begingroup
  \catcode123=1 %
  \catcode125=2 %
  \def\x{LaTeX2e}%
\expandafter\endgroup
\ifcase 0\ifx\install y1\fi\expandafter
         \ifx\csname processbatchFile\endcsname\relax\else1\fi
         \ifx\fmtname\x\else 1\fi\relax
\else\csname fi\endcsname
%</ignore>
%<*install>
\input docstrip.tex
\Msg{************************************************************************}
\Msg{* Installation}
\Msg{* Package: mleftright 2016/05/16 v1.1 Math left/right delim. as open/close (HO)}
\Msg{************************************************************************}

\keepsilent
\askforoverwritefalse

\let\MetaPrefix\relax
\preamble

This is a generated file.

Project: mleftright
Version: 2016/05/16 v1.1

Copyright (C) 2010 by
   Heiko Oberdiek <heiko.oberdiek at googlemail.com>

This work may be distributed and/or modified under the
conditions of the LaTeX Project Public License, either
version 1.3c of this license or (at your option) any later
version. This version of this license is in
   http://www.latex-project.org/lppl/lppl-1-3c.txt
and the latest version of this license is in
   http://www.latex-project.org/lppl.txt
and version 1.3 or later is part of all distributions of
LaTeX version 2005/12/01 or later.

This work has the LPPL maintenance status "maintained".

This Current Maintainer of this work is Heiko Oberdiek.

The Base Interpreter refers to any `TeX-Format',
because some files are installed in TDS:tex/generic//.

This work consists of the main source file mleftright.dtx
and the derived files
   mleftright.sty, mleftright.pdf, mleftright.ins, mleftright.drv,
   mleftright-test1.tex.

\endpreamble
\let\MetaPrefix\DoubleperCent

\generate{%
  \file{mleftright.ins}{\from{mleftright.dtx}{install}}%
  \file{mleftright.drv}{\from{mleftright.dtx}{driver}}%
  \usedir{tex/generic/oberdiek}%
  \file{mleftright.sty}{\from{mleftright.dtx}{package}}%
  \usedir{doc/latex/oberdiek/test}%
  \file{mleftright-test1.tex}{\from{mleftright.dtx}{test1}}%
  \nopreamble
  \nopostamble
  \usedir{source/latex/oberdiek/catalogue}%
  \file{mleftright.xml}{\from{mleftright.dtx}{catalogue}}%
}

\catcode32=13\relax% active space
\let =\space%
\Msg{************************************************************************}
\Msg{*}
\Msg{* To finish the installation you have to move the following}
\Msg{* file into a directory searched by TeX:}
\Msg{*}
\Msg{*     mleftright.sty}
\Msg{*}
\Msg{* To produce the documentation run the file `mleftright.drv'}
\Msg{* through LaTeX.}
\Msg{*}
\Msg{* Happy TeXing!}
\Msg{*}
\Msg{************************************************************************}

\endbatchfile
%</install>
%<*ignore>
\fi
%</ignore>
%<*driver>
\NeedsTeXFormat{LaTeX2e}
\ProvidesFile{mleftright.drv}%
  [2016/05/16 v1.1 Math left/right delim. as open/close (HO)]%
\documentclass{ltxdoc}
\usepackage{holtxdoc}[2011/11/22]
\usepackage{mleftright}[2016/05/16]
\begin{document}
  \DocInput{mleftright.dtx}%
\end{document}
%</driver>
% \fi
%
%
% \CharacterTable
%  {Upper-case    \A\B\C\D\E\F\G\H\I\J\K\L\M\N\O\P\Q\R\S\T\U\V\W\X\Y\Z
%   Lower-case    \a\b\c\d\e\f\g\h\i\j\k\l\m\n\o\p\q\r\s\t\u\v\w\x\y\z
%   Digits        \0\1\2\3\4\5\6\7\8\9
%   Exclamation   \!     Double quote  \"     Hash (number) \#
%   Dollar        \$     Percent       \%     Ampersand     \&
%   Acute accent  \'     Left paren    \(     Right paren   \)
%   Asterisk      \*     Plus          \+     Comma         \,
%   Minus         \-     Point         \.     Solidus       \/
%   Colon         \:     Semicolon     \;     Less than     \<
%   Equals        \=     Greater than  \>     Question mark \?
%   Commercial at \@     Left bracket  \[     Backslash     \\
%   Right bracket \]     Circumflex    \^     Underscore    \_
%   Grave accent  \`     Left brace    \{     Vertical bar  \|
%   Right brace   \}     Tilde         \~}
%
% \GetFileInfo{mleftright.drv}
%
% \title{The \xpackage{mleftright} package}
% \date{2016/05/16 v1.1}
% \author{Heiko Oberdiek\thanks
% {Please report any issues at https://github.com/ho-tex/oberdiek/issues}\\
% \xemail{heiko.oberdiek at googlemail.com}}
%
% \maketitle
%
% \begin{abstract}
% \TeX\ sets subformulas by \cs{left} and \cs{right} as inner formulas
% with additional surrounding spaces in some situations. This package
% provides \cs{mleft} and \cs{mright} that call \cs{left} and \cs{right},
% but the delimiters will act as normal \cs{mathopen} and \cs{mathclose}
% delimiters without the additional space of an inner formula.
% \end{abstract}
%
% \tableofcontents
%
% \section{Documentation}
%
% The package is a result of a thread in the newsgroup \textsf{comp.text.tex}
% with the subject \textit{spacing after \cs{right}\texttt{)}
% and before \cs{left}\texttt{)}} \cite{dave}.
% The problem: \cs{left} and \cs{right} adjust the size of the
% delimiters automatically. However, \TeX\ treats the whole expression
% as inner formula. In some circumstances \TeX\ adds extra space
% before or after an inner formula.
% Example:
% \begin{quote}
%   \thinmuskip=1.5\thinmuskip
%   \begin{tabular}{@{}l@{\quad$\Rightarrow$\quad}l@{}}
%     |$\sin(x^2), x$|
%     & $\sin(x^2), x$\\
%     |$\sin\left(x^2\right), x$|
%     & $\sin\left(x^2\right), x$\\
%   ^^A  \multicolumn{1}{@{}r@{\quad$\Rightarrow$\quad}}{^^A
%   ^^A    \itshape with exaggerated spacing^^A
%   ^^A  }
%   ^^A  & $\thinmuskip=4\thinmuskip
%   ^^A    \sin\left(x^2\right){,}\mskip.25\thinmuskip x$\\
%     |$\sin\mleft(x^2\mright), x$|
%     & $\sin\mleft(x^2\mright), x$\\
%   \end{tabular}\\*[.5ex]
%   (\cs{mleft} and \cs{mright} are provided by this package.)
% \end{quote}
%
% In the newsgroup Donald Arseneau answered with clever macros \cite{arseneau}:
% \begin{quote}
%\begin{verbatim}
%\newcommand\lft{\mathopen{}\left}
%\newcommand\rgt{\aftergroup\mathclose\aftergroup{\aftergroup}\right}
%\end{verbatim}
% \end{quote}
% However one problem remains, a following subscript or superscript
% is not applied to the right delimiter but the empty
% \cs{mathclose}.
% Thus Philipp Stephani provided an improvement \cite{stephani}:
%\begin{quote}
%\begin{verbatim}
%\mathopen{} \mathclose{\left\| A^2 \right\|}_2
%\end{verbatim}
%\end{quote}
% Heiko Oberdiek converted this into macro form \cite{oberdiek}:
%\begin{quote}
%\begin{verbatim}
%\newcommand\lft{\mathopen{}\mathclose\bgroup\left}
%\newcommand\rgt{\aftergroup\egroup\right}
%\end{verbatim}
%\end{quote}
%
% The package uses longer macro names \cs{mleft} and \cs{mright}
% to avoid name clashes. Also it adds some checks for error conditions.
%
% \subsection{Use}
%
% \begin{declcs}{mleft}\meta{delimL} \dots\unkern\ \cs{mright}\meta{delimR}
% \end{declcs}
% Macros \cs{mleft} and \cs{mright} are used in the same way as
% \cs{left} and \cs{right}. Also \cs{middle} can be used inbetween if
% \eTeX\ is present.
%
% \begin{declcs}{mleftright}
% \end{declcs}
% Macro \cs{mleftright} redefines \cs{left} as \cs{mleft} and
% \cs{right} as \cs{mright}. The redefinition is local to the group.
%
% \begin{declcs}{mleftrightrestore}
% \end{declcs}
% Macro \cs{mleftright} restores \cs{left} and \cs{right} with
% the original meaning if they were previously redefined by
% \cs{mleftright} (also locally).
%
%
% \StopEventually{
% }
%
% \section{Implementation}
%    \begin{macrocode}
%<*package>
%    \end{macrocode}
%    Reload check, especially if the package is not used with \LaTeX.
%    \begin{macrocode}
\begingroup\catcode61\catcode48\catcode32=10\relax%
  \catcode13=5 % ^^M
  \endlinechar=13 %
  \catcode35=6 % #
  \catcode39=12 % '
  \catcode44=12 % ,
  \catcode45=12 % -
  \catcode46=12 % .
  \catcode58=12 % :
  \catcode64=11 % @
  \catcode123=1 % {
  \catcode125=2 % }
  \expandafter\let\expandafter\x\csname ver@mleftright.sty\endcsname
  \ifx\x\relax % plain-TeX, first loading
  \else
    \def\empty{}%
    \ifx\x\empty % LaTeX, first loading,
      % variable is initialized, but \ProvidesPackage not yet seen
    \else
      \expandafter\ifx\csname PackageInfo\endcsname\relax
        \def\x#1#2{%
          \immediate\write-1{Package #1 Info: #2.}%
        }%
      \else
        \def\x#1#2{\PackageInfo{#1}{#2, stopped}}%
      \fi
      \x{mleftright}{The package is already loaded}%
      \aftergroup\endinput
    \fi
  \fi
\endgroup%
%    \end{macrocode}
%    Package identification:
%    \begin{macrocode}
\begingroup\catcode61\catcode48\catcode32=10\relax%
  \catcode13=5 % ^^M
  \endlinechar=13 %
  \catcode35=6 % #
  \catcode39=12 % '
  \catcode40=12 % (
  \catcode41=12 % )
  \catcode44=12 % ,
  \catcode45=12 % -
  \catcode46=12 % .
  \catcode47=12 % /
  \catcode58=12 % :
  \catcode64=11 % @
  \catcode91=12 % [
  \catcode93=12 % ]
  \catcode123=1 % {
  \catcode125=2 % }
  \expandafter\ifx\csname ProvidesPackage\endcsname\relax
    \def\x#1#2#3[#4]{\endgroup
      \immediate\write-1{Package: #3 #4}%
      \xdef#1{#4}%
    }%
  \else
    \def\x#1#2[#3]{\endgroup
      #2[{#3}]%
      \ifx#1\@undefined
        \xdef#1{#3}%
      \fi
      \ifx#1\relax
        \xdef#1{#3}%
      \fi
    }%
  \fi
\expandafter\x\csname ver@mleftright.sty\endcsname
\ProvidesPackage{mleftright}%
  [2016/05/16 v1.1 Math left/right delim. as open/close (HO)]%
%    \end{macrocode}
%
%    \begin{macrocode}
\begingroup\catcode61\catcode48\catcode32=10\relax%
  \catcode13=5 % ^^M
  \endlinechar=13 %
  \catcode123=1 % {
  \catcode125=2 % }
  \catcode64=11 % @
  \def\x{\endgroup
    \expandafter\edef\csname mleftright@AtEnd\endcsname{%
      \endlinechar=\the\endlinechar\relax
      \catcode13=\the\catcode13\relax
      \catcode32=\the\catcode32\relax
      \catcode35=\the\catcode35\relax
      \catcode61=\the\catcode61\relax
      \catcode64=\the\catcode64\relax
      \catcode123=\the\catcode123\relax
      \catcode125=\the\catcode125\relax
    }%
  }%
\x\catcode61\catcode48\catcode32=10\relax%
\catcode13=5 % ^^M
\endlinechar=13 %
\catcode35=6 % #
\catcode64=11 % @
\catcode123=1 % {
\catcode125=2 % }
\def\TMP@EnsureCode#1#2{%
  \edef\mleftright@AtEnd{%
    \mleftright@AtEnd
    \catcode#1=\the\catcode#1\relax
  }%
  \catcode#1=#2\relax
}
\TMP@EnsureCode{38}{4}% &
\TMP@EnsureCode{39}{12}% '
\TMP@EnsureCode{40}{12}% (
\TMP@EnsureCode{41}{12}% )
\TMP@EnsureCode{42}{12}% *
\TMP@EnsureCode{43}{12}% +
\TMP@EnsureCode{44}{12}% ,
\TMP@EnsureCode{45}{12}% -
\TMP@EnsureCode{46}{12}% .
\TMP@EnsureCode{47}{12}% /
\TMP@EnsureCode{60}{12}% <
\TMP@EnsureCode{91}{12}% [
\TMP@EnsureCode{93}{12}% ]
\edef\mleftright@AtEnd{%
  \mleftright@AtEnd
  \escapechar\the\escapechar\relax
  \noexpand\endinput
}
\escapechar=92 %
%    \end{macrocode}
%
%    \begin{macrocode}
\begingroup\expandafter\expandafter\expandafter\endgroup
\expandafter\ifx\csname RequirePackage\endcsname\relax
  \input infwarerr.sty\relax
  \input ltxcmds.sty\relax
\else
  \RequirePackage{infwarerr}[2010/04/08]%
  \RequirePackage{ltxcmds}[2010/04/26]%
\fi
%    \end{macrocode}
%
%    The original commands \cs{left} and \cs{right}
%    are saved and later used in \cs{mleft} and
%    \cs{mright} in order to deal with:
%    \begin{quote}
%\begin{verbatim}
%\let\left\mleft
%\let\right\mright
%\end{verbatim}
%    \end{quote}
%    \begin{macro}{\mleftright@OrgLeft}
%    \begin{macrocode}
\let\mleftright@OrgLeft\left
%    \end{macrocode}
%    \end{macro}
%    \begin{macro}{\mleftright@OrgRight}
%    \begin{macrocode}
\let\mleftright@OrgRight\right
%    \end{macrocode}
%    \end{macro}
%
%    \begin{macro}{\mleftright@Def}
%    Macro \cs{mleftright@Def} defines a macro as robust macro
%    if \eTeX\ or \LaTeX\ is available.
%    \begin{macrocode}
\ltx@IfUndefined{protected}{%
  \ltx@IfUndefined{DeclareRobustCommand}{%
    \def\mleftright@Def{\def}%
  }{%
    \def\mleftright@Def{\DeclareRobustCommand*}%
  }%
}{%
  \def\mleftright@Def{\protected\def}%
}
\edef\mleftright@Def#1{%
  \noexpand\ltx@IfUndefined{%
    \noexpand\expandafter\noexpand\ltx@gobble\noexpand\string#1%
  }{%
    \expandafter\noexpand\mleftright@Def#1%
  }{%
    \noexpand\@PackageError{mleftright}{%
      Command \noexpand\string#1 already defined%
    }\noexpand\@ehd
    \noexpand\ltx@gobble
  }%
}
%    \end{macrocode}
%    \end{macro}
%
%    In case of \eTeX\ the group status after the left symbol
%    is saved and later checked at the beginning of \cs{mright}.
%    \begin{macrocode}
\ltx@IfUndefined{currentgrouplevel}{%
  \catcode38=14 % & = comment
}{%
  \catcode38=9 % & = ignore
}
%    \end{macrocode}
%
%    \begin{macro}{\mleftright@GroupLevel}
%    \begin{macrocode}
& \def\mleftright@GroupLevel{-1}%
%    \end{macrocode}
%    \end{macro}
%
%    \begin{macro}{\mleftright@WrongGroup}
%    \begin{macrocode}
& \def\mleftright@WrongGroup#1(#2){%
&   \ifnum\mleftright@GroupLevel<\ltx@zero
&     \@PackageError{mleftright}{%
&       Missing previous \string\mleft
&     }\@ehc
&   \else
&     \@PackageError{mleftright}{%
&       Unexpected group status for \string\mright%
&       \ifnum\mleftright@GroupLevel=#1 %
&       \else
&         .\MessageBreak
&         Group level is #1, %
&           expected is \mleftright@GroupLevel
&       \fi
&       \ifnum16=#2 %
&       \else
&         .\MessageBreak
&         Group type is #2 (%
&         \ifcase#2 %
&           bottom level%
&           \expandafter\expandafter\expandafter\ltx@gobblefour
&           \expandafter\ltx@gobbletwo
&         \or simple%
&         \or hbox%
&         \or adjusted hbox%
&         \or vbox%
&         \or vtop%
&         \or align%
&         \or no align%
&         \or output%
&         \or math%
&         \or disc%
&         \or insert%
&         \or vcenter%
&         \or math choice%
&         \or semi simple%
&         \or math shift%
&         \or math left%
&         \else
&           unknown%
&         \fi
&         \space group),\MessageBreak
&         expected is 16 (math left group)%
&       \fi
&     }\@ehd
&   \fi
& }%
%    \end{macrocode}
%    \end{macro}
%
%    \begin{macro}{\mleft}
%    \begin{macrocode}
\mleftright@Def\mleft{%
  \mathopen{}\mathclose\bgroup
& \edef\mleftright@GroupLevel{\the\numexpr\the\currentgrouplevel+1}%
  \mleftright@OrgLeft
}
%    \end{macrocode}
%    \end{macro}
%    \begin{macro}{\mright}
%    \begin{macrocode}
\mleftright@Def\mright{%
& \ifnum\mleftright@GroupLevel=\currentgrouplevel
&   \ifnum16=\currentgrouptype
      \aftergroup\egroup
&   \else
&     \expandafter\mleftright@WrongGroup
&     \the\expandafter\currentgrouplevel
&     \expandafter(\the\currentgrouptype)%
&   \fi
& \else
&   \expandafter\mleftright@WrongGroup
&   \the\expandafter\currentgrouplevel
&   \expandafter(\the\currentgrouptype)%
& \fi
  \mleftright@OrgRight
}
%    \end{macrocode}
%    \end{macro}
%
%    \begin{macro}{\mleftright}
%    \begin{macrocode}
\mleftright@Def\mleftright{%
  \let\left\mleft
  \let\right\mright
}
%    \end{macrocode}
%    \end{macro}
%
%    \begin{macro}{\mleftrightrestore}
%    \begin{macrocode}
\mleftright@Def\mleftrightrestore{%
  \ifx\left\mleft
    \let\left\mleftright@OrgLeft
  \fi
  \ifx\right\mright
    \let\right\mleftright@OrgRight
  \fi
}
%    \end{macrocode}
%    \end{macro}
%
%    \begin{macrocode}
\mleftright@AtEnd%
%</package>
%    \end{macrocode}
%
% \section{Test}
%
% \subsection{Catcode checks for loading}
%
%    \begin{macrocode}
%<*test1>
%    \end{macrocode}
%    \begin{macrocode}
\catcode`\{=1 %
\catcode`\}=2 %
\catcode`\#=6 %
\catcode`\@=11 %
\expandafter\ifx\csname count@\endcsname\relax
  \countdef\count@=255 %
\fi
\expandafter\ifx\csname @gobble\endcsname\relax
  \long\def\@gobble#1{}%
\fi
\expandafter\ifx\csname @firstofone\endcsname\relax
  \long\def\@firstofone#1{#1}%
\fi
\expandafter\ifx\csname loop\endcsname\relax
  \expandafter\@firstofone
\else
  \expandafter\@gobble
\fi
{%
  \def\loop#1\repeat{%
    \def\body{#1}%
    \iterate
  }%
  \def\iterate{%
    \body
      \let\next\iterate
    \else
      \let\next\relax
    \fi
    \next
  }%
  \let\repeat=\fi
}%
\def\RestoreCatcodes{}
\count@=0 %
\loop
  \edef\RestoreCatcodes{%
    \RestoreCatcodes
    \catcode\the\count@=\the\catcode\count@\relax
  }%
\ifnum\count@<255 %
  \advance\count@ 1 %
\repeat

\def\RangeCatcodeInvalid#1#2{%
  \count@=#1\relax
  \loop
    \catcode\count@=15 %
  \ifnum\count@<#2\relax
    \advance\count@ 1 %
  \repeat
}
\def\RangeCatcodeCheck#1#2#3{%
  \count@=#1\relax
  \loop
    \ifnum#3=\catcode\count@
    \else
      \errmessage{%
        Character \the\count@\space
        with wrong catcode \the\catcode\count@\space
        instead of \number#3%
      }%
    \fi
  \ifnum\count@<#2\relax
    \advance\count@ 1 %
  \repeat
}
\def\space{ }
\expandafter\ifx\csname LoadCommand\endcsname\relax
  \def\LoadCommand{\input mleftright.sty\relax}%
\fi
\def\Test{%
  \RangeCatcodeInvalid{0}{47}%
  \RangeCatcodeInvalid{58}{64}%
  \RangeCatcodeInvalid{91}{96}%
  \RangeCatcodeInvalid{123}{255}%
  \catcode`\@=12 %
  \catcode`\\=0 %
  \catcode`\%=14 %
  \LoadCommand
  \RangeCatcodeCheck{0}{36}{15}%
  \RangeCatcodeCheck{37}{37}{14}%
  \RangeCatcodeCheck{38}{47}{15}%
  \RangeCatcodeCheck{48}{57}{12}%
  \RangeCatcodeCheck{58}{63}{15}%
  \RangeCatcodeCheck{64}{64}{12}%
  \RangeCatcodeCheck{65}{90}{11}%
  \RangeCatcodeCheck{91}{91}{15}%
  \RangeCatcodeCheck{92}{92}{0}%
  \RangeCatcodeCheck{93}{96}{15}%
  \RangeCatcodeCheck{97}{122}{11}%
  \RangeCatcodeCheck{123}{255}{15}%
  \RestoreCatcodes
}
\Test
\csname @@end\endcsname
\end
%    \end{macrocode}
%    \begin{macrocode}
%</test1>
%    \end{macrocode}
%
% \section{Installation}
%
% \subsection{Download}
%
% \paragraph{Package.} This package is available on
% CTAN\footnote{\url{http://ctan.org/pkg/mleftright}}:
% \begin{description}
% \item[\CTAN{macros/latex/contrib/oberdiek/mleftright.dtx}] The source file.
% \item[\CTAN{macros/latex/contrib/oberdiek/mleftright.pdf}] Documentation.
% \end{description}
%
%
% \paragraph{Bundle.} All the packages of the bundle `oberdiek'
% are also available in a TDS compliant ZIP archive. There
% the packages are already unpacked and the documentation files
% are generated. The files and directories obey the TDS standard.
% \begin{description}
% \item[\CTAN{install/macros/latex/contrib/oberdiek.tds.zip}]
% \end{description}
% \emph{TDS} refers to the standard ``A Directory Structure
% for \TeX\ Files'' (\CTAN{tds/tds.pdf}). Directories
% with \xfile{texmf} in their name are usually organized this way.
%
% \subsection{Bundle installation}
%
% \paragraph{Unpacking.} Unpack the \xfile{oberdiek.tds.zip} in the
% TDS tree (also known as \xfile{texmf} tree) of your choice.
% Example (linux):
% \begin{quote}
%   |unzip oberdiek.tds.zip -d ~/texmf|
% \end{quote}
%
% \paragraph{Script installation.}
% Check the directory \xfile{TDS:scripts/oberdiek/} for
% scripts that need further installation steps.
% Package \xpackage{attachfile2} comes with the Perl script
% \xfile{pdfatfi.pl} that should be installed in such a way
% that it can be called as \texttt{pdfatfi}.
% Example (linux):
% \begin{quote}
%   |chmod +x scripts/oberdiek/pdfatfi.pl|\\
%   |cp scripts/oberdiek/pdfatfi.pl /usr/local/bin/|
% \end{quote}
%
% \subsection{Package installation}
%
% \paragraph{Unpacking.} The \xfile{.dtx} file is a self-extracting
% \docstrip\ archive. The files are extracted by running the
% \xfile{.dtx} through \plainTeX:
% \begin{quote}
%   \verb|tex mleftright.dtx|
% \end{quote}
%
% \paragraph{TDS.} Now the different files must be moved into
% the different directories in your installation TDS tree
% (also known as \xfile{texmf} tree):
% \begin{quote}
% \def\t{^^A
% \begin{tabular}{@{}>{\ttfamily}l@{ $\rightarrow$ }>{\ttfamily}l@{}}
%   mleftright.sty & tex/generic/oberdiek/mleftright.sty\\
%   mleftright.pdf & doc/latex/oberdiek/mleftright.pdf\\
%   test/mleftright-test1.tex & doc/latex/oberdiek/test/mleftright-test1.tex\\
%   mleftright.dtx & source/latex/oberdiek/mleftright.dtx\\
% \end{tabular}^^A
% }^^A
% \sbox0{\t}^^A
% \ifdim\wd0>\linewidth
%   \begingroup
%     \advance\linewidth by\leftmargin
%     \advance\linewidth by\rightmargin
%   \edef\x{\endgroup
%     \def\noexpand\lw{\the\linewidth}^^A
%   }\x
%   \def\lwbox{^^A
%     \leavevmode
%     \hbox to \linewidth{^^A
%       \kern-\leftmargin\relax
%       \hss
%       \usebox0
%       \hss
%       \kern-\rightmargin\relax
%     }^^A
%   }^^A
%   \ifdim\wd0>\lw
%     \sbox0{\small\t}^^A
%     \ifdim\wd0>\linewidth
%       \ifdim\wd0>\lw
%         \sbox0{\footnotesize\t}^^A
%         \ifdim\wd0>\linewidth
%           \ifdim\wd0>\lw
%             \sbox0{\scriptsize\t}^^A
%             \ifdim\wd0>\linewidth
%               \ifdim\wd0>\lw
%                 \sbox0{\tiny\t}^^A
%                 \ifdim\wd0>\linewidth
%                   \lwbox
%                 \else
%                   \usebox0
%                 \fi
%               \else
%                 \lwbox
%               \fi
%             \else
%               \usebox0
%             \fi
%           \else
%             \lwbox
%           \fi
%         \else
%           \usebox0
%         \fi
%       \else
%         \lwbox
%       \fi
%     \else
%       \usebox0
%     \fi
%   \else
%     \lwbox
%   \fi
% \else
%   \usebox0
% \fi
% \end{quote}
% If you have a \xfile{docstrip.cfg} that configures and enables \docstrip's
% TDS installing feature, then some files can already be in the right
% place, see the documentation of \docstrip.
%
% \subsection{Refresh file name databases}
%
% If your \TeX~distribution
% (\teTeX, \mikTeX, \dots) relies on file name databases, you must refresh
% these. For example, \teTeX\ users run \verb|texhash| or
% \verb|mktexlsr|.
%
% \subsection{Some details for the interested}
%
% \paragraph{Attached source.}
%
% The PDF documentation on CTAN also includes the
% \xfile{.dtx} source file. It can be extracted by
% AcrobatReader 6 or higher. Another option is \textsf{pdftk},
% e.g. unpack the file into the current directory:
% \begin{quote}
%   \verb|pdftk mleftright.pdf unpack_files output .|
% \end{quote}
%
% \paragraph{Unpacking with \LaTeX.}
% The \xfile{.dtx} chooses its action depending on the format:
% \begin{description}
% \item[\plainTeX:] Run \docstrip\ and extract the files.
% \item[\LaTeX:] Generate the documentation.
% \end{description}
% If you insist on using \LaTeX\ for \docstrip\ (really,
% \docstrip\ does not need \LaTeX), then inform the autodetect routine
% about your intention:
% \begin{quote}
%   \verb|latex \let\install=y% \iffalse meta-comment
%
% File: mleftright.dtx
% Version: 2016/05/16 v1.1
% Info: Math left/right delim. as open/close
%
% Copyright (C) 2010 by
%    Heiko Oberdiek <heiko.oberdiek at googlemail.com>
%    2016
%    https://github.com/ho-tex/oberdiek/issues
%
% This work may be distributed and/or modified under the
% conditions of the LaTeX Project Public License, either
% version 1.3c of this license or (at your option) any later
% version. This version of this license is in
%    http://www.latex-project.org/lppl/lppl-1-3c.txt
% and the latest version of this license is in
%    http://www.latex-project.org/lppl.txt
% and version 1.3 or later is part of all distributions of
% LaTeX version 2005/12/01 or later.
%
% This work has the LPPL maintenance status "maintained".
%
% This Current Maintainer of this work is Heiko Oberdiek.
%
% The Base Interpreter refers to any `TeX-Format',
% because some files are installed in TDS:tex/generic//.
%
% This work consists of the main source file mleftright.dtx
% and the derived files
%    mleftright.sty, mleftright.pdf, mleftright.ins, mleftright.drv,
%    mleftright-test1.tex.
%
% Distribution:
%    CTAN:macros/latex/contrib/oberdiek/mleftright.dtx
%    CTAN:macros/latex/contrib/oberdiek/mleftright.pdf
%
% Unpacking:
%    (a) If mleftright.ins is present:
%           tex mleftright.ins
%    (b) Without mleftright.ins:
%           tex mleftright.dtx
%    (c) If you insist on using LaTeX
%           latex \let\install=y% \iffalse meta-comment
%
% File: mleftright.dtx
% Version: 2016/05/16 v1.1
% Info: Math left/right delim. as open/close
%
% Copyright (C) 2010 by
%    Heiko Oberdiek <heiko.oberdiek at googlemail.com>
%    2016
%    https://github.com/ho-tex/oberdiek/issues
%
% This work may be distributed and/or modified under the
% conditions of the LaTeX Project Public License, either
% version 1.3c of this license or (at your option) any later
% version. This version of this license is in
%    http://www.latex-project.org/lppl/lppl-1-3c.txt
% and the latest version of this license is in
%    http://www.latex-project.org/lppl.txt
% and version 1.3 or later is part of all distributions of
% LaTeX version 2005/12/01 or later.
%
% This work has the LPPL maintenance status "maintained".
%
% This Current Maintainer of this work is Heiko Oberdiek.
%
% The Base Interpreter refers to any `TeX-Format',
% because some files are installed in TDS:tex/generic//.
%
% This work consists of the main source file mleftright.dtx
% and the derived files
%    mleftright.sty, mleftright.pdf, mleftright.ins, mleftright.drv,
%    mleftright-test1.tex.
%
% Distribution:
%    CTAN:macros/latex/contrib/oberdiek/mleftright.dtx
%    CTAN:macros/latex/contrib/oberdiek/mleftright.pdf
%
% Unpacking:
%    (a) If mleftright.ins is present:
%           tex mleftright.ins
%    (b) Without mleftright.ins:
%           tex mleftright.dtx
%    (c) If you insist on using LaTeX
%           latex \let\install=y\input{mleftright.dtx}
%        (quote the arguments according to the demands of your shell)
%
% Documentation:
%    (a) If mleftright.drv is present:
%           latex mleftright.drv
%    (b) Without mleftright.drv:
%           latex mleftright.dtx; ...
%    The class ltxdoc loads the configuration file ltxdoc.cfg
%    if available. Here you can specify further options, e.g.
%    use A4 as paper format:
%       \PassOptionsToClass{a4paper}{article}
%
%    Programm calls to get the documentation (example):
%       pdflatex mleftright.dtx
%       makeindex -s gind.ist mleftright.idx
%       pdflatex mleftright.dtx
%       makeindex -s gind.ist mleftright.idx
%       pdflatex mleftright.dtx
%
% Installation:
%    TDS:tex/generic/oberdiek/mleftright.sty
%    TDS:doc/latex/oberdiek/mleftright.pdf
%    TDS:doc/latex/oberdiek/test/mleftright-test1.tex
%    TDS:source/latex/oberdiek/mleftright.dtx
%
%<*ignore>
\begingroup
  \catcode123=1 %
  \catcode125=2 %
  \def\x{LaTeX2e}%
\expandafter\endgroup
\ifcase 0\ifx\install y1\fi\expandafter
         \ifx\csname processbatchFile\endcsname\relax\else1\fi
         \ifx\fmtname\x\else 1\fi\relax
\else\csname fi\endcsname
%</ignore>
%<*install>
\input docstrip.tex
\Msg{************************************************************************}
\Msg{* Installation}
\Msg{* Package: mleftright 2016/05/16 v1.1 Math left/right delim. as open/close (HO)}
\Msg{************************************************************************}

\keepsilent
\askforoverwritefalse

\let\MetaPrefix\relax
\preamble

This is a generated file.

Project: mleftright
Version: 2016/05/16 v1.1

Copyright (C) 2010 by
   Heiko Oberdiek <heiko.oberdiek at googlemail.com>

This work may be distributed and/or modified under the
conditions of the LaTeX Project Public License, either
version 1.3c of this license or (at your option) any later
version. This version of this license is in
   http://www.latex-project.org/lppl/lppl-1-3c.txt
and the latest version of this license is in
   http://www.latex-project.org/lppl.txt
and version 1.3 or later is part of all distributions of
LaTeX version 2005/12/01 or later.

This work has the LPPL maintenance status "maintained".

This Current Maintainer of this work is Heiko Oberdiek.

The Base Interpreter refers to any `TeX-Format',
because some files are installed in TDS:tex/generic//.

This work consists of the main source file mleftright.dtx
and the derived files
   mleftright.sty, mleftright.pdf, mleftright.ins, mleftright.drv,
   mleftright-test1.tex.

\endpreamble
\let\MetaPrefix\DoubleperCent

\generate{%
  \file{mleftright.ins}{\from{mleftright.dtx}{install}}%
  \file{mleftright.drv}{\from{mleftright.dtx}{driver}}%
  \usedir{tex/generic/oberdiek}%
  \file{mleftright.sty}{\from{mleftright.dtx}{package}}%
  \usedir{doc/latex/oberdiek/test}%
  \file{mleftright-test1.tex}{\from{mleftright.dtx}{test1}}%
  \nopreamble
  \nopostamble
  \usedir{source/latex/oberdiek/catalogue}%
  \file{mleftright.xml}{\from{mleftright.dtx}{catalogue}}%
}

\catcode32=13\relax% active space
\let =\space%
\Msg{************************************************************************}
\Msg{*}
\Msg{* To finish the installation you have to move the following}
\Msg{* file into a directory searched by TeX:}
\Msg{*}
\Msg{*     mleftright.sty}
\Msg{*}
\Msg{* To produce the documentation run the file `mleftright.drv'}
\Msg{* through LaTeX.}
\Msg{*}
\Msg{* Happy TeXing!}
\Msg{*}
\Msg{************************************************************************}

\endbatchfile
%</install>
%<*ignore>
\fi
%</ignore>
%<*driver>
\NeedsTeXFormat{LaTeX2e}
\ProvidesFile{mleftright.drv}%
  [2016/05/16 v1.1 Math left/right delim. as open/close (HO)]%
\documentclass{ltxdoc}
\usepackage{holtxdoc}[2011/11/22]
\usepackage{mleftright}[2016/05/16]
\begin{document}
  \DocInput{mleftright.dtx}%
\end{document}
%</driver>
% \fi
%
%
% \CharacterTable
%  {Upper-case    \A\B\C\D\E\F\G\H\I\J\K\L\M\N\O\P\Q\R\S\T\U\V\W\X\Y\Z
%   Lower-case    \a\b\c\d\e\f\g\h\i\j\k\l\m\n\o\p\q\r\s\t\u\v\w\x\y\z
%   Digits        \0\1\2\3\4\5\6\7\8\9
%   Exclamation   \!     Double quote  \"     Hash (number) \#
%   Dollar        \$     Percent       \%     Ampersand     \&
%   Acute accent  \'     Left paren    \(     Right paren   \)
%   Asterisk      \*     Plus          \+     Comma         \,
%   Minus         \-     Point         \.     Solidus       \/
%   Colon         \:     Semicolon     \;     Less than     \<
%   Equals        \=     Greater than  \>     Question mark \?
%   Commercial at \@     Left bracket  \[     Backslash     \\
%   Right bracket \]     Circumflex    \^     Underscore    \_
%   Grave accent  \`     Left brace    \{     Vertical bar  \|
%   Right brace   \}     Tilde         \~}
%
% \GetFileInfo{mleftright.drv}
%
% \title{The \xpackage{mleftright} package}
% \date{2016/05/16 v1.1}
% \author{Heiko Oberdiek\thanks
% {Please report any issues at https://github.com/ho-tex/oberdiek/issues}\\
% \xemail{heiko.oberdiek at googlemail.com}}
%
% \maketitle
%
% \begin{abstract}
% \TeX\ sets subformulas by \cs{left} and \cs{right} as inner formulas
% with additional surrounding spaces in some situations. This package
% provides \cs{mleft} and \cs{mright} that call \cs{left} and \cs{right},
% but the delimiters will act as normal \cs{mathopen} and \cs{mathclose}
% delimiters without the additional space of an inner formula.
% \end{abstract}
%
% \tableofcontents
%
% \section{Documentation}
%
% The package is a result of a thread in the newsgroup \textsf{comp.text.tex}
% with the subject \textit{spacing after \cs{right}\texttt{)}
% and before \cs{left}\texttt{)}} \cite{dave}.
% The problem: \cs{left} and \cs{right} adjust the size of the
% delimiters automatically. However, \TeX\ treats the whole expression
% as inner formula. In some circumstances \TeX\ adds extra space
% before or after an inner formula.
% Example:
% \begin{quote}
%   \thinmuskip=1.5\thinmuskip
%   \begin{tabular}{@{}l@{\quad$\Rightarrow$\quad}l@{}}
%     |$\sin(x^2), x$|
%     & $\sin(x^2), x$\\
%     |$\sin\left(x^2\right), x$|
%     & $\sin\left(x^2\right), x$\\
%   ^^A  \multicolumn{1}{@{}r@{\quad$\Rightarrow$\quad}}{^^A
%   ^^A    \itshape with exaggerated spacing^^A
%   ^^A  }
%   ^^A  & $\thinmuskip=4\thinmuskip
%   ^^A    \sin\left(x^2\right){,}\mskip.25\thinmuskip x$\\
%     |$\sin\mleft(x^2\mright), x$|
%     & $\sin\mleft(x^2\mright), x$\\
%   \end{tabular}\\*[.5ex]
%   (\cs{mleft} and \cs{mright} are provided by this package.)
% \end{quote}
%
% In the newsgroup Donald Arseneau answered with clever macros \cite{arseneau}:
% \begin{quote}
%\begin{verbatim}
%\newcommand\lft{\mathopen{}\left}
%\newcommand\rgt{\aftergroup\mathclose\aftergroup{\aftergroup}\right}
%\end{verbatim}
% \end{quote}
% However one problem remains, a following subscript or superscript
% is not applied to the right delimiter but the empty
% \cs{mathclose}.
% Thus Philipp Stephani provided an improvement \cite{stephani}:
%\begin{quote}
%\begin{verbatim}
%\mathopen{} \mathclose{\left\| A^2 \right\|}_2
%\end{verbatim}
%\end{quote}
% Heiko Oberdiek converted this into macro form \cite{oberdiek}:
%\begin{quote}
%\begin{verbatim}
%\newcommand\lft{\mathopen{}\mathclose\bgroup\left}
%\newcommand\rgt{\aftergroup\egroup\right}
%\end{verbatim}
%\end{quote}
%
% The package uses longer macro names \cs{mleft} and \cs{mright}
% to avoid name clashes. Also it adds some checks for error conditions.
%
% \subsection{Use}
%
% \begin{declcs}{mleft}\meta{delimL} \dots\unkern\ \cs{mright}\meta{delimR}
% \end{declcs}
% Macros \cs{mleft} and \cs{mright} are used in the same way as
% \cs{left} and \cs{right}. Also \cs{middle} can be used inbetween if
% \eTeX\ is present.
%
% \begin{declcs}{mleftright}
% \end{declcs}
% Macro \cs{mleftright} redefines \cs{left} as \cs{mleft} and
% \cs{right} as \cs{mright}. The redefinition is local to the group.
%
% \begin{declcs}{mleftrightrestore}
% \end{declcs}
% Macro \cs{mleftright} restores \cs{left} and \cs{right} with
% the original meaning if they were previously redefined by
% \cs{mleftright} (also locally).
%
%
% \StopEventually{
% }
%
% \section{Implementation}
%    \begin{macrocode}
%<*package>
%    \end{macrocode}
%    Reload check, especially if the package is not used with \LaTeX.
%    \begin{macrocode}
\begingroup\catcode61\catcode48\catcode32=10\relax%
  \catcode13=5 % ^^M
  \endlinechar=13 %
  \catcode35=6 % #
  \catcode39=12 % '
  \catcode44=12 % ,
  \catcode45=12 % -
  \catcode46=12 % .
  \catcode58=12 % :
  \catcode64=11 % @
  \catcode123=1 % {
  \catcode125=2 % }
  \expandafter\let\expandafter\x\csname ver@mleftright.sty\endcsname
  \ifx\x\relax % plain-TeX, first loading
  \else
    \def\empty{}%
    \ifx\x\empty % LaTeX, first loading,
      % variable is initialized, but \ProvidesPackage not yet seen
    \else
      \expandafter\ifx\csname PackageInfo\endcsname\relax
        \def\x#1#2{%
          \immediate\write-1{Package #1 Info: #2.}%
        }%
      \else
        \def\x#1#2{\PackageInfo{#1}{#2, stopped}}%
      \fi
      \x{mleftright}{The package is already loaded}%
      \aftergroup\endinput
    \fi
  \fi
\endgroup%
%    \end{macrocode}
%    Package identification:
%    \begin{macrocode}
\begingroup\catcode61\catcode48\catcode32=10\relax%
  \catcode13=5 % ^^M
  \endlinechar=13 %
  \catcode35=6 % #
  \catcode39=12 % '
  \catcode40=12 % (
  \catcode41=12 % )
  \catcode44=12 % ,
  \catcode45=12 % -
  \catcode46=12 % .
  \catcode47=12 % /
  \catcode58=12 % :
  \catcode64=11 % @
  \catcode91=12 % [
  \catcode93=12 % ]
  \catcode123=1 % {
  \catcode125=2 % }
  \expandafter\ifx\csname ProvidesPackage\endcsname\relax
    \def\x#1#2#3[#4]{\endgroup
      \immediate\write-1{Package: #3 #4}%
      \xdef#1{#4}%
    }%
  \else
    \def\x#1#2[#3]{\endgroup
      #2[{#3}]%
      \ifx#1\@undefined
        \xdef#1{#3}%
      \fi
      \ifx#1\relax
        \xdef#1{#3}%
      \fi
    }%
  \fi
\expandafter\x\csname ver@mleftright.sty\endcsname
\ProvidesPackage{mleftright}%
  [2016/05/16 v1.1 Math left/right delim. as open/close (HO)]%
%    \end{macrocode}
%
%    \begin{macrocode}
\begingroup\catcode61\catcode48\catcode32=10\relax%
  \catcode13=5 % ^^M
  \endlinechar=13 %
  \catcode123=1 % {
  \catcode125=2 % }
  \catcode64=11 % @
  \def\x{\endgroup
    \expandafter\edef\csname mleftright@AtEnd\endcsname{%
      \endlinechar=\the\endlinechar\relax
      \catcode13=\the\catcode13\relax
      \catcode32=\the\catcode32\relax
      \catcode35=\the\catcode35\relax
      \catcode61=\the\catcode61\relax
      \catcode64=\the\catcode64\relax
      \catcode123=\the\catcode123\relax
      \catcode125=\the\catcode125\relax
    }%
  }%
\x\catcode61\catcode48\catcode32=10\relax%
\catcode13=5 % ^^M
\endlinechar=13 %
\catcode35=6 % #
\catcode64=11 % @
\catcode123=1 % {
\catcode125=2 % }
\def\TMP@EnsureCode#1#2{%
  \edef\mleftright@AtEnd{%
    \mleftright@AtEnd
    \catcode#1=\the\catcode#1\relax
  }%
  \catcode#1=#2\relax
}
\TMP@EnsureCode{38}{4}% &
\TMP@EnsureCode{39}{12}% '
\TMP@EnsureCode{40}{12}% (
\TMP@EnsureCode{41}{12}% )
\TMP@EnsureCode{42}{12}% *
\TMP@EnsureCode{43}{12}% +
\TMP@EnsureCode{44}{12}% ,
\TMP@EnsureCode{45}{12}% -
\TMP@EnsureCode{46}{12}% .
\TMP@EnsureCode{47}{12}% /
\TMP@EnsureCode{60}{12}% <
\TMP@EnsureCode{91}{12}% [
\TMP@EnsureCode{93}{12}% ]
\edef\mleftright@AtEnd{%
  \mleftright@AtEnd
  \escapechar\the\escapechar\relax
  \noexpand\endinput
}
\escapechar=92 %
%    \end{macrocode}
%
%    \begin{macrocode}
\begingroup\expandafter\expandafter\expandafter\endgroup
\expandafter\ifx\csname RequirePackage\endcsname\relax
  \input infwarerr.sty\relax
  \input ltxcmds.sty\relax
\else
  \RequirePackage{infwarerr}[2010/04/08]%
  \RequirePackage{ltxcmds}[2010/04/26]%
\fi
%    \end{macrocode}
%
%    The original commands \cs{left} and \cs{right}
%    are saved and later used in \cs{mleft} and
%    \cs{mright} in order to deal with:
%    \begin{quote}
%\begin{verbatim}
%\let\left\mleft
%\let\right\mright
%\end{verbatim}
%    \end{quote}
%    \begin{macro}{\mleftright@OrgLeft}
%    \begin{macrocode}
\let\mleftright@OrgLeft\left
%    \end{macrocode}
%    \end{macro}
%    \begin{macro}{\mleftright@OrgRight}
%    \begin{macrocode}
\let\mleftright@OrgRight\right
%    \end{macrocode}
%    \end{macro}
%
%    \begin{macro}{\mleftright@Def}
%    Macro \cs{mleftright@Def} defines a macro as robust macro
%    if \eTeX\ or \LaTeX\ is available.
%    \begin{macrocode}
\ltx@IfUndefined{protected}{%
  \ltx@IfUndefined{DeclareRobustCommand}{%
    \def\mleftright@Def{\def}%
  }{%
    \def\mleftright@Def{\DeclareRobustCommand*}%
  }%
}{%
  \def\mleftright@Def{\protected\def}%
}
\edef\mleftright@Def#1{%
  \noexpand\ltx@IfUndefined{%
    \noexpand\expandafter\noexpand\ltx@gobble\noexpand\string#1%
  }{%
    \expandafter\noexpand\mleftright@Def#1%
  }{%
    \noexpand\@PackageError{mleftright}{%
      Command \noexpand\string#1 already defined%
    }\noexpand\@ehd
    \noexpand\ltx@gobble
  }%
}
%    \end{macrocode}
%    \end{macro}
%
%    In case of \eTeX\ the group status after the left symbol
%    is saved and later checked at the beginning of \cs{mright}.
%    \begin{macrocode}
\ltx@IfUndefined{currentgrouplevel}{%
  \catcode38=14 % & = comment
}{%
  \catcode38=9 % & = ignore
}
%    \end{macrocode}
%
%    \begin{macro}{\mleftright@GroupLevel}
%    \begin{macrocode}
& \def\mleftright@GroupLevel{-1}%
%    \end{macrocode}
%    \end{macro}
%
%    \begin{macro}{\mleftright@WrongGroup}
%    \begin{macrocode}
& \def\mleftright@WrongGroup#1(#2){%
&   \ifnum\mleftright@GroupLevel<\ltx@zero
&     \@PackageError{mleftright}{%
&       Missing previous \string\mleft
&     }\@ehc
&   \else
&     \@PackageError{mleftright}{%
&       Unexpected group status for \string\mright%
&       \ifnum\mleftright@GroupLevel=#1 %
&       \else
&         .\MessageBreak
&         Group level is #1, %
&           expected is \mleftright@GroupLevel
&       \fi
&       \ifnum16=#2 %
&       \else
&         .\MessageBreak
&         Group type is #2 (%
&         \ifcase#2 %
&           bottom level%
&           \expandafter\expandafter\expandafter\ltx@gobblefour
&           \expandafter\ltx@gobbletwo
&         \or simple%
&         \or hbox%
&         \or adjusted hbox%
&         \or vbox%
&         \or vtop%
&         \or align%
&         \or no align%
&         \or output%
&         \or math%
&         \or disc%
&         \or insert%
&         \or vcenter%
&         \or math choice%
&         \or semi simple%
&         \or math shift%
&         \or math left%
&         \else
&           unknown%
&         \fi
&         \space group),\MessageBreak
&         expected is 16 (math left group)%
&       \fi
&     }\@ehd
&   \fi
& }%
%    \end{macrocode}
%    \end{macro}
%
%    \begin{macro}{\mleft}
%    \begin{macrocode}
\mleftright@Def\mleft{%
  \mathopen{}\mathclose\bgroup
& \edef\mleftright@GroupLevel{\the\numexpr\the\currentgrouplevel+1}%
  \mleftright@OrgLeft
}
%    \end{macrocode}
%    \end{macro}
%    \begin{macro}{\mright}
%    \begin{macrocode}
\mleftright@Def\mright{%
& \ifnum\mleftright@GroupLevel=\currentgrouplevel
&   \ifnum16=\currentgrouptype
      \aftergroup\egroup
&   \else
&     \expandafter\mleftright@WrongGroup
&     \the\expandafter\currentgrouplevel
&     \expandafter(\the\currentgrouptype)%
&   \fi
& \else
&   \expandafter\mleftright@WrongGroup
&   \the\expandafter\currentgrouplevel
&   \expandafter(\the\currentgrouptype)%
& \fi
  \mleftright@OrgRight
}
%    \end{macrocode}
%    \end{macro}
%
%    \begin{macro}{\mleftright}
%    \begin{macrocode}
\mleftright@Def\mleftright{%
  \let\left\mleft
  \let\right\mright
}
%    \end{macrocode}
%    \end{macro}
%
%    \begin{macro}{\mleftrightrestore}
%    \begin{macrocode}
\mleftright@Def\mleftrightrestore{%
  \ifx\left\mleft
    \let\left\mleftright@OrgLeft
  \fi
  \ifx\right\mright
    \let\right\mleftright@OrgRight
  \fi
}
%    \end{macrocode}
%    \end{macro}
%
%    \begin{macrocode}
\mleftright@AtEnd%
%</package>
%    \end{macrocode}
%
% \section{Test}
%
% \subsection{Catcode checks for loading}
%
%    \begin{macrocode}
%<*test1>
%    \end{macrocode}
%    \begin{macrocode}
\catcode`\{=1 %
\catcode`\}=2 %
\catcode`\#=6 %
\catcode`\@=11 %
\expandafter\ifx\csname count@\endcsname\relax
  \countdef\count@=255 %
\fi
\expandafter\ifx\csname @gobble\endcsname\relax
  \long\def\@gobble#1{}%
\fi
\expandafter\ifx\csname @firstofone\endcsname\relax
  \long\def\@firstofone#1{#1}%
\fi
\expandafter\ifx\csname loop\endcsname\relax
  \expandafter\@firstofone
\else
  \expandafter\@gobble
\fi
{%
  \def\loop#1\repeat{%
    \def\body{#1}%
    \iterate
  }%
  \def\iterate{%
    \body
      \let\next\iterate
    \else
      \let\next\relax
    \fi
    \next
  }%
  \let\repeat=\fi
}%
\def\RestoreCatcodes{}
\count@=0 %
\loop
  \edef\RestoreCatcodes{%
    \RestoreCatcodes
    \catcode\the\count@=\the\catcode\count@\relax
  }%
\ifnum\count@<255 %
  \advance\count@ 1 %
\repeat

\def\RangeCatcodeInvalid#1#2{%
  \count@=#1\relax
  \loop
    \catcode\count@=15 %
  \ifnum\count@<#2\relax
    \advance\count@ 1 %
  \repeat
}
\def\RangeCatcodeCheck#1#2#3{%
  \count@=#1\relax
  \loop
    \ifnum#3=\catcode\count@
    \else
      \errmessage{%
        Character \the\count@\space
        with wrong catcode \the\catcode\count@\space
        instead of \number#3%
      }%
    \fi
  \ifnum\count@<#2\relax
    \advance\count@ 1 %
  \repeat
}
\def\space{ }
\expandafter\ifx\csname LoadCommand\endcsname\relax
  \def\LoadCommand{\input mleftright.sty\relax}%
\fi
\def\Test{%
  \RangeCatcodeInvalid{0}{47}%
  \RangeCatcodeInvalid{58}{64}%
  \RangeCatcodeInvalid{91}{96}%
  \RangeCatcodeInvalid{123}{255}%
  \catcode`\@=12 %
  \catcode`\\=0 %
  \catcode`\%=14 %
  \LoadCommand
  \RangeCatcodeCheck{0}{36}{15}%
  \RangeCatcodeCheck{37}{37}{14}%
  \RangeCatcodeCheck{38}{47}{15}%
  \RangeCatcodeCheck{48}{57}{12}%
  \RangeCatcodeCheck{58}{63}{15}%
  \RangeCatcodeCheck{64}{64}{12}%
  \RangeCatcodeCheck{65}{90}{11}%
  \RangeCatcodeCheck{91}{91}{15}%
  \RangeCatcodeCheck{92}{92}{0}%
  \RangeCatcodeCheck{93}{96}{15}%
  \RangeCatcodeCheck{97}{122}{11}%
  \RangeCatcodeCheck{123}{255}{15}%
  \RestoreCatcodes
}
\Test
\csname @@end\endcsname
\end
%    \end{macrocode}
%    \begin{macrocode}
%</test1>
%    \end{macrocode}
%
% \section{Installation}
%
% \subsection{Download}
%
% \paragraph{Package.} This package is available on
% CTAN\footnote{\url{http://ctan.org/pkg/mleftright}}:
% \begin{description}
% \item[\CTAN{macros/latex/contrib/oberdiek/mleftright.dtx}] The source file.
% \item[\CTAN{macros/latex/contrib/oberdiek/mleftright.pdf}] Documentation.
% \end{description}
%
%
% \paragraph{Bundle.} All the packages of the bundle `oberdiek'
% are also available in a TDS compliant ZIP archive. There
% the packages are already unpacked and the documentation files
% are generated. The files and directories obey the TDS standard.
% \begin{description}
% \item[\CTAN{install/macros/latex/contrib/oberdiek.tds.zip}]
% \end{description}
% \emph{TDS} refers to the standard ``A Directory Structure
% for \TeX\ Files'' (\CTAN{tds/tds.pdf}). Directories
% with \xfile{texmf} in their name are usually organized this way.
%
% \subsection{Bundle installation}
%
% \paragraph{Unpacking.} Unpack the \xfile{oberdiek.tds.zip} in the
% TDS tree (also known as \xfile{texmf} tree) of your choice.
% Example (linux):
% \begin{quote}
%   |unzip oberdiek.tds.zip -d ~/texmf|
% \end{quote}
%
% \paragraph{Script installation.}
% Check the directory \xfile{TDS:scripts/oberdiek/} for
% scripts that need further installation steps.
% Package \xpackage{attachfile2} comes with the Perl script
% \xfile{pdfatfi.pl} that should be installed in such a way
% that it can be called as \texttt{pdfatfi}.
% Example (linux):
% \begin{quote}
%   |chmod +x scripts/oberdiek/pdfatfi.pl|\\
%   |cp scripts/oberdiek/pdfatfi.pl /usr/local/bin/|
% \end{quote}
%
% \subsection{Package installation}
%
% \paragraph{Unpacking.} The \xfile{.dtx} file is a self-extracting
% \docstrip\ archive. The files are extracted by running the
% \xfile{.dtx} through \plainTeX:
% \begin{quote}
%   \verb|tex mleftright.dtx|
% \end{quote}
%
% \paragraph{TDS.} Now the different files must be moved into
% the different directories in your installation TDS tree
% (also known as \xfile{texmf} tree):
% \begin{quote}
% \def\t{^^A
% \begin{tabular}{@{}>{\ttfamily}l@{ $\rightarrow$ }>{\ttfamily}l@{}}
%   mleftright.sty & tex/generic/oberdiek/mleftright.sty\\
%   mleftright.pdf & doc/latex/oberdiek/mleftright.pdf\\
%   test/mleftright-test1.tex & doc/latex/oberdiek/test/mleftright-test1.tex\\
%   mleftright.dtx & source/latex/oberdiek/mleftright.dtx\\
% \end{tabular}^^A
% }^^A
% \sbox0{\t}^^A
% \ifdim\wd0>\linewidth
%   \begingroup
%     \advance\linewidth by\leftmargin
%     \advance\linewidth by\rightmargin
%   \edef\x{\endgroup
%     \def\noexpand\lw{\the\linewidth}^^A
%   }\x
%   \def\lwbox{^^A
%     \leavevmode
%     \hbox to \linewidth{^^A
%       \kern-\leftmargin\relax
%       \hss
%       \usebox0
%       \hss
%       \kern-\rightmargin\relax
%     }^^A
%   }^^A
%   \ifdim\wd0>\lw
%     \sbox0{\small\t}^^A
%     \ifdim\wd0>\linewidth
%       \ifdim\wd0>\lw
%         \sbox0{\footnotesize\t}^^A
%         \ifdim\wd0>\linewidth
%           \ifdim\wd0>\lw
%             \sbox0{\scriptsize\t}^^A
%             \ifdim\wd0>\linewidth
%               \ifdim\wd0>\lw
%                 \sbox0{\tiny\t}^^A
%                 \ifdim\wd0>\linewidth
%                   \lwbox
%                 \else
%                   \usebox0
%                 \fi
%               \else
%                 \lwbox
%               \fi
%             \else
%               \usebox0
%             \fi
%           \else
%             \lwbox
%           \fi
%         \else
%           \usebox0
%         \fi
%       \else
%         \lwbox
%       \fi
%     \else
%       \usebox0
%     \fi
%   \else
%     \lwbox
%   \fi
% \else
%   \usebox0
% \fi
% \end{quote}
% If you have a \xfile{docstrip.cfg} that configures and enables \docstrip's
% TDS installing feature, then some files can already be in the right
% place, see the documentation of \docstrip.
%
% \subsection{Refresh file name databases}
%
% If your \TeX~distribution
% (\teTeX, \mikTeX, \dots) relies on file name databases, you must refresh
% these. For example, \teTeX\ users run \verb|texhash| or
% \verb|mktexlsr|.
%
% \subsection{Some details for the interested}
%
% \paragraph{Attached source.}
%
% The PDF documentation on CTAN also includes the
% \xfile{.dtx} source file. It can be extracted by
% AcrobatReader 6 or higher. Another option is \textsf{pdftk},
% e.g. unpack the file into the current directory:
% \begin{quote}
%   \verb|pdftk mleftright.pdf unpack_files output .|
% \end{quote}
%
% \paragraph{Unpacking with \LaTeX.}
% The \xfile{.dtx} chooses its action depending on the format:
% \begin{description}
% \item[\plainTeX:] Run \docstrip\ and extract the files.
% \item[\LaTeX:] Generate the documentation.
% \end{description}
% If you insist on using \LaTeX\ for \docstrip\ (really,
% \docstrip\ does not need \LaTeX), then inform the autodetect routine
% about your intention:
% \begin{quote}
%   \verb|latex \let\install=y\input{mleftright.dtx}|
% \end{quote}
% Do not forget to quote the argument according to the demands
% of your shell.
%
% \paragraph{Generating the documentation.}
% You can use both the \xfile{.dtx} or the \xfile{.drv} to generate
% the documentation. The process can be configured by the
% configuration file \xfile{ltxdoc.cfg}. For instance, put this
% line into this file, if you want to have A4 as paper format:
% \begin{quote}
%   \verb|\PassOptionsToClass{a4paper}{article}|
% \end{quote}
% An example follows how to generate the
% documentation with pdf\LaTeX:
% \begin{quote}
%\begin{verbatim}
%pdflatex mleftright.dtx
%makeindex -s gind.ist mleftright.idx
%pdflatex mleftright.dtx
%makeindex -s gind.ist mleftright.idx
%pdflatex mleftright.dtx
%\end{verbatim}
% \end{quote}
%
% \section{Catalogue}
%
% The following XML file can be used as source for the
% \href{http://mirror.ctan.org/help/Catalogue/catalogue.html}{\TeX\ Catalogue}.
% The elements \texttt{caption} and \texttt{description} are imported
% from the original XML file from the Catalogue.
% The name of the XML file in the Catalogue is \xfile{mleftright.xml}.
%    \begin{macrocode}
%<*catalogue>
<?xml version='1.0' encoding='us-ascii'?>
<!DOCTYPE entry SYSTEM 'catalogue.dtd'>
<entry datestamp='$Date$' modifier='$Author$' id='mleftright'>
  <name>mleftright</name>
  <caption>Variants of delimiters that act as maths open/close.</caption>
  <authorref id='auth:oberdiek'/>
  <copyright owner='Heiko Oberdiek' year='2010'/>
  <license type='lppl1.3'/>
  <version number='1.1'/>
  <description>
    The package defines variants <tt>\mleft</tt> and <tt>\mright</tt>
    of <tt>\left</tt> and <tt>\right</tt>, that make the delimiters
    act as <tt>\mathopen</tt> and <tt>\mathclose</tt>.  These commands
    address spacing difficulties in subformulas.
    <p/>
    The package is part of the <xref refid='oberdiek'>oberdiek</xref> bundle.
  </description>
  <documentation details='Package documentation'
      href='ctan:/macros/latex/contrib/oberdiek/mleftright.pdf'/>
  <ctan file='true' path='/macros/latex/contrib/oberdiek/mleftright.dtx'/>
  <miktex location='oberdiek'/>
  <texlive location='oberdiek'/>
  <install path='/macros/latex/contrib/oberdiek/oberdiek.tds.zip'/>
</entry>
%</catalogue>
%    \end{macrocode}
%
% \section{Acknowledgement}
%
% \begin{description}
% \item[Donald Arsenau:]
% He provided the main trick and the first macros.
% \item[Philipp Stephani:]
% He solved the subscript problem.
% \end{description}
%
% \begin{thebibliography}{9}
% \raggedright
% \bibitem{dave}
%   Dave94705,
%   \textit{spacing after \cs{right}\texttt{)} and before \cs{left}\texttt{)}},
%   newsgroup comp.text.tex,
%   Message-ID: \texttt{\small 5d264909-7c3d-4c9d-9b22-434178b2bf90@g21g2000prn.googlegroups.com},
%   2010-08-12.
%   \newblock
%   {\small\url{http://groups.google.com/group/comp.text.tex/msg/e5b6833da7dc29bf}}
%
% \bibitem{arseneau}
%   Donald Arseneau,
%   \textit{Re: spacing after \cs{right}\texttt) and before \cs{left}\texttt)},
%   newsgroup comp.text.tex,
%   Message-ID: \texttt{\small yfivd6svl8y.fsf@mutant.triumf.ca},
%   2010-08-30.
%   \newblock
%   {\small\url{http://groups.google.com/group/comp.text.tex/msg/e0b2e4386e5d04e4}}
%
% \bibitem{stephani}
%   Philipp Stephani,
%   \textit{Re: spacing after \cs{right}\texttt) and before \cs{left}\texttt)},
%   newsgroup comp.text.tex,
%   Message-ID: \texttt{\small 4c8c8c1e\$0\$6981\$9b4e6d93@newsspool4.arcor-online.net},
%   2010-09-12.
%   \newblock
%   {\small\url{http://groups.google.com/group/comp.text.tex/msg/87ac1f61321de3ef}}
%
% \bibitem{oberdiek}
%   Heiko Oberdiek,
%   \textit{Re: spacing after \cs{right}\texttt) and before \cs{left}\texttt)},
%   newsgroup comp.text.tex,
%   Message-ID: \texttt{\small i6jcc2\$8of\$1@news.eternal-september.org},
%   2010-09-12.
%   \newblock
%   {\small\url{http://groups.google.com/group/comp.text.tex/msg/257aa6119bef878b}}
%
% \end{thebibliography}
%
% \begin{History}
%   \begin{Version}{2010/09/25 v1.0}
%   \item
%     The first version.
%   \end{Version}
%   \begin{Version}{2016/05/16 v1.1}
%   \item
%     Documentation updates.
%   \end{Version}
% \end{History}
%
% \PrintIndex
%
% \Finale
\endinput

%        (quote the arguments according to the demands of your shell)
%
% Documentation:
%    (a) If mleftright.drv is present:
%           latex mleftright.drv
%    (b) Without mleftright.drv:
%           latex mleftright.dtx; ...
%    The class ltxdoc loads the configuration file ltxdoc.cfg
%    if available. Here you can specify further options, e.g.
%    use A4 as paper format:
%       \PassOptionsToClass{a4paper}{article}
%
%    Programm calls to get the documentation (example):
%       pdflatex mleftright.dtx
%       makeindex -s gind.ist mleftright.idx
%       pdflatex mleftright.dtx
%       makeindex -s gind.ist mleftright.idx
%       pdflatex mleftright.dtx
%
% Installation:
%    TDS:tex/generic/oberdiek/mleftright.sty
%    TDS:doc/latex/oberdiek/mleftright.pdf
%    TDS:doc/latex/oberdiek/test/mleftright-test1.tex
%    TDS:source/latex/oberdiek/mleftright.dtx
%
%<*ignore>
\begingroup
  \catcode123=1 %
  \catcode125=2 %
  \def\x{LaTeX2e}%
\expandafter\endgroup
\ifcase 0\ifx\install y1\fi\expandafter
         \ifx\csname processbatchFile\endcsname\relax\else1\fi
         \ifx\fmtname\x\else 1\fi\relax
\else\csname fi\endcsname
%</ignore>
%<*install>
\input docstrip.tex
\Msg{************************************************************************}
\Msg{* Installation}
\Msg{* Package: mleftright 2016/05/16 v1.1 Math left/right delim. as open/close (HO)}
\Msg{************************************************************************}

\keepsilent
\askforoverwritefalse

\let\MetaPrefix\relax
\preamble

This is a generated file.

Project: mleftright
Version: 2016/05/16 v1.1

Copyright (C) 2010 by
   Heiko Oberdiek <heiko.oberdiek at googlemail.com>

This work may be distributed and/or modified under the
conditions of the LaTeX Project Public License, either
version 1.3c of this license or (at your option) any later
version. This version of this license is in
   http://www.latex-project.org/lppl/lppl-1-3c.txt
and the latest version of this license is in
   http://www.latex-project.org/lppl.txt
and version 1.3 or later is part of all distributions of
LaTeX version 2005/12/01 or later.

This work has the LPPL maintenance status "maintained".

This Current Maintainer of this work is Heiko Oberdiek.

The Base Interpreter refers to any `TeX-Format',
because some files are installed in TDS:tex/generic//.

This work consists of the main source file mleftright.dtx
and the derived files
   mleftright.sty, mleftright.pdf, mleftright.ins, mleftright.drv,
   mleftright-test1.tex.

\endpreamble
\let\MetaPrefix\DoubleperCent

\generate{%
  \file{mleftright.ins}{\from{mleftright.dtx}{install}}%
  \file{mleftright.drv}{\from{mleftright.dtx}{driver}}%
  \usedir{tex/generic/oberdiek}%
  \file{mleftright.sty}{\from{mleftright.dtx}{package}}%
  \usedir{doc/latex/oberdiek/test}%
  \file{mleftright-test1.tex}{\from{mleftright.dtx}{test1}}%
  \nopreamble
  \nopostamble
  \usedir{source/latex/oberdiek/catalogue}%
  \file{mleftright.xml}{\from{mleftright.dtx}{catalogue}}%
}

\catcode32=13\relax% active space
\let =\space%
\Msg{************************************************************************}
\Msg{*}
\Msg{* To finish the installation you have to move the following}
\Msg{* file into a directory searched by TeX:}
\Msg{*}
\Msg{*     mleftright.sty}
\Msg{*}
\Msg{* To produce the documentation run the file `mleftright.drv'}
\Msg{* through LaTeX.}
\Msg{*}
\Msg{* Happy TeXing!}
\Msg{*}
\Msg{************************************************************************}

\endbatchfile
%</install>
%<*ignore>
\fi
%</ignore>
%<*driver>
\NeedsTeXFormat{LaTeX2e}
\ProvidesFile{mleftright.drv}%
  [2016/05/16 v1.1 Math left/right delim. as open/close (HO)]%
\documentclass{ltxdoc}
\usepackage{holtxdoc}[2011/11/22]
\usepackage{mleftright}[2016/05/16]
\begin{document}
  \DocInput{mleftright.dtx}%
\end{document}
%</driver>
% \fi
%
%
% \CharacterTable
%  {Upper-case    \A\B\C\D\E\F\G\H\I\J\K\L\M\N\O\P\Q\R\S\T\U\V\W\X\Y\Z
%   Lower-case    \a\b\c\d\e\f\g\h\i\j\k\l\m\n\o\p\q\r\s\t\u\v\w\x\y\z
%   Digits        \0\1\2\3\4\5\6\7\8\9
%   Exclamation   \!     Double quote  \"     Hash (number) \#
%   Dollar        \$     Percent       \%     Ampersand     \&
%   Acute accent  \'     Left paren    \(     Right paren   \)
%   Asterisk      \*     Plus          \+     Comma         \,
%   Minus         \-     Point         \.     Solidus       \/
%   Colon         \:     Semicolon     \;     Less than     \<
%   Equals        \=     Greater than  \>     Question mark \?
%   Commercial at \@     Left bracket  \[     Backslash     \\
%   Right bracket \]     Circumflex    \^     Underscore    \_
%   Grave accent  \`     Left brace    \{     Vertical bar  \|
%   Right brace   \}     Tilde         \~}
%
% \GetFileInfo{mleftright.drv}
%
% \title{The \xpackage{mleftright} package}
% \date{2016/05/16 v1.1}
% \author{Heiko Oberdiek\thanks
% {Please report any issues at https://github.com/ho-tex/oberdiek/issues}\\
% \xemail{heiko.oberdiek at googlemail.com}}
%
% \maketitle
%
% \begin{abstract}
% \TeX\ sets subformulas by \cs{left} and \cs{right} as inner formulas
% with additional surrounding spaces in some situations. This package
% provides \cs{mleft} and \cs{mright} that call \cs{left} and \cs{right},
% but the delimiters will act as normal \cs{mathopen} and \cs{mathclose}
% delimiters without the additional space of an inner formula.
% \end{abstract}
%
% \tableofcontents
%
% \section{Documentation}
%
% The package is a result of a thread in the newsgroup \textsf{comp.text.tex}
% with the subject \textit{spacing after \cs{right}\texttt{)}
% and before \cs{left}\texttt{)}} \cite{dave}.
% The problem: \cs{left} and \cs{right} adjust the size of the
% delimiters automatically. However, \TeX\ treats the whole expression
% as inner formula. In some circumstances \TeX\ adds extra space
% before or after an inner formula.
% Example:
% \begin{quote}
%   \thinmuskip=1.5\thinmuskip
%   \begin{tabular}{@{}l@{\quad$\Rightarrow$\quad}l@{}}
%     |$\sin(x^2), x$|
%     & $\sin(x^2), x$\\
%     |$\sin\left(x^2\right), x$|
%     & $\sin\left(x^2\right), x$\\
%   ^^A  \multicolumn{1}{@{}r@{\quad$\Rightarrow$\quad}}{^^A
%   ^^A    \itshape with exaggerated spacing^^A
%   ^^A  }
%   ^^A  & $\thinmuskip=4\thinmuskip
%   ^^A    \sin\left(x^2\right){,}\mskip.25\thinmuskip x$\\
%     |$\sin\mleft(x^2\mright), x$|
%     & $\sin\mleft(x^2\mright), x$\\
%   \end{tabular}\\*[.5ex]
%   (\cs{mleft} and \cs{mright} are provided by this package.)
% \end{quote}
%
% In the newsgroup Donald Arseneau answered with clever macros \cite{arseneau}:
% \begin{quote}
%\begin{verbatim}
%\newcommand\lft{\mathopen{}\left}
%\newcommand\rgt{\aftergroup\mathclose\aftergroup{\aftergroup}\right}
%\end{verbatim}
% \end{quote}
% However one problem remains, a following subscript or superscript
% is not applied to the right delimiter but the empty
% \cs{mathclose}.
% Thus Philipp Stephani provided an improvement \cite{stephani}:
%\begin{quote}
%\begin{verbatim}
%\mathopen{} \mathclose{\left\| A^2 \right\|}_2
%\end{verbatim}
%\end{quote}
% Heiko Oberdiek converted this into macro form \cite{oberdiek}:
%\begin{quote}
%\begin{verbatim}
%\newcommand\lft{\mathopen{}\mathclose\bgroup\left}
%\newcommand\rgt{\aftergroup\egroup\right}
%\end{verbatim}
%\end{quote}
%
% The package uses longer macro names \cs{mleft} and \cs{mright}
% to avoid name clashes. Also it adds some checks for error conditions.
%
% \subsection{Use}
%
% \begin{declcs}{mleft}\meta{delimL} \dots\unkern\ \cs{mright}\meta{delimR}
% \end{declcs}
% Macros \cs{mleft} and \cs{mright} are used in the same way as
% \cs{left} and \cs{right}. Also \cs{middle} can be used inbetween if
% \eTeX\ is present.
%
% \begin{declcs}{mleftright}
% \end{declcs}
% Macro \cs{mleftright} redefines \cs{left} as \cs{mleft} and
% \cs{right} as \cs{mright}. The redefinition is local to the group.
%
% \begin{declcs}{mleftrightrestore}
% \end{declcs}
% Macro \cs{mleftright} restores \cs{left} and \cs{right} with
% the original meaning if they were previously redefined by
% \cs{mleftright} (also locally).
%
%
% \StopEventually{
% }
%
% \section{Implementation}
%    \begin{macrocode}
%<*package>
%    \end{macrocode}
%    Reload check, especially if the package is not used with \LaTeX.
%    \begin{macrocode}
\begingroup\catcode61\catcode48\catcode32=10\relax%
  \catcode13=5 % ^^M
  \endlinechar=13 %
  \catcode35=6 % #
  \catcode39=12 % '
  \catcode44=12 % ,
  \catcode45=12 % -
  \catcode46=12 % .
  \catcode58=12 % :
  \catcode64=11 % @
  \catcode123=1 % {
  \catcode125=2 % }
  \expandafter\let\expandafter\x\csname ver@mleftright.sty\endcsname
  \ifx\x\relax % plain-TeX, first loading
  \else
    \def\empty{}%
    \ifx\x\empty % LaTeX, first loading,
      % variable is initialized, but \ProvidesPackage not yet seen
    \else
      \expandafter\ifx\csname PackageInfo\endcsname\relax
        \def\x#1#2{%
          \immediate\write-1{Package #1 Info: #2.}%
        }%
      \else
        \def\x#1#2{\PackageInfo{#1}{#2, stopped}}%
      \fi
      \x{mleftright}{The package is already loaded}%
      \aftergroup\endinput
    \fi
  \fi
\endgroup%
%    \end{macrocode}
%    Package identification:
%    \begin{macrocode}
\begingroup\catcode61\catcode48\catcode32=10\relax%
  \catcode13=5 % ^^M
  \endlinechar=13 %
  \catcode35=6 % #
  \catcode39=12 % '
  \catcode40=12 % (
  \catcode41=12 % )
  \catcode44=12 % ,
  \catcode45=12 % -
  \catcode46=12 % .
  \catcode47=12 % /
  \catcode58=12 % :
  \catcode64=11 % @
  \catcode91=12 % [
  \catcode93=12 % ]
  \catcode123=1 % {
  \catcode125=2 % }
  \expandafter\ifx\csname ProvidesPackage\endcsname\relax
    \def\x#1#2#3[#4]{\endgroup
      \immediate\write-1{Package: #3 #4}%
      \xdef#1{#4}%
    }%
  \else
    \def\x#1#2[#3]{\endgroup
      #2[{#3}]%
      \ifx#1\@undefined
        \xdef#1{#3}%
      \fi
      \ifx#1\relax
        \xdef#1{#3}%
      \fi
    }%
  \fi
\expandafter\x\csname ver@mleftright.sty\endcsname
\ProvidesPackage{mleftright}%
  [2016/05/16 v1.1 Math left/right delim. as open/close (HO)]%
%    \end{macrocode}
%
%    \begin{macrocode}
\begingroup\catcode61\catcode48\catcode32=10\relax%
  \catcode13=5 % ^^M
  \endlinechar=13 %
  \catcode123=1 % {
  \catcode125=2 % }
  \catcode64=11 % @
  \def\x{\endgroup
    \expandafter\edef\csname mleftright@AtEnd\endcsname{%
      \endlinechar=\the\endlinechar\relax
      \catcode13=\the\catcode13\relax
      \catcode32=\the\catcode32\relax
      \catcode35=\the\catcode35\relax
      \catcode61=\the\catcode61\relax
      \catcode64=\the\catcode64\relax
      \catcode123=\the\catcode123\relax
      \catcode125=\the\catcode125\relax
    }%
  }%
\x\catcode61\catcode48\catcode32=10\relax%
\catcode13=5 % ^^M
\endlinechar=13 %
\catcode35=6 % #
\catcode64=11 % @
\catcode123=1 % {
\catcode125=2 % }
\def\TMP@EnsureCode#1#2{%
  \edef\mleftright@AtEnd{%
    \mleftright@AtEnd
    \catcode#1=\the\catcode#1\relax
  }%
  \catcode#1=#2\relax
}
\TMP@EnsureCode{38}{4}% &
\TMP@EnsureCode{39}{12}% '
\TMP@EnsureCode{40}{12}% (
\TMP@EnsureCode{41}{12}% )
\TMP@EnsureCode{42}{12}% *
\TMP@EnsureCode{43}{12}% +
\TMP@EnsureCode{44}{12}% ,
\TMP@EnsureCode{45}{12}% -
\TMP@EnsureCode{46}{12}% .
\TMP@EnsureCode{47}{12}% /
\TMP@EnsureCode{60}{12}% <
\TMP@EnsureCode{91}{12}% [
\TMP@EnsureCode{93}{12}% ]
\edef\mleftright@AtEnd{%
  \mleftright@AtEnd
  \escapechar\the\escapechar\relax
  \noexpand\endinput
}
\escapechar=92 %
%    \end{macrocode}
%
%    \begin{macrocode}
\begingroup\expandafter\expandafter\expandafter\endgroup
\expandafter\ifx\csname RequirePackage\endcsname\relax
  \input infwarerr.sty\relax
  \input ltxcmds.sty\relax
\else
  \RequirePackage{infwarerr}[2010/04/08]%
  \RequirePackage{ltxcmds}[2010/04/26]%
\fi
%    \end{macrocode}
%
%    The original commands \cs{left} and \cs{right}
%    are saved and later used in \cs{mleft} and
%    \cs{mright} in order to deal with:
%    \begin{quote}
%\begin{verbatim}
%\let\left\mleft
%\let\right\mright
%\end{verbatim}
%    \end{quote}
%    \begin{macro}{\mleftright@OrgLeft}
%    \begin{macrocode}
\let\mleftright@OrgLeft\left
%    \end{macrocode}
%    \end{macro}
%    \begin{macro}{\mleftright@OrgRight}
%    \begin{macrocode}
\let\mleftright@OrgRight\right
%    \end{macrocode}
%    \end{macro}
%
%    \begin{macro}{\mleftright@Def}
%    Macro \cs{mleftright@Def} defines a macro as robust macro
%    if \eTeX\ or \LaTeX\ is available.
%    \begin{macrocode}
\ltx@IfUndefined{protected}{%
  \ltx@IfUndefined{DeclareRobustCommand}{%
    \def\mleftright@Def{\def}%
  }{%
    \def\mleftright@Def{\DeclareRobustCommand*}%
  }%
}{%
  \def\mleftright@Def{\protected\def}%
}
\edef\mleftright@Def#1{%
  \noexpand\ltx@IfUndefined{%
    \noexpand\expandafter\noexpand\ltx@gobble\noexpand\string#1%
  }{%
    \expandafter\noexpand\mleftright@Def#1%
  }{%
    \noexpand\@PackageError{mleftright}{%
      Command \noexpand\string#1 already defined%
    }\noexpand\@ehd
    \noexpand\ltx@gobble
  }%
}
%    \end{macrocode}
%    \end{macro}
%
%    In case of \eTeX\ the group status after the left symbol
%    is saved and later checked at the beginning of \cs{mright}.
%    \begin{macrocode}
\ltx@IfUndefined{currentgrouplevel}{%
  \catcode38=14 % & = comment
}{%
  \catcode38=9 % & = ignore
}
%    \end{macrocode}
%
%    \begin{macro}{\mleftright@GroupLevel}
%    \begin{macrocode}
& \def\mleftright@GroupLevel{-1}%
%    \end{macrocode}
%    \end{macro}
%
%    \begin{macro}{\mleftright@WrongGroup}
%    \begin{macrocode}
& \def\mleftright@WrongGroup#1(#2){%
&   \ifnum\mleftright@GroupLevel<\ltx@zero
&     \@PackageError{mleftright}{%
&       Missing previous \string\mleft
&     }\@ehc
&   \else
&     \@PackageError{mleftright}{%
&       Unexpected group status for \string\mright%
&       \ifnum\mleftright@GroupLevel=#1 %
&       \else
&         .\MessageBreak
&         Group level is #1, %
&           expected is \mleftright@GroupLevel
&       \fi
&       \ifnum16=#2 %
&       \else
&         .\MessageBreak
&         Group type is #2 (%
&         \ifcase#2 %
&           bottom level%
&           \expandafter\expandafter\expandafter\ltx@gobblefour
&           \expandafter\ltx@gobbletwo
&         \or simple%
&         \or hbox%
&         \or adjusted hbox%
&         \or vbox%
&         \or vtop%
&         \or align%
&         \or no align%
&         \or output%
&         \or math%
&         \or disc%
&         \or insert%
&         \or vcenter%
&         \or math choice%
&         \or semi simple%
&         \or math shift%
&         \or math left%
&         \else
&           unknown%
&         \fi
&         \space group),\MessageBreak
&         expected is 16 (math left group)%
&       \fi
&     }\@ehd
&   \fi
& }%
%    \end{macrocode}
%    \end{macro}
%
%    \begin{macro}{\mleft}
%    \begin{macrocode}
\mleftright@Def\mleft{%
  \mathopen{}\mathclose\bgroup
& \edef\mleftright@GroupLevel{\the\numexpr\the\currentgrouplevel+1}%
  \mleftright@OrgLeft
}
%    \end{macrocode}
%    \end{macro}
%    \begin{macro}{\mright}
%    \begin{macrocode}
\mleftright@Def\mright{%
& \ifnum\mleftright@GroupLevel=\currentgrouplevel
&   \ifnum16=\currentgrouptype
      \aftergroup\egroup
&   \else
&     \expandafter\mleftright@WrongGroup
&     \the\expandafter\currentgrouplevel
&     \expandafter(\the\currentgrouptype)%
&   \fi
& \else
&   \expandafter\mleftright@WrongGroup
&   \the\expandafter\currentgrouplevel
&   \expandafter(\the\currentgrouptype)%
& \fi
  \mleftright@OrgRight
}
%    \end{macrocode}
%    \end{macro}
%
%    \begin{macro}{\mleftright}
%    \begin{macrocode}
\mleftright@Def\mleftright{%
  \let\left\mleft
  \let\right\mright
}
%    \end{macrocode}
%    \end{macro}
%
%    \begin{macro}{\mleftrightrestore}
%    \begin{macrocode}
\mleftright@Def\mleftrightrestore{%
  \ifx\left\mleft
    \let\left\mleftright@OrgLeft
  \fi
  \ifx\right\mright
    \let\right\mleftright@OrgRight
  \fi
}
%    \end{macrocode}
%    \end{macro}
%
%    \begin{macrocode}
\mleftright@AtEnd%
%</package>
%    \end{macrocode}
%
% \section{Test}
%
% \subsection{Catcode checks for loading}
%
%    \begin{macrocode}
%<*test1>
%    \end{macrocode}
%    \begin{macrocode}
\catcode`\{=1 %
\catcode`\}=2 %
\catcode`\#=6 %
\catcode`\@=11 %
\expandafter\ifx\csname count@\endcsname\relax
  \countdef\count@=255 %
\fi
\expandafter\ifx\csname @gobble\endcsname\relax
  \long\def\@gobble#1{}%
\fi
\expandafter\ifx\csname @firstofone\endcsname\relax
  \long\def\@firstofone#1{#1}%
\fi
\expandafter\ifx\csname loop\endcsname\relax
  \expandafter\@firstofone
\else
  \expandafter\@gobble
\fi
{%
  \def\loop#1\repeat{%
    \def\body{#1}%
    \iterate
  }%
  \def\iterate{%
    \body
      \let\next\iterate
    \else
      \let\next\relax
    \fi
    \next
  }%
  \let\repeat=\fi
}%
\def\RestoreCatcodes{}
\count@=0 %
\loop
  \edef\RestoreCatcodes{%
    \RestoreCatcodes
    \catcode\the\count@=\the\catcode\count@\relax
  }%
\ifnum\count@<255 %
  \advance\count@ 1 %
\repeat

\def\RangeCatcodeInvalid#1#2{%
  \count@=#1\relax
  \loop
    \catcode\count@=15 %
  \ifnum\count@<#2\relax
    \advance\count@ 1 %
  \repeat
}
\def\RangeCatcodeCheck#1#2#3{%
  \count@=#1\relax
  \loop
    \ifnum#3=\catcode\count@
    \else
      \errmessage{%
        Character \the\count@\space
        with wrong catcode \the\catcode\count@\space
        instead of \number#3%
      }%
    \fi
  \ifnum\count@<#2\relax
    \advance\count@ 1 %
  \repeat
}
\def\space{ }
\expandafter\ifx\csname LoadCommand\endcsname\relax
  \def\LoadCommand{\input mleftright.sty\relax}%
\fi
\def\Test{%
  \RangeCatcodeInvalid{0}{47}%
  \RangeCatcodeInvalid{58}{64}%
  \RangeCatcodeInvalid{91}{96}%
  \RangeCatcodeInvalid{123}{255}%
  \catcode`\@=12 %
  \catcode`\\=0 %
  \catcode`\%=14 %
  \LoadCommand
  \RangeCatcodeCheck{0}{36}{15}%
  \RangeCatcodeCheck{37}{37}{14}%
  \RangeCatcodeCheck{38}{47}{15}%
  \RangeCatcodeCheck{48}{57}{12}%
  \RangeCatcodeCheck{58}{63}{15}%
  \RangeCatcodeCheck{64}{64}{12}%
  \RangeCatcodeCheck{65}{90}{11}%
  \RangeCatcodeCheck{91}{91}{15}%
  \RangeCatcodeCheck{92}{92}{0}%
  \RangeCatcodeCheck{93}{96}{15}%
  \RangeCatcodeCheck{97}{122}{11}%
  \RangeCatcodeCheck{123}{255}{15}%
  \RestoreCatcodes
}
\Test
\csname @@end\endcsname
\end
%    \end{macrocode}
%    \begin{macrocode}
%</test1>
%    \end{macrocode}
%
% \section{Installation}
%
% \subsection{Download}
%
% \paragraph{Package.} This package is available on
% CTAN\footnote{\url{http://ctan.org/pkg/mleftright}}:
% \begin{description}
% \item[\CTAN{macros/latex/contrib/oberdiek/mleftright.dtx}] The source file.
% \item[\CTAN{macros/latex/contrib/oberdiek/mleftright.pdf}] Documentation.
% \end{description}
%
%
% \paragraph{Bundle.} All the packages of the bundle `oberdiek'
% are also available in a TDS compliant ZIP archive. There
% the packages are already unpacked and the documentation files
% are generated. The files and directories obey the TDS standard.
% \begin{description}
% \item[\CTAN{install/macros/latex/contrib/oberdiek.tds.zip}]
% \end{description}
% \emph{TDS} refers to the standard ``A Directory Structure
% for \TeX\ Files'' (\CTAN{tds/tds.pdf}). Directories
% with \xfile{texmf} in their name are usually organized this way.
%
% \subsection{Bundle installation}
%
% \paragraph{Unpacking.} Unpack the \xfile{oberdiek.tds.zip} in the
% TDS tree (also known as \xfile{texmf} tree) of your choice.
% Example (linux):
% \begin{quote}
%   |unzip oberdiek.tds.zip -d ~/texmf|
% \end{quote}
%
% \paragraph{Script installation.}
% Check the directory \xfile{TDS:scripts/oberdiek/} for
% scripts that need further installation steps.
% Package \xpackage{attachfile2} comes with the Perl script
% \xfile{pdfatfi.pl} that should be installed in such a way
% that it can be called as \texttt{pdfatfi}.
% Example (linux):
% \begin{quote}
%   |chmod +x scripts/oberdiek/pdfatfi.pl|\\
%   |cp scripts/oberdiek/pdfatfi.pl /usr/local/bin/|
% \end{quote}
%
% \subsection{Package installation}
%
% \paragraph{Unpacking.} The \xfile{.dtx} file is a self-extracting
% \docstrip\ archive. The files are extracted by running the
% \xfile{.dtx} through \plainTeX:
% \begin{quote}
%   \verb|tex mleftright.dtx|
% \end{quote}
%
% \paragraph{TDS.} Now the different files must be moved into
% the different directories in your installation TDS tree
% (also known as \xfile{texmf} tree):
% \begin{quote}
% \def\t{^^A
% \begin{tabular}{@{}>{\ttfamily}l@{ $\rightarrow$ }>{\ttfamily}l@{}}
%   mleftright.sty & tex/generic/oberdiek/mleftright.sty\\
%   mleftright.pdf & doc/latex/oberdiek/mleftright.pdf\\
%   test/mleftright-test1.tex & doc/latex/oberdiek/test/mleftright-test1.tex\\
%   mleftright.dtx & source/latex/oberdiek/mleftright.dtx\\
% \end{tabular}^^A
% }^^A
% \sbox0{\t}^^A
% \ifdim\wd0>\linewidth
%   \begingroup
%     \advance\linewidth by\leftmargin
%     \advance\linewidth by\rightmargin
%   \edef\x{\endgroup
%     \def\noexpand\lw{\the\linewidth}^^A
%   }\x
%   \def\lwbox{^^A
%     \leavevmode
%     \hbox to \linewidth{^^A
%       \kern-\leftmargin\relax
%       \hss
%       \usebox0
%       \hss
%       \kern-\rightmargin\relax
%     }^^A
%   }^^A
%   \ifdim\wd0>\lw
%     \sbox0{\small\t}^^A
%     \ifdim\wd0>\linewidth
%       \ifdim\wd0>\lw
%         \sbox0{\footnotesize\t}^^A
%         \ifdim\wd0>\linewidth
%           \ifdim\wd0>\lw
%             \sbox0{\scriptsize\t}^^A
%             \ifdim\wd0>\linewidth
%               \ifdim\wd0>\lw
%                 \sbox0{\tiny\t}^^A
%                 \ifdim\wd0>\linewidth
%                   \lwbox
%                 \else
%                   \usebox0
%                 \fi
%               \else
%                 \lwbox
%               \fi
%             \else
%               \usebox0
%             \fi
%           \else
%             \lwbox
%           \fi
%         \else
%           \usebox0
%         \fi
%       \else
%         \lwbox
%       \fi
%     \else
%       \usebox0
%     \fi
%   \else
%     \lwbox
%   \fi
% \else
%   \usebox0
% \fi
% \end{quote}
% If you have a \xfile{docstrip.cfg} that configures and enables \docstrip's
% TDS installing feature, then some files can already be in the right
% place, see the documentation of \docstrip.
%
% \subsection{Refresh file name databases}
%
% If your \TeX~distribution
% (\teTeX, \mikTeX, \dots) relies on file name databases, you must refresh
% these. For example, \teTeX\ users run \verb|texhash| or
% \verb|mktexlsr|.
%
% \subsection{Some details for the interested}
%
% \paragraph{Attached source.}
%
% The PDF documentation on CTAN also includes the
% \xfile{.dtx} source file. It can be extracted by
% AcrobatReader 6 or higher. Another option is \textsf{pdftk},
% e.g. unpack the file into the current directory:
% \begin{quote}
%   \verb|pdftk mleftright.pdf unpack_files output .|
% \end{quote}
%
% \paragraph{Unpacking with \LaTeX.}
% The \xfile{.dtx} chooses its action depending on the format:
% \begin{description}
% \item[\plainTeX:] Run \docstrip\ and extract the files.
% \item[\LaTeX:] Generate the documentation.
% \end{description}
% If you insist on using \LaTeX\ for \docstrip\ (really,
% \docstrip\ does not need \LaTeX), then inform the autodetect routine
% about your intention:
% \begin{quote}
%   \verb|latex \let\install=y% \iffalse meta-comment
%
% File: mleftright.dtx
% Version: 2016/05/16 v1.1
% Info: Math left/right delim. as open/close
%
% Copyright (C) 2010 by
%    Heiko Oberdiek <heiko.oberdiek at googlemail.com>
%    2016
%    https://github.com/ho-tex/oberdiek/issues
%
% This work may be distributed and/or modified under the
% conditions of the LaTeX Project Public License, either
% version 1.3c of this license or (at your option) any later
% version. This version of this license is in
%    http://www.latex-project.org/lppl/lppl-1-3c.txt
% and the latest version of this license is in
%    http://www.latex-project.org/lppl.txt
% and version 1.3 or later is part of all distributions of
% LaTeX version 2005/12/01 or later.
%
% This work has the LPPL maintenance status "maintained".
%
% This Current Maintainer of this work is Heiko Oberdiek.
%
% The Base Interpreter refers to any `TeX-Format',
% because some files are installed in TDS:tex/generic//.
%
% This work consists of the main source file mleftright.dtx
% and the derived files
%    mleftright.sty, mleftright.pdf, mleftright.ins, mleftright.drv,
%    mleftright-test1.tex.
%
% Distribution:
%    CTAN:macros/latex/contrib/oberdiek/mleftright.dtx
%    CTAN:macros/latex/contrib/oberdiek/mleftright.pdf
%
% Unpacking:
%    (a) If mleftright.ins is present:
%           tex mleftright.ins
%    (b) Without mleftright.ins:
%           tex mleftright.dtx
%    (c) If you insist on using LaTeX
%           latex \let\install=y\input{mleftright.dtx}
%        (quote the arguments according to the demands of your shell)
%
% Documentation:
%    (a) If mleftright.drv is present:
%           latex mleftright.drv
%    (b) Without mleftright.drv:
%           latex mleftright.dtx; ...
%    The class ltxdoc loads the configuration file ltxdoc.cfg
%    if available. Here you can specify further options, e.g.
%    use A4 as paper format:
%       \PassOptionsToClass{a4paper}{article}
%
%    Programm calls to get the documentation (example):
%       pdflatex mleftright.dtx
%       makeindex -s gind.ist mleftright.idx
%       pdflatex mleftright.dtx
%       makeindex -s gind.ist mleftright.idx
%       pdflatex mleftright.dtx
%
% Installation:
%    TDS:tex/generic/oberdiek/mleftright.sty
%    TDS:doc/latex/oberdiek/mleftright.pdf
%    TDS:doc/latex/oberdiek/test/mleftright-test1.tex
%    TDS:source/latex/oberdiek/mleftright.dtx
%
%<*ignore>
\begingroup
  \catcode123=1 %
  \catcode125=2 %
  \def\x{LaTeX2e}%
\expandafter\endgroup
\ifcase 0\ifx\install y1\fi\expandafter
         \ifx\csname processbatchFile\endcsname\relax\else1\fi
         \ifx\fmtname\x\else 1\fi\relax
\else\csname fi\endcsname
%</ignore>
%<*install>
\input docstrip.tex
\Msg{************************************************************************}
\Msg{* Installation}
\Msg{* Package: mleftright 2016/05/16 v1.1 Math left/right delim. as open/close (HO)}
\Msg{************************************************************************}

\keepsilent
\askforoverwritefalse

\let\MetaPrefix\relax
\preamble

This is a generated file.

Project: mleftright
Version: 2016/05/16 v1.1

Copyright (C) 2010 by
   Heiko Oberdiek <heiko.oberdiek at googlemail.com>

This work may be distributed and/or modified under the
conditions of the LaTeX Project Public License, either
version 1.3c of this license or (at your option) any later
version. This version of this license is in
   http://www.latex-project.org/lppl/lppl-1-3c.txt
and the latest version of this license is in
   http://www.latex-project.org/lppl.txt
and version 1.3 or later is part of all distributions of
LaTeX version 2005/12/01 or later.

This work has the LPPL maintenance status "maintained".

This Current Maintainer of this work is Heiko Oberdiek.

The Base Interpreter refers to any `TeX-Format',
because some files are installed in TDS:tex/generic//.

This work consists of the main source file mleftright.dtx
and the derived files
   mleftright.sty, mleftright.pdf, mleftright.ins, mleftright.drv,
   mleftright-test1.tex.

\endpreamble
\let\MetaPrefix\DoubleperCent

\generate{%
  \file{mleftright.ins}{\from{mleftright.dtx}{install}}%
  \file{mleftright.drv}{\from{mleftright.dtx}{driver}}%
  \usedir{tex/generic/oberdiek}%
  \file{mleftright.sty}{\from{mleftright.dtx}{package}}%
  \usedir{doc/latex/oberdiek/test}%
  \file{mleftright-test1.tex}{\from{mleftright.dtx}{test1}}%
  \nopreamble
  \nopostamble
  \usedir{source/latex/oberdiek/catalogue}%
  \file{mleftright.xml}{\from{mleftright.dtx}{catalogue}}%
}

\catcode32=13\relax% active space
\let =\space%
\Msg{************************************************************************}
\Msg{*}
\Msg{* To finish the installation you have to move the following}
\Msg{* file into a directory searched by TeX:}
\Msg{*}
\Msg{*     mleftright.sty}
\Msg{*}
\Msg{* To produce the documentation run the file `mleftright.drv'}
\Msg{* through LaTeX.}
\Msg{*}
\Msg{* Happy TeXing!}
\Msg{*}
\Msg{************************************************************************}

\endbatchfile
%</install>
%<*ignore>
\fi
%</ignore>
%<*driver>
\NeedsTeXFormat{LaTeX2e}
\ProvidesFile{mleftright.drv}%
  [2016/05/16 v1.1 Math left/right delim. as open/close (HO)]%
\documentclass{ltxdoc}
\usepackage{holtxdoc}[2011/11/22]
\usepackage{mleftright}[2016/05/16]
\begin{document}
  \DocInput{mleftright.dtx}%
\end{document}
%</driver>
% \fi
%
%
% \CharacterTable
%  {Upper-case    \A\B\C\D\E\F\G\H\I\J\K\L\M\N\O\P\Q\R\S\T\U\V\W\X\Y\Z
%   Lower-case    \a\b\c\d\e\f\g\h\i\j\k\l\m\n\o\p\q\r\s\t\u\v\w\x\y\z
%   Digits        \0\1\2\3\4\5\6\7\8\9
%   Exclamation   \!     Double quote  \"     Hash (number) \#
%   Dollar        \$     Percent       \%     Ampersand     \&
%   Acute accent  \'     Left paren    \(     Right paren   \)
%   Asterisk      \*     Plus          \+     Comma         \,
%   Minus         \-     Point         \.     Solidus       \/
%   Colon         \:     Semicolon     \;     Less than     \<
%   Equals        \=     Greater than  \>     Question mark \?
%   Commercial at \@     Left bracket  \[     Backslash     \\
%   Right bracket \]     Circumflex    \^     Underscore    \_
%   Grave accent  \`     Left brace    \{     Vertical bar  \|
%   Right brace   \}     Tilde         \~}
%
% \GetFileInfo{mleftright.drv}
%
% \title{The \xpackage{mleftright} package}
% \date{2016/05/16 v1.1}
% \author{Heiko Oberdiek\thanks
% {Please report any issues at https://github.com/ho-tex/oberdiek/issues}\\
% \xemail{heiko.oberdiek at googlemail.com}}
%
% \maketitle
%
% \begin{abstract}
% \TeX\ sets subformulas by \cs{left} and \cs{right} as inner formulas
% with additional surrounding spaces in some situations. This package
% provides \cs{mleft} and \cs{mright} that call \cs{left} and \cs{right},
% but the delimiters will act as normal \cs{mathopen} and \cs{mathclose}
% delimiters without the additional space of an inner formula.
% \end{abstract}
%
% \tableofcontents
%
% \section{Documentation}
%
% The package is a result of a thread in the newsgroup \textsf{comp.text.tex}
% with the subject \textit{spacing after \cs{right}\texttt{)}
% and before \cs{left}\texttt{)}} \cite{dave}.
% The problem: \cs{left} and \cs{right} adjust the size of the
% delimiters automatically. However, \TeX\ treats the whole expression
% as inner formula. In some circumstances \TeX\ adds extra space
% before or after an inner formula.
% Example:
% \begin{quote}
%   \thinmuskip=1.5\thinmuskip
%   \begin{tabular}{@{}l@{\quad$\Rightarrow$\quad}l@{}}
%     |$\sin(x^2), x$|
%     & $\sin(x^2), x$\\
%     |$\sin\left(x^2\right), x$|
%     & $\sin\left(x^2\right), x$\\
%   ^^A  \multicolumn{1}{@{}r@{\quad$\Rightarrow$\quad}}{^^A
%   ^^A    \itshape with exaggerated spacing^^A
%   ^^A  }
%   ^^A  & $\thinmuskip=4\thinmuskip
%   ^^A    \sin\left(x^2\right){,}\mskip.25\thinmuskip x$\\
%     |$\sin\mleft(x^2\mright), x$|
%     & $\sin\mleft(x^2\mright), x$\\
%   \end{tabular}\\*[.5ex]
%   (\cs{mleft} and \cs{mright} are provided by this package.)
% \end{quote}
%
% In the newsgroup Donald Arseneau answered with clever macros \cite{arseneau}:
% \begin{quote}
%\begin{verbatim}
%\newcommand\lft{\mathopen{}\left}
%\newcommand\rgt{\aftergroup\mathclose\aftergroup{\aftergroup}\right}
%\end{verbatim}
% \end{quote}
% However one problem remains, a following subscript or superscript
% is not applied to the right delimiter but the empty
% \cs{mathclose}.
% Thus Philipp Stephani provided an improvement \cite{stephani}:
%\begin{quote}
%\begin{verbatim}
%\mathopen{} \mathclose{\left\| A^2 \right\|}_2
%\end{verbatim}
%\end{quote}
% Heiko Oberdiek converted this into macro form \cite{oberdiek}:
%\begin{quote}
%\begin{verbatim}
%\newcommand\lft{\mathopen{}\mathclose\bgroup\left}
%\newcommand\rgt{\aftergroup\egroup\right}
%\end{verbatim}
%\end{quote}
%
% The package uses longer macro names \cs{mleft} and \cs{mright}
% to avoid name clashes. Also it adds some checks for error conditions.
%
% \subsection{Use}
%
% \begin{declcs}{mleft}\meta{delimL} \dots\unkern\ \cs{mright}\meta{delimR}
% \end{declcs}
% Macros \cs{mleft} and \cs{mright} are used in the same way as
% \cs{left} and \cs{right}. Also \cs{middle} can be used inbetween if
% \eTeX\ is present.
%
% \begin{declcs}{mleftright}
% \end{declcs}
% Macro \cs{mleftright} redefines \cs{left} as \cs{mleft} and
% \cs{right} as \cs{mright}. The redefinition is local to the group.
%
% \begin{declcs}{mleftrightrestore}
% \end{declcs}
% Macro \cs{mleftright} restores \cs{left} and \cs{right} with
% the original meaning if they were previously redefined by
% \cs{mleftright} (also locally).
%
%
% \StopEventually{
% }
%
% \section{Implementation}
%    \begin{macrocode}
%<*package>
%    \end{macrocode}
%    Reload check, especially if the package is not used with \LaTeX.
%    \begin{macrocode}
\begingroup\catcode61\catcode48\catcode32=10\relax%
  \catcode13=5 % ^^M
  \endlinechar=13 %
  \catcode35=6 % #
  \catcode39=12 % '
  \catcode44=12 % ,
  \catcode45=12 % -
  \catcode46=12 % .
  \catcode58=12 % :
  \catcode64=11 % @
  \catcode123=1 % {
  \catcode125=2 % }
  \expandafter\let\expandafter\x\csname ver@mleftright.sty\endcsname
  \ifx\x\relax % plain-TeX, first loading
  \else
    \def\empty{}%
    \ifx\x\empty % LaTeX, first loading,
      % variable is initialized, but \ProvidesPackage not yet seen
    \else
      \expandafter\ifx\csname PackageInfo\endcsname\relax
        \def\x#1#2{%
          \immediate\write-1{Package #1 Info: #2.}%
        }%
      \else
        \def\x#1#2{\PackageInfo{#1}{#2, stopped}}%
      \fi
      \x{mleftright}{The package is already loaded}%
      \aftergroup\endinput
    \fi
  \fi
\endgroup%
%    \end{macrocode}
%    Package identification:
%    \begin{macrocode}
\begingroup\catcode61\catcode48\catcode32=10\relax%
  \catcode13=5 % ^^M
  \endlinechar=13 %
  \catcode35=6 % #
  \catcode39=12 % '
  \catcode40=12 % (
  \catcode41=12 % )
  \catcode44=12 % ,
  \catcode45=12 % -
  \catcode46=12 % .
  \catcode47=12 % /
  \catcode58=12 % :
  \catcode64=11 % @
  \catcode91=12 % [
  \catcode93=12 % ]
  \catcode123=1 % {
  \catcode125=2 % }
  \expandafter\ifx\csname ProvidesPackage\endcsname\relax
    \def\x#1#2#3[#4]{\endgroup
      \immediate\write-1{Package: #3 #4}%
      \xdef#1{#4}%
    }%
  \else
    \def\x#1#2[#3]{\endgroup
      #2[{#3}]%
      \ifx#1\@undefined
        \xdef#1{#3}%
      \fi
      \ifx#1\relax
        \xdef#1{#3}%
      \fi
    }%
  \fi
\expandafter\x\csname ver@mleftright.sty\endcsname
\ProvidesPackage{mleftright}%
  [2016/05/16 v1.1 Math left/right delim. as open/close (HO)]%
%    \end{macrocode}
%
%    \begin{macrocode}
\begingroup\catcode61\catcode48\catcode32=10\relax%
  \catcode13=5 % ^^M
  \endlinechar=13 %
  \catcode123=1 % {
  \catcode125=2 % }
  \catcode64=11 % @
  \def\x{\endgroup
    \expandafter\edef\csname mleftright@AtEnd\endcsname{%
      \endlinechar=\the\endlinechar\relax
      \catcode13=\the\catcode13\relax
      \catcode32=\the\catcode32\relax
      \catcode35=\the\catcode35\relax
      \catcode61=\the\catcode61\relax
      \catcode64=\the\catcode64\relax
      \catcode123=\the\catcode123\relax
      \catcode125=\the\catcode125\relax
    }%
  }%
\x\catcode61\catcode48\catcode32=10\relax%
\catcode13=5 % ^^M
\endlinechar=13 %
\catcode35=6 % #
\catcode64=11 % @
\catcode123=1 % {
\catcode125=2 % }
\def\TMP@EnsureCode#1#2{%
  \edef\mleftright@AtEnd{%
    \mleftright@AtEnd
    \catcode#1=\the\catcode#1\relax
  }%
  \catcode#1=#2\relax
}
\TMP@EnsureCode{38}{4}% &
\TMP@EnsureCode{39}{12}% '
\TMP@EnsureCode{40}{12}% (
\TMP@EnsureCode{41}{12}% )
\TMP@EnsureCode{42}{12}% *
\TMP@EnsureCode{43}{12}% +
\TMP@EnsureCode{44}{12}% ,
\TMP@EnsureCode{45}{12}% -
\TMP@EnsureCode{46}{12}% .
\TMP@EnsureCode{47}{12}% /
\TMP@EnsureCode{60}{12}% <
\TMP@EnsureCode{91}{12}% [
\TMP@EnsureCode{93}{12}% ]
\edef\mleftright@AtEnd{%
  \mleftright@AtEnd
  \escapechar\the\escapechar\relax
  \noexpand\endinput
}
\escapechar=92 %
%    \end{macrocode}
%
%    \begin{macrocode}
\begingroup\expandafter\expandafter\expandafter\endgroup
\expandafter\ifx\csname RequirePackage\endcsname\relax
  \input infwarerr.sty\relax
  \input ltxcmds.sty\relax
\else
  \RequirePackage{infwarerr}[2010/04/08]%
  \RequirePackage{ltxcmds}[2010/04/26]%
\fi
%    \end{macrocode}
%
%    The original commands \cs{left} and \cs{right}
%    are saved and later used in \cs{mleft} and
%    \cs{mright} in order to deal with:
%    \begin{quote}
%\begin{verbatim}
%\let\left\mleft
%\let\right\mright
%\end{verbatim}
%    \end{quote}
%    \begin{macro}{\mleftright@OrgLeft}
%    \begin{macrocode}
\let\mleftright@OrgLeft\left
%    \end{macrocode}
%    \end{macro}
%    \begin{macro}{\mleftright@OrgRight}
%    \begin{macrocode}
\let\mleftright@OrgRight\right
%    \end{macrocode}
%    \end{macro}
%
%    \begin{macro}{\mleftright@Def}
%    Macro \cs{mleftright@Def} defines a macro as robust macro
%    if \eTeX\ or \LaTeX\ is available.
%    \begin{macrocode}
\ltx@IfUndefined{protected}{%
  \ltx@IfUndefined{DeclareRobustCommand}{%
    \def\mleftright@Def{\def}%
  }{%
    \def\mleftright@Def{\DeclareRobustCommand*}%
  }%
}{%
  \def\mleftright@Def{\protected\def}%
}
\edef\mleftright@Def#1{%
  \noexpand\ltx@IfUndefined{%
    \noexpand\expandafter\noexpand\ltx@gobble\noexpand\string#1%
  }{%
    \expandafter\noexpand\mleftright@Def#1%
  }{%
    \noexpand\@PackageError{mleftright}{%
      Command \noexpand\string#1 already defined%
    }\noexpand\@ehd
    \noexpand\ltx@gobble
  }%
}
%    \end{macrocode}
%    \end{macro}
%
%    In case of \eTeX\ the group status after the left symbol
%    is saved and later checked at the beginning of \cs{mright}.
%    \begin{macrocode}
\ltx@IfUndefined{currentgrouplevel}{%
  \catcode38=14 % & = comment
}{%
  \catcode38=9 % & = ignore
}
%    \end{macrocode}
%
%    \begin{macro}{\mleftright@GroupLevel}
%    \begin{macrocode}
& \def\mleftright@GroupLevel{-1}%
%    \end{macrocode}
%    \end{macro}
%
%    \begin{macro}{\mleftright@WrongGroup}
%    \begin{macrocode}
& \def\mleftright@WrongGroup#1(#2){%
&   \ifnum\mleftright@GroupLevel<\ltx@zero
&     \@PackageError{mleftright}{%
&       Missing previous \string\mleft
&     }\@ehc
&   \else
&     \@PackageError{mleftright}{%
&       Unexpected group status for \string\mright%
&       \ifnum\mleftright@GroupLevel=#1 %
&       \else
&         .\MessageBreak
&         Group level is #1, %
&           expected is \mleftright@GroupLevel
&       \fi
&       \ifnum16=#2 %
&       \else
&         .\MessageBreak
&         Group type is #2 (%
&         \ifcase#2 %
&           bottom level%
&           \expandafter\expandafter\expandafter\ltx@gobblefour
&           \expandafter\ltx@gobbletwo
&         \or simple%
&         \or hbox%
&         \or adjusted hbox%
&         \or vbox%
&         \or vtop%
&         \or align%
&         \or no align%
&         \or output%
&         \or math%
&         \or disc%
&         \or insert%
&         \or vcenter%
&         \or math choice%
&         \or semi simple%
&         \or math shift%
&         \or math left%
&         \else
&           unknown%
&         \fi
&         \space group),\MessageBreak
&         expected is 16 (math left group)%
&       \fi
&     }\@ehd
&   \fi
& }%
%    \end{macrocode}
%    \end{macro}
%
%    \begin{macro}{\mleft}
%    \begin{macrocode}
\mleftright@Def\mleft{%
  \mathopen{}\mathclose\bgroup
& \edef\mleftright@GroupLevel{\the\numexpr\the\currentgrouplevel+1}%
  \mleftright@OrgLeft
}
%    \end{macrocode}
%    \end{macro}
%    \begin{macro}{\mright}
%    \begin{macrocode}
\mleftright@Def\mright{%
& \ifnum\mleftright@GroupLevel=\currentgrouplevel
&   \ifnum16=\currentgrouptype
      \aftergroup\egroup
&   \else
&     \expandafter\mleftright@WrongGroup
&     \the\expandafter\currentgrouplevel
&     \expandafter(\the\currentgrouptype)%
&   \fi
& \else
&   \expandafter\mleftright@WrongGroup
&   \the\expandafter\currentgrouplevel
&   \expandafter(\the\currentgrouptype)%
& \fi
  \mleftright@OrgRight
}
%    \end{macrocode}
%    \end{macro}
%
%    \begin{macro}{\mleftright}
%    \begin{macrocode}
\mleftright@Def\mleftright{%
  \let\left\mleft
  \let\right\mright
}
%    \end{macrocode}
%    \end{macro}
%
%    \begin{macro}{\mleftrightrestore}
%    \begin{macrocode}
\mleftright@Def\mleftrightrestore{%
  \ifx\left\mleft
    \let\left\mleftright@OrgLeft
  \fi
  \ifx\right\mright
    \let\right\mleftright@OrgRight
  \fi
}
%    \end{macrocode}
%    \end{macro}
%
%    \begin{macrocode}
\mleftright@AtEnd%
%</package>
%    \end{macrocode}
%
% \section{Test}
%
% \subsection{Catcode checks for loading}
%
%    \begin{macrocode}
%<*test1>
%    \end{macrocode}
%    \begin{macrocode}
\catcode`\{=1 %
\catcode`\}=2 %
\catcode`\#=6 %
\catcode`\@=11 %
\expandafter\ifx\csname count@\endcsname\relax
  \countdef\count@=255 %
\fi
\expandafter\ifx\csname @gobble\endcsname\relax
  \long\def\@gobble#1{}%
\fi
\expandafter\ifx\csname @firstofone\endcsname\relax
  \long\def\@firstofone#1{#1}%
\fi
\expandafter\ifx\csname loop\endcsname\relax
  \expandafter\@firstofone
\else
  \expandafter\@gobble
\fi
{%
  \def\loop#1\repeat{%
    \def\body{#1}%
    \iterate
  }%
  \def\iterate{%
    \body
      \let\next\iterate
    \else
      \let\next\relax
    \fi
    \next
  }%
  \let\repeat=\fi
}%
\def\RestoreCatcodes{}
\count@=0 %
\loop
  \edef\RestoreCatcodes{%
    \RestoreCatcodes
    \catcode\the\count@=\the\catcode\count@\relax
  }%
\ifnum\count@<255 %
  \advance\count@ 1 %
\repeat

\def\RangeCatcodeInvalid#1#2{%
  \count@=#1\relax
  \loop
    \catcode\count@=15 %
  \ifnum\count@<#2\relax
    \advance\count@ 1 %
  \repeat
}
\def\RangeCatcodeCheck#1#2#3{%
  \count@=#1\relax
  \loop
    \ifnum#3=\catcode\count@
    \else
      \errmessage{%
        Character \the\count@\space
        with wrong catcode \the\catcode\count@\space
        instead of \number#3%
      }%
    \fi
  \ifnum\count@<#2\relax
    \advance\count@ 1 %
  \repeat
}
\def\space{ }
\expandafter\ifx\csname LoadCommand\endcsname\relax
  \def\LoadCommand{\input mleftright.sty\relax}%
\fi
\def\Test{%
  \RangeCatcodeInvalid{0}{47}%
  \RangeCatcodeInvalid{58}{64}%
  \RangeCatcodeInvalid{91}{96}%
  \RangeCatcodeInvalid{123}{255}%
  \catcode`\@=12 %
  \catcode`\\=0 %
  \catcode`\%=14 %
  \LoadCommand
  \RangeCatcodeCheck{0}{36}{15}%
  \RangeCatcodeCheck{37}{37}{14}%
  \RangeCatcodeCheck{38}{47}{15}%
  \RangeCatcodeCheck{48}{57}{12}%
  \RangeCatcodeCheck{58}{63}{15}%
  \RangeCatcodeCheck{64}{64}{12}%
  \RangeCatcodeCheck{65}{90}{11}%
  \RangeCatcodeCheck{91}{91}{15}%
  \RangeCatcodeCheck{92}{92}{0}%
  \RangeCatcodeCheck{93}{96}{15}%
  \RangeCatcodeCheck{97}{122}{11}%
  \RangeCatcodeCheck{123}{255}{15}%
  \RestoreCatcodes
}
\Test
\csname @@end\endcsname
\end
%    \end{macrocode}
%    \begin{macrocode}
%</test1>
%    \end{macrocode}
%
% \section{Installation}
%
% \subsection{Download}
%
% \paragraph{Package.} This package is available on
% CTAN\footnote{\url{http://ctan.org/pkg/mleftright}}:
% \begin{description}
% \item[\CTAN{macros/latex/contrib/oberdiek/mleftright.dtx}] The source file.
% \item[\CTAN{macros/latex/contrib/oberdiek/mleftright.pdf}] Documentation.
% \end{description}
%
%
% \paragraph{Bundle.} All the packages of the bundle `oberdiek'
% are also available in a TDS compliant ZIP archive. There
% the packages are already unpacked and the documentation files
% are generated. The files and directories obey the TDS standard.
% \begin{description}
% \item[\CTAN{install/macros/latex/contrib/oberdiek.tds.zip}]
% \end{description}
% \emph{TDS} refers to the standard ``A Directory Structure
% for \TeX\ Files'' (\CTAN{tds/tds.pdf}). Directories
% with \xfile{texmf} in their name are usually organized this way.
%
% \subsection{Bundle installation}
%
% \paragraph{Unpacking.} Unpack the \xfile{oberdiek.tds.zip} in the
% TDS tree (also known as \xfile{texmf} tree) of your choice.
% Example (linux):
% \begin{quote}
%   |unzip oberdiek.tds.zip -d ~/texmf|
% \end{quote}
%
% \paragraph{Script installation.}
% Check the directory \xfile{TDS:scripts/oberdiek/} for
% scripts that need further installation steps.
% Package \xpackage{attachfile2} comes with the Perl script
% \xfile{pdfatfi.pl} that should be installed in such a way
% that it can be called as \texttt{pdfatfi}.
% Example (linux):
% \begin{quote}
%   |chmod +x scripts/oberdiek/pdfatfi.pl|\\
%   |cp scripts/oberdiek/pdfatfi.pl /usr/local/bin/|
% \end{quote}
%
% \subsection{Package installation}
%
% \paragraph{Unpacking.} The \xfile{.dtx} file is a self-extracting
% \docstrip\ archive. The files are extracted by running the
% \xfile{.dtx} through \plainTeX:
% \begin{quote}
%   \verb|tex mleftright.dtx|
% \end{quote}
%
% \paragraph{TDS.} Now the different files must be moved into
% the different directories in your installation TDS tree
% (also known as \xfile{texmf} tree):
% \begin{quote}
% \def\t{^^A
% \begin{tabular}{@{}>{\ttfamily}l@{ $\rightarrow$ }>{\ttfamily}l@{}}
%   mleftright.sty & tex/generic/oberdiek/mleftright.sty\\
%   mleftright.pdf & doc/latex/oberdiek/mleftright.pdf\\
%   test/mleftright-test1.tex & doc/latex/oberdiek/test/mleftright-test1.tex\\
%   mleftright.dtx & source/latex/oberdiek/mleftright.dtx\\
% \end{tabular}^^A
% }^^A
% \sbox0{\t}^^A
% \ifdim\wd0>\linewidth
%   \begingroup
%     \advance\linewidth by\leftmargin
%     \advance\linewidth by\rightmargin
%   \edef\x{\endgroup
%     \def\noexpand\lw{\the\linewidth}^^A
%   }\x
%   \def\lwbox{^^A
%     \leavevmode
%     \hbox to \linewidth{^^A
%       \kern-\leftmargin\relax
%       \hss
%       \usebox0
%       \hss
%       \kern-\rightmargin\relax
%     }^^A
%   }^^A
%   \ifdim\wd0>\lw
%     \sbox0{\small\t}^^A
%     \ifdim\wd0>\linewidth
%       \ifdim\wd0>\lw
%         \sbox0{\footnotesize\t}^^A
%         \ifdim\wd0>\linewidth
%           \ifdim\wd0>\lw
%             \sbox0{\scriptsize\t}^^A
%             \ifdim\wd0>\linewidth
%               \ifdim\wd0>\lw
%                 \sbox0{\tiny\t}^^A
%                 \ifdim\wd0>\linewidth
%                   \lwbox
%                 \else
%                   \usebox0
%                 \fi
%               \else
%                 \lwbox
%               \fi
%             \else
%               \usebox0
%             \fi
%           \else
%             \lwbox
%           \fi
%         \else
%           \usebox0
%         \fi
%       \else
%         \lwbox
%       \fi
%     \else
%       \usebox0
%     \fi
%   \else
%     \lwbox
%   \fi
% \else
%   \usebox0
% \fi
% \end{quote}
% If you have a \xfile{docstrip.cfg} that configures and enables \docstrip's
% TDS installing feature, then some files can already be in the right
% place, see the documentation of \docstrip.
%
% \subsection{Refresh file name databases}
%
% If your \TeX~distribution
% (\teTeX, \mikTeX, \dots) relies on file name databases, you must refresh
% these. For example, \teTeX\ users run \verb|texhash| or
% \verb|mktexlsr|.
%
% \subsection{Some details for the interested}
%
% \paragraph{Attached source.}
%
% The PDF documentation on CTAN also includes the
% \xfile{.dtx} source file. It can be extracted by
% AcrobatReader 6 or higher. Another option is \textsf{pdftk},
% e.g. unpack the file into the current directory:
% \begin{quote}
%   \verb|pdftk mleftright.pdf unpack_files output .|
% \end{quote}
%
% \paragraph{Unpacking with \LaTeX.}
% The \xfile{.dtx} chooses its action depending on the format:
% \begin{description}
% \item[\plainTeX:] Run \docstrip\ and extract the files.
% \item[\LaTeX:] Generate the documentation.
% \end{description}
% If you insist on using \LaTeX\ for \docstrip\ (really,
% \docstrip\ does not need \LaTeX), then inform the autodetect routine
% about your intention:
% \begin{quote}
%   \verb|latex \let\install=y\input{mleftright.dtx}|
% \end{quote}
% Do not forget to quote the argument according to the demands
% of your shell.
%
% \paragraph{Generating the documentation.}
% You can use both the \xfile{.dtx} or the \xfile{.drv} to generate
% the documentation. The process can be configured by the
% configuration file \xfile{ltxdoc.cfg}. For instance, put this
% line into this file, if you want to have A4 as paper format:
% \begin{quote}
%   \verb|\PassOptionsToClass{a4paper}{article}|
% \end{quote}
% An example follows how to generate the
% documentation with pdf\LaTeX:
% \begin{quote}
%\begin{verbatim}
%pdflatex mleftright.dtx
%makeindex -s gind.ist mleftright.idx
%pdflatex mleftright.dtx
%makeindex -s gind.ist mleftright.idx
%pdflatex mleftright.dtx
%\end{verbatim}
% \end{quote}
%
% \section{Catalogue}
%
% The following XML file can be used as source for the
% \href{http://mirror.ctan.org/help/Catalogue/catalogue.html}{\TeX\ Catalogue}.
% The elements \texttt{caption} and \texttt{description} are imported
% from the original XML file from the Catalogue.
% The name of the XML file in the Catalogue is \xfile{mleftright.xml}.
%    \begin{macrocode}
%<*catalogue>
<?xml version='1.0' encoding='us-ascii'?>
<!DOCTYPE entry SYSTEM 'catalogue.dtd'>
<entry datestamp='$Date$' modifier='$Author$' id='mleftright'>
  <name>mleftright</name>
  <caption>Variants of delimiters that act as maths open/close.</caption>
  <authorref id='auth:oberdiek'/>
  <copyright owner='Heiko Oberdiek' year='2010'/>
  <license type='lppl1.3'/>
  <version number='1.1'/>
  <description>
    The package defines variants <tt>\mleft</tt> and <tt>\mright</tt>
    of <tt>\left</tt> and <tt>\right</tt>, that make the delimiters
    act as <tt>\mathopen</tt> and <tt>\mathclose</tt>.  These commands
    address spacing difficulties in subformulas.
    <p/>
    The package is part of the <xref refid='oberdiek'>oberdiek</xref> bundle.
  </description>
  <documentation details='Package documentation'
      href='ctan:/macros/latex/contrib/oberdiek/mleftright.pdf'/>
  <ctan file='true' path='/macros/latex/contrib/oberdiek/mleftright.dtx'/>
  <miktex location='oberdiek'/>
  <texlive location='oberdiek'/>
  <install path='/macros/latex/contrib/oberdiek/oberdiek.tds.zip'/>
</entry>
%</catalogue>
%    \end{macrocode}
%
% \section{Acknowledgement}
%
% \begin{description}
% \item[Donald Arsenau:]
% He provided the main trick and the first macros.
% \item[Philipp Stephani:]
% He solved the subscript problem.
% \end{description}
%
% \begin{thebibliography}{9}
% \raggedright
% \bibitem{dave}
%   Dave94705,
%   \textit{spacing after \cs{right}\texttt{)} and before \cs{left}\texttt{)}},
%   newsgroup comp.text.tex,
%   Message-ID: \texttt{\small 5d264909-7c3d-4c9d-9b22-434178b2bf90@g21g2000prn.googlegroups.com},
%   2010-08-12.
%   \newblock
%   {\small\url{http://groups.google.com/group/comp.text.tex/msg/e5b6833da7dc29bf}}
%
% \bibitem{arseneau}
%   Donald Arseneau,
%   \textit{Re: spacing after \cs{right}\texttt) and before \cs{left}\texttt)},
%   newsgroup comp.text.tex,
%   Message-ID: \texttt{\small yfivd6svl8y.fsf@mutant.triumf.ca},
%   2010-08-30.
%   \newblock
%   {\small\url{http://groups.google.com/group/comp.text.tex/msg/e0b2e4386e5d04e4}}
%
% \bibitem{stephani}
%   Philipp Stephani,
%   \textit{Re: spacing after \cs{right}\texttt) and before \cs{left}\texttt)},
%   newsgroup comp.text.tex,
%   Message-ID: \texttt{\small 4c8c8c1e\$0\$6981\$9b4e6d93@newsspool4.arcor-online.net},
%   2010-09-12.
%   \newblock
%   {\small\url{http://groups.google.com/group/comp.text.tex/msg/87ac1f61321de3ef}}
%
% \bibitem{oberdiek}
%   Heiko Oberdiek,
%   \textit{Re: spacing after \cs{right}\texttt) and before \cs{left}\texttt)},
%   newsgroup comp.text.tex,
%   Message-ID: \texttt{\small i6jcc2\$8of\$1@news.eternal-september.org},
%   2010-09-12.
%   \newblock
%   {\small\url{http://groups.google.com/group/comp.text.tex/msg/257aa6119bef878b}}
%
% \end{thebibliography}
%
% \begin{History}
%   \begin{Version}{2010/09/25 v1.0}
%   \item
%     The first version.
%   \end{Version}
%   \begin{Version}{2016/05/16 v1.1}
%   \item
%     Documentation updates.
%   \end{Version}
% \end{History}
%
% \PrintIndex
%
% \Finale
\endinput
|
% \end{quote}
% Do not forget to quote the argument according to the demands
% of your shell.
%
% \paragraph{Generating the documentation.}
% You can use both the \xfile{.dtx} or the \xfile{.drv} to generate
% the documentation. The process can be configured by the
% configuration file \xfile{ltxdoc.cfg}. For instance, put this
% line into this file, if you want to have A4 as paper format:
% \begin{quote}
%   \verb|\PassOptionsToClass{a4paper}{article}|
% \end{quote}
% An example follows how to generate the
% documentation with pdf\LaTeX:
% \begin{quote}
%\begin{verbatim}
%pdflatex mleftright.dtx
%makeindex -s gind.ist mleftright.idx
%pdflatex mleftright.dtx
%makeindex -s gind.ist mleftright.idx
%pdflatex mleftright.dtx
%\end{verbatim}
% \end{quote}
%
% \section{Catalogue}
%
% The following XML file can be used as source for the
% \href{http://mirror.ctan.org/help/Catalogue/catalogue.html}{\TeX\ Catalogue}.
% The elements \texttt{caption} and \texttt{description} are imported
% from the original XML file from the Catalogue.
% The name of the XML file in the Catalogue is \xfile{mleftright.xml}.
%    \begin{macrocode}
%<*catalogue>
<?xml version='1.0' encoding='us-ascii'?>
<!DOCTYPE entry SYSTEM 'catalogue.dtd'>
<entry datestamp='$Date$' modifier='$Author$' id='mleftright'>
  <name>mleftright</name>
  <caption>Variants of delimiters that act as maths open/close.</caption>
  <authorref id='auth:oberdiek'/>
  <copyright owner='Heiko Oberdiek' year='2010'/>
  <license type='lppl1.3'/>
  <version number='1.1'/>
  <description>
    The package defines variants <tt>\mleft</tt> and <tt>\mright</tt>
    of <tt>\left</tt> and <tt>\right</tt>, that make the delimiters
    act as <tt>\mathopen</tt> and <tt>\mathclose</tt>.  These commands
    address spacing difficulties in subformulas.
    <p/>
    The package is part of the <xref refid='oberdiek'>oberdiek</xref> bundle.
  </description>
  <documentation details='Package documentation'
      href='ctan:/macros/latex/contrib/oberdiek/mleftright.pdf'/>
  <ctan file='true' path='/macros/latex/contrib/oberdiek/mleftright.dtx'/>
  <miktex location='oberdiek'/>
  <texlive location='oberdiek'/>
  <install path='/macros/latex/contrib/oberdiek/oberdiek.tds.zip'/>
</entry>
%</catalogue>
%    \end{macrocode}
%
% \section{Acknowledgement}
%
% \begin{description}
% \item[Donald Arsenau:]
% He provided the main trick and the first macros.
% \item[Philipp Stephani:]
% He solved the subscript problem.
% \end{description}
%
% \begin{thebibliography}{9}
% \raggedright
% \bibitem{dave}
%   Dave94705,
%   \textit{spacing after \cs{right}\texttt{)} and before \cs{left}\texttt{)}},
%   newsgroup comp.text.tex,
%   Message-ID: \texttt{\small 5d264909-7c3d-4c9d-9b22-434178b2bf90@g21g2000prn.googlegroups.com},
%   2010-08-12.
%   \newblock
%   {\small\url{http://groups.google.com/group/comp.text.tex/msg/e5b6833da7dc29bf}}
%
% \bibitem{arseneau}
%   Donald Arseneau,
%   \textit{Re: spacing after \cs{right}\texttt) and before \cs{left}\texttt)},
%   newsgroup comp.text.tex,
%   Message-ID: \texttt{\small yfivd6svl8y.fsf@mutant.triumf.ca},
%   2010-08-30.
%   \newblock
%   {\small\url{http://groups.google.com/group/comp.text.tex/msg/e0b2e4386e5d04e4}}
%
% \bibitem{stephani}
%   Philipp Stephani,
%   \textit{Re: spacing after \cs{right}\texttt) and before \cs{left}\texttt)},
%   newsgroup comp.text.tex,
%   Message-ID: \texttt{\small 4c8c8c1e\$0\$6981\$9b4e6d93@newsspool4.arcor-online.net},
%   2010-09-12.
%   \newblock
%   {\small\url{http://groups.google.com/group/comp.text.tex/msg/87ac1f61321de3ef}}
%
% \bibitem{oberdiek}
%   Heiko Oberdiek,
%   \textit{Re: spacing after \cs{right}\texttt) and before \cs{left}\texttt)},
%   newsgroup comp.text.tex,
%   Message-ID: \texttt{\small i6jcc2\$8of\$1@news.eternal-september.org},
%   2010-09-12.
%   \newblock
%   {\small\url{http://groups.google.com/group/comp.text.tex/msg/257aa6119bef878b}}
%
% \end{thebibliography}
%
% \begin{History}
%   \begin{Version}{2010/09/25 v1.0}
%   \item
%     The first version.
%   \end{Version}
%   \begin{Version}{2016/05/16 v1.1}
%   \item
%     Documentation updates.
%   \end{Version}
% \end{History}
%
% \PrintIndex
%
% \Finale
\endinput
|
% \end{quote}
% Do not forget to quote the argument according to the demands
% of your shell.
%
% \paragraph{Generating the documentation.}
% You can use both the \xfile{.dtx} or the \xfile{.drv} to generate
% the documentation. The process can be configured by the
% configuration file \xfile{ltxdoc.cfg}. For instance, put this
% line into this file, if you want to have A4 as paper format:
% \begin{quote}
%   \verb|\PassOptionsToClass{a4paper}{article}|
% \end{quote}
% An example follows how to generate the
% documentation with pdf\LaTeX:
% \begin{quote}
%\begin{verbatim}
%pdflatex mleftright.dtx
%makeindex -s gind.ist mleftright.idx
%pdflatex mleftright.dtx
%makeindex -s gind.ist mleftright.idx
%pdflatex mleftright.dtx
%\end{verbatim}
% \end{quote}
%
% \section{Catalogue}
%
% The following XML file can be used as source for the
% \href{http://mirror.ctan.org/help/Catalogue/catalogue.html}{\TeX\ Catalogue}.
% The elements \texttt{caption} and \texttt{description} are imported
% from the original XML file from the Catalogue.
% The name of the XML file in the Catalogue is \xfile{mleftright.xml}.
%    \begin{macrocode}
%<*catalogue>
<?xml version='1.0' encoding='us-ascii'?>
<!DOCTYPE entry SYSTEM 'catalogue.dtd'>
<entry datestamp='$Date$' modifier='$Author$' id='mleftright'>
  <name>mleftright</name>
  <caption>Variants of delimiters that act as maths open/close.</caption>
  <authorref id='auth:oberdiek'/>
  <copyright owner='Heiko Oberdiek' year='2010'/>
  <license type='lppl1.3'/>
  <version number='1.1'/>
  <description>
    The package defines variants <tt>\mleft</tt> and <tt>\mright</tt>
    of <tt>\left</tt> and <tt>\right</tt>, that make the delimiters
    act as <tt>\mathopen</tt> and <tt>\mathclose</tt>.  These commands
    address spacing difficulties in subformulas.
    <p/>
    The package is part of the <xref refid='oberdiek'>oberdiek</xref> bundle.
  </description>
  <documentation details='Package documentation'
      href='ctan:/macros/latex/contrib/oberdiek/mleftright.pdf'/>
  <ctan file='true' path='/macros/latex/contrib/oberdiek/mleftright.dtx'/>
  <miktex location='oberdiek'/>
  <texlive location='oberdiek'/>
  <install path='/macros/latex/contrib/oberdiek/oberdiek.tds.zip'/>
</entry>
%</catalogue>
%    \end{macrocode}
%
% \section{Acknowledgement}
%
% \begin{description}
% \item[Donald Arsenau:]
% He provided the main trick and the first macros.
% \item[Philipp Stephani:]
% He solved the subscript problem.
% \end{description}
%
% \begin{thebibliography}{9}
% \raggedright
% \bibitem{dave}
%   Dave94705,
%   \textit{spacing after \cs{right}\texttt{)} and before \cs{left}\texttt{)}},
%   newsgroup comp.text.tex,
%   Message-ID: \texttt{\small 5d264909-7c3d-4c9d-9b22-434178b2bf90@g21g2000prn.googlegroups.com},
%   2010-08-12.
%   \newblock
%   {\small\url{http://groups.google.com/group/comp.text.tex/msg/e5b6833da7dc29bf}}
%
% \bibitem{arseneau}
%   Donald Arseneau,
%   \textit{Re: spacing after \cs{right}\texttt) and before \cs{left}\texttt)},
%   newsgroup comp.text.tex,
%   Message-ID: \texttt{\small yfivd6svl8y.fsf@mutant.triumf.ca},
%   2010-08-30.
%   \newblock
%   {\small\url{http://groups.google.com/group/comp.text.tex/msg/e0b2e4386e5d04e4}}
%
% \bibitem{stephani}
%   Philipp Stephani,
%   \textit{Re: spacing after \cs{right}\texttt) and before \cs{left}\texttt)},
%   newsgroup comp.text.tex,
%   Message-ID: \texttt{\small 4c8c8c1e\$0\$6981\$9b4e6d93@newsspool4.arcor-online.net},
%   2010-09-12.
%   \newblock
%   {\small\url{http://groups.google.com/group/comp.text.tex/msg/87ac1f61321de3ef}}
%
% \bibitem{oberdiek}
%   Heiko Oberdiek,
%   \textit{Re: spacing after \cs{right}\texttt) and before \cs{left}\texttt)},
%   newsgroup comp.text.tex,
%   Message-ID: \texttt{\small i6jcc2\$8of\$1@news.eternal-september.org},
%   2010-09-12.
%   \newblock
%   {\small\url{http://groups.google.com/group/comp.text.tex/msg/257aa6119bef878b}}
%
% \end{thebibliography}
%
% \begin{History}
%   \begin{Version}{2010/09/25 v1.0}
%   \item
%     The first version.
%   \end{Version}
%   \begin{Version}{2016/05/16 v1.1}
%   \item
%     Documentation updates.
%   \end{Version}
% \end{History}
%
% \PrintIndex
%
% \Finale
\endinput

%        (quote the arguments according to the demands of your shell)
%
% Documentation:
%    (a) If mleftright.drv is present:
%           latex mleftright.drv
%    (b) Without mleftright.drv:
%           latex mleftright.dtx; ...
%    The class ltxdoc loads the configuration file ltxdoc.cfg
%    if available. Here you can specify further options, e.g.
%    use A4 as paper format:
%       \PassOptionsToClass{a4paper}{article}
%
%    Programm calls to get the documentation (example):
%       pdflatex mleftright.dtx
%       makeindex -s gind.ist mleftright.idx
%       pdflatex mleftright.dtx
%       makeindex -s gind.ist mleftright.idx
%       pdflatex mleftright.dtx
%
% Installation:
%    TDS:tex/generic/oberdiek/mleftright.sty
%    TDS:doc/latex/oberdiek/mleftright.pdf
%    TDS:doc/latex/oberdiek/test/mleftright-test1.tex
%    TDS:source/latex/oberdiek/mleftright.dtx
%
%<*ignore>
\begingroup
  \catcode123=1 %
  \catcode125=2 %
  \def\x{LaTeX2e}%
\expandafter\endgroup
\ifcase 0\ifx\install y1\fi\expandafter
         \ifx\csname processbatchFile\endcsname\relax\else1\fi
         \ifx\fmtname\x\else 1\fi\relax
\else\csname fi\endcsname
%</ignore>
%<*install>
\input docstrip.tex
\Msg{************************************************************************}
\Msg{* Installation}
\Msg{* Package: mleftright 2016/05/16 v1.1 Math left/right delim. as open/close (HO)}
\Msg{************************************************************************}

\keepsilent
\askforoverwritefalse

\let\MetaPrefix\relax
\preamble

This is a generated file.

Project: mleftright
Version: 2016/05/16 v1.1

Copyright (C) 2010 by
   Heiko Oberdiek <heiko.oberdiek at googlemail.com>

This work may be distributed and/or modified under the
conditions of the LaTeX Project Public License, either
version 1.3c of this license or (at your option) any later
version. This version of this license is in
   http://www.latex-project.org/lppl/lppl-1-3c.txt
and the latest version of this license is in
   http://www.latex-project.org/lppl.txt
and version 1.3 or later is part of all distributions of
LaTeX version 2005/12/01 or later.

This work has the LPPL maintenance status "maintained".

This Current Maintainer of this work is Heiko Oberdiek.

The Base Interpreter refers to any `TeX-Format',
because some files are installed in TDS:tex/generic//.

This work consists of the main source file mleftright.dtx
and the derived files
   mleftright.sty, mleftright.pdf, mleftright.ins, mleftright.drv,
   mleftright-test1.tex.

\endpreamble
\let\MetaPrefix\DoubleperCent

\generate{%
  \file{mleftright.ins}{\from{mleftright.dtx}{install}}%
  \file{mleftright.drv}{\from{mleftright.dtx}{driver}}%
  \usedir{tex/generic/oberdiek}%
  \file{mleftright.sty}{\from{mleftright.dtx}{package}}%
  \usedir{doc/latex/oberdiek/test}%
  \file{mleftright-test1.tex}{\from{mleftright.dtx}{test1}}%
  \nopreamble
  \nopostamble
  \usedir{source/latex/oberdiek/catalogue}%
  \file{mleftright.xml}{\from{mleftright.dtx}{catalogue}}%
}

\catcode32=13\relax% active space
\let =\space%
\Msg{************************************************************************}
\Msg{*}
\Msg{* To finish the installation you have to move the following}
\Msg{* file into a directory searched by TeX:}
\Msg{*}
\Msg{*     mleftright.sty}
\Msg{*}
\Msg{* To produce the documentation run the file `mleftright.drv'}
\Msg{* through LaTeX.}
\Msg{*}
\Msg{* Happy TeXing!}
\Msg{*}
\Msg{************************************************************************}

\endbatchfile
%</install>
%<*ignore>
\fi
%</ignore>
%<*driver>
\NeedsTeXFormat{LaTeX2e}
\ProvidesFile{mleftright.drv}%
  [2016/05/16 v1.1 Math left/right delim. as open/close (HO)]%
\documentclass{ltxdoc}
\usepackage{holtxdoc}[2011/11/22]
\usepackage{mleftright}[2016/05/16]
\begin{document}
  \DocInput{mleftright.dtx}%
\end{document}
%</driver>
% \fi
%
%
% \CharacterTable
%  {Upper-case    \A\B\C\D\E\F\G\H\I\J\K\L\M\N\O\P\Q\R\S\T\U\V\W\X\Y\Z
%   Lower-case    \a\b\c\d\e\f\g\h\i\j\k\l\m\n\o\p\q\r\s\t\u\v\w\x\y\z
%   Digits        \0\1\2\3\4\5\6\7\8\9
%   Exclamation   \!     Double quote  \"     Hash (number) \#
%   Dollar        \$     Percent       \%     Ampersand     \&
%   Acute accent  \'     Left paren    \(     Right paren   \)
%   Asterisk      \*     Plus          \+     Comma         \,
%   Minus         \-     Point         \.     Solidus       \/
%   Colon         \:     Semicolon     \;     Less than     \<
%   Equals        \=     Greater than  \>     Question mark \?
%   Commercial at \@     Left bracket  \[     Backslash     \\
%   Right bracket \]     Circumflex    \^     Underscore    \_
%   Grave accent  \`     Left brace    \{     Vertical bar  \|
%   Right brace   \}     Tilde         \~}
%
% \GetFileInfo{mleftright.drv}
%
% \title{The \xpackage{mleftright} package}
% \date{2016/05/16 v1.1}
% \author{Heiko Oberdiek\thanks
% {Please report any issues at https://github.com/ho-tex/oberdiek/issues}\\
% \xemail{heiko.oberdiek at googlemail.com}}
%
% \maketitle
%
% \begin{abstract}
% \TeX\ sets subformulas by \cs{left} and \cs{right} as inner formulas
% with additional surrounding spaces in some situations. This package
% provides \cs{mleft} and \cs{mright} that call \cs{left} and \cs{right},
% but the delimiters will act as normal \cs{mathopen} and \cs{mathclose}
% delimiters without the additional space of an inner formula.
% \end{abstract}
%
% \tableofcontents
%
% \section{Documentation}
%
% The package is a result of a thread in the newsgroup \textsf{comp.text.tex}
% with the subject \textit{spacing after \cs{right}\texttt{)}
% and before \cs{left}\texttt{)}} \cite{dave}.
% The problem: \cs{left} and \cs{right} adjust the size of the
% delimiters automatically. However, \TeX\ treats the whole expression
% as inner formula. In some circumstances \TeX\ adds extra space
% before or after an inner formula.
% Example:
% \begin{quote}
%   \thinmuskip=1.5\thinmuskip
%   \begin{tabular}{@{}l@{\quad$\Rightarrow$\quad}l@{}}
%     |$\sin(x^2), x$|
%     & $\sin(x^2), x$\\
%     |$\sin\left(x^2\right), x$|
%     & $\sin\left(x^2\right), x$\\
%   ^^A  \multicolumn{1}{@{}r@{\quad$\Rightarrow$\quad}}{^^A
%   ^^A    \itshape with exaggerated spacing^^A
%   ^^A  }
%   ^^A  & $\thinmuskip=4\thinmuskip
%   ^^A    \sin\left(x^2\right){,}\mskip.25\thinmuskip x$\\
%     |$\sin\mleft(x^2\mright), x$|
%     & $\sin\mleft(x^2\mright), x$\\
%   \end{tabular}\\*[.5ex]
%   (\cs{mleft} and \cs{mright} are provided by this package.)
% \end{quote}
%
% In the newsgroup Donald Arseneau answered with clever macros \cite{arseneau}:
% \begin{quote}
%\begin{verbatim}
%\newcommand\lft{\mathopen{}\left}
%\newcommand\rgt{\aftergroup\mathclose\aftergroup{\aftergroup}\right}
%\end{verbatim}
% \end{quote}
% However one problem remains, a following subscript or superscript
% is not applied to the right delimiter but the empty
% \cs{mathclose}.
% Thus Philipp Stephani provided an improvement \cite{stephani}:
%\begin{quote}
%\begin{verbatim}
%\mathopen{} \mathclose{\left\| A^2 \right\|}_2
%\end{verbatim}
%\end{quote}
% Heiko Oberdiek converted this into macro form \cite{oberdiek}:
%\begin{quote}
%\begin{verbatim}
%\newcommand\lft{\mathopen{}\mathclose\bgroup\left}
%\newcommand\rgt{\aftergroup\egroup\right}
%\end{verbatim}
%\end{quote}
%
% The package uses longer macro names \cs{mleft} and \cs{mright}
% to avoid name clashes. Also it adds some checks for error conditions.
%
% \subsection{Use}
%
% \begin{declcs}{mleft}\meta{delimL} \dots\unkern\ \cs{mright}\meta{delimR}
% \end{declcs}
% Macros \cs{mleft} and \cs{mright} are used in the same way as
% \cs{left} and \cs{right}. Also \cs{middle} can be used inbetween if
% \eTeX\ is present.
%
% \begin{declcs}{mleftright}
% \end{declcs}
% Macro \cs{mleftright} redefines \cs{left} as \cs{mleft} and
% \cs{right} as \cs{mright}. The redefinition is local to the group.
%
% \begin{declcs}{mleftrightrestore}
% \end{declcs}
% Macro \cs{mleftright} restores \cs{left} and \cs{right} with
% the original meaning if they were previously redefined by
% \cs{mleftright} (also locally).
%
%
% \StopEventually{
% }
%
% \section{Implementation}
%    \begin{macrocode}
%<*package>
%    \end{macrocode}
%    Reload check, especially if the package is not used with \LaTeX.
%    \begin{macrocode}
\begingroup\catcode61\catcode48\catcode32=10\relax%
  \catcode13=5 % ^^M
  \endlinechar=13 %
  \catcode35=6 % #
  \catcode39=12 % '
  \catcode44=12 % ,
  \catcode45=12 % -
  \catcode46=12 % .
  \catcode58=12 % :
  \catcode64=11 % @
  \catcode123=1 % {
  \catcode125=2 % }
  \expandafter\let\expandafter\x\csname ver@mleftright.sty\endcsname
  \ifx\x\relax % plain-TeX, first loading
  \else
    \def\empty{}%
    \ifx\x\empty % LaTeX, first loading,
      % variable is initialized, but \ProvidesPackage not yet seen
    \else
      \expandafter\ifx\csname PackageInfo\endcsname\relax
        \def\x#1#2{%
          \immediate\write-1{Package #1 Info: #2.}%
        }%
      \else
        \def\x#1#2{\PackageInfo{#1}{#2, stopped}}%
      \fi
      \x{mleftright}{The package is already loaded}%
      \aftergroup\endinput
    \fi
  \fi
\endgroup%
%    \end{macrocode}
%    Package identification:
%    \begin{macrocode}
\begingroup\catcode61\catcode48\catcode32=10\relax%
  \catcode13=5 % ^^M
  \endlinechar=13 %
  \catcode35=6 % #
  \catcode39=12 % '
  \catcode40=12 % (
  \catcode41=12 % )
  \catcode44=12 % ,
  \catcode45=12 % -
  \catcode46=12 % .
  \catcode47=12 % /
  \catcode58=12 % :
  \catcode64=11 % @
  \catcode91=12 % [
  \catcode93=12 % ]
  \catcode123=1 % {
  \catcode125=2 % }
  \expandafter\ifx\csname ProvidesPackage\endcsname\relax
    \def\x#1#2#3[#4]{\endgroup
      \immediate\write-1{Package: #3 #4}%
      \xdef#1{#4}%
    }%
  \else
    \def\x#1#2[#3]{\endgroup
      #2[{#3}]%
      \ifx#1\@undefined
        \xdef#1{#3}%
      \fi
      \ifx#1\relax
        \xdef#1{#3}%
      \fi
    }%
  \fi
\expandafter\x\csname ver@mleftright.sty\endcsname
\ProvidesPackage{mleftright}%
  [2016/05/16 v1.1 Math left/right delim. as open/close (HO)]%
%    \end{macrocode}
%
%    \begin{macrocode}
\begingroup\catcode61\catcode48\catcode32=10\relax%
  \catcode13=5 % ^^M
  \endlinechar=13 %
  \catcode123=1 % {
  \catcode125=2 % }
  \catcode64=11 % @
  \def\x{\endgroup
    \expandafter\edef\csname mleftright@AtEnd\endcsname{%
      \endlinechar=\the\endlinechar\relax
      \catcode13=\the\catcode13\relax
      \catcode32=\the\catcode32\relax
      \catcode35=\the\catcode35\relax
      \catcode61=\the\catcode61\relax
      \catcode64=\the\catcode64\relax
      \catcode123=\the\catcode123\relax
      \catcode125=\the\catcode125\relax
    }%
  }%
\x\catcode61\catcode48\catcode32=10\relax%
\catcode13=5 % ^^M
\endlinechar=13 %
\catcode35=6 % #
\catcode64=11 % @
\catcode123=1 % {
\catcode125=2 % }
\def\TMP@EnsureCode#1#2{%
  \edef\mleftright@AtEnd{%
    \mleftright@AtEnd
    \catcode#1=\the\catcode#1\relax
  }%
  \catcode#1=#2\relax
}
\TMP@EnsureCode{38}{4}% &
\TMP@EnsureCode{39}{12}% '
\TMP@EnsureCode{40}{12}% (
\TMP@EnsureCode{41}{12}% )
\TMP@EnsureCode{42}{12}% *
\TMP@EnsureCode{43}{12}% +
\TMP@EnsureCode{44}{12}% ,
\TMP@EnsureCode{45}{12}% -
\TMP@EnsureCode{46}{12}% .
\TMP@EnsureCode{47}{12}% /
\TMP@EnsureCode{60}{12}% <
\TMP@EnsureCode{91}{12}% [
\TMP@EnsureCode{93}{12}% ]
\edef\mleftright@AtEnd{%
  \mleftright@AtEnd
  \escapechar\the\escapechar\relax
  \noexpand\endinput
}
\escapechar=92 %
%    \end{macrocode}
%
%    \begin{macrocode}
\begingroup\expandafter\expandafter\expandafter\endgroup
\expandafter\ifx\csname RequirePackage\endcsname\relax
  \input infwarerr.sty\relax
  \input ltxcmds.sty\relax
\else
  \RequirePackage{infwarerr}[2010/04/08]%
  \RequirePackage{ltxcmds}[2010/04/26]%
\fi
%    \end{macrocode}
%
%    The original commands \cs{left} and \cs{right}
%    are saved and later used in \cs{mleft} and
%    \cs{mright} in order to deal with:
%    \begin{quote}
%\begin{verbatim}
%\let\left\mleft
%\let\right\mright
%\end{verbatim}
%    \end{quote}
%    \begin{macro}{\mleftright@OrgLeft}
%    \begin{macrocode}
\let\mleftright@OrgLeft\left
%    \end{macrocode}
%    \end{macro}
%    \begin{macro}{\mleftright@OrgRight}
%    \begin{macrocode}
\let\mleftright@OrgRight\right
%    \end{macrocode}
%    \end{macro}
%
%    \begin{macro}{\mleftright@Def}
%    Macro \cs{mleftright@Def} defines a macro as robust macro
%    if \eTeX\ or \LaTeX\ is available.
%    \begin{macrocode}
\ltx@IfUndefined{protected}{%
  \ltx@IfUndefined{DeclareRobustCommand}{%
    \def\mleftright@Def{\def}%
  }{%
    \def\mleftright@Def{\DeclareRobustCommand*}%
  }%
}{%
  \def\mleftright@Def{\protected\def}%
}
\edef\mleftright@Def#1{%
  \noexpand\ltx@IfUndefined{%
    \noexpand\expandafter\noexpand\ltx@gobble\noexpand\string#1%
  }{%
    \expandafter\noexpand\mleftright@Def#1%
  }{%
    \noexpand\@PackageError{mleftright}{%
      Command \noexpand\string#1 already defined%
    }\noexpand\@ehd
    \noexpand\ltx@gobble
  }%
}
%    \end{macrocode}
%    \end{macro}
%
%    In case of \eTeX\ the group status after the left symbol
%    is saved and later checked at the beginning of \cs{mright}.
%    \begin{macrocode}
\ltx@IfUndefined{currentgrouplevel}{%
  \catcode38=14 % & = comment
}{%
  \catcode38=9 % & = ignore
}
%    \end{macrocode}
%
%    \begin{macro}{\mleftright@GroupLevel}
%    \begin{macrocode}
& \def\mleftright@GroupLevel{-1}%
%    \end{macrocode}
%    \end{macro}
%
%    \begin{macro}{\mleftright@WrongGroup}
%    \begin{macrocode}
& \def\mleftright@WrongGroup#1(#2){%
&   \ifnum\mleftright@GroupLevel<\ltx@zero
&     \@PackageError{mleftright}{%
&       Missing previous \string\mleft
&     }\@ehc
&   \else
&     \@PackageError{mleftright}{%
&       Unexpected group status for \string\mright%
&       \ifnum\mleftright@GroupLevel=#1 %
&       \else
&         .\MessageBreak
&         Group level is #1, %
&           expected is \mleftright@GroupLevel
&       \fi
&       \ifnum16=#2 %
&       \else
&         .\MessageBreak
&         Group type is #2 (%
&         \ifcase#2 %
&           bottom level%
&           \expandafter\expandafter\expandafter\ltx@gobblefour
&           \expandafter\ltx@gobbletwo
&         \or simple%
&         \or hbox%
&         \or adjusted hbox%
&         \or vbox%
&         \or vtop%
&         \or align%
&         \or no align%
&         \or output%
&         \or math%
&         \or disc%
&         \or insert%
&         \or vcenter%
&         \or math choice%
&         \or semi simple%
&         \or math shift%
&         \or math left%
&         \else
&           unknown%
&         \fi
&         \space group),\MessageBreak
&         expected is 16 (math left group)%
&       \fi
&     }\@ehd
&   \fi
& }%
%    \end{macrocode}
%    \end{macro}
%
%    \begin{macro}{\mleft}
%    \begin{macrocode}
\mleftright@Def\mleft{%
  \mathopen{}\mathclose\bgroup
& \edef\mleftright@GroupLevel{\the\numexpr\the\currentgrouplevel+1}%
  \mleftright@OrgLeft
}
%    \end{macrocode}
%    \end{macro}
%    \begin{macro}{\mright}
%    \begin{macrocode}
\mleftright@Def\mright{%
& \ifnum\mleftright@GroupLevel=\currentgrouplevel
&   \ifnum16=\currentgrouptype
      \aftergroup\egroup
&   \else
&     \expandafter\mleftright@WrongGroup
&     \the\expandafter\currentgrouplevel
&     \expandafter(\the\currentgrouptype)%
&   \fi
& \else
&   \expandafter\mleftright@WrongGroup
&   \the\expandafter\currentgrouplevel
&   \expandafter(\the\currentgrouptype)%
& \fi
  \mleftright@OrgRight
}
%    \end{macrocode}
%    \end{macro}
%
%    \begin{macro}{\mleftright}
%    \begin{macrocode}
\mleftright@Def\mleftright{%
  \let\left\mleft
  \let\right\mright
}
%    \end{macrocode}
%    \end{macro}
%
%    \begin{macro}{\mleftrightrestore}
%    \begin{macrocode}
\mleftright@Def\mleftrightrestore{%
  \ifx\left\mleft
    \let\left\mleftright@OrgLeft
  \fi
  \ifx\right\mright
    \let\right\mleftright@OrgRight
  \fi
}
%    \end{macrocode}
%    \end{macro}
%
%    \begin{macrocode}
\mleftright@AtEnd%
%</package>
%    \end{macrocode}
%
% \section{Test}
%
% \subsection{Catcode checks for loading}
%
%    \begin{macrocode}
%<*test1>
%    \end{macrocode}
%    \begin{macrocode}
\catcode`\{=1 %
\catcode`\}=2 %
\catcode`\#=6 %
\catcode`\@=11 %
\expandafter\ifx\csname count@\endcsname\relax
  \countdef\count@=255 %
\fi
\expandafter\ifx\csname @gobble\endcsname\relax
  \long\def\@gobble#1{}%
\fi
\expandafter\ifx\csname @firstofone\endcsname\relax
  \long\def\@firstofone#1{#1}%
\fi
\expandafter\ifx\csname loop\endcsname\relax
  \expandafter\@firstofone
\else
  \expandafter\@gobble
\fi
{%
  \def\loop#1\repeat{%
    \def\body{#1}%
    \iterate
  }%
  \def\iterate{%
    \body
      \let\next\iterate
    \else
      \let\next\relax
    \fi
    \next
  }%
  \let\repeat=\fi
}%
\def\RestoreCatcodes{}
\count@=0 %
\loop
  \edef\RestoreCatcodes{%
    \RestoreCatcodes
    \catcode\the\count@=\the\catcode\count@\relax
  }%
\ifnum\count@<255 %
  \advance\count@ 1 %
\repeat

\def\RangeCatcodeInvalid#1#2{%
  \count@=#1\relax
  \loop
    \catcode\count@=15 %
  \ifnum\count@<#2\relax
    \advance\count@ 1 %
  \repeat
}
\def\RangeCatcodeCheck#1#2#3{%
  \count@=#1\relax
  \loop
    \ifnum#3=\catcode\count@
    \else
      \errmessage{%
        Character \the\count@\space
        with wrong catcode \the\catcode\count@\space
        instead of \number#3%
      }%
    \fi
  \ifnum\count@<#2\relax
    \advance\count@ 1 %
  \repeat
}
\def\space{ }
\expandafter\ifx\csname LoadCommand\endcsname\relax
  \def\LoadCommand{\input mleftright.sty\relax}%
\fi
\def\Test{%
  \RangeCatcodeInvalid{0}{47}%
  \RangeCatcodeInvalid{58}{64}%
  \RangeCatcodeInvalid{91}{96}%
  \RangeCatcodeInvalid{123}{255}%
  \catcode`\@=12 %
  \catcode`\\=0 %
  \catcode`\%=14 %
  \LoadCommand
  \RangeCatcodeCheck{0}{36}{15}%
  \RangeCatcodeCheck{37}{37}{14}%
  \RangeCatcodeCheck{38}{47}{15}%
  \RangeCatcodeCheck{48}{57}{12}%
  \RangeCatcodeCheck{58}{63}{15}%
  \RangeCatcodeCheck{64}{64}{12}%
  \RangeCatcodeCheck{65}{90}{11}%
  \RangeCatcodeCheck{91}{91}{15}%
  \RangeCatcodeCheck{92}{92}{0}%
  \RangeCatcodeCheck{93}{96}{15}%
  \RangeCatcodeCheck{97}{122}{11}%
  \RangeCatcodeCheck{123}{255}{15}%
  \RestoreCatcodes
}
\Test
\csname @@end\endcsname
\end
%    \end{macrocode}
%    \begin{macrocode}
%</test1>
%    \end{macrocode}
%
% \section{Installation}
%
% \subsection{Download}
%
% \paragraph{Package.} This package is available on
% CTAN\footnote{\url{http://ctan.org/pkg/mleftright}}:
% \begin{description}
% \item[\CTAN{macros/latex/contrib/oberdiek/mleftright.dtx}] The source file.
% \item[\CTAN{macros/latex/contrib/oberdiek/mleftright.pdf}] Documentation.
% \end{description}
%
%
% \paragraph{Bundle.} All the packages of the bundle `oberdiek'
% are also available in a TDS compliant ZIP archive. There
% the packages are already unpacked and the documentation files
% are generated. The files and directories obey the TDS standard.
% \begin{description}
% \item[\CTAN{install/macros/latex/contrib/oberdiek.tds.zip}]
% \end{description}
% \emph{TDS} refers to the standard ``A Directory Structure
% for \TeX\ Files'' (\CTAN{tds/tds.pdf}). Directories
% with \xfile{texmf} in their name are usually organized this way.
%
% \subsection{Bundle installation}
%
% \paragraph{Unpacking.} Unpack the \xfile{oberdiek.tds.zip} in the
% TDS tree (also known as \xfile{texmf} tree) of your choice.
% Example (linux):
% \begin{quote}
%   |unzip oberdiek.tds.zip -d ~/texmf|
% \end{quote}
%
% \paragraph{Script installation.}
% Check the directory \xfile{TDS:scripts/oberdiek/} for
% scripts that need further installation steps.
% Package \xpackage{attachfile2} comes with the Perl script
% \xfile{pdfatfi.pl} that should be installed in such a way
% that it can be called as \texttt{pdfatfi}.
% Example (linux):
% \begin{quote}
%   |chmod +x scripts/oberdiek/pdfatfi.pl|\\
%   |cp scripts/oberdiek/pdfatfi.pl /usr/local/bin/|
% \end{quote}
%
% \subsection{Package installation}
%
% \paragraph{Unpacking.} The \xfile{.dtx} file is a self-extracting
% \docstrip\ archive. The files are extracted by running the
% \xfile{.dtx} through \plainTeX:
% \begin{quote}
%   \verb|tex mleftright.dtx|
% \end{quote}
%
% \paragraph{TDS.} Now the different files must be moved into
% the different directories in your installation TDS tree
% (also known as \xfile{texmf} tree):
% \begin{quote}
% \def\t{^^A
% \begin{tabular}{@{}>{\ttfamily}l@{ $\rightarrow$ }>{\ttfamily}l@{}}
%   mleftright.sty & tex/generic/oberdiek/mleftright.sty\\
%   mleftright.pdf & doc/latex/oberdiek/mleftright.pdf\\
%   test/mleftright-test1.tex & doc/latex/oberdiek/test/mleftright-test1.tex\\
%   mleftright.dtx & source/latex/oberdiek/mleftright.dtx\\
% \end{tabular}^^A
% }^^A
% \sbox0{\t}^^A
% \ifdim\wd0>\linewidth
%   \begingroup
%     \advance\linewidth by\leftmargin
%     \advance\linewidth by\rightmargin
%   \edef\x{\endgroup
%     \def\noexpand\lw{\the\linewidth}^^A
%   }\x
%   \def\lwbox{^^A
%     \leavevmode
%     \hbox to \linewidth{^^A
%       \kern-\leftmargin\relax
%       \hss
%       \usebox0
%       \hss
%       \kern-\rightmargin\relax
%     }^^A
%   }^^A
%   \ifdim\wd0>\lw
%     \sbox0{\small\t}^^A
%     \ifdim\wd0>\linewidth
%       \ifdim\wd0>\lw
%         \sbox0{\footnotesize\t}^^A
%         \ifdim\wd0>\linewidth
%           \ifdim\wd0>\lw
%             \sbox0{\scriptsize\t}^^A
%             \ifdim\wd0>\linewidth
%               \ifdim\wd0>\lw
%                 \sbox0{\tiny\t}^^A
%                 \ifdim\wd0>\linewidth
%                   \lwbox
%                 \else
%                   \usebox0
%                 \fi
%               \else
%                 \lwbox
%               \fi
%             \else
%               \usebox0
%             \fi
%           \else
%             \lwbox
%           \fi
%         \else
%           \usebox0
%         \fi
%       \else
%         \lwbox
%       \fi
%     \else
%       \usebox0
%     \fi
%   \else
%     \lwbox
%   \fi
% \else
%   \usebox0
% \fi
% \end{quote}
% If you have a \xfile{docstrip.cfg} that configures and enables \docstrip's
% TDS installing feature, then some files can already be in the right
% place, see the documentation of \docstrip.
%
% \subsection{Refresh file name databases}
%
% If your \TeX~distribution
% (\teTeX, \mikTeX, \dots) relies on file name databases, you must refresh
% these. For example, \teTeX\ users run \verb|texhash| or
% \verb|mktexlsr|.
%
% \subsection{Some details for the interested}
%
% \paragraph{Attached source.}
%
% The PDF documentation on CTAN also includes the
% \xfile{.dtx} source file. It can be extracted by
% AcrobatReader 6 or higher. Another option is \textsf{pdftk},
% e.g. unpack the file into the current directory:
% \begin{quote}
%   \verb|pdftk mleftright.pdf unpack_files output .|
% \end{quote}
%
% \paragraph{Unpacking with \LaTeX.}
% The \xfile{.dtx} chooses its action depending on the format:
% \begin{description}
% \item[\plainTeX:] Run \docstrip\ and extract the files.
% \item[\LaTeX:] Generate the documentation.
% \end{description}
% If you insist on using \LaTeX\ for \docstrip\ (really,
% \docstrip\ does not need \LaTeX), then inform the autodetect routine
% about your intention:
% \begin{quote}
%   \verb|latex \let\install=y% \iffalse meta-comment
%
% File: mleftright.dtx
% Version: 2016/05/16 v1.1
% Info: Math left/right delim. as open/close
%
% Copyright (C) 2010 by
%    Heiko Oberdiek <heiko.oberdiek at googlemail.com>
%    2016
%    https://github.com/ho-tex/oberdiek/issues
%
% This work may be distributed and/or modified under the
% conditions of the LaTeX Project Public License, either
% version 1.3c of this license or (at your option) any later
% version. This version of this license is in
%    http://www.latex-project.org/lppl/lppl-1-3c.txt
% and the latest version of this license is in
%    http://www.latex-project.org/lppl.txt
% and version 1.3 or later is part of all distributions of
% LaTeX version 2005/12/01 or later.
%
% This work has the LPPL maintenance status "maintained".
%
% This Current Maintainer of this work is Heiko Oberdiek.
%
% The Base Interpreter refers to any `TeX-Format',
% because some files are installed in TDS:tex/generic//.
%
% This work consists of the main source file mleftright.dtx
% and the derived files
%    mleftright.sty, mleftright.pdf, mleftright.ins, mleftright.drv,
%    mleftright-test1.tex.
%
% Distribution:
%    CTAN:macros/latex/contrib/oberdiek/mleftright.dtx
%    CTAN:macros/latex/contrib/oberdiek/mleftright.pdf
%
% Unpacking:
%    (a) If mleftright.ins is present:
%           tex mleftright.ins
%    (b) Without mleftright.ins:
%           tex mleftright.dtx
%    (c) If you insist on using LaTeX
%           latex \let\install=y% \iffalse meta-comment
%
% File: mleftright.dtx
% Version: 2016/05/16 v1.1
% Info: Math left/right delim. as open/close
%
% Copyright (C) 2010 by
%    Heiko Oberdiek <heiko.oberdiek at googlemail.com>
%    2016
%    https://github.com/ho-tex/oberdiek/issues
%
% This work may be distributed and/or modified under the
% conditions of the LaTeX Project Public License, either
% version 1.3c of this license or (at your option) any later
% version. This version of this license is in
%    http://www.latex-project.org/lppl/lppl-1-3c.txt
% and the latest version of this license is in
%    http://www.latex-project.org/lppl.txt
% and version 1.3 or later is part of all distributions of
% LaTeX version 2005/12/01 or later.
%
% This work has the LPPL maintenance status "maintained".
%
% This Current Maintainer of this work is Heiko Oberdiek.
%
% The Base Interpreter refers to any `TeX-Format',
% because some files are installed in TDS:tex/generic//.
%
% This work consists of the main source file mleftright.dtx
% and the derived files
%    mleftright.sty, mleftright.pdf, mleftright.ins, mleftright.drv,
%    mleftright-test1.tex.
%
% Distribution:
%    CTAN:macros/latex/contrib/oberdiek/mleftright.dtx
%    CTAN:macros/latex/contrib/oberdiek/mleftright.pdf
%
% Unpacking:
%    (a) If mleftright.ins is present:
%           tex mleftright.ins
%    (b) Without mleftright.ins:
%           tex mleftright.dtx
%    (c) If you insist on using LaTeX
%           latex \let\install=y% \iffalse meta-comment
%
% File: mleftright.dtx
% Version: 2016/05/16 v1.1
% Info: Math left/right delim. as open/close
%
% Copyright (C) 2010 by
%    Heiko Oberdiek <heiko.oberdiek at googlemail.com>
%    2016
%    https://github.com/ho-tex/oberdiek/issues
%
% This work may be distributed and/or modified under the
% conditions of the LaTeX Project Public License, either
% version 1.3c of this license or (at your option) any later
% version. This version of this license is in
%    http://www.latex-project.org/lppl/lppl-1-3c.txt
% and the latest version of this license is in
%    http://www.latex-project.org/lppl.txt
% and version 1.3 or later is part of all distributions of
% LaTeX version 2005/12/01 or later.
%
% This work has the LPPL maintenance status "maintained".
%
% This Current Maintainer of this work is Heiko Oberdiek.
%
% The Base Interpreter refers to any `TeX-Format',
% because some files are installed in TDS:tex/generic//.
%
% This work consists of the main source file mleftright.dtx
% and the derived files
%    mleftright.sty, mleftright.pdf, mleftright.ins, mleftright.drv,
%    mleftright-test1.tex.
%
% Distribution:
%    CTAN:macros/latex/contrib/oberdiek/mleftright.dtx
%    CTAN:macros/latex/contrib/oberdiek/mleftright.pdf
%
% Unpacking:
%    (a) If mleftright.ins is present:
%           tex mleftright.ins
%    (b) Without mleftright.ins:
%           tex mleftright.dtx
%    (c) If you insist on using LaTeX
%           latex \let\install=y\input{mleftright.dtx}
%        (quote the arguments according to the demands of your shell)
%
% Documentation:
%    (a) If mleftright.drv is present:
%           latex mleftright.drv
%    (b) Without mleftright.drv:
%           latex mleftright.dtx; ...
%    The class ltxdoc loads the configuration file ltxdoc.cfg
%    if available. Here you can specify further options, e.g.
%    use A4 as paper format:
%       \PassOptionsToClass{a4paper}{article}
%
%    Programm calls to get the documentation (example):
%       pdflatex mleftright.dtx
%       makeindex -s gind.ist mleftright.idx
%       pdflatex mleftright.dtx
%       makeindex -s gind.ist mleftright.idx
%       pdflatex mleftright.dtx
%
% Installation:
%    TDS:tex/generic/oberdiek/mleftright.sty
%    TDS:doc/latex/oberdiek/mleftright.pdf
%    TDS:doc/latex/oberdiek/test/mleftright-test1.tex
%    TDS:source/latex/oberdiek/mleftright.dtx
%
%<*ignore>
\begingroup
  \catcode123=1 %
  \catcode125=2 %
  \def\x{LaTeX2e}%
\expandafter\endgroup
\ifcase 0\ifx\install y1\fi\expandafter
         \ifx\csname processbatchFile\endcsname\relax\else1\fi
         \ifx\fmtname\x\else 1\fi\relax
\else\csname fi\endcsname
%</ignore>
%<*install>
\input docstrip.tex
\Msg{************************************************************************}
\Msg{* Installation}
\Msg{* Package: mleftright 2016/05/16 v1.1 Math left/right delim. as open/close (HO)}
\Msg{************************************************************************}

\keepsilent
\askforoverwritefalse

\let\MetaPrefix\relax
\preamble

This is a generated file.

Project: mleftright
Version: 2016/05/16 v1.1

Copyright (C) 2010 by
   Heiko Oberdiek <heiko.oberdiek at googlemail.com>

This work may be distributed and/or modified under the
conditions of the LaTeX Project Public License, either
version 1.3c of this license or (at your option) any later
version. This version of this license is in
   http://www.latex-project.org/lppl/lppl-1-3c.txt
and the latest version of this license is in
   http://www.latex-project.org/lppl.txt
and version 1.3 or later is part of all distributions of
LaTeX version 2005/12/01 or later.

This work has the LPPL maintenance status "maintained".

This Current Maintainer of this work is Heiko Oberdiek.

The Base Interpreter refers to any `TeX-Format',
because some files are installed in TDS:tex/generic//.

This work consists of the main source file mleftright.dtx
and the derived files
   mleftright.sty, mleftright.pdf, mleftright.ins, mleftright.drv,
   mleftright-test1.tex.

\endpreamble
\let\MetaPrefix\DoubleperCent

\generate{%
  \file{mleftright.ins}{\from{mleftright.dtx}{install}}%
  \file{mleftright.drv}{\from{mleftright.dtx}{driver}}%
  \usedir{tex/generic/oberdiek}%
  \file{mleftright.sty}{\from{mleftright.dtx}{package}}%
  \usedir{doc/latex/oberdiek/test}%
  \file{mleftright-test1.tex}{\from{mleftright.dtx}{test1}}%
  \nopreamble
  \nopostamble
  \usedir{source/latex/oberdiek/catalogue}%
  \file{mleftright.xml}{\from{mleftright.dtx}{catalogue}}%
}

\catcode32=13\relax% active space
\let =\space%
\Msg{************************************************************************}
\Msg{*}
\Msg{* To finish the installation you have to move the following}
\Msg{* file into a directory searched by TeX:}
\Msg{*}
\Msg{*     mleftright.sty}
\Msg{*}
\Msg{* To produce the documentation run the file `mleftright.drv'}
\Msg{* through LaTeX.}
\Msg{*}
\Msg{* Happy TeXing!}
\Msg{*}
\Msg{************************************************************************}

\endbatchfile
%</install>
%<*ignore>
\fi
%</ignore>
%<*driver>
\NeedsTeXFormat{LaTeX2e}
\ProvidesFile{mleftright.drv}%
  [2016/05/16 v1.1 Math left/right delim. as open/close (HO)]%
\documentclass{ltxdoc}
\usepackage{holtxdoc}[2011/11/22]
\usepackage{mleftright}[2016/05/16]
\begin{document}
  \DocInput{mleftright.dtx}%
\end{document}
%</driver>
% \fi
%
%
% \CharacterTable
%  {Upper-case    \A\B\C\D\E\F\G\H\I\J\K\L\M\N\O\P\Q\R\S\T\U\V\W\X\Y\Z
%   Lower-case    \a\b\c\d\e\f\g\h\i\j\k\l\m\n\o\p\q\r\s\t\u\v\w\x\y\z
%   Digits        \0\1\2\3\4\5\6\7\8\9
%   Exclamation   \!     Double quote  \"     Hash (number) \#
%   Dollar        \$     Percent       \%     Ampersand     \&
%   Acute accent  \'     Left paren    \(     Right paren   \)
%   Asterisk      \*     Plus          \+     Comma         \,
%   Minus         \-     Point         \.     Solidus       \/
%   Colon         \:     Semicolon     \;     Less than     \<
%   Equals        \=     Greater than  \>     Question mark \?
%   Commercial at \@     Left bracket  \[     Backslash     \\
%   Right bracket \]     Circumflex    \^     Underscore    \_
%   Grave accent  \`     Left brace    \{     Vertical bar  \|
%   Right brace   \}     Tilde         \~}
%
% \GetFileInfo{mleftright.drv}
%
% \title{The \xpackage{mleftright} package}
% \date{2016/05/16 v1.1}
% \author{Heiko Oberdiek\thanks
% {Please report any issues at https://github.com/ho-tex/oberdiek/issues}\\
% \xemail{heiko.oberdiek at googlemail.com}}
%
% \maketitle
%
% \begin{abstract}
% \TeX\ sets subformulas by \cs{left} and \cs{right} as inner formulas
% with additional surrounding spaces in some situations. This package
% provides \cs{mleft} and \cs{mright} that call \cs{left} and \cs{right},
% but the delimiters will act as normal \cs{mathopen} and \cs{mathclose}
% delimiters without the additional space of an inner formula.
% \end{abstract}
%
% \tableofcontents
%
% \section{Documentation}
%
% The package is a result of a thread in the newsgroup \textsf{comp.text.tex}
% with the subject \textit{spacing after \cs{right}\texttt{)}
% and before \cs{left}\texttt{)}} \cite{dave}.
% The problem: \cs{left} and \cs{right} adjust the size of the
% delimiters automatically. However, \TeX\ treats the whole expression
% as inner formula. In some circumstances \TeX\ adds extra space
% before or after an inner formula.
% Example:
% \begin{quote}
%   \thinmuskip=1.5\thinmuskip
%   \begin{tabular}{@{}l@{\quad$\Rightarrow$\quad}l@{}}
%     |$\sin(x^2), x$|
%     & $\sin(x^2), x$\\
%     |$\sin\left(x^2\right), x$|
%     & $\sin\left(x^2\right), x$\\
%   ^^A  \multicolumn{1}{@{}r@{\quad$\Rightarrow$\quad}}{^^A
%   ^^A    \itshape with exaggerated spacing^^A
%   ^^A  }
%   ^^A  & $\thinmuskip=4\thinmuskip
%   ^^A    \sin\left(x^2\right){,}\mskip.25\thinmuskip x$\\
%     |$\sin\mleft(x^2\mright), x$|
%     & $\sin\mleft(x^2\mright), x$\\
%   \end{tabular}\\*[.5ex]
%   (\cs{mleft} and \cs{mright} are provided by this package.)
% \end{quote}
%
% In the newsgroup Donald Arseneau answered with clever macros \cite{arseneau}:
% \begin{quote}
%\begin{verbatim}
%\newcommand\lft{\mathopen{}\left}
%\newcommand\rgt{\aftergroup\mathclose\aftergroup{\aftergroup}\right}
%\end{verbatim}
% \end{quote}
% However one problem remains, a following subscript or superscript
% is not applied to the right delimiter but the empty
% \cs{mathclose}.
% Thus Philipp Stephani provided an improvement \cite{stephani}:
%\begin{quote}
%\begin{verbatim}
%\mathopen{} \mathclose{\left\| A^2 \right\|}_2
%\end{verbatim}
%\end{quote}
% Heiko Oberdiek converted this into macro form \cite{oberdiek}:
%\begin{quote}
%\begin{verbatim}
%\newcommand\lft{\mathopen{}\mathclose\bgroup\left}
%\newcommand\rgt{\aftergroup\egroup\right}
%\end{verbatim}
%\end{quote}
%
% The package uses longer macro names \cs{mleft} and \cs{mright}
% to avoid name clashes. Also it adds some checks for error conditions.
%
% \subsection{Use}
%
% \begin{declcs}{mleft}\meta{delimL} \dots\unkern\ \cs{mright}\meta{delimR}
% \end{declcs}
% Macros \cs{mleft} and \cs{mright} are used in the same way as
% \cs{left} and \cs{right}. Also \cs{middle} can be used inbetween if
% \eTeX\ is present.
%
% \begin{declcs}{mleftright}
% \end{declcs}
% Macro \cs{mleftright} redefines \cs{left} as \cs{mleft} and
% \cs{right} as \cs{mright}. The redefinition is local to the group.
%
% \begin{declcs}{mleftrightrestore}
% \end{declcs}
% Macro \cs{mleftright} restores \cs{left} and \cs{right} with
% the original meaning if they were previously redefined by
% \cs{mleftright} (also locally).
%
%
% \StopEventually{
% }
%
% \section{Implementation}
%    \begin{macrocode}
%<*package>
%    \end{macrocode}
%    Reload check, especially if the package is not used with \LaTeX.
%    \begin{macrocode}
\begingroup\catcode61\catcode48\catcode32=10\relax%
  \catcode13=5 % ^^M
  \endlinechar=13 %
  \catcode35=6 % #
  \catcode39=12 % '
  \catcode44=12 % ,
  \catcode45=12 % -
  \catcode46=12 % .
  \catcode58=12 % :
  \catcode64=11 % @
  \catcode123=1 % {
  \catcode125=2 % }
  \expandafter\let\expandafter\x\csname ver@mleftright.sty\endcsname
  \ifx\x\relax % plain-TeX, first loading
  \else
    \def\empty{}%
    \ifx\x\empty % LaTeX, first loading,
      % variable is initialized, but \ProvidesPackage not yet seen
    \else
      \expandafter\ifx\csname PackageInfo\endcsname\relax
        \def\x#1#2{%
          \immediate\write-1{Package #1 Info: #2.}%
        }%
      \else
        \def\x#1#2{\PackageInfo{#1}{#2, stopped}}%
      \fi
      \x{mleftright}{The package is already loaded}%
      \aftergroup\endinput
    \fi
  \fi
\endgroup%
%    \end{macrocode}
%    Package identification:
%    \begin{macrocode}
\begingroup\catcode61\catcode48\catcode32=10\relax%
  \catcode13=5 % ^^M
  \endlinechar=13 %
  \catcode35=6 % #
  \catcode39=12 % '
  \catcode40=12 % (
  \catcode41=12 % )
  \catcode44=12 % ,
  \catcode45=12 % -
  \catcode46=12 % .
  \catcode47=12 % /
  \catcode58=12 % :
  \catcode64=11 % @
  \catcode91=12 % [
  \catcode93=12 % ]
  \catcode123=1 % {
  \catcode125=2 % }
  \expandafter\ifx\csname ProvidesPackage\endcsname\relax
    \def\x#1#2#3[#4]{\endgroup
      \immediate\write-1{Package: #3 #4}%
      \xdef#1{#4}%
    }%
  \else
    \def\x#1#2[#3]{\endgroup
      #2[{#3}]%
      \ifx#1\@undefined
        \xdef#1{#3}%
      \fi
      \ifx#1\relax
        \xdef#1{#3}%
      \fi
    }%
  \fi
\expandafter\x\csname ver@mleftright.sty\endcsname
\ProvidesPackage{mleftright}%
  [2016/05/16 v1.1 Math left/right delim. as open/close (HO)]%
%    \end{macrocode}
%
%    \begin{macrocode}
\begingroup\catcode61\catcode48\catcode32=10\relax%
  \catcode13=5 % ^^M
  \endlinechar=13 %
  \catcode123=1 % {
  \catcode125=2 % }
  \catcode64=11 % @
  \def\x{\endgroup
    \expandafter\edef\csname mleftright@AtEnd\endcsname{%
      \endlinechar=\the\endlinechar\relax
      \catcode13=\the\catcode13\relax
      \catcode32=\the\catcode32\relax
      \catcode35=\the\catcode35\relax
      \catcode61=\the\catcode61\relax
      \catcode64=\the\catcode64\relax
      \catcode123=\the\catcode123\relax
      \catcode125=\the\catcode125\relax
    }%
  }%
\x\catcode61\catcode48\catcode32=10\relax%
\catcode13=5 % ^^M
\endlinechar=13 %
\catcode35=6 % #
\catcode64=11 % @
\catcode123=1 % {
\catcode125=2 % }
\def\TMP@EnsureCode#1#2{%
  \edef\mleftright@AtEnd{%
    \mleftright@AtEnd
    \catcode#1=\the\catcode#1\relax
  }%
  \catcode#1=#2\relax
}
\TMP@EnsureCode{38}{4}% &
\TMP@EnsureCode{39}{12}% '
\TMP@EnsureCode{40}{12}% (
\TMP@EnsureCode{41}{12}% )
\TMP@EnsureCode{42}{12}% *
\TMP@EnsureCode{43}{12}% +
\TMP@EnsureCode{44}{12}% ,
\TMP@EnsureCode{45}{12}% -
\TMP@EnsureCode{46}{12}% .
\TMP@EnsureCode{47}{12}% /
\TMP@EnsureCode{60}{12}% <
\TMP@EnsureCode{91}{12}% [
\TMP@EnsureCode{93}{12}% ]
\edef\mleftright@AtEnd{%
  \mleftright@AtEnd
  \escapechar\the\escapechar\relax
  \noexpand\endinput
}
\escapechar=92 %
%    \end{macrocode}
%
%    \begin{macrocode}
\begingroup\expandafter\expandafter\expandafter\endgroup
\expandafter\ifx\csname RequirePackage\endcsname\relax
  \input infwarerr.sty\relax
  \input ltxcmds.sty\relax
\else
  \RequirePackage{infwarerr}[2010/04/08]%
  \RequirePackage{ltxcmds}[2010/04/26]%
\fi
%    \end{macrocode}
%
%    The original commands \cs{left} and \cs{right}
%    are saved and later used in \cs{mleft} and
%    \cs{mright} in order to deal with:
%    \begin{quote}
%\begin{verbatim}
%\let\left\mleft
%\let\right\mright
%\end{verbatim}
%    \end{quote}
%    \begin{macro}{\mleftright@OrgLeft}
%    \begin{macrocode}
\let\mleftright@OrgLeft\left
%    \end{macrocode}
%    \end{macro}
%    \begin{macro}{\mleftright@OrgRight}
%    \begin{macrocode}
\let\mleftright@OrgRight\right
%    \end{macrocode}
%    \end{macro}
%
%    \begin{macro}{\mleftright@Def}
%    Macro \cs{mleftright@Def} defines a macro as robust macro
%    if \eTeX\ or \LaTeX\ is available.
%    \begin{macrocode}
\ltx@IfUndefined{protected}{%
  \ltx@IfUndefined{DeclareRobustCommand}{%
    \def\mleftright@Def{\def}%
  }{%
    \def\mleftright@Def{\DeclareRobustCommand*}%
  }%
}{%
  \def\mleftright@Def{\protected\def}%
}
\edef\mleftright@Def#1{%
  \noexpand\ltx@IfUndefined{%
    \noexpand\expandafter\noexpand\ltx@gobble\noexpand\string#1%
  }{%
    \expandafter\noexpand\mleftright@Def#1%
  }{%
    \noexpand\@PackageError{mleftright}{%
      Command \noexpand\string#1 already defined%
    }\noexpand\@ehd
    \noexpand\ltx@gobble
  }%
}
%    \end{macrocode}
%    \end{macro}
%
%    In case of \eTeX\ the group status after the left symbol
%    is saved and later checked at the beginning of \cs{mright}.
%    \begin{macrocode}
\ltx@IfUndefined{currentgrouplevel}{%
  \catcode38=14 % & = comment
}{%
  \catcode38=9 % & = ignore
}
%    \end{macrocode}
%
%    \begin{macro}{\mleftright@GroupLevel}
%    \begin{macrocode}
& \def\mleftright@GroupLevel{-1}%
%    \end{macrocode}
%    \end{macro}
%
%    \begin{macro}{\mleftright@WrongGroup}
%    \begin{macrocode}
& \def\mleftright@WrongGroup#1(#2){%
&   \ifnum\mleftright@GroupLevel<\ltx@zero
&     \@PackageError{mleftright}{%
&       Missing previous \string\mleft
&     }\@ehc
&   \else
&     \@PackageError{mleftright}{%
&       Unexpected group status for \string\mright%
&       \ifnum\mleftright@GroupLevel=#1 %
&       \else
&         .\MessageBreak
&         Group level is #1, %
&           expected is \mleftright@GroupLevel
&       \fi
&       \ifnum16=#2 %
&       \else
&         .\MessageBreak
&         Group type is #2 (%
&         \ifcase#2 %
&           bottom level%
&           \expandafter\expandafter\expandafter\ltx@gobblefour
&           \expandafter\ltx@gobbletwo
&         \or simple%
&         \or hbox%
&         \or adjusted hbox%
&         \or vbox%
&         \or vtop%
&         \or align%
&         \or no align%
&         \or output%
&         \or math%
&         \or disc%
&         \or insert%
&         \or vcenter%
&         \or math choice%
&         \or semi simple%
&         \or math shift%
&         \or math left%
&         \else
&           unknown%
&         \fi
&         \space group),\MessageBreak
&         expected is 16 (math left group)%
&       \fi
&     }\@ehd
&   \fi
& }%
%    \end{macrocode}
%    \end{macro}
%
%    \begin{macro}{\mleft}
%    \begin{macrocode}
\mleftright@Def\mleft{%
  \mathopen{}\mathclose\bgroup
& \edef\mleftright@GroupLevel{\the\numexpr\the\currentgrouplevel+1}%
  \mleftright@OrgLeft
}
%    \end{macrocode}
%    \end{macro}
%    \begin{macro}{\mright}
%    \begin{macrocode}
\mleftright@Def\mright{%
& \ifnum\mleftright@GroupLevel=\currentgrouplevel
&   \ifnum16=\currentgrouptype
      \aftergroup\egroup
&   \else
&     \expandafter\mleftright@WrongGroup
&     \the\expandafter\currentgrouplevel
&     \expandafter(\the\currentgrouptype)%
&   \fi
& \else
&   \expandafter\mleftright@WrongGroup
&   \the\expandafter\currentgrouplevel
&   \expandafter(\the\currentgrouptype)%
& \fi
  \mleftright@OrgRight
}
%    \end{macrocode}
%    \end{macro}
%
%    \begin{macro}{\mleftright}
%    \begin{macrocode}
\mleftright@Def\mleftright{%
  \let\left\mleft
  \let\right\mright
}
%    \end{macrocode}
%    \end{macro}
%
%    \begin{macro}{\mleftrightrestore}
%    \begin{macrocode}
\mleftright@Def\mleftrightrestore{%
  \ifx\left\mleft
    \let\left\mleftright@OrgLeft
  \fi
  \ifx\right\mright
    \let\right\mleftright@OrgRight
  \fi
}
%    \end{macrocode}
%    \end{macro}
%
%    \begin{macrocode}
\mleftright@AtEnd%
%</package>
%    \end{macrocode}
%
% \section{Test}
%
% \subsection{Catcode checks for loading}
%
%    \begin{macrocode}
%<*test1>
%    \end{macrocode}
%    \begin{macrocode}
\catcode`\{=1 %
\catcode`\}=2 %
\catcode`\#=6 %
\catcode`\@=11 %
\expandafter\ifx\csname count@\endcsname\relax
  \countdef\count@=255 %
\fi
\expandafter\ifx\csname @gobble\endcsname\relax
  \long\def\@gobble#1{}%
\fi
\expandafter\ifx\csname @firstofone\endcsname\relax
  \long\def\@firstofone#1{#1}%
\fi
\expandafter\ifx\csname loop\endcsname\relax
  \expandafter\@firstofone
\else
  \expandafter\@gobble
\fi
{%
  \def\loop#1\repeat{%
    \def\body{#1}%
    \iterate
  }%
  \def\iterate{%
    \body
      \let\next\iterate
    \else
      \let\next\relax
    \fi
    \next
  }%
  \let\repeat=\fi
}%
\def\RestoreCatcodes{}
\count@=0 %
\loop
  \edef\RestoreCatcodes{%
    \RestoreCatcodes
    \catcode\the\count@=\the\catcode\count@\relax
  }%
\ifnum\count@<255 %
  \advance\count@ 1 %
\repeat

\def\RangeCatcodeInvalid#1#2{%
  \count@=#1\relax
  \loop
    \catcode\count@=15 %
  \ifnum\count@<#2\relax
    \advance\count@ 1 %
  \repeat
}
\def\RangeCatcodeCheck#1#2#3{%
  \count@=#1\relax
  \loop
    \ifnum#3=\catcode\count@
    \else
      \errmessage{%
        Character \the\count@\space
        with wrong catcode \the\catcode\count@\space
        instead of \number#3%
      }%
    \fi
  \ifnum\count@<#2\relax
    \advance\count@ 1 %
  \repeat
}
\def\space{ }
\expandafter\ifx\csname LoadCommand\endcsname\relax
  \def\LoadCommand{\input mleftright.sty\relax}%
\fi
\def\Test{%
  \RangeCatcodeInvalid{0}{47}%
  \RangeCatcodeInvalid{58}{64}%
  \RangeCatcodeInvalid{91}{96}%
  \RangeCatcodeInvalid{123}{255}%
  \catcode`\@=12 %
  \catcode`\\=0 %
  \catcode`\%=14 %
  \LoadCommand
  \RangeCatcodeCheck{0}{36}{15}%
  \RangeCatcodeCheck{37}{37}{14}%
  \RangeCatcodeCheck{38}{47}{15}%
  \RangeCatcodeCheck{48}{57}{12}%
  \RangeCatcodeCheck{58}{63}{15}%
  \RangeCatcodeCheck{64}{64}{12}%
  \RangeCatcodeCheck{65}{90}{11}%
  \RangeCatcodeCheck{91}{91}{15}%
  \RangeCatcodeCheck{92}{92}{0}%
  \RangeCatcodeCheck{93}{96}{15}%
  \RangeCatcodeCheck{97}{122}{11}%
  \RangeCatcodeCheck{123}{255}{15}%
  \RestoreCatcodes
}
\Test
\csname @@end\endcsname
\end
%    \end{macrocode}
%    \begin{macrocode}
%</test1>
%    \end{macrocode}
%
% \section{Installation}
%
% \subsection{Download}
%
% \paragraph{Package.} This package is available on
% CTAN\footnote{\url{http://ctan.org/pkg/mleftright}}:
% \begin{description}
% \item[\CTAN{macros/latex/contrib/oberdiek/mleftright.dtx}] The source file.
% \item[\CTAN{macros/latex/contrib/oberdiek/mleftright.pdf}] Documentation.
% \end{description}
%
%
% \paragraph{Bundle.} All the packages of the bundle `oberdiek'
% are also available in a TDS compliant ZIP archive. There
% the packages are already unpacked and the documentation files
% are generated. The files and directories obey the TDS standard.
% \begin{description}
% \item[\CTAN{install/macros/latex/contrib/oberdiek.tds.zip}]
% \end{description}
% \emph{TDS} refers to the standard ``A Directory Structure
% for \TeX\ Files'' (\CTAN{tds/tds.pdf}). Directories
% with \xfile{texmf} in their name are usually organized this way.
%
% \subsection{Bundle installation}
%
% \paragraph{Unpacking.} Unpack the \xfile{oberdiek.tds.zip} in the
% TDS tree (also known as \xfile{texmf} tree) of your choice.
% Example (linux):
% \begin{quote}
%   |unzip oberdiek.tds.zip -d ~/texmf|
% \end{quote}
%
% \paragraph{Script installation.}
% Check the directory \xfile{TDS:scripts/oberdiek/} for
% scripts that need further installation steps.
% Package \xpackage{attachfile2} comes with the Perl script
% \xfile{pdfatfi.pl} that should be installed in such a way
% that it can be called as \texttt{pdfatfi}.
% Example (linux):
% \begin{quote}
%   |chmod +x scripts/oberdiek/pdfatfi.pl|\\
%   |cp scripts/oberdiek/pdfatfi.pl /usr/local/bin/|
% \end{quote}
%
% \subsection{Package installation}
%
% \paragraph{Unpacking.} The \xfile{.dtx} file is a self-extracting
% \docstrip\ archive. The files are extracted by running the
% \xfile{.dtx} through \plainTeX:
% \begin{quote}
%   \verb|tex mleftright.dtx|
% \end{quote}
%
% \paragraph{TDS.} Now the different files must be moved into
% the different directories in your installation TDS tree
% (also known as \xfile{texmf} tree):
% \begin{quote}
% \def\t{^^A
% \begin{tabular}{@{}>{\ttfamily}l@{ $\rightarrow$ }>{\ttfamily}l@{}}
%   mleftright.sty & tex/generic/oberdiek/mleftright.sty\\
%   mleftright.pdf & doc/latex/oberdiek/mleftright.pdf\\
%   test/mleftright-test1.tex & doc/latex/oberdiek/test/mleftright-test1.tex\\
%   mleftright.dtx & source/latex/oberdiek/mleftright.dtx\\
% \end{tabular}^^A
% }^^A
% \sbox0{\t}^^A
% \ifdim\wd0>\linewidth
%   \begingroup
%     \advance\linewidth by\leftmargin
%     \advance\linewidth by\rightmargin
%   \edef\x{\endgroup
%     \def\noexpand\lw{\the\linewidth}^^A
%   }\x
%   \def\lwbox{^^A
%     \leavevmode
%     \hbox to \linewidth{^^A
%       \kern-\leftmargin\relax
%       \hss
%       \usebox0
%       \hss
%       \kern-\rightmargin\relax
%     }^^A
%   }^^A
%   \ifdim\wd0>\lw
%     \sbox0{\small\t}^^A
%     \ifdim\wd0>\linewidth
%       \ifdim\wd0>\lw
%         \sbox0{\footnotesize\t}^^A
%         \ifdim\wd0>\linewidth
%           \ifdim\wd0>\lw
%             \sbox0{\scriptsize\t}^^A
%             \ifdim\wd0>\linewidth
%               \ifdim\wd0>\lw
%                 \sbox0{\tiny\t}^^A
%                 \ifdim\wd0>\linewidth
%                   \lwbox
%                 \else
%                   \usebox0
%                 \fi
%               \else
%                 \lwbox
%               \fi
%             \else
%               \usebox0
%             \fi
%           \else
%             \lwbox
%           \fi
%         \else
%           \usebox0
%         \fi
%       \else
%         \lwbox
%       \fi
%     \else
%       \usebox0
%     \fi
%   \else
%     \lwbox
%   \fi
% \else
%   \usebox0
% \fi
% \end{quote}
% If you have a \xfile{docstrip.cfg} that configures and enables \docstrip's
% TDS installing feature, then some files can already be in the right
% place, see the documentation of \docstrip.
%
% \subsection{Refresh file name databases}
%
% If your \TeX~distribution
% (\teTeX, \mikTeX, \dots) relies on file name databases, you must refresh
% these. For example, \teTeX\ users run \verb|texhash| or
% \verb|mktexlsr|.
%
% \subsection{Some details for the interested}
%
% \paragraph{Attached source.}
%
% The PDF documentation on CTAN also includes the
% \xfile{.dtx} source file. It can be extracted by
% AcrobatReader 6 or higher. Another option is \textsf{pdftk},
% e.g. unpack the file into the current directory:
% \begin{quote}
%   \verb|pdftk mleftright.pdf unpack_files output .|
% \end{quote}
%
% \paragraph{Unpacking with \LaTeX.}
% The \xfile{.dtx} chooses its action depending on the format:
% \begin{description}
% \item[\plainTeX:] Run \docstrip\ and extract the files.
% \item[\LaTeX:] Generate the documentation.
% \end{description}
% If you insist on using \LaTeX\ for \docstrip\ (really,
% \docstrip\ does not need \LaTeX), then inform the autodetect routine
% about your intention:
% \begin{quote}
%   \verb|latex \let\install=y\input{mleftright.dtx}|
% \end{quote}
% Do not forget to quote the argument according to the demands
% of your shell.
%
% \paragraph{Generating the documentation.}
% You can use both the \xfile{.dtx} or the \xfile{.drv} to generate
% the documentation. The process can be configured by the
% configuration file \xfile{ltxdoc.cfg}. For instance, put this
% line into this file, if you want to have A4 as paper format:
% \begin{quote}
%   \verb|\PassOptionsToClass{a4paper}{article}|
% \end{quote}
% An example follows how to generate the
% documentation with pdf\LaTeX:
% \begin{quote}
%\begin{verbatim}
%pdflatex mleftright.dtx
%makeindex -s gind.ist mleftright.idx
%pdflatex mleftright.dtx
%makeindex -s gind.ist mleftright.idx
%pdflatex mleftright.dtx
%\end{verbatim}
% \end{quote}
%
% \section{Catalogue}
%
% The following XML file can be used as source for the
% \href{http://mirror.ctan.org/help/Catalogue/catalogue.html}{\TeX\ Catalogue}.
% The elements \texttt{caption} and \texttt{description} are imported
% from the original XML file from the Catalogue.
% The name of the XML file in the Catalogue is \xfile{mleftright.xml}.
%    \begin{macrocode}
%<*catalogue>
<?xml version='1.0' encoding='us-ascii'?>
<!DOCTYPE entry SYSTEM 'catalogue.dtd'>
<entry datestamp='$Date$' modifier='$Author$' id='mleftright'>
  <name>mleftright</name>
  <caption>Variants of delimiters that act as maths open/close.</caption>
  <authorref id='auth:oberdiek'/>
  <copyright owner='Heiko Oberdiek' year='2010'/>
  <license type='lppl1.3'/>
  <version number='1.1'/>
  <description>
    The package defines variants <tt>\mleft</tt> and <tt>\mright</tt>
    of <tt>\left</tt> and <tt>\right</tt>, that make the delimiters
    act as <tt>\mathopen</tt> and <tt>\mathclose</tt>.  These commands
    address spacing difficulties in subformulas.
    <p/>
    The package is part of the <xref refid='oberdiek'>oberdiek</xref> bundle.
  </description>
  <documentation details='Package documentation'
      href='ctan:/macros/latex/contrib/oberdiek/mleftright.pdf'/>
  <ctan file='true' path='/macros/latex/contrib/oberdiek/mleftright.dtx'/>
  <miktex location='oberdiek'/>
  <texlive location='oberdiek'/>
  <install path='/macros/latex/contrib/oberdiek/oberdiek.tds.zip'/>
</entry>
%</catalogue>
%    \end{macrocode}
%
% \section{Acknowledgement}
%
% \begin{description}
% \item[Donald Arsenau:]
% He provided the main trick and the first macros.
% \item[Philipp Stephani:]
% He solved the subscript problem.
% \end{description}
%
% \begin{thebibliography}{9}
% \raggedright
% \bibitem{dave}
%   Dave94705,
%   \textit{spacing after \cs{right}\texttt{)} and before \cs{left}\texttt{)}},
%   newsgroup comp.text.tex,
%   Message-ID: \texttt{\small 5d264909-7c3d-4c9d-9b22-434178b2bf90@g21g2000prn.googlegroups.com},
%   2010-08-12.
%   \newblock
%   {\small\url{http://groups.google.com/group/comp.text.tex/msg/e5b6833da7dc29bf}}
%
% \bibitem{arseneau}
%   Donald Arseneau,
%   \textit{Re: spacing after \cs{right}\texttt) and before \cs{left}\texttt)},
%   newsgroup comp.text.tex,
%   Message-ID: \texttt{\small yfivd6svl8y.fsf@mutant.triumf.ca},
%   2010-08-30.
%   \newblock
%   {\small\url{http://groups.google.com/group/comp.text.tex/msg/e0b2e4386e5d04e4}}
%
% \bibitem{stephani}
%   Philipp Stephani,
%   \textit{Re: spacing after \cs{right}\texttt) and before \cs{left}\texttt)},
%   newsgroup comp.text.tex,
%   Message-ID: \texttt{\small 4c8c8c1e\$0\$6981\$9b4e6d93@newsspool4.arcor-online.net},
%   2010-09-12.
%   \newblock
%   {\small\url{http://groups.google.com/group/comp.text.tex/msg/87ac1f61321de3ef}}
%
% \bibitem{oberdiek}
%   Heiko Oberdiek,
%   \textit{Re: spacing after \cs{right}\texttt) and before \cs{left}\texttt)},
%   newsgroup comp.text.tex,
%   Message-ID: \texttt{\small i6jcc2\$8of\$1@news.eternal-september.org},
%   2010-09-12.
%   \newblock
%   {\small\url{http://groups.google.com/group/comp.text.tex/msg/257aa6119bef878b}}
%
% \end{thebibliography}
%
% \begin{History}
%   \begin{Version}{2010/09/25 v1.0}
%   \item
%     The first version.
%   \end{Version}
%   \begin{Version}{2016/05/16 v1.1}
%   \item
%     Documentation updates.
%   \end{Version}
% \end{History}
%
% \PrintIndex
%
% \Finale
\endinput

%        (quote the arguments according to the demands of your shell)
%
% Documentation:
%    (a) If mleftright.drv is present:
%           latex mleftright.drv
%    (b) Without mleftright.drv:
%           latex mleftright.dtx; ...
%    The class ltxdoc loads the configuration file ltxdoc.cfg
%    if available. Here you can specify further options, e.g.
%    use A4 as paper format:
%       \PassOptionsToClass{a4paper}{article}
%
%    Programm calls to get the documentation (example):
%       pdflatex mleftright.dtx
%       makeindex -s gind.ist mleftright.idx
%       pdflatex mleftright.dtx
%       makeindex -s gind.ist mleftright.idx
%       pdflatex mleftright.dtx
%
% Installation:
%    TDS:tex/generic/oberdiek/mleftright.sty
%    TDS:doc/latex/oberdiek/mleftright.pdf
%    TDS:doc/latex/oberdiek/test/mleftright-test1.tex
%    TDS:source/latex/oberdiek/mleftright.dtx
%
%<*ignore>
\begingroup
  \catcode123=1 %
  \catcode125=2 %
  \def\x{LaTeX2e}%
\expandafter\endgroup
\ifcase 0\ifx\install y1\fi\expandafter
         \ifx\csname processbatchFile\endcsname\relax\else1\fi
         \ifx\fmtname\x\else 1\fi\relax
\else\csname fi\endcsname
%</ignore>
%<*install>
\input docstrip.tex
\Msg{************************************************************************}
\Msg{* Installation}
\Msg{* Package: mleftright 2016/05/16 v1.1 Math left/right delim. as open/close (HO)}
\Msg{************************************************************************}

\keepsilent
\askforoverwritefalse

\let\MetaPrefix\relax
\preamble

This is a generated file.

Project: mleftright
Version: 2016/05/16 v1.1

Copyright (C) 2010 by
   Heiko Oberdiek <heiko.oberdiek at googlemail.com>

This work may be distributed and/or modified under the
conditions of the LaTeX Project Public License, either
version 1.3c of this license or (at your option) any later
version. This version of this license is in
   http://www.latex-project.org/lppl/lppl-1-3c.txt
and the latest version of this license is in
   http://www.latex-project.org/lppl.txt
and version 1.3 or later is part of all distributions of
LaTeX version 2005/12/01 or later.

This work has the LPPL maintenance status "maintained".

This Current Maintainer of this work is Heiko Oberdiek.

The Base Interpreter refers to any `TeX-Format',
because some files are installed in TDS:tex/generic//.

This work consists of the main source file mleftright.dtx
and the derived files
   mleftright.sty, mleftright.pdf, mleftright.ins, mleftright.drv,
   mleftright-test1.tex.

\endpreamble
\let\MetaPrefix\DoubleperCent

\generate{%
  \file{mleftright.ins}{\from{mleftright.dtx}{install}}%
  \file{mleftright.drv}{\from{mleftright.dtx}{driver}}%
  \usedir{tex/generic/oberdiek}%
  \file{mleftright.sty}{\from{mleftright.dtx}{package}}%
  \usedir{doc/latex/oberdiek/test}%
  \file{mleftright-test1.tex}{\from{mleftright.dtx}{test1}}%
  \nopreamble
  \nopostamble
  \usedir{source/latex/oberdiek/catalogue}%
  \file{mleftright.xml}{\from{mleftright.dtx}{catalogue}}%
}

\catcode32=13\relax% active space
\let =\space%
\Msg{************************************************************************}
\Msg{*}
\Msg{* To finish the installation you have to move the following}
\Msg{* file into a directory searched by TeX:}
\Msg{*}
\Msg{*     mleftright.sty}
\Msg{*}
\Msg{* To produce the documentation run the file `mleftright.drv'}
\Msg{* through LaTeX.}
\Msg{*}
\Msg{* Happy TeXing!}
\Msg{*}
\Msg{************************************************************************}

\endbatchfile
%</install>
%<*ignore>
\fi
%</ignore>
%<*driver>
\NeedsTeXFormat{LaTeX2e}
\ProvidesFile{mleftright.drv}%
  [2016/05/16 v1.1 Math left/right delim. as open/close (HO)]%
\documentclass{ltxdoc}
\usepackage{holtxdoc}[2011/11/22]
\usepackage{mleftright}[2016/05/16]
\begin{document}
  \DocInput{mleftright.dtx}%
\end{document}
%</driver>
% \fi
%
%
% \CharacterTable
%  {Upper-case    \A\B\C\D\E\F\G\H\I\J\K\L\M\N\O\P\Q\R\S\T\U\V\W\X\Y\Z
%   Lower-case    \a\b\c\d\e\f\g\h\i\j\k\l\m\n\o\p\q\r\s\t\u\v\w\x\y\z
%   Digits        \0\1\2\3\4\5\6\7\8\9
%   Exclamation   \!     Double quote  \"     Hash (number) \#
%   Dollar        \$     Percent       \%     Ampersand     \&
%   Acute accent  \'     Left paren    \(     Right paren   \)
%   Asterisk      \*     Plus          \+     Comma         \,
%   Minus         \-     Point         \.     Solidus       \/
%   Colon         \:     Semicolon     \;     Less than     \<
%   Equals        \=     Greater than  \>     Question mark \?
%   Commercial at \@     Left bracket  \[     Backslash     \\
%   Right bracket \]     Circumflex    \^     Underscore    \_
%   Grave accent  \`     Left brace    \{     Vertical bar  \|
%   Right brace   \}     Tilde         \~}
%
% \GetFileInfo{mleftright.drv}
%
% \title{The \xpackage{mleftright} package}
% \date{2016/05/16 v1.1}
% \author{Heiko Oberdiek\thanks
% {Please report any issues at https://github.com/ho-tex/oberdiek/issues}\\
% \xemail{heiko.oberdiek at googlemail.com}}
%
% \maketitle
%
% \begin{abstract}
% \TeX\ sets subformulas by \cs{left} and \cs{right} as inner formulas
% with additional surrounding spaces in some situations. This package
% provides \cs{mleft} and \cs{mright} that call \cs{left} and \cs{right},
% but the delimiters will act as normal \cs{mathopen} and \cs{mathclose}
% delimiters without the additional space of an inner formula.
% \end{abstract}
%
% \tableofcontents
%
% \section{Documentation}
%
% The package is a result of a thread in the newsgroup \textsf{comp.text.tex}
% with the subject \textit{spacing after \cs{right}\texttt{)}
% and before \cs{left}\texttt{)}} \cite{dave}.
% The problem: \cs{left} and \cs{right} adjust the size of the
% delimiters automatically. However, \TeX\ treats the whole expression
% as inner formula. In some circumstances \TeX\ adds extra space
% before or after an inner formula.
% Example:
% \begin{quote}
%   \thinmuskip=1.5\thinmuskip
%   \begin{tabular}{@{}l@{\quad$\Rightarrow$\quad}l@{}}
%     |$\sin(x^2), x$|
%     & $\sin(x^2), x$\\
%     |$\sin\left(x^2\right), x$|
%     & $\sin\left(x^2\right), x$\\
%   ^^A  \multicolumn{1}{@{}r@{\quad$\Rightarrow$\quad}}{^^A
%   ^^A    \itshape with exaggerated spacing^^A
%   ^^A  }
%   ^^A  & $\thinmuskip=4\thinmuskip
%   ^^A    \sin\left(x^2\right){,}\mskip.25\thinmuskip x$\\
%     |$\sin\mleft(x^2\mright), x$|
%     & $\sin\mleft(x^2\mright), x$\\
%   \end{tabular}\\*[.5ex]
%   (\cs{mleft} and \cs{mright} are provided by this package.)
% \end{quote}
%
% In the newsgroup Donald Arseneau answered with clever macros \cite{arseneau}:
% \begin{quote}
%\begin{verbatim}
%\newcommand\lft{\mathopen{}\left}
%\newcommand\rgt{\aftergroup\mathclose\aftergroup{\aftergroup}\right}
%\end{verbatim}
% \end{quote}
% However one problem remains, a following subscript or superscript
% is not applied to the right delimiter but the empty
% \cs{mathclose}.
% Thus Philipp Stephani provided an improvement \cite{stephani}:
%\begin{quote}
%\begin{verbatim}
%\mathopen{} \mathclose{\left\| A^2 \right\|}_2
%\end{verbatim}
%\end{quote}
% Heiko Oberdiek converted this into macro form \cite{oberdiek}:
%\begin{quote}
%\begin{verbatim}
%\newcommand\lft{\mathopen{}\mathclose\bgroup\left}
%\newcommand\rgt{\aftergroup\egroup\right}
%\end{verbatim}
%\end{quote}
%
% The package uses longer macro names \cs{mleft} and \cs{mright}
% to avoid name clashes. Also it adds some checks for error conditions.
%
% \subsection{Use}
%
% \begin{declcs}{mleft}\meta{delimL} \dots\unkern\ \cs{mright}\meta{delimR}
% \end{declcs}
% Macros \cs{mleft} and \cs{mright} are used in the same way as
% \cs{left} and \cs{right}. Also \cs{middle} can be used inbetween if
% \eTeX\ is present.
%
% \begin{declcs}{mleftright}
% \end{declcs}
% Macro \cs{mleftright} redefines \cs{left} as \cs{mleft} and
% \cs{right} as \cs{mright}. The redefinition is local to the group.
%
% \begin{declcs}{mleftrightrestore}
% \end{declcs}
% Macro \cs{mleftright} restores \cs{left} and \cs{right} with
% the original meaning if they were previously redefined by
% \cs{mleftright} (also locally).
%
%
% \StopEventually{
% }
%
% \section{Implementation}
%    \begin{macrocode}
%<*package>
%    \end{macrocode}
%    Reload check, especially if the package is not used with \LaTeX.
%    \begin{macrocode}
\begingroup\catcode61\catcode48\catcode32=10\relax%
  \catcode13=5 % ^^M
  \endlinechar=13 %
  \catcode35=6 % #
  \catcode39=12 % '
  \catcode44=12 % ,
  \catcode45=12 % -
  \catcode46=12 % .
  \catcode58=12 % :
  \catcode64=11 % @
  \catcode123=1 % {
  \catcode125=2 % }
  \expandafter\let\expandafter\x\csname ver@mleftright.sty\endcsname
  \ifx\x\relax % plain-TeX, first loading
  \else
    \def\empty{}%
    \ifx\x\empty % LaTeX, first loading,
      % variable is initialized, but \ProvidesPackage not yet seen
    \else
      \expandafter\ifx\csname PackageInfo\endcsname\relax
        \def\x#1#2{%
          \immediate\write-1{Package #1 Info: #2.}%
        }%
      \else
        \def\x#1#2{\PackageInfo{#1}{#2, stopped}}%
      \fi
      \x{mleftright}{The package is already loaded}%
      \aftergroup\endinput
    \fi
  \fi
\endgroup%
%    \end{macrocode}
%    Package identification:
%    \begin{macrocode}
\begingroup\catcode61\catcode48\catcode32=10\relax%
  \catcode13=5 % ^^M
  \endlinechar=13 %
  \catcode35=6 % #
  \catcode39=12 % '
  \catcode40=12 % (
  \catcode41=12 % )
  \catcode44=12 % ,
  \catcode45=12 % -
  \catcode46=12 % .
  \catcode47=12 % /
  \catcode58=12 % :
  \catcode64=11 % @
  \catcode91=12 % [
  \catcode93=12 % ]
  \catcode123=1 % {
  \catcode125=2 % }
  \expandafter\ifx\csname ProvidesPackage\endcsname\relax
    \def\x#1#2#3[#4]{\endgroup
      \immediate\write-1{Package: #3 #4}%
      \xdef#1{#4}%
    }%
  \else
    \def\x#1#2[#3]{\endgroup
      #2[{#3}]%
      \ifx#1\@undefined
        \xdef#1{#3}%
      \fi
      \ifx#1\relax
        \xdef#1{#3}%
      \fi
    }%
  \fi
\expandafter\x\csname ver@mleftright.sty\endcsname
\ProvidesPackage{mleftright}%
  [2016/05/16 v1.1 Math left/right delim. as open/close (HO)]%
%    \end{macrocode}
%
%    \begin{macrocode}
\begingroup\catcode61\catcode48\catcode32=10\relax%
  \catcode13=5 % ^^M
  \endlinechar=13 %
  \catcode123=1 % {
  \catcode125=2 % }
  \catcode64=11 % @
  \def\x{\endgroup
    \expandafter\edef\csname mleftright@AtEnd\endcsname{%
      \endlinechar=\the\endlinechar\relax
      \catcode13=\the\catcode13\relax
      \catcode32=\the\catcode32\relax
      \catcode35=\the\catcode35\relax
      \catcode61=\the\catcode61\relax
      \catcode64=\the\catcode64\relax
      \catcode123=\the\catcode123\relax
      \catcode125=\the\catcode125\relax
    }%
  }%
\x\catcode61\catcode48\catcode32=10\relax%
\catcode13=5 % ^^M
\endlinechar=13 %
\catcode35=6 % #
\catcode64=11 % @
\catcode123=1 % {
\catcode125=2 % }
\def\TMP@EnsureCode#1#2{%
  \edef\mleftright@AtEnd{%
    \mleftright@AtEnd
    \catcode#1=\the\catcode#1\relax
  }%
  \catcode#1=#2\relax
}
\TMP@EnsureCode{38}{4}% &
\TMP@EnsureCode{39}{12}% '
\TMP@EnsureCode{40}{12}% (
\TMP@EnsureCode{41}{12}% )
\TMP@EnsureCode{42}{12}% *
\TMP@EnsureCode{43}{12}% +
\TMP@EnsureCode{44}{12}% ,
\TMP@EnsureCode{45}{12}% -
\TMP@EnsureCode{46}{12}% .
\TMP@EnsureCode{47}{12}% /
\TMP@EnsureCode{60}{12}% <
\TMP@EnsureCode{91}{12}% [
\TMP@EnsureCode{93}{12}% ]
\edef\mleftright@AtEnd{%
  \mleftright@AtEnd
  \escapechar\the\escapechar\relax
  \noexpand\endinput
}
\escapechar=92 %
%    \end{macrocode}
%
%    \begin{macrocode}
\begingroup\expandafter\expandafter\expandafter\endgroup
\expandafter\ifx\csname RequirePackage\endcsname\relax
  \input infwarerr.sty\relax
  \input ltxcmds.sty\relax
\else
  \RequirePackage{infwarerr}[2010/04/08]%
  \RequirePackage{ltxcmds}[2010/04/26]%
\fi
%    \end{macrocode}
%
%    The original commands \cs{left} and \cs{right}
%    are saved and later used in \cs{mleft} and
%    \cs{mright} in order to deal with:
%    \begin{quote}
%\begin{verbatim}
%\let\left\mleft
%\let\right\mright
%\end{verbatim}
%    \end{quote}
%    \begin{macro}{\mleftright@OrgLeft}
%    \begin{macrocode}
\let\mleftright@OrgLeft\left
%    \end{macrocode}
%    \end{macro}
%    \begin{macro}{\mleftright@OrgRight}
%    \begin{macrocode}
\let\mleftright@OrgRight\right
%    \end{macrocode}
%    \end{macro}
%
%    \begin{macro}{\mleftright@Def}
%    Macro \cs{mleftright@Def} defines a macro as robust macro
%    if \eTeX\ or \LaTeX\ is available.
%    \begin{macrocode}
\ltx@IfUndefined{protected}{%
  \ltx@IfUndefined{DeclareRobustCommand}{%
    \def\mleftright@Def{\def}%
  }{%
    \def\mleftright@Def{\DeclareRobustCommand*}%
  }%
}{%
  \def\mleftright@Def{\protected\def}%
}
\edef\mleftright@Def#1{%
  \noexpand\ltx@IfUndefined{%
    \noexpand\expandafter\noexpand\ltx@gobble\noexpand\string#1%
  }{%
    \expandafter\noexpand\mleftright@Def#1%
  }{%
    \noexpand\@PackageError{mleftright}{%
      Command \noexpand\string#1 already defined%
    }\noexpand\@ehd
    \noexpand\ltx@gobble
  }%
}
%    \end{macrocode}
%    \end{macro}
%
%    In case of \eTeX\ the group status after the left symbol
%    is saved and later checked at the beginning of \cs{mright}.
%    \begin{macrocode}
\ltx@IfUndefined{currentgrouplevel}{%
  \catcode38=14 % & = comment
}{%
  \catcode38=9 % & = ignore
}
%    \end{macrocode}
%
%    \begin{macro}{\mleftright@GroupLevel}
%    \begin{macrocode}
& \def\mleftright@GroupLevel{-1}%
%    \end{macrocode}
%    \end{macro}
%
%    \begin{macro}{\mleftright@WrongGroup}
%    \begin{macrocode}
& \def\mleftright@WrongGroup#1(#2){%
&   \ifnum\mleftright@GroupLevel<\ltx@zero
&     \@PackageError{mleftright}{%
&       Missing previous \string\mleft
&     }\@ehc
&   \else
&     \@PackageError{mleftright}{%
&       Unexpected group status for \string\mright%
&       \ifnum\mleftright@GroupLevel=#1 %
&       \else
&         .\MessageBreak
&         Group level is #1, %
&           expected is \mleftright@GroupLevel
&       \fi
&       \ifnum16=#2 %
&       \else
&         .\MessageBreak
&         Group type is #2 (%
&         \ifcase#2 %
&           bottom level%
&           \expandafter\expandafter\expandafter\ltx@gobblefour
&           \expandafter\ltx@gobbletwo
&         \or simple%
&         \or hbox%
&         \or adjusted hbox%
&         \or vbox%
&         \or vtop%
&         \or align%
&         \or no align%
&         \or output%
&         \or math%
&         \or disc%
&         \or insert%
&         \or vcenter%
&         \or math choice%
&         \or semi simple%
&         \or math shift%
&         \or math left%
&         \else
&           unknown%
&         \fi
&         \space group),\MessageBreak
&         expected is 16 (math left group)%
&       \fi
&     }\@ehd
&   \fi
& }%
%    \end{macrocode}
%    \end{macro}
%
%    \begin{macro}{\mleft}
%    \begin{macrocode}
\mleftright@Def\mleft{%
  \mathopen{}\mathclose\bgroup
& \edef\mleftright@GroupLevel{\the\numexpr\the\currentgrouplevel+1}%
  \mleftright@OrgLeft
}
%    \end{macrocode}
%    \end{macro}
%    \begin{macro}{\mright}
%    \begin{macrocode}
\mleftright@Def\mright{%
& \ifnum\mleftright@GroupLevel=\currentgrouplevel
&   \ifnum16=\currentgrouptype
      \aftergroup\egroup
&   \else
&     \expandafter\mleftright@WrongGroup
&     \the\expandafter\currentgrouplevel
&     \expandafter(\the\currentgrouptype)%
&   \fi
& \else
&   \expandafter\mleftright@WrongGroup
&   \the\expandafter\currentgrouplevel
&   \expandafter(\the\currentgrouptype)%
& \fi
  \mleftright@OrgRight
}
%    \end{macrocode}
%    \end{macro}
%
%    \begin{macro}{\mleftright}
%    \begin{macrocode}
\mleftright@Def\mleftright{%
  \let\left\mleft
  \let\right\mright
}
%    \end{macrocode}
%    \end{macro}
%
%    \begin{macro}{\mleftrightrestore}
%    \begin{macrocode}
\mleftright@Def\mleftrightrestore{%
  \ifx\left\mleft
    \let\left\mleftright@OrgLeft
  \fi
  \ifx\right\mright
    \let\right\mleftright@OrgRight
  \fi
}
%    \end{macrocode}
%    \end{macro}
%
%    \begin{macrocode}
\mleftright@AtEnd%
%</package>
%    \end{macrocode}
%
% \section{Test}
%
% \subsection{Catcode checks for loading}
%
%    \begin{macrocode}
%<*test1>
%    \end{macrocode}
%    \begin{macrocode}
\catcode`\{=1 %
\catcode`\}=2 %
\catcode`\#=6 %
\catcode`\@=11 %
\expandafter\ifx\csname count@\endcsname\relax
  \countdef\count@=255 %
\fi
\expandafter\ifx\csname @gobble\endcsname\relax
  \long\def\@gobble#1{}%
\fi
\expandafter\ifx\csname @firstofone\endcsname\relax
  \long\def\@firstofone#1{#1}%
\fi
\expandafter\ifx\csname loop\endcsname\relax
  \expandafter\@firstofone
\else
  \expandafter\@gobble
\fi
{%
  \def\loop#1\repeat{%
    \def\body{#1}%
    \iterate
  }%
  \def\iterate{%
    \body
      \let\next\iterate
    \else
      \let\next\relax
    \fi
    \next
  }%
  \let\repeat=\fi
}%
\def\RestoreCatcodes{}
\count@=0 %
\loop
  \edef\RestoreCatcodes{%
    \RestoreCatcodes
    \catcode\the\count@=\the\catcode\count@\relax
  }%
\ifnum\count@<255 %
  \advance\count@ 1 %
\repeat

\def\RangeCatcodeInvalid#1#2{%
  \count@=#1\relax
  \loop
    \catcode\count@=15 %
  \ifnum\count@<#2\relax
    \advance\count@ 1 %
  \repeat
}
\def\RangeCatcodeCheck#1#2#3{%
  \count@=#1\relax
  \loop
    \ifnum#3=\catcode\count@
    \else
      \errmessage{%
        Character \the\count@\space
        with wrong catcode \the\catcode\count@\space
        instead of \number#3%
      }%
    \fi
  \ifnum\count@<#2\relax
    \advance\count@ 1 %
  \repeat
}
\def\space{ }
\expandafter\ifx\csname LoadCommand\endcsname\relax
  \def\LoadCommand{\input mleftright.sty\relax}%
\fi
\def\Test{%
  \RangeCatcodeInvalid{0}{47}%
  \RangeCatcodeInvalid{58}{64}%
  \RangeCatcodeInvalid{91}{96}%
  \RangeCatcodeInvalid{123}{255}%
  \catcode`\@=12 %
  \catcode`\\=0 %
  \catcode`\%=14 %
  \LoadCommand
  \RangeCatcodeCheck{0}{36}{15}%
  \RangeCatcodeCheck{37}{37}{14}%
  \RangeCatcodeCheck{38}{47}{15}%
  \RangeCatcodeCheck{48}{57}{12}%
  \RangeCatcodeCheck{58}{63}{15}%
  \RangeCatcodeCheck{64}{64}{12}%
  \RangeCatcodeCheck{65}{90}{11}%
  \RangeCatcodeCheck{91}{91}{15}%
  \RangeCatcodeCheck{92}{92}{0}%
  \RangeCatcodeCheck{93}{96}{15}%
  \RangeCatcodeCheck{97}{122}{11}%
  \RangeCatcodeCheck{123}{255}{15}%
  \RestoreCatcodes
}
\Test
\csname @@end\endcsname
\end
%    \end{macrocode}
%    \begin{macrocode}
%</test1>
%    \end{macrocode}
%
% \section{Installation}
%
% \subsection{Download}
%
% \paragraph{Package.} This package is available on
% CTAN\footnote{\url{http://ctan.org/pkg/mleftright}}:
% \begin{description}
% \item[\CTAN{macros/latex/contrib/oberdiek/mleftright.dtx}] The source file.
% \item[\CTAN{macros/latex/contrib/oberdiek/mleftright.pdf}] Documentation.
% \end{description}
%
%
% \paragraph{Bundle.} All the packages of the bundle `oberdiek'
% are also available in a TDS compliant ZIP archive. There
% the packages are already unpacked and the documentation files
% are generated. The files and directories obey the TDS standard.
% \begin{description}
% \item[\CTAN{install/macros/latex/contrib/oberdiek.tds.zip}]
% \end{description}
% \emph{TDS} refers to the standard ``A Directory Structure
% for \TeX\ Files'' (\CTAN{tds/tds.pdf}). Directories
% with \xfile{texmf} in their name are usually organized this way.
%
% \subsection{Bundle installation}
%
% \paragraph{Unpacking.} Unpack the \xfile{oberdiek.tds.zip} in the
% TDS tree (also known as \xfile{texmf} tree) of your choice.
% Example (linux):
% \begin{quote}
%   |unzip oberdiek.tds.zip -d ~/texmf|
% \end{quote}
%
% \paragraph{Script installation.}
% Check the directory \xfile{TDS:scripts/oberdiek/} for
% scripts that need further installation steps.
% Package \xpackage{attachfile2} comes with the Perl script
% \xfile{pdfatfi.pl} that should be installed in such a way
% that it can be called as \texttt{pdfatfi}.
% Example (linux):
% \begin{quote}
%   |chmod +x scripts/oberdiek/pdfatfi.pl|\\
%   |cp scripts/oberdiek/pdfatfi.pl /usr/local/bin/|
% \end{quote}
%
% \subsection{Package installation}
%
% \paragraph{Unpacking.} The \xfile{.dtx} file is a self-extracting
% \docstrip\ archive. The files are extracted by running the
% \xfile{.dtx} through \plainTeX:
% \begin{quote}
%   \verb|tex mleftright.dtx|
% \end{quote}
%
% \paragraph{TDS.} Now the different files must be moved into
% the different directories in your installation TDS tree
% (also known as \xfile{texmf} tree):
% \begin{quote}
% \def\t{^^A
% \begin{tabular}{@{}>{\ttfamily}l@{ $\rightarrow$ }>{\ttfamily}l@{}}
%   mleftright.sty & tex/generic/oberdiek/mleftright.sty\\
%   mleftright.pdf & doc/latex/oberdiek/mleftright.pdf\\
%   test/mleftright-test1.tex & doc/latex/oberdiek/test/mleftright-test1.tex\\
%   mleftright.dtx & source/latex/oberdiek/mleftright.dtx\\
% \end{tabular}^^A
% }^^A
% \sbox0{\t}^^A
% \ifdim\wd0>\linewidth
%   \begingroup
%     \advance\linewidth by\leftmargin
%     \advance\linewidth by\rightmargin
%   \edef\x{\endgroup
%     \def\noexpand\lw{\the\linewidth}^^A
%   }\x
%   \def\lwbox{^^A
%     \leavevmode
%     \hbox to \linewidth{^^A
%       \kern-\leftmargin\relax
%       \hss
%       \usebox0
%       \hss
%       \kern-\rightmargin\relax
%     }^^A
%   }^^A
%   \ifdim\wd0>\lw
%     \sbox0{\small\t}^^A
%     \ifdim\wd0>\linewidth
%       \ifdim\wd0>\lw
%         \sbox0{\footnotesize\t}^^A
%         \ifdim\wd0>\linewidth
%           \ifdim\wd0>\lw
%             \sbox0{\scriptsize\t}^^A
%             \ifdim\wd0>\linewidth
%               \ifdim\wd0>\lw
%                 \sbox0{\tiny\t}^^A
%                 \ifdim\wd0>\linewidth
%                   \lwbox
%                 \else
%                   \usebox0
%                 \fi
%               \else
%                 \lwbox
%               \fi
%             \else
%               \usebox0
%             \fi
%           \else
%             \lwbox
%           \fi
%         \else
%           \usebox0
%         \fi
%       \else
%         \lwbox
%       \fi
%     \else
%       \usebox0
%     \fi
%   \else
%     \lwbox
%   \fi
% \else
%   \usebox0
% \fi
% \end{quote}
% If you have a \xfile{docstrip.cfg} that configures and enables \docstrip's
% TDS installing feature, then some files can already be in the right
% place, see the documentation of \docstrip.
%
% \subsection{Refresh file name databases}
%
% If your \TeX~distribution
% (\teTeX, \mikTeX, \dots) relies on file name databases, you must refresh
% these. For example, \teTeX\ users run \verb|texhash| or
% \verb|mktexlsr|.
%
% \subsection{Some details for the interested}
%
% \paragraph{Attached source.}
%
% The PDF documentation on CTAN also includes the
% \xfile{.dtx} source file. It can be extracted by
% AcrobatReader 6 or higher. Another option is \textsf{pdftk},
% e.g. unpack the file into the current directory:
% \begin{quote}
%   \verb|pdftk mleftright.pdf unpack_files output .|
% \end{quote}
%
% \paragraph{Unpacking with \LaTeX.}
% The \xfile{.dtx} chooses its action depending on the format:
% \begin{description}
% \item[\plainTeX:] Run \docstrip\ and extract the files.
% \item[\LaTeX:] Generate the documentation.
% \end{description}
% If you insist on using \LaTeX\ for \docstrip\ (really,
% \docstrip\ does not need \LaTeX), then inform the autodetect routine
% about your intention:
% \begin{quote}
%   \verb|latex \let\install=y% \iffalse meta-comment
%
% File: mleftright.dtx
% Version: 2016/05/16 v1.1
% Info: Math left/right delim. as open/close
%
% Copyright (C) 2010 by
%    Heiko Oberdiek <heiko.oberdiek at googlemail.com>
%    2016
%    https://github.com/ho-tex/oberdiek/issues
%
% This work may be distributed and/or modified under the
% conditions of the LaTeX Project Public License, either
% version 1.3c of this license or (at your option) any later
% version. This version of this license is in
%    http://www.latex-project.org/lppl/lppl-1-3c.txt
% and the latest version of this license is in
%    http://www.latex-project.org/lppl.txt
% and version 1.3 or later is part of all distributions of
% LaTeX version 2005/12/01 or later.
%
% This work has the LPPL maintenance status "maintained".
%
% This Current Maintainer of this work is Heiko Oberdiek.
%
% The Base Interpreter refers to any `TeX-Format',
% because some files are installed in TDS:tex/generic//.
%
% This work consists of the main source file mleftright.dtx
% and the derived files
%    mleftright.sty, mleftright.pdf, mleftright.ins, mleftright.drv,
%    mleftright-test1.tex.
%
% Distribution:
%    CTAN:macros/latex/contrib/oberdiek/mleftright.dtx
%    CTAN:macros/latex/contrib/oberdiek/mleftright.pdf
%
% Unpacking:
%    (a) If mleftright.ins is present:
%           tex mleftright.ins
%    (b) Without mleftright.ins:
%           tex mleftright.dtx
%    (c) If you insist on using LaTeX
%           latex \let\install=y\input{mleftright.dtx}
%        (quote the arguments according to the demands of your shell)
%
% Documentation:
%    (a) If mleftright.drv is present:
%           latex mleftright.drv
%    (b) Without mleftright.drv:
%           latex mleftright.dtx; ...
%    The class ltxdoc loads the configuration file ltxdoc.cfg
%    if available. Here you can specify further options, e.g.
%    use A4 as paper format:
%       \PassOptionsToClass{a4paper}{article}
%
%    Programm calls to get the documentation (example):
%       pdflatex mleftright.dtx
%       makeindex -s gind.ist mleftright.idx
%       pdflatex mleftright.dtx
%       makeindex -s gind.ist mleftright.idx
%       pdflatex mleftright.dtx
%
% Installation:
%    TDS:tex/generic/oberdiek/mleftright.sty
%    TDS:doc/latex/oberdiek/mleftright.pdf
%    TDS:doc/latex/oberdiek/test/mleftright-test1.tex
%    TDS:source/latex/oberdiek/mleftright.dtx
%
%<*ignore>
\begingroup
  \catcode123=1 %
  \catcode125=2 %
  \def\x{LaTeX2e}%
\expandafter\endgroup
\ifcase 0\ifx\install y1\fi\expandafter
         \ifx\csname processbatchFile\endcsname\relax\else1\fi
         \ifx\fmtname\x\else 1\fi\relax
\else\csname fi\endcsname
%</ignore>
%<*install>
\input docstrip.tex
\Msg{************************************************************************}
\Msg{* Installation}
\Msg{* Package: mleftright 2016/05/16 v1.1 Math left/right delim. as open/close (HO)}
\Msg{************************************************************************}

\keepsilent
\askforoverwritefalse

\let\MetaPrefix\relax
\preamble

This is a generated file.

Project: mleftright
Version: 2016/05/16 v1.1

Copyright (C) 2010 by
   Heiko Oberdiek <heiko.oberdiek at googlemail.com>

This work may be distributed and/or modified under the
conditions of the LaTeX Project Public License, either
version 1.3c of this license or (at your option) any later
version. This version of this license is in
   http://www.latex-project.org/lppl/lppl-1-3c.txt
and the latest version of this license is in
   http://www.latex-project.org/lppl.txt
and version 1.3 or later is part of all distributions of
LaTeX version 2005/12/01 or later.

This work has the LPPL maintenance status "maintained".

This Current Maintainer of this work is Heiko Oberdiek.

The Base Interpreter refers to any `TeX-Format',
because some files are installed in TDS:tex/generic//.

This work consists of the main source file mleftright.dtx
and the derived files
   mleftright.sty, mleftright.pdf, mleftright.ins, mleftright.drv,
   mleftright-test1.tex.

\endpreamble
\let\MetaPrefix\DoubleperCent

\generate{%
  \file{mleftright.ins}{\from{mleftright.dtx}{install}}%
  \file{mleftright.drv}{\from{mleftright.dtx}{driver}}%
  \usedir{tex/generic/oberdiek}%
  \file{mleftright.sty}{\from{mleftright.dtx}{package}}%
  \usedir{doc/latex/oberdiek/test}%
  \file{mleftright-test1.tex}{\from{mleftright.dtx}{test1}}%
  \nopreamble
  \nopostamble
  \usedir{source/latex/oberdiek/catalogue}%
  \file{mleftright.xml}{\from{mleftright.dtx}{catalogue}}%
}

\catcode32=13\relax% active space
\let =\space%
\Msg{************************************************************************}
\Msg{*}
\Msg{* To finish the installation you have to move the following}
\Msg{* file into a directory searched by TeX:}
\Msg{*}
\Msg{*     mleftright.sty}
\Msg{*}
\Msg{* To produce the documentation run the file `mleftright.drv'}
\Msg{* through LaTeX.}
\Msg{*}
\Msg{* Happy TeXing!}
\Msg{*}
\Msg{************************************************************************}

\endbatchfile
%</install>
%<*ignore>
\fi
%</ignore>
%<*driver>
\NeedsTeXFormat{LaTeX2e}
\ProvidesFile{mleftright.drv}%
  [2016/05/16 v1.1 Math left/right delim. as open/close (HO)]%
\documentclass{ltxdoc}
\usepackage{holtxdoc}[2011/11/22]
\usepackage{mleftright}[2016/05/16]
\begin{document}
  \DocInput{mleftright.dtx}%
\end{document}
%</driver>
% \fi
%
%
% \CharacterTable
%  {Upper-case    \A\B\C\D\E\F\G\H\I\J\K\L\M\N\O\P\Q\R\S\T\U\V\W\X\Y\Z
%   Lower-case    \a\b\c\d\e\f\g\h\i\j\k\l\m\n\o\p\q\r\s\t\u\v\w\x\y\z
%   Digits        \0\1\2\3\4\5\6\7\8\9
%   Exclamation   \!     Double quote  \"     Hash (number) \#
%   Dollar        \$     Percent       \%     Ampersand     \&
%   Acute accent  \'     Left paren    \(     Right paren   \)
%   Asterisk      \*     Plus          \+     Comma         \,
%   Minus         \-     Point         \.     Solidus       \/
%   Colon         \:     Semicolon     \;     Less than     \<
%   Equals        \=     Greater than  \>     Question mark \?
%   Commercial at \@     Left bracket  \[     Backslash     \\
%   Right bracket \]     Circumflex    \^     Underscore    \_
%   Grave accent  \`     Left brace    \{     Vertical bar  \|
%   Right brace   \}     Tilde         \~}
%
% \GetFileInfo{mleftright.drv}
%
% \title{The \xpackage{mleftright} package}
% \date{2016/05/16 v1.1}
% \author{Heiko Oberdiek\thanks
% {Please report any issues at https://github.com/ho-tex/oberdiek/issues}\\
% \xemail{heiko.oberdiek at googlemail.com}}
%
% \maketitle
%
% \begin{abstract}
% \TeX\ sets subformulas by \cs{left} and \cs{right} as inner formulas
% with additional surrounding spaces in some situations. This package
% provides \cs{mleft} and \cs{mright} that call \cs{left} and \cs{right},
% but the delimiters will act as normal \cs{mathopen} and \cs{mathclose}
% delimiters without the additional space of an inner formula.
% \end{abstract}
%
% \tableofcontents
%
% \section{Documentation}
%
% The package is a result of a thread in the newsgroup \textsf{comp.text.tex}
% with the subject \textit{spacing after \cs{right}\texttt{)}
% and before \cs{left}\texttt{)}} \cite{dave}.
% The problem: \cs{left} and \cs{right} adjust the size of the
% delimiters automatically. However, \TeX\ treats the whole expression
% as inner formula. In some circumstances \TeX\ adds extra space
% before or after an inner formula.
% Example:
% \begin{quote}
%   \thinmuskip=1.5\thinmuskip
%   \begin{tabular}{@{}l@{\quad$\Rightarrow$\quad}l@{}}
%     |$\sin(x^2), x$|
%     & $\sin(x^2), x$\\
%     |$\sin\left(x^2\right), x$|
%     & $\sin\left(x^2\right), x$\\
%   ^^A  \multicolumn{1}{@{}r@{\quad$\Rightarrow$\quad}}{^^A
%   ^^A    \itshape with exaggerated spacing^^A
%   ^^A  }
%   ^^A  & $\thinmuskip=4\thinmuskip
%   ^^A    \sin\left(x^2\right){,}\mskip.25\thinmuskip x$\\
%     |$\sin\mleft(x^2\mright), x$|
%     & $\sin\mleft(x^2\mright), x$\\
%   \end{tabular}\\*[.5ex]
%   (\cs{mleft} and \cs{mright} are provided by this package.)
% \end{quote}
%
% In the newsgroup Donald Arseneau answered with clever macros \cite{arseneau}:
% \begin{quote}
%\begin{verbatim}
%\newcommand\lft{\mathopen{}\left}
%\newcommand\rgt{\aftergroup\mathclose\aftergroup{\aftergroup}\right}
%\end{verbatim}
% \end{quote}
% However one problem remains, a following subscript or superscript
% is not applied to the right delimiter but the empty
% \cs{mathclose}.
% Thus Philipp Stephani provided an improvement \cite{stephani}:
%\begin{quote}
%\begin{verbatim}
%\mathopen{} \mathclose{\left\| A^2 \right\|}_2
%\end{verbatim}
%\end{quote}
% Heiko Oberdiek converted this into macro form \cite{oberdiek}:
%\begin{quote}
%\begin{verbatim}
%\newcommand\lft{\mathopen{}\mathclose\bgroup\left}
%\newcommand\rgt{\aftergroup\egroup\right}
%\end{verbatim}
%\end{quote}
%
% The package uses longer macro names \cs{mleft} and \cs{mright}
% to avoid name clashes. Also it adds some checks for error conditions.
%
% \subsection{Use}
%
% \begin{declcs}{mleft}\meta{delimL} \dots\unkern\ \cs{mright}\meta{delimR}
% \end{declcs}
% Macros \cs{mleft} and \cs{mright} are used in the same way as
% \cs{left} and \cs{right}. Also \cs{middle} can be used inbetween if
% \eTeX\ is present.
%
% \begin{declcs}{mleftright}
% \end{declcs}
% Macro \cs{mleftright} redefines \cs{left} as \cs{mleft} and
% \cs{right} as \cs{mright}. The redefinition is local to the group.
%
% \begin{declcs}{mleftrightrestore}
% \end{declcs}
% Macro \cs{mleftright} restores \cs{left} and \cs{right} with
% the original meaning if they were previously redefined by
% \cs{mleftright} (also locally).
%
%
% \StopEventually{
% }
%
% \section{Implementation}
%    \begin{macrocode}
%<*package>
%    \end{macrocode}
%    Reload check, especially if the package is not used with \LaTeX.
%    \begin{macrocode}
\begingroup\catcode61\catcode48\catcode32=10\relax%
  \catcode13=5 % ^^M
  \endlinechar=13 %
  \catcode35=6 % #
  \catcode39=12 % '
  \catcode44=12 % ,
  \catcode45=12 % -
  \catcode46=12 % .
  \catcode58=12 % :
  \catcode64=11 % @
  \catcode123=1 % {
  \catcode125=2 % }
  \expandafter\let\expandafter\x\csname ver@mleftright.sty\endcsname
  \ifx\x\relax % plain-TeX, first loading
  \else
    \def\empty{}%
    \ifx\x\empty % LaTeX, first loading,
      % variable is initialized, but \ProvidesPackage not yet seen
    \else
      \expandafter\ifx\csname PackageInfo\endcsname\relax
        \def\x#1#2{%
          \immediate\write-1{Package #1 Info: #2.}%
        }%
      \else
        \def\x#1#2{\PackageInfo{#1}{#2, stopped}}%
      \fi
      \x{mleftright}{The package is already loaded}%
      \aftergroup\endinput
    \fi
  \fi
\endgroup%
%    \end{macrocode}
%    Package identification:
%    \begin{macrocode}
\begingroup\catcode61\catcode48\catcode32=10\relax%
  \catcode13=5 % ^^M
  \endlinechar=13 %
  \catcode35=6 % #
  \catcode39=12 % '
  \catcode40=12 % (
  \catcode41=12 % )
  \catcode44=12 % ,
  \catcode45=12 % -
  \catcode46=12 % .
  \catcode47=12 % /
  \catcode58=12 % :
  \catcode64=11 % @
  \catcode91=12 % [
  \catcode93=12 % ]
  \catcode123=1 % {
  \catcode125=2 % }
  \expandafter\ifx\csname ProvidesPackage\endcsname\relax
    \def\x#1#2#3[#4]{\endgroup
      \immediate\write-1{Package: #3 #4}%
      \xdef#1{#4}%
    }%
  \else
    \def\x#1#2[#3]{\endgroup
      #2[{#3}]%
      \ifx#1\@undefined
        \xdef#1{#3}%
      \fi
      \ifx#1\relax
        \xdef#1{#3}%
      \fi
    }%
  \fi
\expandafter\x\csname ver@mleftright.sty\endcsname
\ProvidesPackage{mleftright}%
  [2016/05/16 v1.1 Math left/right delim. as open/close (HO)]%
%    \end{macrocode}
%
%    \begin{macrocode}
\begingroup\catcode61\catcode48\catcode32=10\relax%
  \catcode13=5 % ^^M
  \endlinechar=13 %
  \catcode123=1 % {
  \catcode125=2 % }
  \catcode64=11 % @
  \def\x{\endgroup
    \expandafter\edef\csname mleftright@AtEnd\endcsname{%
      \endlinechar=\the\endlinechar\relax
      \catcode13=\the\catcode13\relax
      \catcode32=\the\catcode32\relax
      \catcode35=\the\catcode35\relax
      \catcode61=\the\catcode61\relax
      \catcode64=\the\catcode64\relax
      \catcode123=\the\catcode123\relax
      \catcode125=\the\catcode125\relax
    }%
  }%
\x\catcode61\catcode48\catcode32=10\relax%
\catcode13=5 % ^^M
\endlinechar=13 %
\catcode35=6 % #
\catcode64=11 % @
\catcode123=1 % {
\catcode125=2 % }
\def\TMP@EnsureCode#1#2{%
  \edef\mleftright@AtEnd{%
    \mleftright@AtEnd
    \catcode#1=\the\catcode#1\relax
  }%
  \catcode#1=#2\relax
}
\TMP@EnsureCode{38}{4}% &
\TMP@EnsureCode{39}{12}% '
\TMP@EnsureCode{40}{12}% (
\TMP@EnsureCode{41}{12}% )
\TMP@EnsureCode{42}{12}% *
\TMP@EnsureCode{43}{12}% +
\TMP@EnsureCode{44}{12}% ,
\TMP@EnsureCode{45}{12}% -
\TMP@EnsureCode{46}{12}% .
\TMP@EnsureCode{47}{12}% /
\TMP@EnsureCode{60}{12}% <
\TMP@EnsureCode{91}{12}% [
\TMP@EnsureCode{93}{12}% ]
\edef\mleftright@AtEnd{%
  \mleftright@AtEnd
  \escapechar\the\escapechar\relax
  \noexpand\endinput
}
\escapechar=92 %
%    \end{macrocode}
%
%    \begin{macrocode}
\begingroup\expandafter\expandafter\expandafter\endgroup
\expandafter\ifx\csname RequirePackage\endcsname\relax
  \input infwarerr.sty\relax
  \input ltxcmds.sty\relax
\else
  \RequirePackage{infwarerr}[2010/04/08]%
  \RequirePackage{ltxcmds}[2010/04/26]%
\fi
%    \end{macrocode}
%
%    The original commands \cs{left} and \cs{right}
%    are saved and later used in \cs{mleft} and
%    \cs{mright} in order to deal with:
%    \begin{quote}
%\begin{verbatim}
%\let\left\mleft
%\let\right\mright
%\end{verbatim}
%    \end{quote}
%    \begin{macro}{\mleftright@OrgLeft}
%    \begin{macrocode}
\let\mleftright@OrgLeft\left
%    \end{macrocode}
%    \end{macro}
%    \begin{macro}{\mleftright@OrgRight}
%    \begin{macrocode}
\let\mleftright@OrgRight\right
%    \end{macrocode}
%    \end{macro}
%
%    \begin{macro}{\mleftright@Def}
%    Macro \cs{mleftright@Def} defines a macro as robust macro
%    if \eTeX\ or \LaTeX\ is available.
%    \begin{macrocode}
\ltx@IfUndefined{protected}{%
  \ltx@IfUndefined{DeclareRobustCommand}{%
    \def\mleftright@Def{\def}%
  }{%
    \def\mleftright@Def{\DeclareRobustCommand*}%
  }%
}{%
  \def\mleftright@Def{\protected\def}%
}
\edef\mleftright@Def#1{%
  \noexpand\ltx@IfUndefined{%
    \noexpand\expandafter\noexpand\ltx@gobble\noexpand\string#1%
  }{%
    \expandafter\noexpand\mleftright@Def#1%
  }{%
    \noexpand\@PackageError{mleftright}{%
      Command \noexpand\string#1 already defined%
    }\noexpand\@ehd
    \noexpand\ltx@gobble
  }%
}
%    \end{macrocode}
%    \end{macro}
%
%    In case of \eTeX\ the group status after the left symbol
%    is saved and later checked at the beginning of \cs{mright}.
%    \begin{macrocode}
\ltx@IfUndefined{currentgrouplevel}{%
  \catcode38=14 % & = comment
}{%
  \catcode38=9 % & = ignore
}
%    \end{macrocode}
%
%    \begin{macro}{\mleftright@GroupLevel}
%    \begin{macrocode}
& \def\mleftright@GroupLevel{-1}%
%    \end{macrocode}
%    \end{macro}
%
%    \begin{macro}{\mleftright@WrongGroup}
%    \begin{macrocode}
& \def\mleftright@WrongGroup#1(#2){%
&   \ifnum\mleftright@GroupLevel<\ltx@zero
&     \@PackageError{mleftright}{%
&       Missing previous \string\mleft
&     }\@ehc
&   \else
&     \@PackageError{mleftright}{%
&       Unexpected group status for \string\mright%
&       \ifnum\mleftright@GroupLevel=#1 %
&       \else
&         .\MessageBreak
&         Group level is #1, %
&           expected is \mleftright@GroupLevel
&       \fi
&       \ifnum16=#2 %
&       \else
&         .\MessageBreak
&         Group type is #2 (%
&         \ifcase#2 %
&           bottom level%
&           \expandafter\expandafter\expandafter\ltx@gobblefour
&           \expandafter\ltx@gobbletwo
&         \or simple%
&         \or hbox%
&         \or adjusted hbox%
&         \or vbox%
&         \or vtop%
&         \or align%
&         \or no align%
&         \or output%
&         \or math%
&         \or disc%
&         \or insert%
&         \or vcenter%
&         \or math choice%
&         \or semi simple%
&         \or math shift%
&         \or math left%
&         \else
&           unknown%
&         \fi
&         \space group),\MessageBreak
&         expected is 16 (math left group)%
&       \fi
&     }\@ehd
&   \fi
& }%
%    \end{macrocode}
%    \end{macro}
%
%    \begin{macro}{\mleft}
%    \begin{macrocode}
\mleftright@Def\mleft{%
  \mathopen{}\mathclose\bgroup
& \edef\mleftright@GroupLevel{\the\numexpr\the\currentgrouplevel+1}%
  \mleftright@OrgLeft
}
%    \end{macrocode}
%    \end{macro}
%    \begin{macro}{\mright}
%    \begin{macrocode}
\mleftright@Def\mright{%
& \ifnum\mleftright@GroupLevel=\currentgrouplevel
&   \ifnum16=\currentgrouptype
      \aftergroup\egroup
&   \else
&     \expandafter\mleftright@WrongGroup
&     \the\expandafter\currentgrouplevel
&     \expandafter(\the\currentgrouptype)%
&   \fi
& \else
&   \expandafter\mleftright@WrongGroup
&   \the\expandafter\currentgrouplevel
&   \expandafter(\the\currentgrouptype)%
& \fi
  \mleftright@OrgRight
}
%    \end{macrocode}
%    \end{macro}
%
%    \begin{macro}{\mleftright}
%    \begin{macrocode}
\mleftright@Def\mleftright{%
  \let\left\mleft
  \let\right\mright
}
%    \end{macrocode}
%    \end{macro}
%
%    \begin{macro}{\mleftrightrestore}
%    \begin{macrocode}
\mleftright@Def\mleftrightrestore{%
  \ifx\left\mleft
    \let\left\mleftright@OrgLeft
  \fi
  \ifx\right\mright
    \let\right\mleftright@OrgRight
  \fi
}
%    \end{macrocode}
%    \end{macro}
%
%    \begin{macrocode}
\mleftright@AtEnd%
%</package>
%    \end{macrocode}
%
% \section{Test}
%
% \subsection{Catcode checks for loading}
%
%    \begin{macrocode}
%<*test1>
%    \end{macrocode}
%    \begin{macrocode}
\catcode`\{=1 %
\catcode`\}=2 %
\catcode`\#=6 %
\catcode`\@=11 %
\expandafter\ifx\csname count@\endcsname\relax
  \countdef\count@=255 %
\fi
\expandafter\ifx\csname @gobble\endcsname\relax
  \long\def\@gobble#1{}%
\fi
\expandafter\ifx\csname @firstofone\endcsname\relax
  \long\def\@firstofone#1{#1}%
\fi
\expandafter\ifx\csname loop\endcsname\relax
  \expandafter\@firstofone
\else
  \expandafter\@gobble
\fi
{%
  \def\loop#1\repeat{%
    \def\body{#1}%
    \iterate
  }%
  \def\iterate{%
    \body
      \let\next\iterate
    \else
      \let\next\relax
    \fi
    \next
  }%
  \let\repeat=\fi
}%
\def\RestoreCatcodes{}
\count@=0 %
\loop
  \edef\RestoreCatcodes{%
    \RestoreCatcodes
    \catcode\the\count@=\the\catcode\count@\relax
  }%
\ifnum\count@<255 %
  \advance\count@ 1 %
\repeat

\def\RangeCatcodeInvalid#1#2{%
  \count@=#1\relax
  \loop
    \catcode\count@=15 %
  \ifnum\count@<#2\relax
    \advance\count@ 1 %
  \repeat
}
\def\RangeCatcodeCheck#1#2#3{%
  \count@=#1\relax
  \loop
    \ifnum#3=\catcode\count@
    \else
      \errmessage{%
        Character \the\count@\space
        with wrong catcode \the\catcode\count@\space
        instead of \number#3%
      }%
    \fi
  \ifnum\count@<#2\relax
    \advance\count@ 1 %
  \repeat
}
\def\space{ }
\expandafter\ifx\csname LoadCommand\endcsname\relax
  \def\LoadCommand{\input mleftright.sty\relax}%
\fi
\def\Test{%
  \RangeCatcodeInvalid{0}{47}%
  \RangeCatcodeInvalid{58}{64}%
  \RangeCatcodeInvalid{91}{96}%
  \RangeCatcodeInvalid{123}{255}%
  \catcode`\@=12 %
  \catcode`\\=0 %
  \catcode`\%=14 %
  \LoadCommand
  \RangeCatcodeCheck{0}{36}{15}%
  \RangeCatcodeCheck{37}{37}{14}%
  \RangeCatcodeCheck{38}{47}{15}%
  \RangeCatcodeCheck{48}{57}{12}%
  \RangeCatcodeCheck{58}{63}{15}%
  \RangeCatcodeCheck{64}{64}{12}%
  \RangeCatcodeCheck{65}{90}{11}%
  \RangeCatcodeCheck{91}{91}{15}%
  \RangeCatcodeCheck{92}{92}{0}%
  \RangeCatcodeCheck{93}{96}{15}%
  \RangeCatcodeCheck{97}{122}{11}%
  \RangeCatcodeCheck{123}{255}{15}%
  \RestoreCatcodes
}
\Test
\csname @@end\endcsname
\end
%    \end{macrocode}
%    \begin{macrocode}
%</test1>
%    \end{macrocode}
%
% \section{Installation}
%
% \subsection{Download}
%
% \paragraph{Package.} This package is available on
% CTAN\footnote{\url{http://ctan.org/pkg/mleftright}}:
% \begin{description}
% \item[\CTAN{macros/latex/contrib/oberdiek/mleftright.dtx}] The source file.
% \item[\CTAN{macros/latex/contrib/oberdiek/mleftright.pdf}] Documentation.
% \end{description}
%
%
% \paragraph{Bundle.} All the packages of the bundle `oberdiek'
% are also available in a TDS compliant ZIP archive. There
% the packages are already unpacked and the documentation files
% are generated. The files and directories obey the TDS standard.
% \begin{description}
% \item[\CTAN{install/macros/latex/contrib/oberdiek.tds.zip}]
% \end{description}
% \emph{TDS} refers to the standard ``A Directory Structure
% for \TeX\ Files'' (\CTAN{tds/tds.pdf}). Directories
% with \xfile{texmf} in their name are usually organized this way.
%
% \subsection{Bundle installation}
%
% \paragraph{Unpacking.} Unpack the \xfile{oberdiek.tds.zip} in the
% TDS tree (also known as \xfile{texmf} tree) of your choice.
% Example (linux):
% \begin{quote}
%   |unzip oberdiek.tds.zip -d ~/texmf|
% \end{quote}
%
% \paragraph{Script installation.}
% Check the directory \xfile{TDS:scripts/oberdiek/} for
% scripts that need further installation steps.
% Package \xpackage{attachfile2} comes with the Perl script
% \xfile{pdfatfi.pl} that should be installed in such a way
% that it can be called as \texttt{pdfatfi}.
% Example (linux):
% \begin{quote}
%   |chmod +x scripts/oberdiek/pdfatfi.pl|\\
%   |cp scripts/oberdiek/pdfatfi.pl /usr/local/bin/|
% \end{quote}
%
% \subsection{Package installation}
%
% \paragraph{Unpacking.} The \xfile{.dtx} file is a self-extracting
% \docstrip\ archive. The files are extracted by running the
% \xfile{.dtx} through \plainTeX:
% \begin{quote}
%   \verb|tex mleftright.dtx|
% \end{quote}
%
% \paragraph{TDS.} Now the different files must be moved into
% the different directories in your installation TDS tree
% (also known as \xfile{texmf} tree):
% \begin{quote}
% \def\t{^^A
% \begin{tabular}{@{}>{\ttfamily}l@{ $\rightarrow$ }>{\ttfamily}l@{}}
%   mleftright.sty & tex/generic/oberdiek/mleftright.sty\\
%   mleftright.pdf & doc/latex/oberdiek/mleftright.pdf\\
%   test/mleftright-test1.tex & doc/latex/oberdiek/test/mleftright-test1.tex\\
%   mleftright.dtx & source/latex/oberdiek/mleftright.dtx\\
% \end{tabular}^^A
% }^^A
% \sbox0{\t}^^A
% \ifdim\wd0>\linewidth
%   \begingroup
%     \advance\linewidth by\leftmargin
%     \advance\linewidth by\rightmargin
%   \edef\x{\endgroup
%     \def\noexpand\lw{\the\linewidth}^^A
%   }\x
%   \def\lwbox{^^A
%     \leavevmode
%     \hbox to \linewidth{^^A
%       \kern-\leftmargin\relax
%       \hss
%       \usebox0
%       \hss
%       \kern-\rightmargin\relax
%     }^^A
%   }^^A
%   \ifdim\wd0>\lw
%     \sbox0{\small\t}^^A
%     \ifdim\wd0>\linewidth
%       \ifdim\wd0>\lw
%         \sbox0{\footnotesize\t}^^A
%         \ifdim\wd0>\linewidth
%           \ifdim\wd0>\lw
%             \sbox0{\scriptsize\t}^^A
%             \ifdim\wd0>\linewidth
%               \ifdim\wd0>\lw
%                 \sbox0{\tiny\t}^^A
%                 \ifdim\wd0>\linewidth
%                   \lwbox
%                 \else
%                   \usebox0
%                 \fi
%               \else
%                 \lwbox
%               \fi
%             \else
%               \usebox0
%             \fi
%           \else
%             \lwbox
%           \fi
%         \else
%           \usebox0
%         \fi
%       \else
%         \lwbox
%       \fi
%     \else
%       \usebox0
%     \fi
%   \else
%     \lwbox
%   \fi
% \else
%   \usebox0
% \fi
% \end{quote}
% If you have a \xfile{docstrip.cfg} that configures and enables \docstrip's
% TDS installing feature, then some files can already be in the right
% place, see the documentation of \docstrip.
%
% \subsection{Refresh file name databases}
%
% If your \TeX~distribution
% (\teTeX, \mikTeX, \dots) relies on file name databases, you must refresh
% these. For example, \teTeX\ users run \verb|texhash| or
% \verb|mktexlsr|.
%
% \subsection{Some details for the interested}
%
% \paragraph{Attached source.}
%
% The PDF documentation on CTAN also includes the
% \xfile{.dtx} source file. It can be extracted by
% AcrobatReader 6 or higher. Another option is \textsf{pdftk},
% e.g. unpack the file into the current directory:
% \begin{quote}
%   \verb|pdftk mleftright.pdf unpack_files output .|
% \end{quote}
%
% \paragraph{Unpacking with \LaTeX.}
% The \xfile{.dtx} chooses its action depending on the format:
% \begin{description}
% \item[\plainTeX:] Run \docstrip\ and extract the files.
% \item[\LaTeX:] Generate the documentation.
% \end{description}
% If you insist on using \LaTeX\ for \docstrip\ (really,
% \docstrip\ does not need \LaTeX), then inform the autodetect routine
% about your intention:
% \begin{quote}
%   \verb|latex \let\install=y\input{mleftright.dtx}|
% \end{quote}
% Do not forget to quote the argument according to the demands
% of your shell.
%
% \paragraph{Generating the documentation.}
% You can use both the \xfile{.dtx} or the \xfile{.drv} to generate
% the documentation. The process can be configured by the
% configuration file \xfile{ltxdoc.cfg}. For instance, put this
% line into this file, if you want to have A4 as paper format:
% \begin{quote}
%   \verb|\PassOptionsToClass{a4paper}{article}|
% \end{quote}
% An example follows how to generate the
% documentation with pdf\LaTeX:
% \begin{quote}
%\begin{verbatim}
%pdflatex mleftright.dtx
%makeindex -s gind.ist mleftright.idx
%pdflatex mleftright.dtx
%makeindex -s gind.ist mleftright.idx
%pdflatex mleftright.dtx
%\end{verbatim}
% \end{quote}
%
% \section{Catalogue}
%
% The following XML file can be used as source for the
% \href{http://mirror.ctan.org/help/Catalogue/catalogue.html}{\TeX\ Catalogue}.
% The elements \texttt{caption} and \texttt{description} are imported
% from the original XML file from the Catalogue.
% The name of the XML file in the Catalogue is \xfile{mleftright.xml}.
%    \begin{macrocode}
%<*catalogue>
<?xml version='1.0' encoding='us-ascii'?>
<!DOCTYPE entry SYSTEM 'catalogue.dtd'>
<entry datestamp='$Date$' modifier='$Author$' id='mleftright'>
  <name>mleftright</name>
  <caption>Variants of delimiters that act as maths open/close.</caption>
  <authorref id='auth:oberdiek'/>
  <copyright owner='Heiko Oberdiek' year='2010'/>
  <license type='lppl1.3'/>
  <version number='1.1'/>
  <description>
    The package defines variants <tt>\mleft</tt> and <tt>\mright</tt>
    of <tt>\left</tt> and <tt>\right</tt>, that make the delimiters
    act as <tt>\mathopen</tt> and <tt>\mathclose</tt>.  These commands
    address spacing difficulties in subformulas.
    <p/>
    The package is part of the <xref refid='oberdiek'>oberdiek</xref> bundle.
  </description>
  <documentation details='Package documentation'
      href='ctan:/macros/latex/contrib/oberdiek/mleftright.pdf'/>
  <ctan file='true' path='/macros/latex/contrib/oberdiek/mleftright.dtx'/>
  <miktex location='oberdiek'/>
  <texlive location='oberdiek'/>
  <install path='/macros/latex/contrib/oberdiek/oberdiek.tds.zip'/>
</entry>
%</catalogue>
%    \end{macrocode}
%
% \section{Acknowledgement}
%
% \begin{description}
% \item[Donald Arsenau:]
% He provided the main trick and the first macros.
% \item[Philipp Stephani:]
% He solved the subscript problem.
% \end{description}
%
% \begin{thebibliography}{9}
% \raggedright
% \bibitem{dave}
%   Dave94705,
%   \textit{spacing after \cs{right}\texttt{)} and before \cs{left}\texttt{)}},
%   newsgroup comp.text.tex,
%   Message-ID: \texttt{\small 5d264909-7c3d-4c9d-9b22-434178b2bf90@g21g2000prn.googlegroups.com},
%   2010-08-12.
%   \newblock
%   {\small\url{http://groups.google.com/group/comp.text.tex/msg/e5b6833da7dc29bf}}
%
% \bibitem{arseneau}
%   Donald Arseneau,
%   \textit{Re: spacing after \cs{right}\texttt) and before \cs{left}\texttt)},
%   newsgroup comp.text.tex,
%   Message-ID: \texttt{\small yfivd6svl8y.fsf@mutant.triumf.ca},
%   2010-08-30.
%   \newblock
%   {\small\url{http://groups.google.com/group/comp.text.tex/msg/e0b2e4386e5d04e4}}
%
% \bibitem{stephani}
%   Philipp Stephani,
%   \textit{Re: spacing after \cs{right}\texttt) and before \cs{left}\texttt)},
%   newsgroup comp.text.tex,
%   Message-ID: \texttt{\small 4c8c8c1e\$0\$6981\$9b4e6d93@newsspool4.arcor-online.net},
%   2010-09-12.
%   \newblock
%   {\small\url{http://groups.google.com/group/comp.text.tex/msg/87ac1f61321de3ef}}
%
% \bibitem{oberdiek}
%   Heiko Oberdiek,
%   \textit{Re: spacing after \cs{right}\texttt) and before \cs{left}\texttt)},
%   newsgroup comp.text.tex,
%   Message-ID: \texttt{\small i6jcc2\$8of\$1@news.eternal-september.org},
%   2010-09-12.
%   \newblock
%   {\small\url{http://groups.google.com/group/comp.text.tex/msg/257aa6119bef878b}}
%
% \end{thebibliography}
%
% \begin{History}
%   \begin{Version}{2010/09/25 v1.0}
%   \item
%     The first version.
%   \end{Version}
%   \begin{Version}{2016/05/16 v1.1}
%   \item
%     Documentation updates.
%   \end{Version}
% \end{History}
%
% \PrintIndex
%
% \Finale
\endinput
|
% \end{quote}
% Do not forget to quote the argument according to the demands
% of your shell.
%
% \paragraph{Generating the documentation.}
% You can use both the \xfile{.dtx} or the \xfile{.drv} to generate
% the documentation. The process can be configured by the
% configuration file \xfile{ltxdoc.cfg}. For instance, put this
% line into this file, if you want to have A4 as paper format:
% \begin{quote}
%   \verb|\PassOptionsToClass{a4paper}{article}|
% \end{quote}
% An example follows how to generate the
% documentation with pdf\LaTeX:
% \begin{quote}
%\begin{verbatim}
%pdflatex mleftright.dtx
%makeindex -s gind.ist mleftright.idx
%pdflatex mleftright.dtx
%makeindex -s gind.ist mleftright.idx
%pdflatex mleftright.dtx
%\end{verbatim}
% \end{quote}
%
% \section{Catalogue}
%
% The following XML file can be used as source for the
% \href{http://mirror.ctan.org/help/Catalogue/catalogue.html}{\TeX\ Catalogue}.
% The elements \texttt{caption} and \texttt{description} are imported
% from the original XML file from the Catalogue.
% The name of the XML file in the Catalogue is \xfile{mleftright.xml}.
%    \begin{macrocode}
%<*catalogue>
<?xml version='1.0' encoding='us-ascii'?>
<!DOCTYPE entry SYSTEM 'catalogue.dtd'>
<entry datestamp='$Date$' modifier='$Author$' id='mleftright'>
  <name>mleftright</name>
  <caption>Variants of delimiters that act as maths open/close.</caption>
  <authorref id='auth:oberdiek'/>
  <copyright owner='Heiko Oberdiek' year='2010'/>
  <license type='lppl1.3'/>
  <version number='1.1'/>
  <description>
    The package defines variants <tt>\mleft</tt> and <tt>\mright</tt>
    of <tt>\left</tt> and <tt>\right</tt>, that make the delimiters
    act as <tt>\mathopen</tt> and <tt>\mathclose</tt>.  These commands
    address spacing difficulties in subformulas.
    <p/>
    The package is part of the <xref refid='oberdiek'>oberdiek</xref> bundle.
  </description>
  <documentation details='Package documentation'
      href='ctan:/macros/latex/contrib/oberdiek/mleftright.pdf'/>
  <ctan file='true' path='/macros/latex/contrib/oberdiek/mleftright.dtx'/>
  <miktex location='oberdiek'/>
  <texlive location='oberdiek'/>
  <install path='/macros/latex/contrib/oberdiek/oberdiek.tds.zip'/>
</entry>
%</catalogue>
%    \end{macrocode}
%
% \section{Acknowledgement}
%
% \begin{description}
% \item[Donald Arsenau:]
% He provided the main trick and the first macros.
% \item[Philipp Stephani:]
% He solved the subscript problem.
% \end{description}
%
% \begin{thebibliography}{9}
% \raggedright
% \bibitem{dave}
%   Dave94705,
%   \textit{spacing after \cs{right}\texttt{)} and before \cs{left}\texttt{)}},
%   newsgroup comp.text.tex,
%   Message-ID: \texttt{\small 5d264909-7c3d-4c9d-9b22-434178b2bf90@g21g2000prn.googlegroups.com},
%   2010-08-12.
%   \newblock
%   {\small\url{http://groups.google.com/group/comp.text.tex/msg/e5b6833da7dc29bf}}
%
% \bibitem{arseneau}
%   Donald Arseneau,
%   \textit{Re: spacing after \cs{right}\texttt) and before \cs{left}\texttt)},
%   newsgroup comp.text.tex,
%   Message-ID: \texttt{\small yfivd6svl8y.fsf@mutant.triumf.ca},
%   2010-08-30.
%   \newblock
%   {\small\url{http://groups.google.com/group/comp.text.tex/msg/e0b2e4386e5d04e4}}
%
% \bibitem{stephani}
%   Philipp Stephani,
%   \textit{Re: spacing after \cs{right}\texttt) and before \cs{left}\texttt)},
%   newsgroup comp.text.tex,
%   Message-ID: \texttt{\small 4c8c8c1e\$0\$6981\$9b4e6d93@newsspool4.arcor-online.net},
%   2010-09-12.
%   \newblock
%   {\small\url{http://groups.google.com/group/comp.text.tex/msg/87ac1f61321de3ef}}
%
% \bibitem{oberdiek}
%   Heiko Oberdiek,
%   \textit{Re: spacing after \cs{right}\texttt) and before \cs{left}\texttt)},
%   newsgroup comp.text.tex,
%   Message-ID: \texttt{\small i6jcc2\$8of\$1@news.eternal-september.org},
%   2010-09-12.
%   \newblock
%   {\small\url{http://groups.google.com/group/comp.text.tex/msg/257aa6119bef878b}}
%
% \end{thebibliography}
%
% \begin{History}
%   \begin{Version}{2010/09/25 v1.0}
%   \item
%     The first version.
%   \end{Version}
%   \begin{Version}{2016/05/16 v1.1}
%   \item
%     Documentation updates.
%   \end{Version}
% \end{History}
%
% \PrintIndex
%
% \Finale
\endinput

%        (quote the arguments according to the demands of your shell)
%
% Documentation:
%    (a) If mleftright.drv is present:
%           latex mleftright.drv
%    (b) Without mleftright.drv:
%           latex mleftright.dtx; ...
%    The class ltxdoc loads the configuration file ltxdoc.cfg
%    if available. Here you can specify further options, e.g.
%    use A4 as paper format:
%       \PassOptionsToClass{a4paper}{article}
%
%    Programm calls to get the documentation (example):
%       pdflatex mleftright.dtx
%       makeindex -s gind.ist mleftright.idx
%       pdflatex mleftright.dtx
%       makeindex -s gind.ist mleftright.idx
%       pdflatex mleftright.dtx
%
% Installation:
%    TDS:tex/generic/oberdiek/mleftright.sty
%    TDS:doc/latex/oberdiek/mleftright.pdf
%    TDS:doc/latex/oberdiek/test/mleftright-test1.tex
%    TDS:source/latex/oberdiek/mleftright.dtx
%
%<*ignore>
\begingroup
  \catcode123=1 %
  \catcode125=2 %
  \def\x{LaTeX2e}%
\expandafter\endgroup
\ifcase 0\ifx\install y1\fi\expandafter
         \ifx\csname processbatchFile\endcsname\relax\else1\fi
         \ifx\fmtname\x\else 1\fi\relax
\else\csname fi\endcsname
%</ignore>
%<*install>
\input docstrip.tex
\Msg{************************************************************************}
\Msg{* Installation}
\Msg{* Package: mleftright 2016/05/16 v1.1 Math left/right delim. as open/close (HO)}
\Msg{************************************************************************}

\keepsilent
\askforoverwritefalse

\let\MetaPrefix\relax
\preamble

This is a generated file.

Project: mleftright
Version: 2016/05/16 v1.1

Copyright (C) 2010 by
   Heiko Oberdiek <heiko.oberdiek at googlemail.com>

This work may be distributed and/or modified under the
conditions of the LaTeX Project Public License, either
version 1.3c of this license or (at your option) any later
version. This version of this license is in
   http://www.latex-project.org/lppl/lppl-1-3c.txt
and the latest version of this license is in
   http://www.latex-project.org/lppl.txt
and version 1.3 or later is part of all distributions of
LaTeX version 2005/12/01 or later.

This work has the LPPL maintenance status "maintained".

This Current Maintainer of this work is Heiko Oberdiek.

The Base Interpreter refers to any `TeX-Format',
because some files are installed in TDS:tex/generic//.

This work consists of the main source file mleftright.dtx
and the derived files
   mleftright.sty, mleftright.pdf, mleftright.ins, mleftright.drv,
   mleftright-test1.tex.

\endpreamble
\let\MetaPrefix\DoubleperCent

\generate{%
  \file{mleftright.ins}{\from{mleftright.dtx}{install}}%
  \file{mleftright.drv}{\from{mleftright.dtx}{driver}}%
  \usedir{tex/generic/oberdiek}%
  \file{mleftright.sty}{\from{mleftright.dtx}{package}}%
  \usedir{doc/latex/oberdiek/test}%
  \file{mleftright-test1.tex}{\from{mleftright.dtx}{test1}}%
  \nopreamble
  \nopostamble
  \usedir{source/latex/oberdiek/catalogue}%
  \file{mleftright.xml}{\from{mleftright.dtx}{catalogue}}%
}

\catcode32=13\relax% active space
\let =\space%
\Msg{************************************************************************}
\Msg{*}
\Msg{* To finish the installation you have to move the following}
\Msg{* file into a directory searched by TeX:}
\Msg{*}
\Msg{*     mleftright.sty}
\Msg{*}
\Msg{* To produce the documentation run the file `mleftright.drv'}
\Msg{* through LaTeX.}
\Msg{*}
\Msg{* Happy TeXing!}
\Msg{*}
\Msg{************************************************************************}

\endbatchfile
%</install>
%<*ignore>
\fi
%</ignore>
%<*driver>
\NeedsTeXFormat{LaTeX2e}
\ProvidesFile{mleftright.drv}%
  [2016/05/16 v1.1 Math left/right delim. as open/close (HO)]%
\documentclass{ltxdoc}
\usepackage{holtxdoc}[2011/11/22]
\usepackage{mleftright}[2016/05/16]
\begin{document}
  \DocInput{mleftright.dtx}%
\end{document}
%</driver>
% \fi
%
%
% \CharacterTable
%  {Upper-case    \A\B\C\D\E\F\G\H\I\J\K\L\M\N\O\P\Q\R\S\T\U\V\W\X\Y\Z
%   Lower-case    \a\b\c\d\e\f\g\h\i\j\k\l\m\n\o\p\q\r\s\t\u\v\w\x\y\z
%   Digits        \0\1\2\3\4\5\6\7\8\9
%   Exclamation   \!     Double quote  \"     Hash (number) \#
%   Dollar        \$     Percent       \%     Ampersand     \&
%   Acute accent  \'     Left paren    \(     Right paren   \)
%   Asterisk      \*     Plus          \+     Comma         \,
%   Minus         \-     Point         \.     Solidus       \/
%   Colon         \:     Semicolon     \;     Less than     \<
%   Equals        \=     Greater than  \>     Question mark \?
%   Commercial at \@     Left bracket  \[     Backslash     \\
%   Right bracket \]     Circumflex    \^     Underscore    \_
%   Grave accent  \`     Left brace    \{     Vertical bar  \|
%   Right brace   \}     Tilde         \~}
%
% \GetFileInfo{mleftright.drv}
%
% \title{The \xpackage{mleftright} package}
% \date{2016/05/16 v1.1}
% \author{Heiko Oberdiek\thanks
% {Please report any issues at https://github.com/ho-tex/oberdiek/issues}\\
% \xemail{heiko.oberdiek at googlemail.com}}
%
% \maketitle
%
% \begin{abstract}
% \TeX\ sets subformulas by \cs{left} and \cs{right} as inner formulas
% with additional surrounding spaces in some situations. This package
% provides \cs{mleft} and \cs{mright} that call \cs{left} and \cs{right},
% but the delimiters will act as normal \cs{mathopen} and \cs{mathclose}
% delimiters without the additional space of an inner formula.
% \end{abstract}
%
% \tableofcontents
%
% \section{Documentation}
%
% The package is a result of a thread in the newsgroup \textsf{comp.text.tex}
% with the subject \textit{spacing after \cs{right}\texttt{)}
% and before \cs{left}\texttt{)}} \cite{dave}.
% The problem: \cs{left} and \cs{right} adjust the size of the
% delimiters automatically. However, \TeX\ treats the whole expression
% as inner formula. In some circumstances \TeX\ adds extra space
% before or after an inner formula.
% Example:
% \begin{quote}
%   \thinmuskip=1.5\thinmuskip
%   \begin{tabular}{@{}l@{\quad$\Rightarrow$\quad}l@{}}
%     |$\sin(x^2), x$|
%     & $\sin(x^2), x$\\
%     |$\sin\left(x^2\right), x$|
%     & $\sin\left(x^2\right), x$\\
%   ^^A  \multicolumn{1}{@{}r@{\quad$\Rightarrow$\quad}}{^^A
%   ^^A    \itshape with exaggerated spacing^^A
%   ^^A  }
%   ^^A  & $\thinmuskip=4\thinmuskip
%   ^^A    \sin\left(x^2\right){,}\mskip.25\thinmuskip x$\\
%     |$\sin\mleft(x^2\mright), x$|
%     & $\sin\mleft(x^2\mright), x$\\
%   \end{tabular}\\*[.5ex]
%   (\cs{mleft} and \cs{mright} are provided by this package.)
% \end{quote}
%
% In the newsgroup Donald Arseneau answered with clever macros \cite{arseneau}:
% \begin{quote}
%\begin{verbatim}
%\newcommand\lft{\mathopen{}\left}
%\newcommand\rgt{\aftergroup\mathclose\aftergroup{\aftergroup}\right}
%\end{verbatim}
% \end{quote}
% However one problem remains, a following subscript or superscript
% is not applied to the right delimiter but the empty
% \cs{mathclose}.
% Thus Philipp Stephani provided an improvement \cite{stephani}:
%\begin{quote}
%\begin{verbatim}
%\mathopen{} \mathclose{\left\| A^2 \right\|}_2
%\end{verbatim}
%\end{quote}
% Heiko Oberdiek converted this into macro form \cite{oberdiek}:
%\begin{quote}
%\begin{verbatim}
%\newcommand\lft{\mathopen{}\mathclose\bgroup\left}
%\newcommand\rgt{\aftergroup\egroup\right}
%\end{verbatim}
%\end{quote}
%
% The package uses longer macro names \cs{mleft} and \cs{mright}
% to avoid name clashes. Also it adds some checks for error conditions.
%
% \subsection{Use}
%
% \begin{declcs}{mleft}\meta{delimL} \dots\unkern\ \cs{mright}\meta{delimR}
% \end{declcs}
% Macros \cs{mleft} and \cs{mright} are used in the same way as
% \cs{left} and \cs{right}. Also \cs{middle} can be used inbetween if
% \eTeX\ is present.
%
% \begin{declcs}{mleftright}
% \end{declcs}
% Macro \cs{mleftright} redefines \cs{left} as \cs{mleft} and
% \cs{right} as \cs{mright}. The redefinition is local to the group.
%
% \begin{declcs}{mleftrightrestore}
% \end{declcs}
% Macro \cs{mleftright} restores \cs{left} and \cs{right} with
% the original meaning if they were previously redefined by
% \cs{mleftright} (also locally).
%
%
% \StopEventually{
% }
%
% \section{Implementation}
%    \begin{macrocode}
%<*package>
%    \end{macrocode}
%    Reload check, especially if the package is not used with \LaTeX.
%    \begin{macrocode}
\begingroup\catcode61\catcode48\catcode32=10\relax%
  \catcode13=5 % ^^M
  \endlinechar=13 %
  \catcode35=6 % #
  \catcode39=12 % '
  \catcode44=12 % ,
  \catcode45=12 % -
  \catcode46=12 % .
  \catcode58=12 % :
  \catcode64=11 % @
  \catcode123=1 % {
  \catcode125=2 % }
  \expandafter\let\expandafter\x\csname ver@mleftright.sty\endcsname
  \ifx\x\relax % plain-TeX, first loading
  \else
    \def\empty{}%
    \ifx\x\empty % LaTeX, first loading,
      % variable is initialized, but \ProvidesPackage not yet seen
    \else
      \expandafter\ifx\csname PackageInfo\endcsname\relax
        \def\x#1#2{%
          \immediate\write-1{Package #1 Info: #2.}%
        }%
      \else
        \def\x#1#2{\PackageInfo{#1}{#2, stopped}}%
      \fi
      \x{mleftright}{The package is already loaded}%
      \aftergroup\endinput
    \fi
  \fi
\endgroup%
%    \end{macrocode}
%    Package identification:
%    \begin{macrocode}
\begingroup\catcode61\catcode48\catcode32=10\relax%
  \catcode13=5 % ^^M
  \endlinechar=13 %
  \catcode35=6 % #
  \catcode39=12 % '
  \catcode40=12 % (
  \catcode41=12 % )
  \catcode44=12 % ,
  \catcode45=12 % -
  \catcode46=12 % .
  \catcode47=12 % /
  \catcode58=12 % :
  \catcode64=11 % @
  \catcode91=12 % [
  \catcode93=12 % ]
  \catcode123=1 % {
  \catcode125=2 % }
  \expandafter\ifx\csname ProvidesPackage\endcsname\relax
    \def\x#1#2#3[#4]{\endgroup
      \immediate\write-1{Package: #3 #4}%
      \xdef#1{#4}%
    }%
  \else
    \def\x#1#2[#3]{\endgroup
      #2[{#3}]%
      \ifx#1\@undefined
        \xdef#1{#3}%
      \fi
      \ifx#1\relax
        \xdef#1{#3}%
      \fi
    }%
  \fi
\expandafter\x\csname ver@mleftright.sty\endcsname
\ProvidesPackage{mleftright}%
  [2016/05/16 v1.1 Math left/right delim. as open/close (HO)]%
%    \end{macrocode}
%
%    \begin{macrocode}
\begingroup\catcode61\catcode48\catcode32=10\relax%
  \catcode13=5 % ^^M
  \endlinechar=13 %
  \catcode123=1 % {
  \catcode125=2 % }
  \catcode64=11 % @
  \def\x{\endgroup
    \expandafter\edef\csname mleftright@AtEnd\endcsname{%
      \endlinechar=\the\endlinechar\relax
      \catcode13=\the\catcode13\relax
      \catcode32=\the\catcode32\relax
      \catcode35=\the\catcode35\relax
      \catcode61=\the\catcode61\relax
      \catcode64=\the\catcode64\relax
      \catcode123=\the\catcode123\relax
      \catcode125=\the\catcode125\relax
    }%
  }%
\x\catcode61\catcode48\catcode32=10\relax%
\catcode13=5 % ^^M
\endlinechar=13 %
\catcode35=6 % #
\catcode64=11 % @
\catcode123=1 % {
\catcode125=2 % }
\def\TMP@EnsureCode#1#2{%
  \edef\mleftright@AtEnd{%
    \mleftright@AtEnd
    \catcode#1=\the\catcode#1\relax
  }%
  \catcode#1=#2\relax
}
\TMP@EnsureCode{38}{4}% &
\TMP@EnsureCode{39}{12}% '
\TMP@EnsureCode{40}{12}% (
\TMP@EnsureCode{41}{12}% )
\TMP@EnsureCode{42}{12}% *
\TMP@EnsureCode{43}{12}% +
\TMP@EnsureCode{44}{12}% ,
\TMP@EnsureCode{45}{12}% -
\TMP@EnsureCode{46}{12}% .
\TMP@EnsureCode{47}{12}% /
\TMP@EnsureCode{60}{12}% <
\TMP@EnsureCode{91}{12}% [
\TMP@EnsureCode{93}{12}% ]
\edef\mleftright@AtEnd{%
  \mleftright@AtEnd
  \escapechar\the\escapechar\relax
  \noexpand\endinput
}
\escapechar=92 %
%    \end{macrocode}
%
%    \begin{macrocode}
\begingroup\expandafter\expandafter\expandafter\endgroup
\expandafter\ifx\csname RequirePackage\endcsname\relax
  \input infwarerr.sty\relax
  \input ltxcmds.sty\relax
\else
  \RequirePackage{infwarerr}[2010/04/08]%
  \RequirePackage{ltxcmds}[2010/04/26]%
\fi
%    \end{macrocode}
%
%    The original commands \cs{left} and \cs{right}
%    are saved and later used in \cs{mleft} and
%    \cs{mright} in order to deal with:
%    \begin{quote}
%\begin{verbatim}
%\let\left\mleft
%\let\right\mright
%\end{verbatim}
%    \end{quote}
%    \begin{macro}{\mleftright@OrgLeft}
%    \begin{macrocode}
\let\mleftright@OrgLeft\left
%    \end{macrocode}
%    \end{macro}
%    \begin{macro}{\mleftright@OrgRight}
%    \begin{macrocode}
\let\mleftright@OrgRight\right
%    \end{macrocode}
%    \end{macro}
%
%    \begin{macro}{\mleftright@Def}
%    Macro \cs{mleftright@Def} defines a macro as robust macro
%    if \eTeX\ or \LaTeX\ is available.
%    \begin{macrocode}
\ltx@IfUndefined{protected}{%
  \ltx@IfUndefined{DeclareRobustCommand}{%
    \def\mleftright@Def{\def}%
  }{%
    \def\mleftright@Def{\DeclareRobustCommand*}%
  }%
}{%
  \def\mleftright@Def{\protected\def}%
}
\edef\mleftright@Def#1{%
  \noexpand\ltx@IfUndefined{%
    \noexpand\expandafter\noexpand\ltx@gobble\noexpand\string#1%
  }{%
    \expandafter\noexpand\mleftright@Def#1%
  }{%
    \noexpand\@PackageError{mleftright}{%
      Command \noexpand\string#1 already defined%
    }\noexpand\@ehd
    \noexpand\ltx@gobble
  }%
}
%    \end{macrocode}
%    \end{macro}
%
%    In case of \eTeX\ the group status after the left symbol
%    is saved and later checked at the beginning of \cs{mright}.
%    \begin{macrocode}
\ltx@IfUndefined{currentgrouplevel}{%
  \catcode38=14 % & = comment
}{%
  \catcode38=9 % & = ignore
}
%    \end{macrocode}
%
%    \begin{macro}{\mleftright@GroupLevel}
%    \begin{macrocode}
& \def\mleftright@GroupLevel{-1}%
%    \end{macrocode}
%    \end{macro}
%
%    \begin{macro}{\mleftright@WrongGroup}
%    \begin{macrocode}
& \def\mleftright@WrongGroup#1(#2){%
&   \ifnum\mleftright@GroupLevel<\ltx@zero
&     \@PackageError{mleftright}{%
&       Missing previous \string\mleft
&     }\@ehc
&   \else
&     \@PackageError{mleftright}{%
&       Unexpected group status for \string\mright%
&       \ifnum\mleftright@GroupLevel=#1 %
&       \else
&         .\MessageBreak
&         Group level is #1, %
&           expected is \mleftright@GroupLevel
&       \fi
&       \ifnum16=#2 %
&       \else
&         .\MessageBreak
&         Group type is #2 (%
&         \ifcase#2 %
&           bottom level%
&           \expandafter\expandafter\expandafter\ltx@gobblefour
&           \expandafter\ltx@gobbletwo
&         \or simple%
&         \or hbox%
&         \or adjusted hbox%
&         \or vbox%
&         \or vtop%
&         \or align%
&         \or no align%
&         \or output%
&         \or math%
&         \or disc%
&         \or insert%
&         \or vcenter%
&         \or math choice%
&         \or semi simple%
&         \or math shift%
&         \or math left%
&         \else
&           unknown%
&         \fi
&         \space group),\MessageBreak
&         expected is 16 (math left group)%
&       \fi
&     }\@ehd
&   \fi
& }%
%    \end{macrocode}
%    \end{macro}
%
%    \begin{macro}{\mleft}
%    \begin{macrocode}
\mleftright@Def\mleft{%
  \mathopen{}\mathclose\bgroup
& \edef\mleftright@GroupLevel{\the\numexpr\the\currentgrouplevel+1}%
  \mleftright@OrgLeft
}
%    \end{macrocode}
%    \end{macro}
%    \begin{macro}{\mright}
%    \begin{macrocode}
\mleftright@Def\mright{%
& \ifnum\mleftright@GroupLevel=\currentgrouplevel
&   \ifnum16=\currentgrouptype
      \aftergroup\egroup
&   \else
&     \expandafter\mleftright@WrongGroup
&     \the\expandafter\currentgrouplevel
&     \expandafter(\the\currentgrouptype)%
&   \fi
& \else
&   \expandafter\mleftright@WrongGroup
&   \the\expandafter\currentgrouplevel
&   \expandafter(\the\currentgrouptype)%
& \fi
  \mleftright@OrgRight
}
%    \end{macrocode}
%    \end{macro}
%
%    \begin{macro}{\mleftright}
%    \begin{macrocode}
\mleftright@Def\mleftright{%
  \let\left\mleft
  \let\right\mright
}
%    \end{macrocode}
%    \end{macro}
%
%    \begin{macro}{\mleftrightrestore}
%    \begin{macrocode}
\mleftright@Def\mleftrightrestore{%
  \ifx\left\mleft
    \let\left\mleftright@OrgLeft
  \fi
  \ifx\right\mright
    \let\right\mleftright@OrgRight
  \fi
}
%    \end{macrocode}
%    \end{macro}
%
%    \begin{macrocode}
\mleftright@AtEnd%
%</package>
%    \end{macrocode}
%
% \section{Test}
%
% \subsection{Catcode checks for loading}
%
%    \begin{macrocode}
%<*test1>
%    \end{macrocode}
%    \begin{macrocode}
\catcode`\{=1 %
\catcode`\}=2 %
\catcode`\#=6 %
\catcode`\@=11 %
\expandafter\ifx\csname count@\endcsname\relax
  \countdef\count@=255 %
\fi
\expandafter\ifx\csname @gobble\endcsname\relax
  \long\def\@gobble#1{}%
\fi
\expandafter\ifx\csname @firstofone\endcsname\relax
  \long\def\@firstofone#1{#1}%
\fi
\expandafter\ifx\csname loop\endcsname\relax
  \expandafter\@firstofone
\else
  \expandafter\@gobble
\fi
{%
  \def\loop#1\repeat{%
    \def\body{#1}%
    \iterate
  }%
  \def\iterate{%
    \body
      \let\next\iterate
    \else
      \let\next\relax
    \fi
    \next
  }%
  \let\repeat=\fi
}%
\def\RestoreCatcodes{}
\count@=0 %
\loop
  \edef\RestoreCatcodes{%
    \RestoreCatcodes
    \catcode\the\count@=\the\catcode\count@\relax
  }%
\ifnum\count@<255 %
  \advance\count@ 1 %
\repeat

\def\RangeCatcodeInvalid#1#2{%
  \count@=#1\relax
  \loop
    \catcode\count@=15 %
  \ifnum\count@<#2\relax
    \advance\count@ 1 %
  \repeat
}
\def\RangeCatcodeCheck#1#2#3{%
  \count@=#1\relax
  \loop
    \ifnum#3=\catcode\count@
    \else
      \errmessage{%
        Character \the\count@\space
        with wrong catcode \the\catcode\count@\space
        instead of \number#3%
      }%
    \fi
  \ifnum\count@<#2\relax
    \advance\count@ 1 %
  \repeat
}
\def\space{ }
\expandafter\ifx\csname LoadCommand\endcsname\relax
  \def\LoadCommand{\input mleftright.sty\relax}%
\fi
\def\Test{%
  \RangeCatcodeInvalid{0}{47}%
  \RangeCatcodeInvalid{58}{64}%
  \RangeCatcodeInvalid{91}{96}%
  \RangeCatcodeInvalid{123}{255}%
  \catcode`\@=12 %
  \catcode`\\=0 %
  \catcode`\%=14 %
  \LoadCommand
  \RangeCatcodeCheck{0}{36}{15}%
  \RangeCatcodeCheck{37}{37}{14}%
  \RangeCatcodeCheck{38}{47}{15}%
  \RangeCatcodeCheck{48}{57}{12}%
  \RangeCatcodeCheck{58}{63}{15}%
  \RangeCatcodeCheck{64}{64}{12}%
  \RangeCatcodeCheck{65}{90}{11}%
  \RangeCatcodeCheck{91}{91}{15}%
  \RangeCatcodeCheck{92}{92}{0}%
  \RangeCatcodeCheck{93}{96}{15}%
  \RangeCatcodeCheck{97}{122}{11}%
  \RangeCatcodeCheck{123}{255}{15}%
  \RestoreCatcodes
}
\Test
\csname @@end\endcsname
\end
%    \end{macrocode}
%    \begin{macrocode}
%</test1>
%    \end{macrocode}
%
% \section{Installation}
%
% \subsection{Download}
%
% \paragraph{Package.} This package is available on
% CTAN\footnote{\url{http://ctan.org/pkg/mleftright}}:
% \begin{description}
% \item[\CTAN{macros/latex/contrib/oberdiek/mleftright.dtx}] The source file.
% \item[\CTAN{macros/latex/contrib/oberdiek/mleftright.pdf}] Documentation.
% \end{description}
%
%
% \paragraph{Bundle.} All the packages of the bundle `oberdiek'
% are also available in a TDS compliant ZIP archive. There
% the packages are already unpacked and the documentation files
% are generated. The files and directories obey the TDS standard.
% \begin{description}
% \item[\CTAN{install/macros/latex/contrib/oberdiek.tds.zip}]
% \end{description}
% \emph{TDS} refers to the standard ``A Directory Structure
% for \TeX\ Files'' (\CTAN{tds/tds.pdf}). Directories
% with \xfile{texmf} in their name are usually organized this way.
%
% \subsection{Bundle installation}
%
% \paragraph{Unpacking.} Unpack the \xfile{oberdiek.tds.zip} in the
% TDS tree (also known as \xfile{texmf} tree) of your choice.
% Example (linux):
% \begin{quote}
%   |unzip oberdiek.tds.zip -d ~/texmf|
% \end{quote}
%
% \paragraph{Script installation.}
% Check the directory \xfile{TDS:scripts/oberdiek/} for
% scripts that need further installation steps.
% Package \xpackage{attachfile2} comes with the Perl script
% \xfile{pdfatfi.pl} that should be installed in such a way
% that it can be called as \texttt{pdfatfi}.
% Example (linux):
% \begin{quote}
%   |chmod +x scripts/oberdiek/pdfatfi.pl|\\
%   |cp scripts/oberdiek/pdfatfi.pl /usr/local/bin/|
% \end{quote}
%
% \subsection{Package installation}
%
% \paragraph{Unpacking.} The \xfile{.dtx} file is a self-extracting
% \docstrip\ archive. The files are extracted by running the
% \xfile{.dtx} through \plainTeX:
% \begin{quote}
%   \verb|tex mleftright.dtx|
% \end{quote}
%
% \paragraph{TDS.} Now the different files must be moved into
% the different directories in your installation TDS tree
% (also known as \xfile{texmf} tree):
% \begin{quote}
% \def\t{^^A
% \begin{tabular}{@{}>{\ttfamily}l@{ $\rightarrow$ }>{\ttfamily}l@{}}
%   mleftright.sty & tex/generic/oberdiek/mleftright.sty\\
%   mleftright.pdf & doc/latex/oberdiek/mleftright.pdf\\
%   test/mleftright-test1.tex & doc/latex/oberdiek/test/mleftright-test1.tex\\
%   mleftright.dtx & source/latex/oberdiek/mleftright.dtx\\
% \end{tabular}^^A
% }^^A
% \sbox0{\t}^^A
% \ifdim\wd0>\linewidth
%   \begingroup
%     \advance\linewidth by\leftmargin
%     \advance\linewidth by\rightmargin
%   \edef\x{\endgroup
%     \def\noexpand\lw{\the\linewidth}^^A
%   }\x
%   \def\lwbox{^^A
%     \leavevmode
%     \hbox to \linewidth{^^A
%       \kern-\leftmargin\relax
%       \hss
%       \usebox0
%       \hss
%       \kern-\rightmargin\relax
%     }^^A
%   }^^A
%   \ifdim\wd0>\lw
%     \sbox0{\small\t}^^A
%     \ifdim\wd0>\linewidth
%       \ifdim\wd0>\lw
%         \sbox0{\footnotesize\t}^^A
%         \ifdim\wd0>\linewidth
%           \ifdim\wd0>\lw
%             \sbox0{\scriptsize\t}^^A
%             \ifdim\wd0>\linewidth
%               \ifdim\wd0>\lw
%                 \sbox0{\tiny\t}^^A
%                 \ifdim\wd0>\linewidth
%                   \lwbox
%                 \else
%                   \usebox0
%                 \fi
%               \else
%                 \lwbox
%               \fi
%             \else
%               \usebox0
%             \fi
%           \else
%             \lwbox
%           \fi
%         \else
%           \usebox0
%         \fi
%       \else
%         \lwbox
%       \fi
%     \else
%       \usebox0
%     \fi
%   \else
%     \lwbox
%   \fi
% \else
%   \usebox0
% \fi
% \end{quote}
% If you have a \xfile{docstrip.cfg} that configures and enables \docstrip's
% TDS installing feature, then some files can already be in the right
% place, see the documentation of \docstrip.
%
% \subsection{Refresh file name databases}
%
% If your \TeX~distribution
% (\teTeX, \mikTeX, \dots) relies on file name databases, you must refresh
% these. For example, \teTeX\ users run \verb|texhash| or
% \verb|mktexlsr|.
%
% \subsection{Some details for the interested}
%
% \paragraph{Attached source.}
%
% The PDF documentation on CTAN also includes the
% \xfile{.dtx} source file. It can be extracted by
% AcrobatReader 6 or higher. Another option is \textsf{pdftk},
% e.g. unpack the file into the current directory:
% \begin{quote}
%   \verb|pdftk mleftright.pdf unpack_files output .|
% \end{quote}
%
% \paragraph{Unpacking with \LaTeX.}
% The \xfile{.dtx} chooses its action depending on the format:
% \begin{description}
% \item[\plainTeX:] Run \docstrip\ and extract the files.
% \item[\LaTeX:] Generate the documentation.
% \end{description}
% If you insist on using \LaTeX\ for \docstrip\ (really,
% \docstrip\ does not need \LaTeX), then inform the autodetect routine
% about your intention:
% \begin{quote}
%   \verb|latex \let\install=y% \iffalse meta-comment
%
% File: mleftright.dtx
% Version: 2016/05/16 v1.1
% Info: Math left/right delim. as open/close
%
% Copyright (C) 2010 by
%    Heiko Oberdiek <heiko.oberdiek at googlemail.com>
%    2016
%    https://github.com/ho-tex/oberdiek/issues
%
% This work may be distributed and/or modified under the
% conditions of the LaTeX Project Public License, either
% version 1.3c of this license or (at your option) any later
% version. This version of this license is in
%    http://www.latex-project.org/lppl/lppl-1-3c.txt
% and the latest version of this license is in
%    http://www.latex-project.org/lppl.txt
% and version 1.3 or later is part of all distributions of
% LaTeX version 2005/12/01 or later.
%
% This work has the LPPL maintenance status "maintained".
%
% This Current Maintainer of this work is Heiko Oberdiek.
%
% The Base Interpreter refers to any `TeX-Format',
% because some files are installed in TDS:tex/generic//.
%
% This work consists of the main source file mleftright.dtx
% and the derived files
%    mleftright.sty, mleftright.pdf, mleftright.ins, mleftright.drv,
%    mleftright-test1.tex.
%
% Distribution:
%    CTAN:macros/latex/contrib/oberdiek/mleftright.dtx
%    CTAN:macros/latex/contrib/oberdiek/mleftright.pdf
%
% Unpacking:
%    (a) If mleftright.ins is present:
%           tex mleftright.ins
%    (b) Without mleftright.ins:
%           tex mleftright.dtx
%    (c) If you insist on using LaTeX
%           latex \let\install=y% \iffalse meta-comment
%
% File: mleftright.dtx
% Version: 2016/05/16 v1.1
% Info: Math left/right delim. as open/close
%
% Copyright (C) 2010 by
%    Heiko Oberdiek <heiko.oberdiek at googlemail.com>
%    2016
%    https://github.com/ho-tex/oberdiek/issues
%
% This work may be distributed and/or modified under the
% conditions of the LaTeX Project Public License, either
% version 1.3c of this license or (at your option) any later
% version. This version of this license is in
%    http://www.latex-project.org/lppl/lppl-1-3c.txt
% and the latest version of this license is in
%    http://www.latex-project.org/lppl.txt
% and version 1.3 or later is part of all distributions of
% LaTeX version 2005/12/01 or later.
%
% This work has the LPPL maintenance status "maintained".
%
% This Current Maintainer of this work is Heiko Oberdiek.
%
% The Base Interpreter refers to any `TeX-Format',
% because some files are installed in TDS:tex/generic//.
%
% This work consists of the main source file mleftright.dtx
% and the derived files
%    mleftright.sty, mleftright.pdf, mleftright.ins, mleftright.drv,
%    mleftright-test1.tex.
%
% Distribution:
%    CTAN:macros/latex/contrib/oberdiek/mleftright.dtx
%    CTAN:macros/latex/contrib/oberdiek/mleftright.pdf
%
% Unpacking:
%    (a) If mleftright.ins is present:
%           tex mleftright.ins
%    (b) Without mleftright.ins:
%           tex mleftright.dtx
%    (c) If you insist on using LaTeX
%           latex \let\install=y\input{mleftright.dtx}
%        (quote the arguments according to the demands of your shell)
%
% Documentation:
%    (a) If mleftright.drv is present:
%           latex mleftright.drv
%    (b) Without mleftright.drv:
%           latex mleftright.dtx; ...
%    The class ltxdoc loads the configuration file ltxdoc.cfg
%    if available. Here you can specify further options, e.g.
%    use A4 as paper format:
%       \PassOptionsToClass{a4paper}{article}
%
%    Programm calls to get the documentation (example):
%       pdflatex mleftright.dtx
%       makeindex -s gind.ist mleftright.idx
%       pdflatex mleftright.dtx
%       makeindex -s gind.ist mleftright.idx
%       pdflatex mleftright.dtx
%
% Installation:
%    TDS:tex/generic/oberdiek/mleftright.sty
%    TDS:doc/latex/oberdiek/mleftright.pdf
%    TDS:doc/latex/oberdiek/test/mleftright-test1.tex
%    TDS:source/latex/oberdiek/mleftright.dtx
%
%<*ignore>
\begingroup
  \catcode123=1 %
  \catcode125=2 %
  \def\x{LaTeX2e}%
\expandafter\endgroup
\ifcase 0\ifx\install y1\fi\expandafter
         \ifx\csname processbatchFile\endcsname\relax\else1\fi
         \ifx\fmtname\x\else 1\fi\relax
\else\csname fi\endcsname
%</ignore>
%<*install>
\input docstrip.tex
\Msg{************************************************************************}
\Msg{* Installation}
\Msg{* Package: mleftright 2016/05/16 v1.1 Math left/right delim. as open/close (HO)}
\Msg{************************************************************************}

\keepsilent
\askforoverwritefalse

\let\MetaPrefix\relax
\preamble

This is a generated file.

Project: mleftright
Version: 2016/05/16 v1.1

Copyright (C) 2010 by
   Heiko Oberdiek <heiko.oberdiek at googlemail.com>

This work may be distributed and/or modified under the
conditions of the LaTeX Project Public License, either
version 1.3c of this license or (at your option) any later
version. This version of this license is in
   http://www.latex-project.org/lppl/lppl-1-3c.txt
and the latest version of this license is in
   http://www.latex-project.org/lppl.txt
and version 1.3 or later is part of all distributions of
LaTeX version 2005/12/01 or later.

This work has the LPPL maintenance status "maintained".

This Current Maintainer of this work is Heiko Oberdiek.

The Base Interpreter refers to any `TeX-Format',
because some files are installed in TDS:tex/generic//.

This work consists of the main source file mleftright.dtx
and the derived files
   mleftright.sty, mleftright.pdf, mleftright.ins, mleftright.drv,
   mleftright-test1.tex.

\endpreamble
\let\MetaPrefix\DoubleperCent

\generate{%
  \file{mleftright.ins}{\from{mleftright.dtx}{install}}%
  \file{mleftright.drv}{\from{mleftright.dtx}{driver}}%
  \usedir{tex/generic/oberdiek}%
  \file{mleftright.sty}{\from{mleftright.dtx}{package}}%
  \usedir{doc/latex/oberdiek/test}%
  \file{mleftright-test1.tex}{\from{mleftright.dtx}{test1}}%
  \nopreamble
  \nopostamble
  \usedir{source/latex/oberdiek/catalogue}%
  \file{mleftright.xml}{\from{mleftright.dtx}{catalogue}}%
}

\catcode32=13\relax% active space
\let =\space%
\Msg{************************************************************************}
\Msg{*}
\Msg{* To finish the installation you have to move the following}
\Msg{* file into a directory searched by TeX:}
\Msg{*}
\Msg{*     mleftright.sty}
\Msg{*}
\Msg{* To produce the documentation run the file `mleftright.drv'}
\Msg{* through LaTeX.}
\Msg{*}
\Msg{* Happy TeXing!}
\Msg{*}
\Msg{************************************************************************}

\endbatchfile
%</install>
%<*ignore>
\fi
%</ignore>
%<*driver>
\NeedsTeXFormat{LaTeX2e}
\ProvidesFile{mleftright.drv}%
  [2016/05/16 v1.1 Math left/right delim. as open/close (HO)]%
\documentclass{ltxdoc}
\usepackage{holtxdoc}[2011/11/22]
\usepackage{mleftright}[2016/05/16]
\begin{document}
  \DocInput{mleftright.dtx}%
\end{document}
%</driver>
% \fi
%
%
% \CharacterTable
%  {Upper-case    \A\B\C\D\E\F\G\H\I\J\K\L\M\N\O\P\Q\R\S\T\U\V\W\X\Y\Z
%   Lower-case    \a\b\c\d\e\f\g\h\i\j\k\l\m\n\o\p\q\r\s\t\u\v\w\x\y\z
%   Digits        \0\1\2\3\4\5\6\7\8\9
%   Exclamation   \!     Double quote  \"     Hash (number) \#
%   Dollar        \$     Percent       \%     Ampersand     \&
%   Acute accent  \'     Left paren    \(     Right paren   \)
%   Asterisk      \*     Plus          \+     Comma         \,
%   Minus         \-     Point         \.     Solidus       \/
%   Colon         \:     Semicolon     \;     Less than     \<
%   Equals        \=     Greater than  \>     Question mark \?
%   Commercial at \@     Left bracket  \[     Backslash     \\
%   Right bracket \]     Circumflex    \^     Underscore    \_
%   Grave accent  \`     Left brace    \{     Vertical bar  \|
%   Right brace   \}     Tilde         \~}
%
% \GetFileInfo{mleftright.drv}
%
% \title{The \xpackage{mleftright} package}
% \date{2016/05/16 v1.1}
% \author{Heiko Oberdiek\thanks
% {Please report any issues at https://github.com/ho-tex/oberdiek/issues}\\
% \xemail{heiko.oberdiek at googlemail.com}}
%
% \maketitle
%
% \begin{abstract}
% \TeX\ sets subformulas by \cs{left} and \cs{right} as inner formulas
% with additional surrounding spaces in some situations. This package
% provides \cs{mleft} and \cs{mright} that call \cs{left} and \cs{right},
% but the delimiters will act as normal \cs{mathopen} and \cs{mathclose}
% delimiters without the additional space of an inner formula.
% \end{abstract}
%
% \tableofcontents
%
% \section{Documentation}
%
% The package is a result of a thread in the newsgroup \textsf{comp.text.tex}
% with the subject \textit{spacing after \cs{right}\texttt{)}
% and before \cs{left}\texttt{)}} \cite{dave}.
% The problem: \cs{left} and \cs{right} adjust the size of the
% delimiters automatically. However, \TeX\ treats the whole expression
% as inner formula. In some circumstances \TeX\ adds extra space
% before or after an inner formula.
% Example:
% \begin{quote}
%   \thinmuskip=1.5\thinmuskip
%   \begin{tabular}{@{}l@{\quad$\Rightarrow$\quad}l@{}}
%     |$\sin(x^2), x$|
%     & $\sin(x^2), x$\\
%     |$\sin\left(x^2\right), x$|
%     & $\sin\left(x^2\right), x$\\
%   ^^A  \multicolumn{1}{@{}r@{\quad$\Rightarrow$\quad}}{^^A
%   ^^A    \itshape with exaggerated spacing^^A
%   ^^A  }
%   ^^A  & $\thinmuskip=4\thinmuskip
%   ^^A    \sin\left(x^2\right){,}\mskip.25\thinmuskip x$\\
%     |$\sin\mleft(x^2\mright), x$|
%     & $\sin\mleft(x^2\mright), x$\\
%   \end{tabular}\\*[.5ex]
%   (\cs{mleft} and \cs{mright} are provided by this package.)
% \end{quote}
%
% In the newsgroup Donald Arseneau answered with clever macros \cite{arseneau}:
% \begin{quote}
%\begin{verbatim}
%\newcommand\lft{\mathopen{}\left}
%\newcommand\rgt{\aftergroup\mathclose\aftergroup{\aftergroup}\right}
%\end{verbatim}
% \end{quote}
% However one problem remains, a following subscript or superscript
% is not applied to the right delimiter but the empty
% \cs{mathclose}.
% Thus Philipp Stephani provided an improvement \cite{stephani}:
%\begin{quote}
%\begin{verbatim}
%\mathopen{} \mathclose{\left\| A^2 \right\|}_2
%\end{verbatim}
%\end{quote}
% Heiko Oberdiek converted this into macro form \cite{oberdiek}:
%\begin{quote}
%\begin{verbatim}
%\newcommand\lft{\mathopen{}\mathclose\bgroup\left}
%\newcommand\rgt{\aftergroup\egroup\right}
%\end{verbatim}
%\end{quote}
%
% The package uses longer macro names \cs{mleft} and \cs{mright}
% to avoid name clashes. Also it adds some checks for error conditions.
%
% \subsection{Use}
%
% \begin{declcs}{mleft}\meta{delimL} \dots\unkern\ \cs{mright}\meta{delimR}
% \end{declcs}
% Macros \cs{mleft} and \cs{mright} are used in the same way as
% \cs{left} and \cs{right}. Also \cs{middle} can be used inbetween if
% \eTeX\ is present.
%
% \begin{declcs}{mleftright}
% \end{declcs}
% Macro \cs{mleftright} redefines \cs{left} as \cs{mleft} and
% \cs{right} as \cs{mright}. The redefinition is local to the group.
%
% \begin{declcs}{mleftrightrestore}
% \end{declcs}
% Macro \cs{mleftright} restores \cs{left} and \cs{right} with
% the original meaning if they were previously redefined by
% \cs{mleftright} (also locally).
%
%
% \StopEventually{
% }
%
% \section{Implementation}
%    \begin{macrocode}
%<*package>
%    \end{macrocode}
%    Reload check, especially if the package is not used with \LaTeX.
%    \begin{macrocode}
\begingroup\catcode61\catcode48\catcode32=10\relax%
  \catcode13=5 % ^^M
  \endlinechar=13 %
  \catcode35=6 % #
  \catcode39=12 % '
  \catcode44=12 % ,
  \catcode45=12 % -
  \catcode46=12 % .
  \catcode58=12 % :
  \catcode64=11 % @
  \catcode123=1 % {
  \catcode125=2 % }
  \expandafter\let\expandafter\x\csname ver@mleftright.sty\endcsname
  \ifx\x\relax % plain-TeX, first loading
  \else
    \def\empty{}%
    \ifx\x\empty % LaTeX, first loading,
      % variable is initialized, but \ProvidesPackage not yet seen
    \else
      \expandafter\ifx\csname PackageInfo\endcsname\relax
        \def\x#1#2{%
          \immediate\write-1{Package #1 Info: #2.}%
        }%
      \else
        \def\x#1#2{\PackageInfo{#1}{#2, stopped}}%
      \fi
      \x{mleftright}{The package is already loaded}%
      \aftergroup\endinput
    \fi
  \fi
\endgroup%
%    \end{macrocode}
%    Package identification:
%    \begin{macrocode}
\begingroup\catcode61\catcode48\catcode32=10\relax%
  \catcode13=5 % ^^M
  \endlinechar=13 %
  \catcode35=6 % #
  \catcode39=12 % '
  \catcode40=12 % (
  \catcode41=12 % )
  \catcode44=12 % ,
  \catcode45=12 % -
  \catcode46=12 % .
  \catcode47=12 % /
  \catcode58=12 % :
  \catcode64=11 % @
  \catcode91=12 % [
  \catcode93=12 % ]
  \catcode123=1 % {
  \catcode125=2 % }
  \expandafter\ifx\csname ProvidesPackage\endcsname\relax
    \def\x#1#2#3[#4]{\endgroup
      \immediate\write-1{Package: #3 #4}%
      \xdef#1{#4}%
    }%
  \else
    \def\x#1#2[#3]{\endgroup
      #2[{#3}]%
      \ifx#1\@undefined
        \xdef#1{#3}%
      \fi
      \ifx#1\relax
        \xdef#1{#3}%
      \fi
    }%
  \fi
\expandafter\x\csname ver@mleftright.sty\endcsname
\ProvidesPackage{mleftright}%
  [2016/05/16 v1.1 Math left/right delim. as open/close (HO)]%
%    \end{macrocode}
%
%    \begin{macrocode}
\begingroup\catcode61\catcode48\catcode32=10\relax%
  \catcode13=5 % ^^M
  \endlinechar=13 %
  \catcode123=1 % {
  \catcode125=2 % }
  \catcode64=11 % @
  \def\x{\endgroup
    \expandafter\edef\csname mleftright@AtEnd\endcsname{%
      \endlinechar=\the\endlinechar\relax
      \catcode13=\the\catcode13\relax
      \catcode32=\the\catcode32\relax
      \catcode35=\the\catcode35\relax
      \catcode61=\the\catcode61\relax
      \catcode64=\the\catcode64\relax
      \catcode123=\the\catcode123\relax
      \catcode125=\the\catcode125\relax
    }%
  }%
\x\catcode61\catcode48\catcode32=10\relax%
\catcode13=5 % ^^M
\endlinechar=13 %
\catcode35=6 % #
\catcode64=11 % @
\catcode123=1 % {
\catcode125=2 % }
\def\TMP@EnsureCode#1#2{%
  \edef\mleftright@AtEnd{%
    \mleftright@AtEnd
    \catcode#1=\the\catcode#1\relax
  }%
  \catcode#1=#2\relax
}
\TMP@EnsureCode{38}{4}% &
\TMP@EnsureCode{39}{12}% '
\TMP@EnsureCode{40}{12}% (
\TMP@EnsureCode{41}{12}% )
\TMP@EnsureCode{42}{12}% *
\TMP@EnsureCode{43}{12}% +
\TMP@EnsureCode{44}{12}% ,
\TMP@EnsureCode{45}{12}% -
\TMP@EnsureCode{46}{12}% .
\TMP@EnsureCode{47}{12}% /
\TMP@EnsureCode{60}{12}% <
\TMP@EnsureCode{91}{12}% [
\TMP@EnsureCode{93}{12}% ]
\edef\mleftright@AtEnd{%
  \mleftright@AtEnd
  \escapechar\the\escapechar\relax
  \noexpand\endinput
}
\escapechar=92 %
%    \end{macrocode}
%
%    \begin{macrocode}
\begingroup\expandafter\expandafter\expandafter\endgroup
\expandafter\ifx\csname RequirePackage\endcsname\relax
  \input infwarerr.sty\relax
  \input ltxcmds.sty\relax
\else
  \RequirePackage{infwarerr}[2010/04/08]%
  \RequirePackage{ltxcmds}[2010/04/26]%
\fi
%    \end{macrocode}
%
%    The original commands \cs{left} and \cs{right}
%    are saved and later used in \cs{mleft} and
%    \cs{mright} in order to deal with:
%    \begin{quote}
%\begin{verbatim}
%\let\left\mleft
%\let\right\mright
%\end{verbatim}
%    \end{quote}
%    \begin{macro}{\mleftright@OrgLeft}
%    \begin{macrocode}
\let\mleftright@OrgLeft\left
%    \end{macrocode}
%    \end{macro}
%    \begin{macro}{\mleftright@OrgRight}
%    \begin{macrocode}
\let\mleftright@OrgRight\right
%    \end{macrocode}
%    \end{macro}
%
%    \begin{macro}{\mleftright@Def}
%    Macro \cs{mleftright@Def} defines a macro as robust macro
%    if \eTeX\ or \LaTeX\ is available.
%    \begin{macrocode}
\ltx@IfUndefined{protected}{%
  \ltx@IfUndefined{DeclareRobustCommand}{%
    \def\mleftright@Def{\def}%
  }{%
    \def\mleftright@Def{\DeclareRobustCommand*}%
  }%
}{%
  \def\mleftright@Def{\protected\def}%
}
\edef\mleftright@Def#1{%
  \noexpand\ltx@IfUndefined{%
    \noexpand\expandafter\noexpand\ltx@gobble\noexpand\string#1%
  }{%
    \expandafter\noexpand\mleftright@Def#1%
  }{%
    \noexpand\@PackageError{mleftright}{%
      Command \noexpand\string#1 already defined%
    }\noexpand\@ehd
    \noexpand\ltx@gobble
  }%
}
%    \end{macrocode}
%    \end{macro}
%
%    In case of \eTeX\ the group status after the left symbol
%    is saved and later checked at the beginning of \cs{mright}.
%    \begin{macrocode}
\ltx@IfUndefined{currentgrouplevel}{%
  \catcode38=14 % & = comment
}{%
  \catcode38=9 % & = ignore
}
%    \end{macrocode}
%
%    \begin{macro}{\mleftright@GroupLevel}
%    \begin{macrocode}
& \def\mleftright@GroupLevel{-1}%
%    \end{macrocode}
%    \end{macro}
%
%    \begin{macro}{\mleftright@WrongGroup}
%    \begin{macrocode}
& \def\mleftright@WrongGroup#1(#2){%
&   \ifnum\mleftright@GroupLevel<\ltx@zero
&     \@PackageError{mleftright}{%
&       Missing previous \string\mleft
&     }\@ehc
&   \else
&     \@PackageError{mleftright}{%
&       Unexpected group status for \string\mright%
&       \ifnum\mleftright@GroupLevel=#1 %
&       \else
&         .\MessageBreak
&         Group level is #1, %
&           expected is \mleftright@GroupLevel
&       \fi
&       \ifnum16=#2 %
&       \else
&         .\MessageBreak
&         Group type is #2 (%
&         \ifcase#2 %
&           bottom level%
&           \expandafter\expandafter\expandafter\ltx@gobblefour
&           \expandafter\ltx@gobbletwo
&         \or simple%
&         \or hbox%
&         \or adjusted hbox%
&         \or vbox%
&         \or vtop%
&         \or align%
&         \or no align%
&         \or output%
&         \or math%
&         \or disc%
&         \or insert%
&         \or vcenter%
&         \or math choice%
&         \or semi simple%
&         \or math shift%
&         \or math left%
&         \else
&           unknown%
&         \fi
&         \space group),\MessageBreak
&         expected is 16 (math left group)%
&       \fi
&     }\@ehd
&   \fi
& }%
%    \end{macrocode}
%    \end{macro}
%
%    \begin{macro}{\mleft}
%    \begin{macrocode}
\mleftright@Def\mleft{%
  \mathopen{}\mathclose\bgroup
& \edef\mleftright@GroupLevel{\the\numexpr\the\currentgrouplevel+1}%
  \mleftright@OrgLeft
}
%    \end{macrocode}
%    \end{macro}
%    \begin{macro}{\mright}
%    \begin{macrocode}
\mleftright@Def\mright{%
& \ifnum\mleftright@GroupLevel=\currentgrouplevel
&   \ifnum16=\currentgrouptype
      \aftergroup\egroup
&   \else
&     \expandafter\mleftright@WrongGroup
&     \the\expandafter\currentgrouplevel
&     \expandafter(\the\currentgrouptype)%
&   \fi
& \else
&   \expandafter\mleftright@WrongGroup
&   \the\expandafter\currentgrouplevel
&   \expandafter(\the\currentgrouptype)%
& \fi
  \mleftright@OrgRight
}
%    \end{macrocode}
%    \end{macro}
%
%    \begin{macro}{\mleftright}
%    \begin{macrocode}
\mleftright@Def\mleftright{%
  \let\left\mleft
  \let\right\mright
}
%    \end{macrocode}
%    \end{macro}
%
%    \begin{macro}{\mleftrightrestore}
%    \begin{macrocode}
\mleftright@Def\mleftrightrestore{%
  \ifx\left\mleft
    \let\left\mleftright@OrgLeft
  \fi
  \ifx\right\mright
    \let\right\mleftright@OrgRight
  \fi
}
%    \end{macrocode}
%    \end{macro}
%
%    \begin{macrocode}
\mleftright@AtEnd%
%</package>
%    \end{macrocode}
%
% \section{Test}
%
% \subsection{Catcode checks for loading}
%
%    \begin{macrocode}
%<*test1>
%    \end{macrocode}
%    \begin{macrocode}
\catcode`\{=1 %
\catcode`\}=2 %
\catcode`\#=6 %
\catcode`\@=11 %
\expandafter\ifx\csname count@\endcsname\relax
  \countdef\count@=255 %
\fi
\expandafter\ifx\csname @gobble\endcsname\relax
  \long\def\@gobble#1{}%
\fi
\expandafter\ifx\csname @firstofone\endcsname\relax
  \long\def\@firstofone#1{#1}%
\fi
\expandafter\ifx\csname loop\endcsname\relax
  \expandafter\@firstofone
\else
  \expandafter\@gobble
\fi
{%
  \def\loop#1\repeat{%
    \def\body{#1}%
    \iterate
  }%
  \def\iterate{%
    \body
      \let\next\iterate
    \else
      \let\next\relax
    \fi
    \next
  }%
  \let\repeat=\fi
}%
\def\RestoreCatcodes{}
\count@=0 %
\loop
  \edef\RestoreCatcodes{%
    \RestoreCatcodes
    \catcode\the\count@=\the\catcode\count@\relax
  }%
\ifnum\count@<255 %
  \advance\count@ 1 %
\repeat

\def\RangeCatcodeInvalid#1#2{%
  \count@=#1\relax
  \loop
    \catcode\count@=15 %
  \ifnum\count@<#2\relax
    \advance\count@ 1 %
  \repeat
}
\def\RangeCatcodeCheck#1#2#3{%
  \count@=#1\relax
  \loop
    \ifnum#3=\catcode\count@
    \else
      \errmessage{%
        Character \the\count@\space
        with wrong catcode \the\catcode\count@\space
        instead of \number#3%
      }%
    \fi
  \ifnum\count@<#2\relax
    \advance\count@ 1 %
  \repeat
}
\def\space{ }
\expandafter\ifx\csname LoadCommand\endcsname\relax
  \def\LoadCommand{\input mleftright.sty\relax}%
\fi
\def\Test{%
  \RangeCatcodeInvalid{0}{47}%
  \RangeCatcodeInvalid{58}{64}%
  \RangeCatcodeInvalid{91}{96}%
  \RangeCatcodeInvalid{123}{255}%
  \catcode`\@=12 %
  \catcode`\\=0 %
  \catcode`\%=14 %
  \LoadCommand
  \RangeCatcodeCheck{0}{36}{15}%
  \RangeCatcodeCheck{37}{37}{14}%
  \RangeCatcodeCheck{38}{47}{15}%
  \RangeCatcodeCheck{48}{57}{12}%
  \RangeCatcodeCheck{58}{63}{15}%
  \RangeCatcodeCheck{64}{64}{12}%
  \RangeCatcodeCheck{65}{90}{11}%
  \RangeCatcodeCheck{91}{91}{15}%
  \RangeCatcodeCheck{92}{92}{0}%
  \RangeCatcodeCheck{93}{96}{15}%
  \RangeCatcodeCheck{97}{122}{11}%
  \RangeCatcodeCheck{123}{255}{15}%
  \RestoreCatcodes
}
\Test
\csname @@end\endcsname
\end
%    \end{macrocode}
%    \begin{macrocode}
%</test1>
%    \end{macrocode}
%
% \section{Installation}
%
% \subsection{Download}
%
% \paragraph{Package.} This package is available on
% CTAN\footnote{\url{http://ctan.org/pkg/mleftright}}:
% \begin{description}
% \item[\CTAN{macros/latex/contrib/oberdiek/mleftright.dtx}] The source file.
% \item[\CTAN{macros/latex/contrib/oberdiek/mleftright.pdf}] Documentation.
% \end{description}
%
%
% \paragraph{Bundle.} All the packages of the bundle `oberdiek'
% are also available in a TDS compliant ZIP archive. There
% the packages are already unpacked and the documentation files
% are generated. The files and directories obey the TDS standard.
% \begin{description}
% \item[\CTAN{install/macros/latex/contrib/oberdiek.tds.zip}]
% \end{description}
% \emph{TDS} refers to the standard ``A Directory Structure
% for \TeX\ Files'' (\CTAN{tds/tds.pdf}). Directories
% with \xfile{texmf} in their name are usually organized this way.
%
% \subsection{Bundle installation}
%
% \paragraph{Unpacking.} Unpack the \xfile{oberdiek.tds.zip} in the
% TDS tree (also known as \xfile{texmf} tree) of your choice.
% Example (linux):
% \begin{quote}
%   |unzip oberdiek.tds.zip -d ~/texmf|
% \end{quote}
%
% \paragraph{Script installation.}
% Check the directory \xfile{TDS:scripts/oberdiek/} for
% scripts that need further installation steps.
% Package \xpackage{attachfile2} comes with the Perl script
% \xfile{pdfatfi.pl} that should be installed in such a way
% that it can be called as \texttt{pdfatfi}.
% Example (linux):
% \begin{quote}
%   |chmod +x scripts/oberdiek/pdfatfi.pl|\\
%   |cp scripts/oberdiek/pdfatfi.pl /usr/local/bin/|
% \end{quote}
%
% \subsection{Package installation}
%
% \paragraph{Unpacking.} The \xfile{.dtx} file is a self-extracting
% \docstrip\ archive. The files are extracted by running the
% \xfile{.dtx} through \plainTeX:
% \begin{quote}
%   \verb|tex mleftright.dtx|
% \end{quote}
%
% \paragraph{TDS.} Now the different files must be moved into
% the different directories in your installation TDS tree
% (also known as \xfile{texmf} tree):
% \begin{quote}
% \def\t{^^A
% \begin{tabular}{@{}>{\ttfamily}l@{ $\rightarrow$ }>{\ttfamily}l@{}}
%   mleftright.sty & tex/generic/oberdiek/mleftright.sty\\
%   mleftright.pdf & doc/latex/oberdiek/mleftright.pdf\\
%   test/mleftright-test1.tex & doc/latex/oberdiek/test/mleftright-test1.tex\\
%   mleftright.dtx & source/latex/oberdiek/mleftright.dtx\\
% \end{tabular}^^A
% }^^A
% \sbox0{\t}^^A
% \ifdim\wd0>\linewidth
%   \begingroup
%     \advance\linewidth by\leftmargin
%     \advance\linewidth by\rightmargin
%   \edef\x{\endgroup
%     \def\noexpand\lw{\the\linewidth}^^A
%   }\x
%   \def\lwbox{^^A
%     \leavevmode
%     \hbox to \linewidth{^^A
%       \kern-\leftmargin\relax
%       \hss
%       \usebox0
%       \hss
%       \kern-\rightmargin\relax
%     }^^A
%   }^^A
%   \ifdim\wd0>\lw
%     \sbox0{\small\t}^^A
%     \ifdim\wd0>\linewidth
%       \ifdim\wd0>\lw
%         \sbox0{\footnotesize\t}^^A
%         \ifdim\wd0>\linewidth
%           \ifdim\wd0>\lw
%             \sbox0{\scriptsize\t}^^A
%             \ifdim\wd0>\linewidth
%               \ifdim\wd0>\lw
%                 \sbox0{\tiny\t}^^A
%                 \ifdim\wd0>\linewidth
%                   \lwbox
%                 \else
%                   \usebox0
%                 \fi
%               \else
%                 \lwbox
%               \fi
%             \else
%               \usebox0
%             \fi
%           \else
%             \lwbox
%           \fi
%         \else
%           \usebox0
%         \fi
%       \else
%         \lwbox
%       \fi
%     \else
%       \usebox0
%     \fi
%   \else
%     \lwbox
%   \fi
% \else
%   \usebox0
% \fi
% \end{quote}
% If you have a \xfile{docstrip.cfg} that configures and enables \docstrip's
% TDS installing feature, then some files can already be in the right
% place, see the documentation of \docstrip.
%
% \subsection{Refresh file name databases}
%
% If your \TeX~distribution
% (\teTeX, \mikTeX, \dots) relies on file name databases, you must refresh
% these. For example, \teTeX\ users run \verb|texhash| or
% \verb|mktexlsr|.
%
% \subsection{Some details for the interested}
%
% \paragraph{Attached source.}
%
% The PDF documentation on CTAN also includes the
% \xfile{.dtx} source file. It can be extracted by
% AcrobatReader 6 or higher. Another option is \textsf{pdftk},
% e.g. unpack the file into the current directory:
% \begin{quote}
%   \verb|pdftk mleftright.pdf unpack_files output .|
% \end{quote}
%
% \paragraph{Unpacking with \LaTeX.}
% The \xfile{.dtx} chooses its action depending on the format:
% \begin{description}
% \item[\plainTeX:] Run \docstrip\ and extract the files.
% \item[\LaTeX:] Generate the documentation.
% \end{description}
% If you insist on using \LaTeX\ for \docstrip\ (really,
% \docstrip\ does not need \LaTeX), then inform the autodetect routine
% about your intention:
% \begin{quote}
%   \verb|latex \let\install=y\input{mleftright.dtx}|
% \end{quote}
% Do not forget to quote the argument according to the demands
% of your shell.
%
% \paragraph{Generating the documentation.}
% You can use both the \xfile{.dtx} or the \xfile{.drv} to generate
% the documentation. The process can be configured by the
% configuration file \xfile{ltxdoc.cfg}. For instance, put this
% line into this file, if you want to have A4 as paper format:
% \begin{quote}
%   \verb|\PassOptionsToClass{a4paper}{article}|
% \end{quote}
% An example follows how to generate the
% documentation with pdf\LaTeX:
% \begin{quote}
%\begin{verbatim}
%pdflatex mleftright.dtx
%makeindex -s gind.ist mleftright.idx
%pdflatex mleftright.dtx
%makeindex -s gind.ist mleftright.idx
%pdflatex mleftright.dtx
%\end{verbatim}
% \end{quote}
%
% \section{Catalogue}
%
% The following XML file can be used as source for the
% \href{http://mirror.ctan.org/help/Catalogue/catalogue.html}{\TeX\ Catalogue}.
% The elements \texttt{caption} and \texttt{description} are imported
% from the original XML file from the Catalogue.
% The name of the XML file in the Catalogue is \xfile{mleftright.xml}.
%    \begin{macrocode}
%<*catalogue>
<?xml version='1.0' encoding='us-ascii'?>
<!DOCTYPE entry SYSTEM 'catalogue.dtd'>
<entry datestamp='$Date$' modifier='$Author$' id='mleftright'>
  <name>mleftright</name>
  <caption>Variants of delimiters that act as maths open/close.</caption>
  <authorref id='auth:oberdiek'/>
  <copyright owner='Heiko Oberdiek' year='2010'/>
  <license type='lppl1.3'/>
  <version number='1.1'/>
  <description>
    The package defines variants <tt>\mleft</tt> and <tt>\mright</tt>
    of <tt>\left</tt> and <tt>\right</tt>, that make the delimiters
    act as <tt>\mathopen</tt> and <tt>\mathclose</tt>.  These commands
    address spacing difficulties in subformulas.
    <p/>
    The package is part of the <xref refid='oberdiek'>oberdiek</xref> bundle.
  </description>
  <documentation details='Package documentation'
      href='ctan:/macros/latex/contrib/oberdiek/mleftright.pdf'/>
  <ctan file='true' path='/macros/latex/contrib/oberdiek/mleftright.dtx'/>
  <miktex location='oberdiek'/>
  <texlive location='oberdiek'/>
  <install path='/macros/latex/contrib/oberdiek/oberdiek.tds.zip'/>
</entry>
%</catalogue>
%    \end{macrocode}
%
% \section{Acknowledgement}
%
% \begin{description}
% \item[Donald Arsenau:]
% He provided the main trick and the first macros.
% \item[Philipp Stephani:]
% He solved the subscript problem.
% \end{description}
%
% \begin{thebibliography}{9}
% \raggedright
% \bibitem{dave}
%   Dave94705,
%   \textit{spacing after \cs{right}\texttt{)} and before \cs{left}\texttt{)}},
%   newsgroup comp.text.tex,
%   Message-ID: \texttt{\small 5d264909-7c3d-4c9d-9b22-434178b2bf90@g21g2000prn.googlegroups.com},
%   2010-08-12.
%   \newblock
%   {\small\url{http://groups.google.com/group/comp.text.tex/msg/e5b6833da7dc29bf}}
%
% \bibitem{arseneau}
%   Donald Arseneau,
%   \textit{Re: spacing after \cs{right}\texttt) and before \cs{left}\texttt)},
%   newsgroup comp.text.tex,
%   Message-ID: \texttt{\small yfivd6svl8y.fsf@mutant.triumf.ca},
%   2010-08-30.
%   \newblock
%   {\small\url{http://groups.google.com/group/comp.text.tex/msg/e0b2e4386e5d04e4}}
%
% \bibitem{stephani}
%   Philipp Stephani,
%   \textit{Re: spacing after \cs{right}\texttt) and before \cs{left}\texttt)},
%   newsgroup comp.text.tex,
%   Message-ID: \texttt{\small 4c8c8c1e\$0\$6981\$9b4e6d93@newsspool4.arcor-online.net},
%   2010-09-12.
%   \newblock
%   {\small\url{http://groups.google.com/group/comp.text.tex/msg/87ac1f61321de3ef}}
%
% \bibitem{oberdiek}
%   Heiko Oberdiek,
%   \textit{Re: spacing after \cs{right}\texttt) and before \cs{left}\texttt)},
%   newsgroup comp.text.tex,
%   Message-ID: \texttt{\small i6jcc2\$8of\$1@news.eternal-september.org},
%   2010-09-12.
%   \newblock
%   {\small\url{http://groups.google.com/group/comp.text.tex/msg/257aa6119bef878b}}
%
% \end{thebibliography}
%
% \begin{History}
%   \begin{Version}{2010/09/25 v1.0}
%   \item
%     The first version.
%   \end{Version}
%   \begin{Version}{2016/05/16 v1.1}
%   \item
%     Documentation updates.
%   \end{Version}
% \end{History}
%
% \PrintIndex
%
% \Finale
\endinput

%        (quote the arguments according to the demands of your shell)
%
% Documentation:
%    (a) If mleftright.drv is present:
%           latex mleftright.drv
%    (b) Without mleftright.drv:
%           latex mleftright.dtx; ...
%    The class ltxdoc loads the configuration file ltxdoc.cfg
%    if available. Here you can specify further options, e.g.
%    use A4 as paper format:
%       \PassOptionsToClass{a4paper}{article}
%
%    Programm calls to get the documentation (example):
%       pdflatex mleftright.dtx
%       makeindex -s gind.ist mleftright.idx
%       pdflatex mleftright.dtx
%       makeindex -s gind.ist mleftright.idx
%       pdflatex mleftright.dtx
%
% Installation:
%    TDS:tex/generic/oberdiek/mleftright.sty
%    TDS:doc/latex/oberdiek/mleftright.pdf
%    TDS:doc/latex/oberdiek/test/mleftright-test1.tex
%    TDS:source/latex/oberdiek/mleftright.dtx
%
%<*ignore>
\begingroup
  \catcode123=1 %
  \catcode125=2 %
  \def\x{LaTeX2e}%
\expandafter\endgroup
\ifcase 0\ifx\install y1\fi\expandafter
         \ifx\csname processbatchFile\endcsname\relax\else1\fi
         \ifx\fmtname\x\else 1\fi\relax
\else\csname fi\endcsname
%</ignore>
%<*install>
\input docstrip.tex
\Msg{************************************************************************}
\Msg{* Installation}
\Msg{* Package: mleftright 2016/05/16 v1.1 Math left/right delim. as open/close (HO)}
\Msg{************************************************************************}

\keepsilent
\askforoverwritefalse

\let\MetaPrefix\relax
\preamble

This is a generated file.

Project: mleftright
Version: 2016/05/16 v1.1

Copyright (C) 2010 by
   Heiko Oberdiek <heiko.oberdiek at googlemail.com>

This work may be distributed and/or modified under the
conditions of the LaTeX Project Public License, either
version 1.3c of this license or (at your option) any later
version. This version of this license is in
   http://www.latex-project.org/lppl/lppl-1-3c.txt
and the latest version of this license is in
   http://www.latex-project.org/lppl.txt
and version 1.3 or later is part of all distributions of
LaTeX version 2005/12/01 or later.

This work has the LPPL maintenance status "maintained".

This Current Maintainer of this work is Heiko Oberdiek.

The Base Interpreter refers to any `TeX-Format',
because some files are installed in TDS:tex/generic//.

This work consists of the main source file mleftright.dtx
and the derived files
   mleftright.sty, mleftright.pdf, mleftright.ins, mleftright.drv,
   mleftright-test1.tex.

\endpreamble
\let\MetaPrefix\DoubleperCent

\generate{%
  \file{mleftright.ins}{\from{mleftright.dtx}{install}}%
  \file{mleftright.drv}{\from{mleftright.dtx}{driver}}%
  \usedir{tex/generic/oberdiek}%
  \file{mleftright.sty}{\from{mleftright.dtx}{package}}%
  \usedir{doc/latex/oberdiek/test}%
  \file{mleftright-test1.tex}{\from{mleftright.dtx}{test1}}%
  \nopreamble
  \nopostamble
  \usedir{source/latex/oberdiek/catalogue}%
  \file{mleftright.xml}{\from{mleftright.dtx}{catalogue}}%
}

\catcode32=13\relax% active space
\let =\space%
\Msg{************************************************************************}
\Msg{*}
\Msg{* To finish the installation you have to move the following}
\Msg{* file into a directory searched by TeX:}
\Msg{*}
\Msg{*     mleftright.sty}
\Msg{*}
\Msg{* To produce the documentation run the file `mleftright.drv'}
\Msg{* through LaTeX.}
\Msg{*}
\Msg{* Happy TeXing!}
\Msg{*}
\Msg{************************************************************************}

\endbatchfile
%</install>
%<*ignore>
\fi
%</ignore>
%<*driver>
\NeedsTeXFormat{LaTeX2e}
\ProvidesFile{mleftright.drv}%
  [2016/05/16 v1.1 Math left/right delim. as open/close (HO)]%
\documentclass{ltxdoc}
\usepackage{holtxdoc}[2011/11/22]
\usepackage{mleftright}[2016/05/16]
\begin{document}
  \DocInput{mleftright.dtx}%
\end{document}
%</driver>
% \fi
%
%
% \CharacterTable
%  {Upper-case    \A\B\C\D\E\F\G\H\I\J\K\L\M\N\O\P\Q\R\S\T\U\V\W\X\Y\Z
%   Lower-case    \a\b\c\d\e\f\g\h\i\j\k\l\m\n\o\p\q\r\s\t\u\v\w\x\y\z
%   Digits        \0\1\2\3\4\5\6\7\8\9
%   Exclamation   \!     Double quote  \"     Hash (number) \#
%   Dollar        \$     Percent       \%     Ampersand     \&
%   Acute accent  \'     Left paren    \(     Right paren   \)
%   Asterisk      \*     Plus          \+     Comma         \,
%   Minus         \-     Point         \.     Solidus       \/
%   Colon         \:     Semicolon     \;     Less than     \<
%   Equals        \=     Greater than  \>     Question mark \?
%   Commercial at \@     Left bracket  \[     Backslash     \\
%   Right bracket \]     Circumflex    \^     Underscore    \_
%   Grave accent  \`     Left brace    \{     Vertical bar  \|
%   Right brace   \}     Tilde         \~}
%
% \GetFileInfo{mleftright.drv}
%
% \title{The \xpackage{mleftright} package}
% \date{2016/05/16 v1.1}
% \author{Heiko Oberdiek\thanks
% {Please report any issues at https://github.com/ho-tex/oberdiek/issues}\\
% \xemail{heiko.oberdiek at googlemail.com}}
%
% \maketitle
%
% \begin{abstract}
% \TeX\ sets subformulas by \cs{left} and \cs{right} as inner formulas
% with additional surrounding spaces in some situations. This package
% provides \cs{mleft} and \cs{mright} that call \cs{left} and \cs{right},
% but the delimiters will act as normal \cs{mathopen} and \cs{mathclose}
% delimiters without the additional space of an inner formula.
% \end{abstract}
%
% \tableofcontents
%
% \section{Documentation}
%
% The package is a result of a thread in the newsgroup \textsf{comp.text.tex}
% with the subject \textit{spacing after \cs{right}\texttt{)}
% and before \cs{left}\texttt{)}} \cite{dave}.
% The problem: \cs{left} and \cs{right} adjust the size of the
% delimiters automatically. However, \TeX\ treats the whole expression
% as inner formula. In some circumstances \TeX\ adds extra space
% before or after an inner formula.
% Example:
% \begin{quote}
%   \thinmuskip=1.5\thinmuskip
%   \begin{tabular}{@{}l@{\quad$\Rightarrow$\quad}l@{}}
%     |$\sin(x^2), x$|
%     & $\sin(x^2), x$\\
%     |$\sin\left(x^2\right), x$|
%     & $\sin\left(x^2\right), x$\\
%   ^^A  \multicolumn{1}{@{}r@{\quad$\Rightarrow$\quad}}{^^A
%   ^^A    \itshape with exaggerated spacing^^A
%   ^^A  }
%   ^^A  & $\thinmuskip=4\thinmuskip
%   ^^A    \sin\left(x^2\right){,}\mskip.25\thinmuskip x$\\
%     |$\sin\mleft(x^2\mright), x$|
%     & $\sin\mleft(x^2\mright), x$\\
%   \end{tabular}\\*[.5ex]
%   (\cs{mleft} and \cs{mright} are provided by this package.)
% \end{quote}
%
% In the newsgroup Donald Arseneau answered with clever macros \cite{arseneau}:
% \begin{quote}
%\begin{verbatim}
%\newcommand\lft{\mathopen{}\left}
%\newcommand\rgt{\aftergroup\mathclose\aftergroup{\aftergroup}\right}
%\end{verbatim}
% \end{quote}
% However one problem remains, a following subscript or superscript
% is not applied to the right delimiter but the empty
% \cs{mathclose}.
% Thus Philipp Stephani provided an improvement \cite{stephani}:
%\begin{quote}
%\begin{verbatim}
%\mathopen{} \mathclose{\left\| A^2 \right\|}_2
%\end{verbatim}
%\end{quote}
% Heiko Oberdiek converted this into macro form \cite{oberdiek}:
%\begin{quote}
%\begin{verbatim}
%\newcommand\lft{\mathopen{}\mathclose\bgroup\left}
%\newcommand\rgt{\aftergroup\egroup\right}
%\end{verbatim}
%\end{quote}
%
% The package uses longer macro names \cs{mleft} and \cs{mright}
% to avoid name clashes. Also it adds some checks for error conditions.
%
% \subsection{Use}
%
% \begin{declcs}{mleft}\meta{delimL} \dots\unkern\ \cs{mright}\meta{delimR}
% \end{declcs}
% Macros \cs{mleft} and \cs{mright} are used in the same way as
% \cs{left} and \cs{right}. Also \cs{middle} can be used inbetween if
% \eTeX\ is present.
%
% \begin{declcs}{mleftright}
% \end{declcs}
% Macro \cs{mleftright} redefines \cs{left} as \cs{mleft} and
% \cs{right} as \cs{mright}. The redefinition is local to the group.
%
% \begin{declcs}{mleftrightrestore}
% \end{declcs}
% Macro \cs{mleftright} restores \cs{left} and \cs{right} with
% the original meaning if they were previously redefined by
% \cs{mleftright} (also locally).
%
%
% \StopEventually{
% }
%
% \section{Implementation}
%    \begin{macrocode}
%<*package>
%    \end{macrocode}
%    Reload check, especially if the package is not used with \LaTeX.
%    \begin{macrocode}
\begingroup\catcode61\catcode48\catcode32=10\relax%
  \catcode13=5 % ^^M
  \endlinechar=13 %
  \catcode35=6 % #
  \catcode39=12 % '
  \catcode44=12 % ,
  \catcode45=12 % -
  \catcode46=12 % .
  \catcode58=12 % :
  \catcode64=11 % @
  \catcode123=1 % {
  \catcode125=2 % }
  \expandafter\let\expandafter\x\csname ver@mleftright.sty\endcsname
  \ifx\x\relax % plain-TeX, first loading
  \else
    \def\empty{}%
    \ifx\x\empty % LaTeX, first loading,
      % variable is initialized, but \ProvidesPackage not yet seen
    \else
      \expandafter\ifx\csname PackageInfo\endcsname\relax
        \def\x#1#2{%
          \immediate\write-1{Package #1 Info: #2.}%
        }%
      \else
        \def\x#1#2{\PackageInfo{#1}{#2, stopped}}%
      \fi
      \x{mleftright}{The package is already loaded}%
      \aftergroup\endinput
    \fi
  \fi
\endgroup%
%    \end{macrocode}
%    Package identification:
%    \begin{macrocode}
\begingroup\catcode61\catcode48\catcode32=10\relax%
  \catcode13=5 % ^^M
  \endlinechar=13 %
  \catcode35=6 % #
  \catcode39=12 % '
  \catcode40=12 % (
  \catcode41=12 % )
  \catcode44=12 % ,
  \catcode45=12 % -
  \catcode46=12 % .
  \catcode47=12 % /
  \catcode58=12 % :
  \catcode64=11 % @
  \catcode91=12 % [
  \catcode93=12 % ]
  \catcode123=1 % {
  \catcode125=2 % }
  \expandafter\ifx\csname ProvidesPackage\endcsname\relax
    \def\x#1#2#3[#4]{\endgroup
      \immediate\write-1{Package: #3 #4}%
      \xdef#1{#4}%
    }%
  \else
    \def\x#1#2[#3]{\endgroup
      #2[{#3}]%
      \ifx#1\@undefined
        \xdef#1{#3}%
      \fi
      \ifx#1\relax
        \xdef#1{#3}%
      \fi
    }%
  \fi
\expandafter\x\csname ver@mleftright.sty\endcsname
\ProvidesPackage{mleftright}%
  [2016/05/16 v1.1 Math left/right delim. as open/close (HO)]%
%    \end{macrocode}
%
%    \begin{macrocode}
\begingroup\catcode61\catcode48\catcode32=10\relax%
  \catcode13=5 % ^^M
  \endlinechar=13 %
  \catcode123=1 % {
  \catcode125=2 % }
  \catcode64=11 % @
  \def\x{\endgroup
    \expandafter\edef\csname mleftright@AtEnd\endcsname{%
      \endlinechar=\the\endlinechar\relax
      \catcode13=\the\catcode13\relax
      \catcode32=\the\catcode32\relax
      \catcode35=\the\catcode35\relax
      \catcode61=\the\catcode61\relax
      \catcode64=\the\catcode64\relax
      \catcode123=\the\catcode123\relax
      \catcode125=\the\catcode125\relax
    }%
  }%
\x\catcode61\catcode48\catcode32=10\relax%
\catcode13=5 % ^^M
\endlinechar=13 %
\catcode35=6 % #
\catcode64=11 % @
\catcode123=1 % {
\catcode125=2 % }
\def\TMP@EnsureCode#1#2{%
  \edef\mleftright@AtEnd{%
    \mleftright@AtEnd
    \catcode#1=\the\catcode#1\relax
  }%
  \catcode#1=#2\relax
}
\TMP@EnsureCode{38}{4}% &
\TMP@EnsureCode{39}{12}% '
\TMP@EnsureCode{40}{12}% (
\TMP@EnsureCode{41}{12}% )
\TMP@EnsureCode{42}{12}% *
\TMP@EnsureCode{43}{12}% +
\TMP@EnsureCode{44}{12}% ,
\TMP@EnsureCode{45}{12}% -
\TMP@EnsureCode{46}{12}% .
\TMP@EnsureCode{47}{12}% /
\TMP@EnsureCode{60}{12}% <
\TMP@EnsureCode{91}{12}% [
\TMP@EnsureCode{93}{12}% ]
\edef\mleftright@AtEnd{%
  \mleftright@AtEnd
  \escapechar\the\escapechar\relax
  \noexpand\endinput
}
\escapechar=92 %
%    \end{macrocode}
%
%    \begin{macrocode}
\begingroup\expandafter\expandafter\expandafter\endgroup
\expandafter\ifx\csname RequirePackage\endcsname\relax
  \input infwarerr.sty\relax
  \input ltxcmds.sty\relax
\else
  \RequirePackage{infwarerr}[2010/04/08]%
  \RequirePackage{ltxcmds}[2010/04/26]%
\fi
%    \end{macrocode}
%
%    The original commands \cs{left} and \cs{right}
%    are saved and later used in \cs{mleft} and
%    \cs{mright} in order to deal with:
%    \begin{quote}
%\begin{verbatim}
%\let\left\mleft
%\let\right\mright
%\end{verbatim}
%    \end{quote}
%    \begin{macro}{\mleftright@OrgLeft}
%    \begin{macrocode}
\let\mleftright@OrgLeft\left
%    \end{macrocode}
%    \end{macro}
%    \begin{macro}{\mleftright@OrgRight}
%    \begin{macrocode}
\let\mleftright@OrgRight\right
%    \end{macrocode}
%    \end{macro}
%
%    \begin{macro}{\mleftright@Def}
%    Macro \cs{mleftright@Def} defines a macro as robust macro
%    if \eTeX\ or \LaTeX\ is available.
%    \begin{macrocode}
\ltx@IfUndefined{protected}{%
  \ltx@IfUndefined{DeclareRobustCommand}{%
    \def\mleftright@Def{\def}%
  }{%
    \def\mleftright@Def{\DeclareRobustCommand*}%
  }%
}{%
  \def\mleftright@Def{\protected\def}%
}
\edef\mleftright@Def#1{%
  \noexpand\ltx@IfUndefined{%
    \noexpand\expandafter\noexpand\ltx@gobble\noexpand\string#1%
  }{%
    \expandafter\noexpand\mleftright@Def#1%
  }{%
    \noexpand\@PackageError{mleftright}{%
      Command \noexpand\string#1 already defined%
    }\noexpand\@ehd
    \noexpand\ltx@gobble
  }%
}
%    \end{macrocode}
%    \end{macro}
%
%    In case of \eTeX\ the group status after the left symbol
%    is saved and later checked at the beginning of \cs{mright}.
%    \begin{macrocode}
\ltx@IfUndefined{currentgrouplevel}{%
  \catcode38=14 % & = comment
}{%
  \catcode38=9 % & = ignore
}
%    \end{macrocode}
%
%    \begin{macro}{\mleftright@GroupLevel}
%    \begin{macrocode}
& \def\mleftright@GroupLevel{-1}%
%    \end{macrocode}
%    \end{macro}
%
%    \begin{macro}{\mleftright@WrongGroup}
%    \begin{macrocode}
& \def\mleftright@WrongGroup#1(#2){%
&   \ifnum\mleftright@GroupLevel<\ltx@zero
&     \@PackageError{mleftright}{%
&       Missing previous \string\mleft
&     }\@ehc
&   \else
&     \@PackageError{mleftright}{%
&       Unexpected group status for \string\mright%
&       \ifnum\mleftright@GroupLevel=#1 %
&       \else
&         .\MessageBreak
&         Group level is #1, %
&           expected is \mleftright@GroupLevel
&       \fi
&       \ifnum16=#2 %
&       \else
&         .\MessageBreak
&         Group type is #2 (%
&         \ifcase#2 %
&           bottom level%
&           \expandafter\expandafter\expandafter\ltx@gobblefour
&           \expandafter\ltx@gobbletwo
&         \or simple%
&         \or hbox%
&         \or adjusted hbox%
&         \or vbox%
&         \or vtop%
&         \or align%
&         \or no align%
&         \or output%
&         \or math%
&         \or disc%
&         \or insert%
&         \or vcenter%
&         \or math choice%
&         \or semi simple%
&         \or math shift%
&         \or math left%
&         \else
&           unknown%
&         \fi
&         \space group),\MessageBreak
&         expected is 16 (math left group)%
&       \fi
&     }\@ehd
&   \fi
& }%
%    \end{macrocode}
%    \end{macro}
%
%    \begin{macro}{\mleft}
%    \begin{macrocode}
\mleftright@Def\mleft{%
  \mathopen{}\mathclose\bgroup
& \edef\mleftright@GroupLevel{\the\numexpr\the\currentgrouplevel+1}%
  \mleftright@OrgLeft
}
%    \end{macrocode}
%    \end{macro}
%    \begin{macro}{\mright}
%    \begin{macrocode}
\mleftright@Def\mright{%
& \ifnum\mleftright@GroupLevel=\currentgrouplevel
&   \ifnum16=\currentgrouptype
      \aftergroup\egroup
&   \else
&     \expandafter\mleftright@WrongGroup
&     \the\expandafter\currentgrouplevel
&     \expandafter(\the\currentgrouptype)%
&   \fi
& \else
&   \expandafter\mleftright@WrongGroup
&   \the\expandafter\currentgrouplevel
&   \expandafter(\the\currentgrouptype)%
& \fi
  \mleftright@OrgRight
}
%    \end{macrocode}
%    \end{macro}
%
%    \begin{macro}{\mleftright}
%    \begin{macrocode}
\mleftright@Def\mleftright{%
  \let\left\mleft
  \let\right\mright
}
%    \end{macrocode}
%    \end{macro}
%
%    \begin{macro}{\mleftrightrestore}
%    \begin{macrocode}
\mleftright@Def\mleftrightrestore{%
  \ifx\left\mleft
    \let\left\mleftright@OrgLeft
  \fi
  \ifx\right\mright
    \let\right\mleftright@OrgRight
  \fi
}
%    \end{macrocode}
%    \end{macro}
%
%    \begin{macrocode}
\mleftright@AtEnd%
%</package>
%    \end{macrocode}
%
% \section{Test}
%
% \subsection{Catcode checks for loading}
%
%    \begin{macrocode}
%<*test1>
%    \end{macrocode}
%    \begin{macrocode}
\catcode`\{=1 %
\catcode`\}=2 %
\catcode`\#=6 %
\catcode`\@=11 %
\expandafter\ifx\csname count@\endcsname\relax
  \countdef\count@=255 %
\fi
\expandafter\ifx\csname @gobble\endcsname\relax
  \long\def\@gobble#1{}%
\fi
\expandafter\ifx\csname @firstofone\endcsname\relax
  \long\def\@firstofone#1{#1}%
\fi
\expandafter\ifx\csname loop\endcsname\relax
  \expandafter\@firstofone
\else
  \expandafter\@gobble
\fi
{%
  \def\loop#1\repeat{%
    \def\body{#1}%
    \iterate
  }%
  \def\iterate{%
    \body
      \let\next\iterate
    \else
      \let\next\relax
    \fi
    \next
  }%
  \let\repeat=\fi
}%
\def\RestoreCatcodes{}
\count@=0 %
\loop
  \edef\RestoreCatcodes{%
    \RestoreCatcodes
    \catcode\the\count@=\the\catcode\count@\relax
  }%
\ifnum\count@<255 %
  \advance\count@ 1 %
\repeat

\def\RangeCatcodeInvalid#1#2{%
  \count@=#1\relax
  \loop
    \catcode\count@=15 %
  \ifnum\count@<#2\relax
    \advance\count@ 1 %
  \repeat
}
\def\RangeCatcodeCheck#1#2#3{%
  \count@=#1\relax
  \loop
    \ifnum#3=\catcode\count@
    \else
      \errmessage{%
        Character \the\count@\space
        with wrong catcode \the\catcode\count@\space
        instead of \number#3%
      }%
    \fi
  \ifnum\count@<#2\relax
    \advance\count@ 1 %
  \repeat
}
\def\space{ }
\expandafter\ifx\csname LoadCommand\endcsname\relax
  \def\LoadCommand{\input mleftright.sty\relax}%
\fi
\def\Test{%
  \RangeCatcodeInvalid{0}{47}%
  \RangeCatcodeInvalid{58}{64}%
  \RangeCatcodeInvalid{91}{96}%
  \RangeCatcodeInvalid{123}{255}%
  \catcode`\@=12 %
  \catcode`\\=0 %
  \catcode`\%=14 %
  \LoadCommand
  \RangeCatcodeCheck{0}{36}{15}%
  \RangeCatcodeCheck{37}{37}{14}%
  \RangeCatcodeCheck{38}{47}{15}%
  \RangeCatcodeCheck{48}{57}{12}%
  \RangeCatcodeCheck{58}{63}{15}%
  \RangeCatcodeCheck{64}{64}{12}%
  \RangeCatcodeCheck{65}{90}{11}%
  \RangeCatcodeCheck{91}{91}{15}%
  \RangeCatcodeCheck{92}{92}{0}%
  \RangeCatcodeCheck{93}{96}{15}%
  \RangeCatcodeCheck{97}{122}{11}%
  \RangeCatcodeCheck{123}{255}{15}%
  \RestoreCatcodes
}
\Test
\csname @@end\endcsname
\end
%    \end{macrocode}
%    \begin{macrocode}
%</test1>
%    \end{macrocode}
%
% \section{Installation}
%
% \subsection{Download}
%
% \paragraph{Package.} This package is available on
% CTAN\footnote{\url{http://ctan.org/pkg/mleftright}}:
% \begin{description}
% \item[\CTAN{macros/latex/contrib/oberdiek/mleftright.dtx}] The source file.
% \item[\CTAN{macros/latex/contrib/oberdiek/mleftright.pdf}] Documentation.
% \end{description}
%
%
% \paragraph{Bundle.} All the packages of the bundle `oberdiek'
% are also available in a TDS compliant ZIP archive. There
% the packages are already unpacked and the documentation files
% are generated. The files and directories obey the TDS standard.
% \begin{description}
% \item[\CTAN{install/macros/latex/contrib/oberdiek.tds.zip}]
% \end{description}
% \emph{TDS} refers to the standard ``A Directory Structure
% for \TeX\ Files'' (\CTAN{tds/tds.pdf}). Directories
% with \xfile{texmf} in their name are usually organized this way.
%
% \subsection{Bundle installation}
%
% \paragraph{Unpacking.} Unpack the \xfile{oberdiek.tds.zip} in the
% TDS tree (also known as \xfile{texmf} tree) of your choice.
% Example (linux):
% \begin{quote}
%   |unzip oberdiek.tds.zip -d ~/texmf|
% \end{quote}
%
% \paragraph{Script installation.}
% Check the directory \xfile{TDS:scripts/oberdiek/} for
% scripts that need further installation steps.
% Package \xpackage{attachfile2} comes with the Perl script
% \xfile{pdfatfi.pl} that should be installed in such a way
% that it can be called as \texttt{pdfatfi}.
% Example (linux):
% \begin{quote}
%   |chmod +x scripts/oberdiek/pdfatfi.pl|\\
%   |cp scripts/oberdiek/pdfatfi.pl /usr/local/bin/|
% \end{quote}
%
% \subsection{Package installation}
%
% \paragraph{Unpacking.} The \xfile{.dtx} file is a self-extracting
% \docstrip\ archive. The files are extracted by running the
% \xfile{.dtx} through \plainTeX:
% \begin{quote}
%   \verb|tex mleftright.dtx|
% \end{quote}
%
% \paragraph{TDS.} Now the different files must be moved into
% the different directories in your installation TDS tree
% (also known as \xfile{texmf} tree):
% \begin{quote}
% \def\t{^^A
% \begin{tabular}{@{}>{\ttfamily}l@{ $\rightarrow$ }>{\ttfamily}l@{}}
%   mleftright.sty & tex/generic/oberdiek/mleftright.sty\\
%   mleftright.pdf & doc/latex/oberdiek/mleftright.pdf\\
%   test/mleftright-test1.tex & doc/latex/oberdiek/test/mleftright-test1.tex\\
%   mleftright.dtx & source/latex/oberdiek/mleftright.dtx\\
% \end{tabular}^^A
% }^^A
% \sbox0{\t}^^A
% \ifdim\wd0>\linewidth
%   \begingroup
%     \advance\linewidth by\leftmargin
%     \advance\linewidth by\rightmargin
%   \edef\x{\endgroup
%     \def\noexpand\lw{\the\linewidth}^^A
%   }\x
%   \def\lwbox{^^A
%     \leavevmode
%     \hbox to \linewidth{^^A
%       \kern-\leftmargin\relax
%       \hss
%       \usebox0
%       \hss
%       \kern-\rightmargin\relax
%     }^^A
%   }^^A
%   \ifdim\wd0>\lw
%     \sbox0{\small\t}^^A
%     \ifdim\wd0>\linewidth
%       \ifdim\wd0>\lw
%         \sbox0{\footnotesize\t}^^A
%         \ifdim\wd0>\linewidth
%           \ifdim\wd0>\lw
%             \sbox0{\scriptsize\t}^^A
%             \ifdim\wd0>\linewidth
%               \ifdim\wd0>\lw
%                 \sbox0{\tiny\t}^^A
%                 \ifdim\wd0>\linewidth
%                   \lwbox
%                 \else
%                   \usebox0
%                 \fi
%               \else
%                 \lwbox
%               \fi
%             \else
%               \usebox0
%             \fi
%           \else
%             \lwbox
%           \fi
%         \else
%           \usebox0
%         \fi
%       \else
%         \lwbox
%       \fi
%     \else
%       \usebox0
%     \fi
%   \else
%     \lwbox
%   \fi
% \else
%   \usebox0
% \fi
% \end{quote}
% If you have a \xfile{docstrip.cfg} that configures and enables \docstrip's
% TDS installing feature, then some files can already be in the right
% place, see the documentation of \docstrip.
%
% \subsection{Refresh file name databases}
%
% If your \TeX~distribution
% (\teTeX, \mikTeX, \dots) relies on file name databases, you must refresh
% these. For example, \teTeX\ users run \verb|texhash| or
% \verb|mktexlsr|.
%
% \subsection{Some details for the interested}
%
% \paragraph{Attached source.}
%
% The PDF documentation on CTAN also includes the
% \xfile{.dtx} source file. It can be extracted by
% AcrobatReader 6 or higher. Another option is \textsf{pdftk},
% e.g. unpack the file into the current directory:
% \begin{quote}
%   \verb|pdftk mleftright.pdf unpack_files output .|
% \end{quote}
%
% \paragraph{Unpacking with \LaTeX.}
% The \xfile{.dtx} chooses its action depending on the format:
% \begin{description}
% \item[\plainTeX:] Run \docstrip\ and extract the files.
% \item[\LaTeX:] Generate the documentation.
% \end{description}
% If you insist on using \LaTeX\ for \docstrip\ (really,
% \docstrip\ does not need \LaTeX), then inform the autodetect routine
% about your intention:
% \begin{quote}
%   \verb|latex \let\install=y% \iffalse meta-comment
%
% File: mleftright.dtx
% Version: 2016/05/16 v1.1
% Info: Math left/right delim. as open/close
%
% Copyright (C) 2010 by
%    Heiko Oberdiek <heiko.oberdiek at googlemail.com>
%    2016
%    https://github.com/ho-tex/oberdiek/issues
%
% This work may be distributed and/or modified under the
% conditions of the LaTeX Project Public License, either
% version 1.3c of this license or (at your option) any later
% version. This version of this license is in
%    http://www.latex-project.org/lppl/lppl-1-3c.txt
% and the latest version of this license is in
%    http://www.latex-project.org/lppl.txt
% and version 1.3 or later is part of all distributions of
% LaTeX version 2005/12/01 or later.
%
% This work has the LPPL maintenance status "maintained".
%
% This Current Maintainer of this work is Heiko Oberdiek.
%
% The Base Interpreter refers to any `TeX-Format',
% because some files are installed in TDS:tex/generic//.
%
% This work consists of the main source file mleftright.dtx
% and the derived files
%    mleftright.sty, mleftright.pdf, mleftright.ins, mleftright.drv,
%    mleftright-test1.tex.
%
% Distribution:
%    CTAN:macros/latex/contrib/oberdiek/mleftright.dtx
%    CTAN:macros/latex/contrib/oberdiek/mleftright.pdf
%
% Unpacking:
%    (a) If mleftright.ins is present:
%           tex mleftright.ins
%    (b) Without mleftright.ins:
%           tex mleftright.dtx
%    (c) If you insist on using LaTeX
%           latex \let\install=y\input{mleftright.dtx}
%        (quote the arguments according to the demands of your shell)
%
% Documentation:
%    (a) If mleftright.drv is present:
%           latex mleftright.drv
%    (b) Without mleftright.drv:
%           latex mleftright.dtx; ...
%    The class ltxdoc loads the configuration file ltxdoc.cfg
%    if available. Here you can specify further options, e.g.
%    use A4 as paper format:
%       \PassOptionsToClass{a4paper}{article}
%
%    Programm calls to get the documentation (example):
%       pdflatex mleftright.dtx
%       makeindex -s gind.ist mleftright.idx
%       pdflatex mleftright.dtx
%       makeindex -s gind.ist mleftright.idx
%       pdflatex mleftright.dtx
%
% Installation:
%    TDS:tex/generic/oberdiek/mleftright.sty
%    TDS:doc/latex/oberdiek/mleftright.pdf
%    TDS:doc/latex/oberdiek/test/mleftright-test1.tex
%    TDS:source/latex/oberdiek/mleftright.dtx
%
%<*ignore>
\begingroup
  \catcode123=1 %
  \catcode125=2 %
  \def\x{LaTeX2e}%
\expandafter\endgroup
\ifcase 0\ifx\install y1\fi\expandafter
         \ifx\csname processbatchFile\endcsname\relax\else1\fi
         \ifx\fmtname\x\else 1\fi\relax
\else\csname fi\endcsname
%</ignore>
%<*install>
\input docstrip.tex
\Msg{************************************************************************}
\Msg{* Installation}
\Msg{* Package: mleftright 2016/05/16 v1.1 Math left/right delim. as open/close (HO)}
\Msg{************************************************************************}

\keepsilent
\askforoverwritefalse

\let\MetaPrefix\relax
\preamble

This is a generated file.

Project: mleftright
Version: 2016/05/16 v1.1

Copyright (C) 2010 by
   Heiko Oberdiek <heiko.oberdiek at googlemail.com>

This work may be distributed and/or modified under the
conditions of the LaTeX Project Public License, either
version 1.3c of this license or (at your option) any later
version. This version of this license is in
   http://www.latex-project.org/lppl/lppl-1-3c.txt
and the latest version of this license is in
   http://www.latex-project.org/lppl.txt
and version 1.3 or later is part of all distributions of
LaTeX version 2005/12/01 or later.

This work has the LPPL maintenance status "maintained".

This Current Maintainer of this work is Heiko Oberdiek.

The Base Interpreter refers to any `TeX-Format',
because some files are installed in TDS:tex/generic//.

This work consists of the main source file mleftright.dtx
and the derived files
   mleftright.sty, mleftright.pdf, mleftright.ins, mleftright.drv,
   mleftright-test1.tex.

\endpreamble
\let\MetaPrefix\DoubleperCent

\generate{%
  \file{mleftright.ins}{\from{mleftright.dtx}{install}}%
  \file{mleftright.drv}{\from{mleftright.dtx}{driver}}%
  \usedir{tex/generic/oberdiek}%
  \file{mleftright.sty}{\from{mleftright.dtx}{package}}%
  \usedir{doc/latex/oberdiek/test}%
  \file{mleftright-test1.tex}{\from{mleftright.dtx}{test1}}%
  \nopreamble
  \nopostamble
  \usedir{source/latex/oberdiek/catalogue}%
  \file{mleftright.xml}{\from{mleftright.dtx}{catalogue}}%
}

\catcode32=13\relax% active space
\let =\space%
\Msg{************************************************************************}
\Msg{*}
\Msg{* To finish the installation you have to move the following}
\Msg{* file into a directory searched by TeX:}
\Msg{*}
\Msg{*     mleftright.sty}
\Msg{*}
\Msg{* To produce the documentation run the file `mleftright.drv'}
\Msg{* through LaTeX.}
\Msg{*}
\Msg{* Happy TeXing!}
\Msg{*}
\Msg{************************************************************************}

\endbatchfile
%</install>
%<*ignore>
\fi
%</ignore>
%<*driver>
\NeedsTeXFormat{LaTeX2e}
\ProvidesFile{mleftright.drv}%
  [2016/05/16 v1.1 Math left/right delim. as open/close (HO)]%
\documentclass{ltxdoc}
\usepackage{holtxdoc}[2011/11/22]
\usepackage{mleftright}[2016/05/16]
\begin{document}
  \DocInput{mleftright.dtx}%
\end{document}
%</driver>
% \fi
%
%
% \CharacterTable
%  {Upper-case    \A\B\C\D\E\F\G\H\I\J\K\L\M\N\O\P\Q\R\S\T\U\V\W\X\Y\Z
%   Lower-case    \a\b\c\d\e\f\g\h\i\j\k\l\m\n\o\p\q\r\s\t\u\v\w\x\y\z
%   Digits        \0\1\2\3\4\5\6\7\8\9
%   Exclamation   \!     Double quote  \"     Hash (number) \#
%   Dollar        \$     Percent       \%     Ampersand     \&
%   Acute accent  \'     Left paren    \(     Right paren   \)
%   Asterisk      \*     Plus          \+     Comma         \,
%   Minus         \-     Point         \.     Solidus       \/
%   Colon         \:     Semicolon     \;     Less than     \<
%   Equals        \=     Greater than  \>     Question mark \?
%   Commercial at \@     Left bracket  \[     Backslash     \\
%   Right bracket \]     Circumflex    \^     Underscore    \_
%   Grave accent  \`     Left brace    \{     Vertical bar  \|
%   Right brace   \}     Tilde         \~}
%
% \GetFileInfo{mleftright.drv}
%
% \title{The \xpackage{mleftright} package}
% \date{2016/05/16 v1.1}
% \author{Heiko Oberdiek\thanks
% {Please report any issues at https://github.com/ho-tex/oberdiek/issues}\\
% \xemail{heiko.oberdiek at googlemail.com}}
%
% \maketitle
%
% \begin{abstract}
% \TeX\ sets subformulas by \cs{left} and \cs{right} as inner formulas
% with additional surrounding spaces in some situations. This package
% provides \cs{mleft} and \cs{mright} that call \cs{left} and \cs{right},
% but the delimiters will act as normal \cs{mathopen} and \cs{mathclose}
% delimiters without the additional space of an inner formula.
% \end{abstract}
%
% \tableofcontents
%
% \section{Documentation}
%
% The package is a result of a thread in the newsgroup \textsf{comp.text.tex}
% with the subject \textit{spacing after \cs{right}\texttt{)}
% and before \cs{left}\texttt{)}} \cite{dave}.
% The problem: \cs{left} and \cs{right} adjust the size of the
% delimiters automatically. However, \TeX\ treats the whole expression
% as inner formula. In some circumstances \TeX\ adds extra space
% before or after an inner formula.
% Example:
% \begin{quote}
%   \thinmuskip=1.5\thinmuskip
%   \begin{tabular}{@{}l@{\quad$\Rightarrow$\quad}l@{}}
%     |$\sin(x^2), x$|
%     & $\sin(x^2), x$\\
%     |$\sin\left(x^2\right), x$|
%     & $\sin\left(x^2\right), x$\\
%   ^^A  \multicolumn{1}{@{}r@{\quad$\Rightarrow$\quad}}{^^A
%   ^^A    \itshape with exaggerated spacing^^A
%   ^^A  }
%   ^^A  & $\thinmuskip=4\thinmuskip
%   ^^A    \sin\left(x^2\right){,}\mskip.25\thinmuskip x$\\
%     |$\sin\mleft(x^2\mright), x$|
%     & $\sin\mleft(x^2\mright), x$\\
%   \end{tabular}\\*[.5ex]
%   (\cs{mleft} and \cs{mright} are provided by this package.)
% \end{quote}
%
% In the newsgroup Donald Arseneau answered with clever macros \cite{arseneau}:
% \begin{quote}
%\begin{verbatim}
%\newcommand\lft{\mathopen{}\left}
%\newcommand\rgt{\aftergroup\mathclose\aftergroup{\aftergroup}\right}
%\end{verbatim}
% \end{quote}
% However one problem remains, a following subscript or superscript
% is not applied to the right delimiter but the empty
% \cs{mathclose}.
% Thus Philipp Stephani provided an improvement \cite{stephani}:
%\begin{quote}
%\begin{verbatim}
%\mathopen{} \mathclose{\left\| A^2 \right\|}_2
%\end{verbatim}
%\end{quote}
% Heiko Oberdiek converted this into macro form \cite{oberdiek}:
%\begin{quote}
%\begin{verbatim}
%\newcommand\lft{\mathopen{}\mathclose\bgroup\left}
%\newcommand\rgt{\aftergroup\egroup\right}
%\end{verbatim}
%\end{quote}
%
% The package uses longer macro names \cs{mleft} and \cs{mright}
% to avoid name clashes. Also it adds some checks for error conditions.
%
% \subsection{Use}
%
% \begin{declcs}{mleft}\meta{delimL} \dots\unkern\ \cs{mright}\meta{delimR}
% \end{declcs}
% Macros \cs{mleft} and \cs{mright} are used in the same way as
% \cs{left} and \cs{right}. Also \cs{middle} can be used inbetween if
% \eTeX\ is present.
%
% \begin{declcs}{mleftright}
% \end{declcs}
% Macro \cs{mleftright} redefines \cs{left} as \cs{mleft} and
% \cs{right} as \cs{mright}. The redefinition is local to the group.
%
% \begin{declcs}{mleftrightrestore}
% \end{declcs}
% Macro \cs{mleftright} restores \cs{left} and \cs{right} with
% the original meaning if they were previously redefined by
% \cs{mleftright} (also locally).
%
%
% \StopEventually{
% }
%
% \section{Implementation}
%    \begin{macrocode}
%<*package>
%    \end{macrocode}
%    Reload check, especially if the package is not used with \LaTeX.
%    \begin{macrocode}
\begingroup\catcode61\catcode48\catcode32=10\relax%
  \catcode13=5 % ^^M
  \endlinechar=13 %
  \catcode35=6 % #
  \catcode39=12 % '
  \catcode44=12 % ,
  \catcode45=12 % -
  \catcode46=12 % .
  \catcode58=12 % :
  \catcode64=11 % @
  \catcode123=1 % {
  \catcode125=2 % }
  \expandafter\let\expandafter\x\csname ver@mleftright.sty\endcsname
  \ifx\x\relax % plain-TeX, first loading
  \else
    \def\empty{}%
    \ifx\x\empty % LaTeX, first loading,
      % variable is initialized, but \ProvidesPackage not yet seen
    \else
      \expandafter\ifx\csname PackageInfo\endcsname\relax
        \def\x#1#2{%
          \immediate\write-1{Package #1 Info: #2.}%
        }%
      \else
        \def\x#1#2{\PackageInfo{#1}{#2, stopped}}%
      \fi
      \x{mleftright}{The package is already loaded}%
      \aftergroup\endinput
    \fi
  \fi
\endgroup%
%    \end{macrocode}
%    Package identification:
%    \begin{macrocode}
\begingroup\catcode61\catcode48\catcode32=10\relax%
  \catcode13=5 % ^^M
  \endlinechar=13 %
  \catcode35=6 % #
  \catcode39=12 % '
  \catcode40=12 % (
  \catcode41=12 % )
  \catcode44=12 % ,
  \catcode45=12 % -
  \catcode46=12 % .
  \catcode47=12 % /
  \catcode58=12 % :
  \catcode64=11 % @
  \catcode91=12 % [
  \catcode93=12 % ]
  \catcode123=1 % {
  \catcode125=2 % }
  \expandafter\ifx\csname ProvidesPackage\endcsname\relax
    \def\x#1#2#3[#4]{\endgroup
      \immediate\write-1{Package: #3 #4}%
      \xdef#1{#4}%
    }%
  \else
    \def\x#1#2[#3]{\endgroup
      #2[{#3}]%
      \ifx#1\@undefined
        \xdef#1{#3}%
      \fi
      \ifx#1\relax
        \xdef#1{#3}%
      \fi
    }%
  \fi
\expandafter\x\csname ver@mleftright.sty\endcsname
\ProvidesPackage{mleftright}%
  [2016/05/16 v1.1 Math left/right delim. as open/close (HO)]%
%    \end{macrocode}
%
%    \begin{macrocode}
\begingroup\catcode61\catcode48\catcode32=10\relax%
  \catcode13=5 % ^^M
  \endlinechar=13 %
  \catcode123=1 % {
  \catcode125=2 % }
  \catcode64=11 % @
  \def\x{\endgroup
    \expandafter\edef\csname mleftright@AtEnd\endcsname{%
      \endlinechar=\the\endlinechar\relax
      \catcode13=\the\catcode13\relax
      \catcode32=\the\catcode32\relax
      \catcode35=\the\catcode35\relax
      \catcode61=\the\catcode61\relax
      \catcode64=\the\catcode64\relax
      \catcode123=\the\catcode123\relax
      \catcode125=\the\catcode125\relax
    }%
  }%
\x\catcode61\catcode48\catcode32=10\relax%
\catcode13=5 % ^^M
\endlinechar=13 %
\catcode35=6 % #
\catcode64=11 % @
\catcode123=1 % {
\catcode125=2 % }
\def\TMP@EnsureCode#1#2{%
  \edef\mleftright@AtEnd{%
    \mleftright@AtEnd
    \catcode#1=\the\catcode#1\relax
  }%
  \catcode#1=#2\relax
}
\TMP@EnsureCode{38}{4}% &
\TMP@EnsureCode{39}{12}% '
\TMP@EnsureCode{40}{12}% (
\TMP@EnsureCode{41}{12}% )
\TMP@EnsureCode{42}{12}% *
\TMP@EnsureCode{43}{12}% +
\TMP@EnsureCode{44}{12}% ,
\TMP@EnsureCode{45}{12}% -
\TMP@EnsureCode{46}{12}% .
\TMP@EnsureCode{47}{12}% /
\TMP@EnsureCode{60}{12}% <
\TMP@EnsureCode{91}{12}% [
\TMP@EnsureCode{93}{12}% ]
\edef\mleftright@AtEnd{%
  \mleftright@AtEnd
  \escapechar\the\escapechar\relax
  \noexpand\endinput
}
\escapechar=92 %
%    \end{macrocode}
%
%    \begin{macrocode}
\begingroup\expandafter\expandafter\expandafter\endgroup
\expandafter\ifx\csname RequirePackage\endcsname\relax
  \input infwarerr.sty\relax
  \input ltxcmds.sty\relax
\else
  \RequirePackage{infwarerr}[2010/04/08]%
  \RequirePackage{ltxcmds}[2010/04/26]%
\fi
%    \end{macrocode}
%
%    The original commands \cs{left} and \cs{right}
%    are saved and later used in \cs{mleft} and
%    \cs{mright} in order to deal with:
%    \begin{quote}
%\begin{verbatim}
%\let\left\mleft
%\let\right\mright
%\end{verbatim}
%    \end{quote}
%    \begin{macro}{\mleftright@OrgLeft}
%    \begin{macrocode}
\let\mleftright@OrgLeft\left
%    \end{macrocode}
%    \end{macro}
%    \begin{macro}{\mleftright@OrgRight}
%    \begin{macrocode}
\let\mleftright@OrgRight\right
%    \end{macrocode}
%    \end{macro}
%
%    \begin{macro}{\mleftright@Def}
%    Macro \cs{mleftright@Def} defines a macro as robust macro
%    if \eTeX\ or \LaTeX\ is available.
%    \begin{macrocode}
\ltx@IfUndefined{protected}{%
  \ltx@IfUndefined{DeclareRobustCommand}{%
    \def\mleftright@Def{\def}%
  }{%
    \def\mleftright@Def{\DeclareRobustCommand*}%
  }%
}{%
  \def\mleftright@Def{\protected\def}%
}
\edef\mleftright@Def#1{%
  \noexpand\ltx@IfUndefined{%
    \noexpand\expandafter\noexpand\ltx@gobble\noexpand\string#1%
  }{%
    \expandafter\noexpand\mleftright@Def#1%
  }{%
    \noexpand\@PackageError{mleftright}{%
      Command \noexpand\string#1 already defined%
    }\noexpand\@ehd
    \noexpand\ltx@gobble
  }%
}
%    \end{macrocode}
%    \end{macro}
%
%    In case of \eTeX\ the group status after the left symbol
%    is saved and later checked at the beginning of \cs{mright}.
%    \begin{macrocode}
\ltx@IfUndefined{currentgrouplevel}{%
  \catcode38=14 % & = comment
}{%
  \catcode38=9 % & = ignore
}
%    \end{macrocode}
%
%    \begin{macro}{\mleftright@GroupLevel}
%    \begin{macrocode}
& \def\mleftright@GroupLevel{-1}%
%    \end{macrocode}
%    \end{macro}
%
%    \begin{macro}{\mleftright@WrongGroup}
%    \begin{macrocode}
& \def\mleftright@WrongGroup#1(#2){%
&   \ifnum\mleftright@GroupLevel<\ltx@zero
&     \@PackageError{mleftright}{%
&       Missing previous \string\mleft
&     }\@ehc
&   \else
&     \@PackageError{mleftright}{%
&       Unexpected group status for \string\mright%
&       \ifnum\mleftright@GroupLevel=#1 %
&       \else
&         .\MessageBreak
&         Group level is #1, %
&           expected is \mleftright@GroupLevel
&       \fi
&       \ifnum16=#2 %
&       \else
&         .\MessageBreak
&         Group type is #2 (%
&         \ifcase#2 %
&           bottom level%
&           \expandafter\expandafter\expandafter\ltx@gobblefour
&           \expandafter\ltx@gobbletwo
&         \or simple%
&         \or hbox%
&         \or adjusted hbox%
&         \or vbox%
&         \or vtop%
&         \or align%
&         \or no align%
&         \or output%
&         \or math%
&         \or disc%
&         \or insert%
&         \or vcenter%
&         \or math choice%
&         \or semi simple%
&         \or math shift%
&         \or math left%
&         \else
&           unknown%
&         \fi
&         \space group),\MessageBreak
&         expected is 16 (math left group)%
&       \fi
&     }\@ehd
&   \fi
& }%
%    \end{macrocode}
%    \end{macro}
%
%    \begin{macro}{\mleft}
%    \begin{macrocode}
\mleftright@Def\mleft{%
  \mathopen{}\mathclose\bgroup
& \edef\mleftright@GroupLevel{\the\numexpr\the\currentgrouplevel+1}%
  \mleftright@OrgLeft
}
%    \end{macrocode}
%    \end{macro}
%    \begin{macro}{\mright}
%    \begin{macrocode}
\mleftright@Def\mright{%
& \ifnum\mleftright@GroupLevel=\currentgrouplevel
&   \ifnum16=\currentgrouptype
      \aftergroup\egroup
&   \else
&     \expandafter\mleftright@WrongGroup
&     \the\expandafter\currentgrouplevel
&     \expandafter(\the\currentgrouptype)%
&   \fi
& \else
&   \expandafter\mleftright@WrongGroup
&   \the\expandafter\currentgrouplevel
&   \expandafter(\the\currentgrouptype)%
& \fi
  \mleftright@OrgRight
}
%    \end{macrocode}
%    \end{macro}
%
%    \begin{macro}{\mleftright}
%    \begin{macrocode}
\mleftright@Def\mleftright{%
  \let\left\mleft
  \let\right\mright
}
%    \end{macrocode}
%    \end{macro}
%
%    \begin{macro}{\mleftrightrestore}
%    \begin{macrocode}
\mleftright@Def\mleftrightrestore{%
  \ifx\left\mleft
    \let\left\mleftright@OrgLeft
  \fi
  \ifx\right\mright
    \let\right\mleftright@OrgRight
  \fi
}
%    \end{macrocode}
%    \end{macro}
%
%    \begin{macrocode}
\mleftright@AtEnd%
%</package>
%    \end{macrocode}
%
% \section{Test}
%
% \subsection{Catcode checks for loading}
%
%    \begin{macrocode}
%<*test1>
%    \end{macrocode}
%    \begin{macrocode}
\catcode`\{=1 %
\catcode`\}=2 %
\catcode`\#=6 %
\catcode`\@=11 %
\expandafter\ifx\csname count@\endcsname\relax
  \countdef\count@=255 %
\fi
\expandafter\ifx\csname @gobble\endcsname\relax
  \long\def\@gobble#1{}%
\fi
\expandafter\ifx\csname @firstofone\endcsname\relax
  \long\def\@firstofone#1{#1}%
\fi
\expandafter\ifx\csname loop\endcsname\relax
  \expandafter\@firstofone
\else
  \expandafter\@gobble
\fi
{%
  \def\loop#1\repeat{%
    \def\body{#1}%
    \iterate
  }%
  \def\iterate{%
    \body
      \let\next\iterate
    \else
      \let\next\relax
    \fi
    \next
  }%
  \let\repeat=\fi
}%
\def\RestoreCatcodes{}
\count@=0 %
\loop
  \edef\RestoreCatcodes{%
    \RestoreCatcodes
    \catcode\the\count@=\the\catcode\count@\relax
  }%
\ifnum\count@<255 %
  \advance\count@ 1 %
\repeat

\def\RangeCatcodeInvalid#1#2{%
  \count@=#1\relax
  \loop
    \catcode\count@=15 %
  \ifnum\count@<#2\relax
    \advance\count@ 1 %
  \repeat
}
\def\RangeCatcodeCheck#1#2#3{%
  \count@=#1\relax
  \loop
    \ifnum#3=\catcode\count@
    \else
      \errmessage{%
        Character \the\count@\space
        with wrong catcode \the\catcode\count@\space
        instead of \number#3%
      }%
    \fi
  \ifnum\count@<#2\relax
    \advance\count@ 1 %
  \repeat
}
\def\space{ }
\expandafter\ifx\csname LoadCommand\endcsname\relax
  \def\LoadCommand{\input mleftright.sty\relax}%
\fi
\def\Test{%
  \RangeCatcodeInvalid{0}{47}%
  \RangeCatcodeInvalid{58}{64}%
  \RangeCatcodeInvalid{91}{96}%
  \RangeCatcodeInvalid{123}{255}%
  \catcode`\@=12 %
  \catcode`\\=0 %
  \catcode`\%=14 %
  \LoadCommand
  \RangeCatcodeCheck{0}{36}{15}%
  \RangeCatcodeCheck{37}{37}{14}%
  \RangeCatcodeCheck{38}{47}{15}%
  \RangeCatcodeCheck{48}{57}{12}%
  \RangeCatcodeCheck{58}{63}{15}%
  \RangeCatcodeCheck{64}{64}{12}%
  \RangeCatcodeCheck{65}{90}{11}%
  \RangeCatcodeCheck{91}{91}{15}%
  \RangeCatcodeCheck{92}{92}{0}%
  \RangeCatcodeCheck{93}{96}{15}%
  \RangeCatcodeCheck{97}{122}{11}%
  \RangeCatcodeCheck{123}{255}{15}%
  \RestoreCatcodes
}
\Test
\csname @@end\endcsname
\end
%    \end{macrocode}
%    \begin{macrocode}
%</test1>
%    \end{macrocode}
%
% \section{Installation}
%
% \subsection{Download}
%
% \paragraph{Package.} This package is available on
% CTAN\footnote{\url{http://ctan.org/pkg/mleftright}}:
% \begin{description}
% \item[\CTAN{macros/latex/contrib/oberdiek/mleftright.dtx}] The source file.
% \item[\CTAN{macros/latex/contrib/oberdiek/mleftright.pdf}] Documentation.
% \end{description}
%
%
% \paragraph{Bundle.} All the packages of the bundle `oberdiek'
% are also available in a TDS compliant ZIP archive. There
% the packages are already unpacked and the documentation files
% are generated. The files and directories obey the TDS standard.
% \begin{description}
% \item[\CTAN{install/macros/latex/contrib/oberdiek.tds.zip}]
% \end{description}
% \emph{TDS} refers to the standard ``A Directory Structure
% for \TeX\ Files'' (\CTAN{tds/tds.pdf}). Directories
% with \xfile{texmf} in their name are usually organized this way.
%
% \subsection{Bundle installation}
%
% \paragraph{Unpacking.} Unpack the \xfile{oberdiek.tds.zip} in the
% TDS tree (also known as \xfile{texmf} tree) of your choice.
% Example (linux):
% \begin{quote}
%   |unzip oberdiek.tds.zip -d ~/texmf|
% \end{quote}
%
% \paragraph{Script installation.}
% Check the directory \xfile{TDS:scripts/oberdiek/} for
% scripts that need further installation steps.
% Package \xpackage{attachfile2} comes with the Perl script
% \xfile{pdfatfi.pl} that should be installed in such a way
% that it can be called as \texttt{pdfatfi}.
% Example (linux):
% \begin{quote}
%   |chmod +x scripts/oberdiek/pdfatfi.pl|\\
%   |cp scripts/oberdiek/pdfatfi.pl /usr/local/bin/|
% \end{quote}
%
% \subsection{Package installation}
%
% \paragraph{Unpacking.} The \xfile{.dtx} file is a self-extracting
% \docstrip\ archive. The files are extracted by running the
% \xfile{.dtx} through \plainTeX:
% \begin{quote}
%   \verb|tex mleftright.dtx|
% \end{quote}
%
% \paragraph{TDS.} Now the different files must be moved into
% the different directories in your installation TDS tree
% (also known as \xfile{texmf} tree):
% \begin{quote}
% \def\t{^^A
% \begin{tabular}{@{}>{\ttfamily}l@{ $\rightarrow$ }>{\ttfamily}l@{}}
%   mleftright.sty & tex/generic/oberdiek/mleftright.sty\\
%   mleftright.pdf & doc/latex/oberdiek/mleftright.pdf\\
%   test/mleftright-test1.tex & doc/latex/oberdiek/test/mleftright-test1.tex\\
%   mleftright.dtx & source/latex/oberdiek/mleftright.dtx\\
% \end{tabular}^^A
% }^^A
% \sbox0{\t}^^A
% \ifdim\wd0>\linewidth
%   \begingroup
%     \advance\linewidth by\leftmargin
%     \advance\linewidth by\rightmargin
%   \edef\x{\endgroup
%     \def\noexpand\lw{\the\linewidth}^^A
%   }\x
%   \def\lwbox{^^A
%     \leavevmode
%     \hbox to \linewidth{^^A
%       \kern-\leftmargin\relax
%       \hss
%       \usebox0
%       \hss
%       \kern-\rightmargin\relax
%     }^^A
%   }^^A
%   \ifdim\wd0>\lw
%     \sbox0{\small\t}^^A
%     \ifdim\wd0>\linewidth
%       \ifdim\wd0>\lw
%         \sbox0{\footnotesize\t}^^A
%         \ifdim\wd0>\linewidth
%           \ifdim\wd0>\lw
%             \sbox0{\scriptsize\t}^^A
%             \ifdim\wd0>\linewidth
%               \ifdim\wd0>\lw
%                 \sbox0{\tiny\t}^^A
%                 \ifdim\wd0>\linewidth
%                   \lwbox
%                 \else
%                   \usebox0
%                 \fi
%               \else
%                 \lwbox
%               \fi
%             \else
%               \usebox0
%             \fi
%           \else
%             \lwbox
%           \fi
%         \else
%           \usebox0
%         \fi
%       \else
%         \lwbox
%       \fi
%     \else
%       \usebox0
%     \fi
%   \else
%     \lwbox
%   \fi
% \else
%   \usebox0
% \fi
% \end{quote}
% If you have a \xfile{docstrip.cfg} that configures and enables \docstrip's
% TDS installing feature, then some files can already be in the right
% place, see the documentation of \docstrip.
%
% \subsection{Refresh file name databases}
%
% If your \TeX~distribution
% (\teTeX, \mikTeX, \dots) relies on file name databases, you must refresh
% these. For example, \teTeX\ users run \verb|texhash| or
% \verb|mktexlsr|.
%
% \subsection{Some details for the interested}
%
% \paragraph{Attached source.}
%
% The PDF documentation on CTAN also includes the
% \xfile{.dtx} source file. It can be extracted by
% AcrobatReader 6 or higher. Another option is \textsf{pdftk},
% e.g. unpack the file into the current directory:
% \begin{quote}
%   \verb|pdftk mleftright.pdf unpack_files output .|
% \end{quote}
%
% \paragraph{Unpacking with \LaTeX.}
% The \xfile{.dtx} chooses its action depending on the format:
% \begin{description}
% \item[\plainTeX:] Run \docstrip\ and extract the files.
% \item[\LaTeX:] Generate the documentation.
% \end{description}
% If you insist on using \LaTeX\ for \docstrip\ (really,
% \docstrip\ does not need \LaTeX), then inform the autodetect routine
% about your intention:
% \begin{quote}
%   \verb|latex \let\install=y\input{mleftright.dtx}|
% \end{quote}
% Do not forget to quote the argument according to the demands
% of your shell.
%
% \paragraph{Generating the documentation.}
% You can use both the \xfile{.dtx} or the \xfile{.drv} to generate
% the documentation. The process can be configured by the
% configuration file \xfile{ltxdoc.cfg}. For instance, put this
% line into this file, if you want to have A4 as paper format:
% \begin{quote}
%   \verb|\PassOptionsToClass{a4paper}{article}|
% \end{quote}
% An example follows how to generate the
% documentation with pdf\LaTeX:
% \begin{quote}
%\begin{verbatim}
%pdflatex mleftright.dtx
%makeindex -s gind.ist mleftright.idx
%pdflatex mleftright.dtx
%makeindex -s gind.ist mleftright.idx
%pdflatex mleftright.dtx
%\end{verbatim}
% \end{quote}
%
% \section{Catalogue}
%
% The following XML file can be used as source for the
% \href{http://mirror.ctan.org/help/Catalogue/catalogue.html}{\TeX\ Catalogue}.
% The elements \texttt{caption} and \texttt{description} are imported
% from the original XML file from the Catalogue.
% The name of the XML file in the Catalogue is \xfile{mleftright.xml}.
%    \begin{macrocode}
%<*catalogue>
<?xml version='1.0' encoding='us-ascii'?>
<!DOCTYPE entry SYSTEM 'catalogue.dtd'>
<entry datestamp='$Date$' modifier='$Author$' id='mleftright'>
  <name>mleftright</name>
  <caption>Variants of delimiters that act as maths open/close.</caption>
  <authorref id='auth:oberdiek'/>
  <copyright owner='Heiko Oberdiek' year='2010'/>
  <license type='lppl1.3'/>
  <version number='1.1'/>
  <description>
    The package defines variants <tt>\mleft</tt> and <tt>\mright</tt>
    of <tt>\left</tt> and <tt>\right</tt>, that make the delimiters
    act as <tt>\mathopen</tt> and <tt>\mathclose</tt>.  These commands
    address spacing difficulties in subformulas.
    <p/>
    The package is part of the <xref refid='oberdiek'>oberdiek</xref> bundle.
  </description>
  <documentation details='Package documentation'
      href='ctan:/macros/latex/contrib/oberdiek/mleftright.pdf'/>
  <ctan file='true' path='/macros/latex/contrib/oberdiek/mleftright.dtx'/>
  <miktex location='oberdiek'/>
  <texlive location='oberdiek'/>
  <install path='/macros/latex/contrib/oberdiek/oberdiek.tds.zip'/>
</entry>
%</catalogue>
%    \end{macrocode}
%
% \section{Acknowledgement}
%
% \begin{description}
% \item[Donald Arsenau:]
% He provided the main trick and the first macros.
% \item[Philipp Stephani:]
% He solved the subscript problem.
% \end{description}
%
% \begin{thebibliography}{9}
% \raggedright
% \bibitem{dave}
%   Dave94705,
%   \textit{spacing after \cs{right}\texttt{)} and before \cs{left}\texttt{)}},
%   newsgroup comp.text.tex,
%   Message-ID: \texttt{\small 5d264909-7c3d-4c9d-9b22-434178b2bf90@g21g2000prn.googlegroups.com},
%   2010-08-12.
%   \newblock
%   {\small\url{http://groups.google.com/group/comp.text.tex/msg/e5b6833da7dc29bf}}
%
% \bibitem{arseneau}
%   Donald Arseneau,
%   \textit{Re: spacing after \cs{right}\texttt) and before \cs{left}\texttt)},
%   newsgroup comp.text.tex,
%   Message-ID: \texttt{\small yfivd6svl8y.fsf@mutant.triumf.ca},
%   2010-08-30.
%   \newblock
%   {\small\url{http://groups.google.com/group/comp.text.tex/msg/e0b2e4386e5d04e4}}
%
% \bibitem{stephani}
%   Philipp Stephani,
%   \textit{Re: spacing after \cs{right}\texttt) and before \cs{left}\texttt)},
%   newsgroup comp.text.tex,
%   Message-ID: \texttt{\small 4c8c8c1e\$0\$6981\$9b4e6d93@newsspool4.arcor-online.net},
%   2010-09-12.
%   \newblock
%   {\small\url{http://groups.google.com/group/comp.text.tex/msg/87ac1f61321de3ef}}
%
% \bibitem{oberdiek}
%   Heiko Oberdiek,
%   \textit{Re: spacing after \cs{right}\texttt) and before \cs{left}\texttt)},
%   newsgroup comp.text.tex,
%   Message-ID: \texttt{\small i6jcc2\$8of\$1@news.eternal-september.org},
%   2010-09-12.
%   \newblock
%   {\small\url{http://groups.google.com/group/comp.text.tex/msg/257aa6119bef878b}}
%
% \end{thebibliography}
%
% \begin{History}
%   \begin{Version}{2010/09/25 v1.0}
%   \item
%     The first version.
%   \end{Version}
%   \begin{Version}{2016/05/16 v1.1}
%   \item
%     Documentation updates.
%   \end{Version}
% \end{History}
%
% \PrintIndex
%
% \Finale
\endinput
|
% \end{quote}
% Do not forget to quote the argument according to the demands
% of your shell.
%
% \paragraph{Generating the documentation.}
% You can use both the \xfile{.dtx} or the \xfile{.drv} to generate
% the documentation. The process can be configured by the
% configuration file \xfile{ltxdoc.cfg}. For instance, put this
% line into this file, if you want to have A4 as paper format:
% \begin{quote}
%   \verb|\PassOptionsToClass{a4paper}{article}|
% \end{quote}
% An example follows how to generate the
% documentation with pdf\LaTeX:
% \begin{quote}
%\begin{verbatim}
%pdflatex mleftright.dtx
%makeindex -s gind.ist mleftright.idx
%pdflatex mleftright.dtx
%makeindex -s gind.ist mleftright.idx
%pdflatex mleftright.dtx
%\end{verbatim}
% \end{quote}
%
% \section{Catalogue}
%
% The following XML file can be used as source for the
% \href{http://mirror.ctan.org/help/Catalogue/catalogue.html}{\TeX\ Catalogue}.
% The elements \texttt{caption} and \texttt{description} are imported
% from the original XML file from the Catalogue.
% The name of the XML file in the Catalogue is \xfile{mleftright.xml}.
%    \begin{macrocode}
%<*catalogue>
<?xml version='1.0' encoding='us-ascii'?>
<!DOCTYPE entry SYSTEM 'catalogue.dtd'>
<entry datestamp='$Date$' modifier='$Author$' id='mleftright'>
  <name>mleftright</name>
  <caption>Variants of delimiters that act as maths open/close.</caption>
  <authorref id='auth:oberdiek'/>
  <copyright owner='Heiko Oberdiek' year='2010'/>
  <license type='lppl1.3'/>
  <version number='1.1'/>
  <description>
    The package defines variants <tt>\mleft</tt> and <tt>\mright</tt>
    of <tt>\left</tt> and <tt>\right</tt>, that make the delimiters
    act as <tt>\mathopen</tt> and <tt>\mathclose</tt>.  These commands
    address spacing difficulties in subformulas.
    <p/>
    The package is part of the <xref refid='oberdiek'>oberdiek</xref> bundle.
  </description>
  <documentation details='Package documentation'
      href='ctan:/macros/latex/contrib/oberdiek/mleftright.pdf'/>
  <ctan file='true' path='/macros/latex/contrib/oberdiek/mleftright.dtx'/>
  <miktex location='oberdiek'/>
  <texlive location='oberdiek'/>
  <install path='/macros/latex/contrib/oberdiek/oberdiek.tds.zip'/>
</entry>
%</catalogue>
%    \end{macrocode}
%
% \section{Acknowledgement}
%
% \begin{description}
% \item[Donald Arsenau:]
% He provided the main trick and the first macros.
% \item[Philipp Stephani:]
% He solved the subscript problem.
% \end{description}
%
% \begin{thebibliography}{9}
% \raggedright
% \bibitem{dave}
%   Dave94705,
%   \textit{spacing after \cs{right}\texttt{)} and before \cs{left}\texttt{)}},
%   newsgroup comp.text.tex,
%   Message-ID: \texttt{\small 5d264909-7c3d-4c9d-9b22-434178b2bf90@g21g2000prn.googlegroups.com},
%   2010-08-12.
%   \newblock
%   {\small\url{http://groups.google.com/group/comp.text.tex/msg/e5b6833da7dc29bf}}
%
% \bibitem{arseneau}
%   Donald Arseneau,
%   \textit{Re: spacing after \cs{right}\texttt) and before \cs{left}\texttt)},
%   newsgroup comp.text.tex,
%   Message-ID: \texttt{\small yfivd6svl8y.fsf@mutant.triumf.ca},
%   2010-08-30.
%   \newblock
%   {\small\url{http://groups.google.com/group/comp.text.tex/msg/e0b2e4386e5d04e4}}
%
% \bibitem{stephani}
%   Philipp Stephani,
%   \textit{Re: spacing after \cs{right}\texttt) and before \cs{left}\texttt)},
%   newsgroup comp.text.tex,
%   Message-ID: \texttt{\small 4c8c8c1e\$0\$6981\$9b4e6d93@newsspool4.arcor-online.net},
%   2010-09-12.
%   \newblock
%   {\small\url{http://groups.google.com/group/comp.text.tex/msg/87ac1f61321de3ef}}
%
% \bibitem{oberdiek}
%   Heiko Oberdiek,
%   \textit{Re: spacing after \cs{right}\texttt) and before \cs{left}\texttt)},
%   newsgroup comp.text.tex,
%   Message-ID: \texttt{\small i6jcc2\$8of\$1@news.eternal-september.org},
%   2010-09-12.
%   \newblock
%   {\small\url{http://groups.google.com/group/comp.text.tex/msg/257aa6119bef878b}}
%
% \end{thebibliography}
%
% \begin{History}
%   \begin{Version}{2010/09/25 v1.0}
%   \item
%     The first version.
%   \end{Version}
%   \begin{Version}{2016/05/16 v1.1}
%   \item
%     Documentation updates.
%   \end{Version}
% \end{History}
%
% \PrintIndex
%
% \Finale
\endinput
|
% \end{quote}
% Do not forget to quote the argument according to the demands
% of your shell.
%
% \paragraph{Generating the documentation.}
% You can use both the \xfile{.dtx} or the \xfile{.drv} to generate
% the documentation. The process can be configured by the
% configuration file \xfile{ltxdoc.cfg}. For instance, put this
% line into this file, if you want to have A4 as paper format:
% \begin{quote}
%   \verb|\PassOptionsToClass{a4paper}{article}|
% \end{quote}
% An example follows how to generate the
% documentation with pdf\LaTeX:
% \begin{quote}
%\begin{verbatim}
%pdflatex mleftright.dtx
%makeindex -s gind.ist mleftright.idx
%pdflatex mleftright.dtx
%makeindex -s gind.ist mleftright.idx
%pdflatex mleftright.dtx
%\end{verbatim}
% \end{quote}
%
% \section{Catalogue}
%
% The following XML file can be used as source for the
% \href{http://mirror.ctan.org/help/Catalogue/catalogue.html}{\TeX\ Catalogue}.
% The elements \texttt{caption} and \texttt{description} are imported
% from the original XML file from the Catalogue.
% The name of the XML file in the Catalogue is \xfile{mleftright.xml}.
%    \begin{macrocode}
%<*catalogue>
<?xml version='1.0' encoding='us-ascii'?>
<!DOCTYPE entry SYSTEM 'catalogue.dtd'>
<entry datestamp='$Date$' modifier='$Author$' id='mleftright'>
  <name>mleftright</name>
  <caption>Variants of delimiters that act as maths open/close.</caption>
  <authorref id='auth:oberdiek'/>
  <copyright owner='Heiko Oberdiek' year='2010'/>
  <license type='lppl1.3'/>
  <version number='1.1'/>
  <description>
    The package defines variants <tt>\mleft</tt> and <tt>\mright</tt>
    of <tt>\left</tt> and <tt>\right</tt>, that make the delimiters
    act as <tt>\mathopen</tt> and <tt>\mathclose</tt>.  These commands
    address spacing difficulties in subformulas.
    <p/>
    The package is part of the <xref refid='oberdiek'>oberdiek</xref> bundle.
  </description>
  <documentation details='Package documentation'
      href='ctan:/macros/latex/contrib/oberdiek/mleftright.pdf'/>
  <ctan file='true' path='/macros/latex/contrib/oberdiek/mleftright.dtx'/>
  <miktex location='oberdiek'/>
  <texlive location='oberdiek'/>
  <install path='/macros/latex/contrib/oberdiek/oberdiek.tds.zip'/>
</entry>
%</catalogue>
%    \end{macrocode}
%
% \section{Acknowledgement}
%
% \begin{description}
% \item[Donald Arsenau:]
% He provided the main trick and the first macros.
% \item[Philipp Stephani:]
% He solved the subscript problem.
% \end{description}
%
% \begin{thebibliography}{9}
% \raggedright
% \bibitem{dave}
%   Dave94705,
%   \textit{spacing after \cs{right}\texttt{)} and before \cs{left}\texttt{)}},
%   newsgroup comp.text.tex,
%   Message-ID: \texttt{\small 5d264909-7c3d-4c9d-9b22-434178b2bf90@g21g2000prn.googlegroups.com},
%   2010-08-12.
%   \newblock
%   {\small\url{http://groups.google.com/group/comp.text.tex/msg/e5b6833da7dc29bf}}
%
% \bibitem{arseneau}
%   Donald Arseneau,
%   \textit{Re: spacing after \cs{right}\texttt) and before \cs{left}\texttt)},
%   newsgroup comp.text.tex,
%   Message-ID: \texttt{\small yfivd6svl8y.fsf@mutant.triumf.ca},
%   2010-08-30.
%   \newblock
%   {\small\url{http://groups.google.com/group/comp.text.tex/msg/e0b2e4386e5d04e4}}
%
% \bibitem{stephani}
%   Philipp Stephani,
%   \textit{Re: spacing after \cs{right}\texttt) and before \cs{left}\texttt)},
%   newsgroup comp.text.tex,
%   Message-ID: \texttt{\small 4c8c8c1e\$0\$6981\$9b4e6d93@newsspool4.arcor-online.net},
%   2010-09-12.
%   \newblock
%   {\small\url{http://groups.google.com/group/comp.text.tex/msg/87ac1f61321de3ef}}
%
% \bibitem{oberdiek}
%   Heiko Oberdiek,
%   \textit{Re: spacing after \cs{right}\texttt) and before \cs{left}\texttt)},
%   newsgroup comp.text.tex,
%   Message-ID: \texttt{\small i6jcc2\$8of\$1@news.eternal-september.org},
%   2010-09-12.
%   \newblock
%   {\small\url{http://groups.google.com/group/comp.text.tex/msg/257aa6119bef878b}}
%
% \end{thebibliography}
%
% \begin{History}
%   \begin{Version}{2010/09/25 v1.0}
%   \item
%     The first version.
%   \end{Version}
%   \begin{Version}{2016/05/16 v1.1}
%   \item
%     Documentation updates.
%   \end{Version}
% \end{History}
%
% \PrintIndex
%
% \Finale
\endinput
|
% \end{quote}
% Do not forget to quote the argument according to the demands
% of your shell.
%
% \paragraph{Generating the documentation.}
% You can use both the \xfile{.dtx} or the \xfile{.drv} to generate
% the documentation. The process can be configured by the
% configuration file \xfile{ltxdoc.cfg}. For instance, put this
% line into this file, if you want to have A4 as paper format:
% \begin{quote}
%   \verb|\PassOptionsToClass{a4paper}{article}|
% \end{quote}
% An example follows how to generate the
% documentation with pdf\LaTeX:
% \begin{quote}
%\begin{verbatim}
%pdflatex mleftright.dtx
%makeindex -s gind.ist mleftright.idx
%pdflatex mleftright.dtx
%makeindex -s gind.ist mleftright.idx
%pdflatex mleftright.dtx
%\end{verbatim}
% \end{quote}
%
% \section{Catalogue}
%
% The following XML file can be used as source for the
% \href{http://mirror.ctan.org/help/Catalogue/catalogue.html}{\TeX\ Catalogue}.
% The elements \texttt{caption} and \texttt{description} are imported
% from the original XML file from the Catalogue.
% The name of the XML file in the Catalogue is \xfile{mleftright.xml}.
%    \begin{macrocode}
%<*catalogue>
<?xml version='1.0' encoding='us-ascii'?>
<!DOCTYPE entry SYSTEM 'catalogue.dtd'>
<entry datestamp='$Date$' modifier='$Author$' id='mleftright'>
  <name>mleftright</name>
  <caption>Variants of delimiters that act as maths open/close.</caption>
  <authorref id='auth:oberdiek'/>
  <copyright owner='Heiko Oberdiek' year='2010'/>
  <license type='lppl1.3'/>
  <version number='1.1'/>
  <description>
    The package defines variants <tt>\mleft</tt> and <tt>\mright</tt>
    of <tt>\left</tt> and <tt>\right</tt>, that make the delimiters
    act as <tt>\mathopen</tt> and <tt>\mathclose</tt>.  These commands
    address spacing difficulties in subformulas.
    <p/>
    The package is part of the <xref refid='oberdiek'>oberdiek</xref> bundle.
  </description>
  <documentation details='Package documentation'
      href='ctan:/macros/latex/contrib/oberdiek/mleftright.pdf'/>
  <ctan file='true' path='/macros/latex/contrib/oberdiek/mleftright.dtx'/>
  <miktex location='oberdiek'/>
  <texlive location='oberdiek'/>
  <install path='/macros/latex/contrib/oberdiek/oberdiek.tds.zip'/>
</entry>
%</catalogue>
%    \end{macrocode}
%
% \section{Acknowledgement}
%
% \begin{description}
% \item[Donald Arsenau:]
% He provided the main trick and the first macros.
% \item[Philipp Stephani:]
% He solved the subscript problem.
% \end{description}
%
% \begin{thebibliography}{9}
% \raggedright
% \bibitem{dave}
%   Dave94705,
%   \textit{spacing after \cs{right}\texttt{)} and before \cs{left}\texttt{)}},
%   newsgroup comp.text.tex,
%   Message-ID: \texttt{\small 5d264909-7c3d-4c9d-9b22-434178b2bf90@g21g2000prn.googlegroups.com},
%   2010-08-12.
%   \newblock
%   {\small\url{http://groups.google.com/group/comp.text.tex/msg/e5b6833da7dc29bf}}
%
% \bibitem{arseneau}
%   Donald Arseneau,
%   \textit{Re: spacing after \cs{right}\texttt) and before \cs{left}\texttt)},
%   newsgroup comp.text.tex,
%   Message-ID: \texttt{\small yfivd6svl8y.fsf@mutant.triumf.ca},
%   2010-08-30.
%   \newblock
%   {\small\url{http://groups.google.com/group/comp.text.tex/msg/e0b2e4386e5d04e4}}
%
% \bibitem{stephani}
%   Philipp Stephani,
%   \textit{Re: spacing after \cs{right}\texttt) and before \cs{left}\texttt)},
%   newsgroup comp.text.tex,
%   Message-ID: \texttt{\small 4c8c8c1e\$0\$6981\$9b4e6d93@newsspool4.arcor-online.net},
%   2010-09-12.
%   \newblock
%   {\small\url{http://groups.google.com/group/comp.text.tex/msg/87ac1f61321de3ef}}
%
% \bibitem{oberdiek}
%   Heiko Oberdiek,
%   \textit{Re: spacing after \cs{right}\texttt) and before \cs{left}\texttt)},
%   newsgroup comp.text.tex,
%   Message-ID: \texttt{\small i6jcc2\$8of\$1@news.eternal-september.org},
%   2010-09-12.
%   \newblock
%   {\small\url{http://groups.google.com/group/comp.text.tex/msg/257aa6119bef878b}}
%
% \end{thebibliography}
%
% \begin{History}
%   \begin{Version}{2010/09/25 v1.0}
%   \item
%     The first version.
%   \end{Version}
%   \begin{Version}{2016/05/16 v1.1}
%   \item
%     Documentation updates.
%   \end{Version}
% \end{History}
%
% \PrintIndex
%
% \Finale
\endinput
